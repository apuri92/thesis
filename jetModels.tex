% !TEX encoding = UTF-8 Unicode
% !TEX root = thesis-ex.tex

While there are a number of different observables that can be measured in heavy ion collisions, the underlying goal of these measurements is to characterize the QGP.
This makes jet energy loss models that combine dynamics of the jet as well as the QGP invaluable.
Since different jet measurements come with their own set of measurement biases and have different sensitivities, it is vital that any viable model be able to describe a variety of observables.
Models can also help guide experimentalists in their searches and suggest new directions of exploration.
Measurements can then be done to constrain such models, helping further describe the jet-QGP interaction.

%In the case of jet energy loss, If a model can predict where the lost energy is redistributed, it is possible for experimentalists to measure an observable that would be sensitive to such effects, helping constrain the model.
%This sets up a healthy feedback loop between the theory and experiemnet communities.

%Mearuements of energy loss have to be coupled with models that describe the loss but also identify where the energy is redistributed. this makes measurements that are able to identify the different 
%Models investigating jet energy loss do not have to stop at just the loss, but also have to have a description of where the lost energy went.
%This makes models like the jet fluid model
%
%There is a constant feedback between the experimental measurements and models that attempt to describe them.
This chapter specifically discusses three different models: the Jet Fluid model, the Hybrid Model, and the Effective Quenching model.
These were chosen because they have been used to describe a wide variety of observables including the jet \RAA, jet fragmentation, and the jet shape.
In particular, the Jet Fluid model and Hybrid model incorporate a rigorous description of the interactions between the jet and the QGP and describe the radial dependence of the modification of charged particles in a jet, the central topic of this thesis. 
The Effective Quenching model is more phenomenological and shows agreement with measured data using only an intuitive functional form for energy loss.

\section{Jet Fluid model}
% !TEX encoding = UTF-8 Unicode
% !TEX root = thesis-ex.tex

This discussion is based on the model introduced in Ref. \cite{Tachibana:2017syd}. This model considers the evolution of the jet and QGP in a coupled manner, considering the energy and transverse momentum exchange between them. In this picture, both the jet and medium are allowed to modify each other; the jet is modified via collisional and radiative processes while the medium evolves hydrodynamically and is modified because it picks up the energy lost by the jet. 

The time evolution of the jet is given 

\begin{align}
f_i (\omega_i, \kTsq_i, t) = \frac{dN_i (\omega_i \kTsq_i, t)}{d \omega_i d\kTsq_i}
\end{align}
where $i$ is the type of parton, $\omega_i$ is its energy, and $\kTsq$ is its transverse momentum with respect to the jet axis. Then the transport equations can be written in terms of :

\begin{align}
\label{eq:jf_transportEq}
\frac{d f_j}{dt} &= \hat{e_j} \frac{\p f_j}{\p \omega_j} + \frac{1}{4} \hat{q_j} \nabla_{\kT}^2 f_j  \\
& + \sum_i \int d\omega_i d\kTsq_i \frac{d\widetilde{\Gamma}_{i\rightarrow j} }{d\omega_j d\kTsq_j dt} f_i \\
& - \sum_i \int d\omega_i d\kTsq_i \frac{d\widetilde{\Gamma}_{j\rightarrow i} }{d\omega_ij d\kTsq_i dt} f_i \\
\end{align}
where the first term is the collisional energy loss, the second term is the transverse momentum broadening, and the last two terms are the medium induced gain and loss radiative processes respectively. The splitting processes are are given by:

\begin{align}
\frac{d\Gamma_{i \rightarrow j}}{d \omega_j d\kTsq_j dt} = \frac{2\alpha_S}{\pi} \hat{q}_g \frac{x P_{i \rightarrow j} (x)}{\omega_j {k_{\rm T}^4}_j} \sin^2 \left(\dfrac{t - t_i}{2\tau_f} \right)
\end{align}
where $P_{i \rightarrow j} $ is the vacuum splitting function for $i \rightarrow j $ with $\omega_j$ being the energy of the radiated parton, $\tau_f$ is the formation time of the radiated parton, and $\kT_j$ is the transverse momentum of the radiated parton with respect to the parent parton. These transport Equations~\ref{eq:jf_transportEq} can be solved numerically and agree with \RAA\ measurements \cite{Aad:2014bxa, Khachatryan:2016jfl, Abelev:2013kqa}. The effects of the medium are included by considering the energy-momentum conservation of the jet-QGP system $ \p_\mu [T_{\rm QGP}^{\mu\nu} + T_{\rm jet}^{\mu\nu}] = 0$. Then the source term $J^\nu(x)$ that describes the energy transfer between the jet and the medium can be defined as $J^\nu(x) \equiv -\p_\mu  T_{\rm jet}^{\mu\nu}$, making the QGP evolution being given by

\begin{align}
 \p_\mu T_{\rm QGP}^{\mu\nu} = j^\nu
\end{align}
which characterizes the energy-momentum transfer between the jet and the QGP. 

An important component of this model is the flow induced by jets. A snapshot of this is shown in Figure~\ref{fig:jf_snapshot}, where the evolution of the energy density of the medium can be seen in a sample event. A single jet travels through the QGP, and can be clearly seen in the lower panels after the energy of the medium has been subtracted out. The V shaped feature seen is the mach cone that is induced by the parton as it moves faster than the medium sound velocity. 

\begin{figure}[htbp]
\begin{center}
\includegraphics[width=0.85\textwidth]{figures/jetMeasurements/JF_snapshot}
\caption{(Top) The time evolution of the energy density of the quark gluon plasma with a jet propagating through it. (Bottom) The time evolution of the energy density in the event after the energy density of the QGP has been subtracted out. Figure taken from \cite{Tachibana:2017syd}.}
\label{fig:jf_snapshot}
\end{center}
\end{figure}

The final jet energy has two components: the jet shower, and the hydrodynamic response. The former as discussed above comprises of the collisional energy loss, momentum broadening, and medium induced radiation. The latter includes the energy lost from the jet shower that thermalizes into the medium and induces conical flow, some of which is still in the jet cone. This compensates some of the energy lost in the shower and can be seen in Figure~\ref{fig:jf_energyLoss}. While the absolute amount of energy lost increases as a function of initial jet energy, the fractional energy loss decreases. Furthermore there is a cone size dependence once the hydrodynamic contributions are included. This is a result of the jet being highly collimated, such that while an increase in the size does not change the energy much, it does affect the hydrodynamic contribution from the medium.
\begin{figure}[htbp]
\begin{center}
\includegraphics[width=0.45\textwidth]{figures/jetMeasurements/JF_energyLoss}
\caption{(Top) The energy lost by a jets of different radii as a function of their initial energy in central \pbpb\ collisions at \sqrtsnn = 2.76 TeV. Figure taken from \cite{Tachibana:2017syd}.}
\label{fig:jf_energyLoss}
\end{center}
\end{figure}

The \RAA\ distributions constructed with this model and compared to data from CMS \cite{Khachatryan:2016jfl} are shown in Figure~\ref{fig:jf_raa}. Including the hydrodynamic contribution decreases the energy loss, hence increasing the \RAA\ value and inducing a cone size dependence to the \RAA. 

\begin{figure}[htbp]
\begin{center}
\includegraphics[width=0.45\textwidth]{figures/jetMeasurements/JF_RAA}
\caption{The nuclear modification factor \RAA\ as a function of jet \pt\ as determined by the Jet-Fluid model and compared to the data measured by CMS \cite{Khachatryan:2016jfl}. The different colors represent different sized jets, with the dashed lines showing the modeled \RAA\ without the hydro-contribution. There is good agreement within the large uncertainties in the data. Figure taken from \cite{Tachibana:2017syd}.}
\label{fig:jf_raa}
\end{center}
\end{figure}


The internal structure of the jet, i.e. how energy is spread within it, can be investigated using the jet shape variable, defined as a per-jet quantity as:

\begin{align}
\rho_{\rm jet} = \frac{1}{\Njet} \sum_{\rm jet} \left[ \frac{1}{\ptjet} \frac{\sum_{\rm trk} \pttrk}{\delta r}  \right]
\end{align}
where the sum is over all jets and for all tracks around a jet in an annulus with mean radius $r$ from the jet axis. The modification in the jet structure then can be defined as:

\begin{align}
R_{\rm AA}^\rho = \dfrac{\rho_{\rm AA} (r) }{\rho_{\rm pp} (r) }
\end{align}
A comparison of the jet shape variable $\rho$ and its modification $R_{\rm AA}^\rho$ to data measured by CMS is shown in Figure~\ref{fig:JF_jetShapemodel}. The individual shower and hydro contributions are seen in Figure~\ref{fig:jf_jetshape}. These indicate that the shower contribution to the jet shape variable is falls steeply as a function of distance from the jet axis while the hydro contribution is fairly constant at large distances. This is because the energy loss from the shower is carried away by the jet induced flow to large angles.
The $R_{\rm AA}^\rho$ distribution in Figure~\ref{fig:jf_jetshapemod}, shows that the core is largely unmodified while the outer part of the jet is broadened. The hydro-contribution mainly has an effect at larger distances from the jet axis. This is consistent with the cone-size dependence seen in Figure~\ref{fig:jf_energyLoss}.


\begin{figure}
\begin{subfigure}{.45\textwidth}
  \centering
\includegraphics[width=0.95\textwidth]{figures/jetMeasurements/JF_jetShape}
\caption{The jet shape as measured by CMS for \pp\ and central \pbpb\ collisions \cite{Chatrchyan:2013kwa} compared to the Jet Fluid model. The shower (blue) and hydro (orange) contributions to the jet shape are highlighted.}
\label{fig:jf_jetshape}
\end{subfigure} \qquad
\begin{subfigure}{.45\textwidth}
  \centering
\includegraphics[width=0.95\textwidth]{figures/jetMeasurements/JF_jetShapeModification}
\caption{The modification of the jet shape between \pp\ and \pbpb\ as measured by CMS \cite{Chatrchyan:2013kwa} and compared to the Jet Fluid model. The dashed line shows the modeled modification without the hydro-contribution.}
\label{fig:jf_jetshapemod}
\end{subfigure}
\caption{Fits to CMS data. Figures taken from \cite{Tachibana:2017syd}.}
\label{fig:JF_jetShapemodel}
\end{figure}



%\begin{align}
%\frac{d f_j}{dt} =& \left( \hat{e_j} \frac{\p}{\p \omega_j} + \frac{1}{4} \hat{q_j} \nabla_{\kT}^2\right) f_j  \\
%& + \sum_i \int d\omega_i d\kTsq_i \frac{d\widetilde{\Gamma}_{i\rightarrow j} (\omega_j, \kTsq_j | \omega_i, \kTsq_i)}{d\omega_j d\kTsq dt} f_i \\
%& - \sum_i \int d\omega_i d\kTsq_i \frac{d\widetilde{\Gamma}_{j\rightarrow i} (\omega_i, \kTsq_i | \omega_j, \kTsq_j)}{d\omega_j d\kTsq dt} f_i \\
%\end{align}

\label{sec:jet_fluid}

\section{Hybrid Model}
% !TEX encoding = UTF-8 Unicode
% !TEX root = thesis-ex.tex

This discussion is based on the work in Refs. \cite{Casalderrey-Solana:2014bpa, Hulcher:2017cpt, Casalderrey-Solana:2016jvj} and describes jet quenching using a hybrid strong/weak model. It uses perturbative QCD to describe the weakly coupled hard process of jet production and holographic calculations of the energy loss of energetic probes to model the strong coupling between the probe and the plasma \cite{Chesler:2015nqz, Chesler:2014jva}. This is a combination of approaches that focus on the following extreme limits: a weakly coupled system at unrealistically high temperatures that can be treated perturbatively \cite{Jacobs:2004qv, Majumder:2010qh} and a system where the coupling constant is large at all energy scales and Gauge/string duality is applicable \cite{CasalderreySolana:2011us}.

In this model, the jet evolves in space time with the lifetime of the parton in the shower being given by \cite{CasalderreySolana:2011gx}.  

\begin{align}
\tau = 2 \frac{E}{Q^2}
\end{align}
where $Q$ is its virtuality and $E$ its energy. This evolution is unaffected before the proper time at which the plasma hydrodynamizes, $\tau_{\text{hydro}} = 0.6$ fm. After this time, the jet-plasma interaction comes into play and the fragments evolve with the energy loss as:

\begin{align}
\frac{1}{E_{\mathrm{in}}} \frac{dE}{dx} = -\frac{4}{\pi} \frac{x^2}{x_{\mathrm{stop}}^2} \frac{1}{\sqrt{x_{\mathrm{stop}}^2 - x^2}}
\end{align}
where $E_{\mathrm{in}}$ is the initial energy of the parton prior to any quenching and $x_{\mathrm{stop}}$ is its stopping distance (jet thermalization distance). The stopping distance can be written as:

\begin{align}
x_\mathrm{stop} = \frac{1}{2\kappa_\mathrm{sc}} \frac{E_\mathrm{in}^{1/3}}{T^{4/3}}
\end{align}
where $\kappa_\mathrm{sc}$ is a dimensionless free parameter associated to the strong coupling and is used to fit to the data. The energy loss is characterized by the strong  $x^2$ dependence for $x \ll x_\mathrm{stop}$. Furthermore, when $x$ is comparable to $x_\mathrm{stop}$, $dE/dx$ depends nontrivially on $E_\mathrm{in}$ and $x$, diverging for $x\rightarrow x_\mathrm{stop}$ and $E\rightarrow0$. The shower is then embedded into a hydrodynamic description of the QGP from Ref. \cite{Hirano:2010je}, and the energy loss expressions are integrated for each parton, from the time it is produced to the time that it splits. The splitting probabilities are taken to be independent of the medium, depending only on the initial energy of the daughter partons. These further lose energy as they propagate through the QGP and split. Then the total energy lost by a parton is dependent on the history of splitting and propagation of its parents, grandparents and so on and so forth. 

The partons further experience kicks transverse to their direction of motion, a phenomena called transverse momentum broadening. This effect is mainly experienced by softer partons that are much more affected by the angular narrowing effects of energy loss, making most measured observables insensitive to the size of this kick. This is directly related to wider jets losing more energy than narrower ones. The wake left in the medium from the partons depositing momentum in the QGP as they propagate through it lends a non-trivial impact to the model predictions. This wake moves in the direction of the jet and is impossible to separate out in experiments, making its inclusion to any model vital. This wake results in a perturbation to the hydrodynamic background, resulting in corrections to the final state hadron spectra. This effect is particularly important for jet substructure observables like jet fragmentation and jet shapes \cite{Casalderrey-Solana:2016jvj}.

A screening effect recently included in the hybrid model is based considering the resolving power of the QGP \cite{Hulcher:2017cpt}. As depicted in Figure~\ref{fig:hm_lres}, the QGP will only resolve daughter partons of a splitting after they are separated by a certain distance $L_\mathrm{res}$. It is only after they are resolved that they will be allowed them to lose energy independent of each other. This delayed quenching results in an enhancement of softer partons at larger angles from the jet axis compared to the case where the daughter partons are resolved immediately after they split from the parent parton. The \Lres\ parameter has the constraint $ 1/(\pi T) < \Lres < 2 /(\pt T$ based on the Debye screening length for the plasma, i.e. the length at which the QGP is able to resolve and screen color charges.

%The effects of including the \Lres parameter in the context of jet observables are discussed below.

\begin{figure}[htbp]
\begin{center}
\includegraphics[width=0.55\textwidth]{figures/jetMeasurements/HM_lres}
\caption{A schematic illustrating the resolving power of the QGP. The daughter partons $2$ and $3$ that come from $1$ need to be separated by \Lres\ before they are treated individually by the plasma. Prior to that separation, they are treated as one effective parton. Figure taken from \cite{Hulcher:2017cpt}.}
\label{fig:hm_lres}
\end{center}
\end{figure}


The free parameter $\kappa_\mathrm{sc}$ is determined by fitting to jet \RAA\ data from CMS \cite{Khachatryan:2016jfl} as shown in Figure~\ref{fig:hm_fitting}. It can be seen that including the \Lres\ parameter does not really affect the jet \RAA\ prediction. The dependence of the \RAA\ on the size of the jet radius can be seen. This is consistent with the expectation that wider jets lose more energy.

\begin{figure}[htbp]
\begin{center}
\includegraphics[width=1\textwidth]{figures/jetMeasurements/HM_raa}
\caption{The hybrid model without (left) and with (right) the \Lres\ parameter, compared to the jet \RAA\ as a function of jet \pt\ in two centrality intervals as measured in Ref. \cite{Khachatryan:2016jfl}. The different colors correspond to different jet radii. The Hybrid Model is fit to the 100--110 GeV point from the data, giving rise to the colored bands. Figure taken from \cite{Hulcher:2017cpt}. }
\label{fig:hm_fitting}
\end{center}
\end{figure}

Fixing the $\kappa_\mathrm{sc}$ parameter allows for predictions of other jet measurements like jet fragmentation and jet shape. Figures~\ref{fig:hm_ff} and \ref{fig:hm_jetshape} show a comparison of the measured and modeled values of the modifications to the jet fragmentation and jet shape respectively.  The model has also been compared to measurements done by ATLAS, ALICE, and STAR \cite{2013220, Abelev:2013kqa, RUSNAK:2014xfa} \cite{}


\begin{figure}
\begin{subfigure}{1\textwidth}
  \centering
\includegraphics[width=1\textwidth]{figures/jetMeasurements/HM_FF}
\caption{The modification to the jet fragmentation from \pp\ to \pbpb\ as a function of $\ln(1/z)$ as measured in Ref. \cite{Chatrchyan:2014ava} compared to the predictions of the hybrid model. The predictions are shown without (left) and with (right) the effect of the wake from the QGP responding to the jet. The different colors correspond to different \Lres\ parameters. Figure taken from \cite{Hulcher:2017cpt}. }
\label{fig:hm_ff}
\end{subfigure} \\ \\ \\
\begin{subfigure}{1\textwidth}
  \centering
\includegraphics[width=1\textwidth]{figures/jetMeasurements/HM_jetShape}
\caption{The modification to the jet shape from \pp\ to \pbpb\ as a function of $r$ as measured in Ref. \cite{Chatrchyan:2013kwa} compared to the predictions of the hybrid model. The predictions are shown without (left) and with (right) the effect of the wake from the QGP responding to the jet. The different colors correspond to different \Lres\ parameters. Figure taken from \cite{Hulcher:2017cpt}. }
\label{fig:hm_jetshape}
\end{subfigure}
\caption{A comparison of measured data, MC, and the analytic calculation of the EQ model. Figure taken from \cite{Spousta:2015fca}}
\label{fig:HM_modification}
\end{figure}


Here it can be seen that adding a medium response and a non-zero \Lres\ parameter affects the prediction. While the hard fragments (see Figure~\ref{}) are unaffected by the medium response, including the soft particles from the wake compensates some of the suppression of soft fragments in \pbpb\ compared to \pp\ collisions. Moreover, including the \Lres\ parameter further compensates the suppression for soft fragments, while reducing the enhancement of the hard fragments. This is a result of allowing more hadrons carrying a smaller fraction of the jet energy (low $z$, high ($\ln(1/z)$) to survive into the final state. The jet shape observable (see Figure~\ref{}) quantifies the radial distribution of energy in terms of annuli around the jet axis. It can be seen that introducing the \Lres\ parameter enhances the probability to find final state hadrons at larger distances from the jet axis. The jet core ($r < 0.05$) is also affected, with the depletion only slowly evolving with an increasing \Lres. One must be careful before making conclusions though, since these modifications are made between jets that are quenched (in \pbpb\ ) and unquenched (in \pp\ ). Taking into account the fact that wider jets lose more energy and that the jet spectrum rapidly falls off, there is a bias for finding narrower quenched jets than unquenched jets. This makes the jet shape after quenching narrower in \pbpb\ compared to \pp. While the model is not fully able to capture the features in the data, including the medium response moves it in the correct direction. It can be suggested that the model is missing a description of the medium induced modification to the hadronization process or that the wakes in the plasma are not equilibrating.










\label{sec:hybrid_model}

\section{Effective Quenching}
% !TEX encoding = UTF-8 Unicode
% !TEX root = thesis-ex.tex

This discussion is based on the model introduced in Ref. \cite{Spousta:2015fca}. This phenomenological model emphasizes the jet \pt\ dependence of the quark to gluon fraction and the difference between quark-jet and gluon-jet quenching. It uses an ``extended'' power law parameterization of the high-\pt\ hadron spectra coupled with a quenching that is based on a non-constant fractional energy loss. This model considers the different color charges carried by quarks and gluons and their different splitting functions, and assumes that gluon jets lose energy at a rate 9/4 times higher than quark jets. The key assumption of the model are:
\begin{itemize}
\item The energy lost by a jet is radiated at large angles and does not appear within the jet cone. This is backed by \cite{Chatrchyan:2011sx}.
\item The fragmentation pattern of the jet is unaffected by the presence of the QGP i.e. they fragment as they would in a vacuum. This is motivated by the idea that the QGP is unable to resolve the internal jet structure and is supported by \cite{Blaizot:2013hx, CasalderreySolana:2012ef}.
\end{itemize} 

The model uses the following extended power-law parameterization to describe the high-\pt\ jet spectra:

\begin{align}
\frac{dn}{d\ptjet} = A \left( \frac{\pt_0}{\ptjet} \right) ^{n+\beta \log(\ptjet / \pt_0)}
\end{align}

where $\pt_0$ is a reference transverse momentum at which $A= dn/d\ptjet$, $\beta$ is the logarithmic derivative of $dn/d\ptjet$ at $\ptjet = \pt_0$. Then considering the different quark and gluon fractions as $f_{q0}$ and $f_{g0} = 1-f_{q0}$ respectively, the combined spectrum for quarks and gluons can be written as:

\begin{align}
 \frac{dN}{d\ptjet} &= A \left[ f_{q0} \left( \frac{\pt_0}{\ptjet} \right)^{n_q+ \beta_q \log(\ptjet / \pt_0)} + (1-f_{q0}) \left( \frac{\pt_0}{\ptjet} \right)^{n_g + \beta_g \log(\ptjet / \pt_0)} \right] \\
\nonumber \\ 
\label{eq:eq_q_frac} f_q (\ptjet) &= \dfrac{1}{1 + \left(\dfrac{1-f_{q0}}{f_{q0}}\right) \left( \dfrac{\pt_0}{\ptjet}\right)^{\Delta n + \Delta \beta \log(\ptjet/\pt_0)} }
\end{align}
where $\Delta n = n_g - n_q$ and $\Delta \beta = \beta_g - \beta_q$. The \pt\ dependence of the quark fraction along with the fit is shown in Figure~\ref{fig:raa_centDep}. The fragmentation functions can also be determined using final-state charged hadrons within a $R=0.4$ jet cone. These are fit to the form \Dz, with fits for the quark and gluon fragmentation shown in Figure~\ref{fig:gluon_fragmentation}.


\begin{align}
D(z) = a \times \frac{(1+dz)^b}{(1+ez)^c} \times e^{-fz}
\label{eq:ff_param}
\end{align}


\begin{figure}
\begin{subfigure}{.45\textwidth}
  \centering
\includegraphics[width=0.8\textwidth]{figures/jetMeasurements/jetQuarkFraction}
\caption{The jet quark fraction as a function of \ptjet\ in different rapidity bins. The points are from \pythia8 simulations and the lines are fits to Equation~\ref{eq:eq_q_frac}.}
\label{fig:raa_centDep}
\end{subfigure} \qquad
\begin{subfigure}{.45\textwidth}
  \centering
\includegraphics[width=0.8\textwidth]{figures/jetMeasurements/gluon_fragmentation}
\caption{A comparison of the \pythia8 quark and gluon fragmentation. The solid lines are the fits from The jet quark fraction as a function of \ptjet\ in different rapidity bins. The points are from \pythia8 simulations and the lines are fits to Equation~\ref{eq:ff_param}.}
\label{fig:gluon_fragmentation}
\end{subfigure}
\caption{Fits to quark fractions and fragmentation functions from \pythia8.  Figure taken from \cite{Spousta:2015fca}}
\label{fig:EQ_pp_models}
\end{figure}


For the quenched spectra, this model assumes a non-constant fractional shift given below as $S$. This approach is based on \cite{baier2001quenching} and is used because of the inability of the constant fractional shift to explain the jet \pt\ dependence of measured \RAA. 

\begin{align}
S = s' \left( \frac{\ptjet}{\pt_0} \right) ^\alpha
\end{align}
where $\alpha$ is an undetermined parameter and $s'$ is the shift for a jet with $\ptjet = \pt_0$. This gives the following quenched high-\pt\ hadron spectra:

\begin{align}
 \frac{dN_Q}{d\ptjet} &= A \Bigg[ f_{q0} \left( \frac{\pt_0}{\ptjet+S_q} \right)^{n_q+ \beta_q \log\big((\ptjet+S_q) / \pt_0\big)} \left(1 + \frac{dS_q}{d\ptjet} \right) \\
& + (1-f_{q0}) \left( \frac{\pt_0}{\ptjet+S_g} \right)^{n_g + \beta_g \log\big((\ptjet+S_g) / \pt_0\big)}  \left(1 + \frac{dS_g}{d\ptjet} \right) \Bigg] \nonumber
\end{align}
Where the $(1+dS/d\ptjet)$ term is a Jacobian to preserve the number of jets. 
Then the \RAA\ can be written as:

\begin{align}
\RAA = f_q & \left(\frac{1}{1 + S_q / \ptjet}\right) ^{n_q + \beta_q \log\big((\ptjet+S_q)/\pt_0\big)}  \frac{\pt_0}{\ptjet}^{} \left( 1+ \frac{dS_q}{d\ptjet} \right) \times  \\
 (1-f_q) & \left(\frac{1}{1 + S_g / \ptjet}\right) ^{n_g + \beta_g \log\big((\ptjet+S_g)/\pt_0\big)}  \frac{\pt_0}{\ptjet}^{} \left( 1+ \frac{dS_g}{d\ptjet} \right)  \nonumber \\
\end{align}
where the flavor fraction is given by Equation~\ref{eq:eq_q_frac}. These can be fit to the measured ATLAS \RAA\ data as shown in Figure~\ref{fig:EQ_RAA} and the parameters $s'$ and $\alpha$ can be extracted as shown in Figure~\ref{fig:eq_param}. 

\begin{figure}[htbp]
\begin{center}
\includegraphics[width=0.55\textwidth]{figures/jetMeasurements/EQ_fitQuality}
\caption{The extracted values of $\alpha$ and $s'$ as a function of \Npart. The first minimization shows fluctuations for $\alpha$ around 0.55, which was then fixed for the second minimization to give an $s'$ that linearly depends on \Npart. Figure taken from \cite{Spousta:2015fca}}
\label{fig:eq_param}
\end{center}
\end{figure}

It can be seen that the analytic fits and the MC are in good agreement. While the fits agree with the data by definition, the robustness of the model can be seen in that it describes the data with a single value for $\alpha$ and a simple centrality dependent shift constant $s'$.  Fits to the \Dz\ distributions are shown in Figure~\ref{fig:EQ_FF} and it can be seen that while the MC and analytic calculation agree well with each other, they are only able to qualitatively capture some features of the data. The enhancement at high $z$ can be explained by an increased quark content of the jet spectrum and subsequent differential quenching for quark and gluon jets. The low $z$ enhancement on the other hand can be considered to be a result of a gluon radiation within the jet or a wake from the medium itself.

\begin{figure}
\begin{subfigure}{1\textwidth}
  \centering
\includegraphics[width=1\textwidth]{figures/jetMeasurements/EQ_RAA}
\caption{A comparison of the \RAA\ as measured by ATLAS for central \pbpb\ collisions in \cite{Aad:2014bxa}, a MC calculation (blue) and the analytic calculation (red) in the EQ model with the extended power-law parameterization and a non-constant fractional energy loss. The different panels are different rapidity intervals.}
\label{fig:EQ_RAA}
\end{subfigure} \\ \\ \\
\begin{subfigure}{1\textwidth}
  \centering
\includegraphics[width=1\textwidth]{figures/jetMeasurements/eq_FF}
\caption{A comparison of the \Rdz\ as measured by ATLAS in \cite{Aad:2014wha}, a MC calculation (blue) and the analytic calculation (red) in the EQ model with the extended power-law parameterization and a non-constant fractional energy loss. The different panels are different centrality intervals.}
\label{fig:EQ_FF}
\end{subfigure}
\caption{A comparison of measured data, MC, and the analytic calculation of the EQ model. Figure taken from \cite{Spousta:2015fca}}
\label{fig:EQ_modification}
\end{figure}


\label{sec:eff_quench}


