% !TEX encoding = UTF-8 Unicode
% !TEX root = thesis-ex.tex


This discussion is based on Ref.~\cite{PhysRevLett.88.022301}.
Done at RHIC by the PHENIX collaboration, this was one of the first experimental measurements of jet quenching that showed the presence of the QGP.
This measurement analyzed high \pt\ charged hadrons and neutral $\pi^0$s ($\pt > 2$GeV) from jets produced in \AuAu\ collisions, collided at $\sqrtsnn = 130$ GeV.
Since jets form early in the collision and experience the evolution of the QGP, they are expected to lose energy due to collisional and radiative losses as discussed in Section~\ref{sec:jets}.
The modifications between the \pp\ and \AuAu\ system was quantified by constructing the nuclear modification factor \RAA, given as:

\begin{align}
\RAA_{\pt} = \frac{(1 / \Nevt) d^2 N^{\rm A+A} / d\pt d\eta}{(\langle N_{\rm binary} \rangle / \sigma_{\rm inel}^{\rm N+N} d^2 \sigma^{\rm N+N} / d\pt d\eta}
\end{align}
where \Nevt\ is the number of \AuAu\ events, $\langle N_{\rm binary} \rangle$ is the average number of binary collisions per event, $\sigma$ is the scattering cross section, and \pt\ and $\eta$ are the kinematics of the charged particle.
The \RAA\ for charged hadrons and neutral pions is shown in Figure~\ref{fig:hadron_raa}.

\begin{figure}
\begin{subfigure}{.49\textwidth}
  \centering
\includegraphics[width=\textwidth]{figures/jetMeasurements/hadron_raa}
\caption{The \RAA\ for charged hadrons and neutral pions in \AuAu\ collisions at $\sqrtsnn = 130$ GeV.
Also shown is the \RAA\ for inclusive cross sections in $\alpha+\alpha$ compared to \pp\ at $\sqrtsnn = 31$ GeV \cite{ANGELIS1987213} and spectra from \pbpb and $\rm{Pb}+{\rm AU}$ compared to \pp\ at $\sqrtsnn = 17$ GeV \cite{PhysRevC.64.034901}.
Figure taken from Ref.~\cite{PhysRevLett.88.022301}.}
\label{fig:hadron_raa}
\end{subfigure} \qquad
\begin{subfigure}{.49\textwidth}
  \centering
\includegraphics[width=\textwidth]{figures/jetMeasurements/photon_raa}
\caption{The \RAA\ for photons in three centrality regions in \AuAu\ collisions at $\sqrtsnn = 200$ GeV.
Figure taken from Ref.~\cite{PhysRevLett.109.152302}.}
\label{fig:photon_raa}
\end{subfigure}
\caption{\RAA\ evaluated for (left) charged hadrons and pions and (right) photons.}
\label{fig:particle_raa}
\end{figure}


%\begin{figure}[htbp]
%\begin{center}
%\includegraphics[width=0.55\textwidth]{figures/jetMeasurements/hadron_raa}
%\caption{The \RAA\ for charged hadrons and neutral pions in \AuAu\ collisions at $\sqrtsnn = 130$ GeV.
%Also shown is the \RAA\ for inclusive cross sections in $\alpha+\alpha$ compared to \pp\ at $\sqrtsnn = 31$ GeV \cite{ANGELIS1987213} and spectra from \pbpb and $\rm{Pb}+{\rm AU}$ compared to \pp\ at $\sqrtsnn = 17$ GeV \cite{PhysRevC.64.034901}.
%Figure taken from Ref.~\cite{PhysRevLett.88.022301}.}
%\label{fig:hadron_raa}
%\end{center}
%\end{figure}

A significant depletion is seen, with the \RAA\ rising for $\pt < 2$ GeV and remaining fairly constant thereafter.
This modification includes both hot nuclear matter effects from the QGP, as well as cold nuclear matter effects like the Cronin effect that can be seen in $p+A$ collisions \cite{PhysRevD.19.764}.

Electroweak probes like photons and Z bosons do not lose energy is the QGP since they do not interact strongly, and their \RAA\ is expected to be closer to unity.
There can be differences though, that are coming from cold nuclear matter effects.
This can be seen in Figure~\ref{fig:photon_raa}