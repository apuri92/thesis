\subsection{General Cuts}
For the analysis of \pp\ and \pPb\ data samples, the first level of filtering is via a Good Runs List (GRL) which is used to clean bad luminosity blocks (lumiblocks). All the data from every run is split up into these lumiblocks, which can hold thousands of events. Th GRL is compiled by the collaboration after data quality studies identifying issues with data-taking conditions have been performed after each run. The next step of filtering is at the event level where there is a minimum of one reconstructed vertex required for an event to pass. Additionally, DAQ errors due to the Scintillator Detector, Tile Calorimeter, and Liquid Argon calorimeters are checked for every event. If any of these detectors are flagged, or a primary vertex is not identified, the event is skipped. Next, events are chosen based on trigger decision.

\subsection{Trigger Selection}

The ATLAS trigger discussed in Section~\ref{sec:trigger} was used to select minimum-bias and jet events. Jet events were selected by the HLT with L1 seeds from jet, minimum bias, and total-energy triggers. In order to efficiently distribute the limited bandwith of the trigger to the various physics streams, a procedure known as seeding was used. This relies on having minimum requirement for a given trigger to be considered for processing. This requirement is usually a smaller threshold or minimum-bias trigger firing, which selects less common events more efficiently. The HLT jet trigger, used both in \pPb\ collisions and \pp\ collisions, refined the selection of minimum-bias, level one total energy (L1TEx), or level one jet triggers (LIJx) with various thresholds. The total-energy trigger required a total transverse energy measured in the calorimeter of greater than 5 GeV. The L1 jet trigger required jets with transverse momenta greater than 12 GeV to be reconstructed at the hardware level. The forward jet triggered \pPb\ events were seeded by minimum-bias events by requiring at least one hit in the MBTS detector on each side of the interaction point at the L1 trigger. The HLT jet trigger operated a jet reconstruction algorithm similar to that applied in the offline analysis and selected events containing jets with transverse energy thresholds of~15~GeV in \pPb\ collisions and up to 85~GeV in \pp\ collisions. In both \pp\ and \pPb\ collisions, the highest threshold jet trigger sampled the full delivered luminosity. The trigger selecting minimum-bias events required a track above 200~MeV in the \pp\ data-taking. For \pPb\ data-taking, the minimum-bias trigger required the same conditions at the L1 level in the MBTS that were used to seed forward jet triggered events. 
          
Table~\ref{tab:pptriggers} lists the triggers used during \pp\ data-taking both in the forward ($3.2<|\eta|<4.4$), and central ($|\eta|<3.2$) regions, the corresponding \pT\ range where the trigger is $99\%$ efficient, and the average prescale used. In \pp\ data-taking, both forward and central triggers are used. Jet trigger efficiencies during \pp\ data-taking for forward and central triggers are shown in Fig.s~\ref{fig:ppeffcent} and ~\ref{fig:ppefffwd}. These efficiencies are obtained by comparing jet spectra of various triggers to spectra of MinBias triggers or other lower \pt\ triggers.   A small inefficiency is seen for the lowest forward jet trigger \textsc{HLT\_J25\_320ETA490\_L1TE5} due to the jet area overlap with the region between forward and central triggers at $|\eta| = 3.2$. 
          
\begin{table}
	\centering
	\begin{tabular}{|| c | c | c || } 
		\hline
		2015 \pp\ Forward ($3.2<|\eta|<4.4$) Trigger & \pT\ Efficiency Range [GeV] & Average Prescale \\ 
		\hline
		\verb|HLT_j25_320eta490_L1TE5| & $28---42$ & 290.476 \\ 
		\verb|HLT_j35_320eta490_L1TE10| & $42---52$ & 74.11  \\ 
		\verb|HLT_j45_320eta490| & $52---65$ & 1.413 \\ 
		\verb|HLT_j55_320eta490| & $65---90$ & 1.413 \\ 
		\hline \hline 
		2015 \pp\ Central ($|\eta|<3.2$) Trigger & \pT\ Efficiency Range [GeV] \\ 
		\hline
		\verb|HLT_j20| & $28---35$ & 5827.311 \\ 
		\verb|HLT_j30_L1TE5| & $35---44.5$ & 297.388 \\ 
		\verb|HLT_j40_L1TE10| & $44.5---59$ & 73.183 \\ 
		\verb|HLT_j50_L1J12| & $59---70$ & 14.225 \\ 
		\verb|HLT_j60_L1J15| & $70---79$ & 10.807 \\ 
		\verb|HLT_j75_L1J20| & $79---89$ & 1.012\\ 
		\verb|HLT_j85| & $89---90$ & 1.002 \\ 
		\hline 		
	\end{tabular}
	\caption{\label{tab:pptriggers} List of \pp\ triggers with associated \pT\ ranges where the trigger is over $99\%$ efficient.}
\end{table}


\begin{table}
	\centering
	\begin{tabular}{|| c | c | c || } 
		\hline
		2016 \pPb\ Forward ($-4.4<\eta<-3.2$) Trigger & \pT\ Efficiency Range [GeV] & Average Prescale \\ 
		\hline
		\verb|HLT_j15_ion_n320eta490_L1MBTS_1_1| & $28---90$ & 1.02 \\ 
		\hline
	\end{tabular}
	\caption{\label{tab:pPbtriggers} Un-prescaled \pPb\ trigger with associated \pT\ ranges where the trigger is over $99\%$ efficient.  }
\end{table}


\begin{table}
	\centering
	\begin{tabular}{|| c || } 
		\hline 
		2016 \pPb\ Min-Bias Trigger \\ 
		\hline
		\verb|HLT_mb_sptrk_L1MBTS_1_OVERLAY|  \\ 
		\verb|HLT_noalg_L1TE5_OVERLAY|  \\ 
		\verb|HLT_noalg_L1TE20_OVERLAY|  \\ 
		\hline
	\end{tabular}
	\caption{\label{tab:pPbOverlaytriggers} List of 2016 \pPb\ triggers used to tag events for the MC data overlay.}
\end{table}


\begin{figure}
	\centerline{
		\includegraphics[width=0.7\textwidth]{output/output_pp_data/eta_eff_32_Eta_0.pdf} }
	\caption{Jet trigger efficiency for \pp\ central triggers in the pseudorapidity range $-3.2<|\eta|<3.2$.}
	\label{fig:ppeffcent}
\end{figure}

\begin{figure}
	\centerline{
		\includegraphics[width=0.7\textwidth]{output/output_pp_data/eta_eff_44_Eta_32.pdf} }
	\caption{Jet trigger efficiency for \pp\ forward triggers in the pseudorapidity range $3.2<|\eta|<4.4$. A small inefficiency is seen for the lowest forward jet trigger \textsc{HLT\_J25\_320ETA490\_L1TE5} due to the jet area overlap with the region between forward and central triggers at $|\eta| = 3.2$. }
	\label{fig:ppefffwd}
\end{figure}

\begin{figure}
	\centerline{
		\includegraphics[width=0.7\textwidth]{output/output_pPb_data/eta_eff_44_Eta_32.pdf} }
	\caption{Jet trigger efficiency for \pPb\ forward triggers in the pseudorapidity range $3.2\eta<4.4$.}
	\label{fig:pPbefffwd}
\end{figure}

During \pPb\ data-taking, only one forward, unprescaled jet trigger was used because the \ystar\ interval from 2.7 to 4.0 for the leading jet corresponds to a pseudorapidity interval from -3.2 to -4.4. The efficiency plot for this forward jet trigger is shown in Fig.~\ref{fig:pPbefffwd}. This trigger was seeded by the L1 MBTS trigger and its corresponding \pt\ range used is shown in Table~\ref{tab:pPbtriggers}. The \pPb\ triggers used to produce the data overlay for the \pPb\ MC are shown in Table~\ref{tab:pPbOverlaytriggers}. For the data overlay, entire events were selected based solely on the MB trigger decision with no requirement on jets.

\FloatBarrier

To check that the performance of jet triggers was consistent across runs in \pp\ and \pPb\ data-taking, the number of jets in some \ptone\ and \ystarone\ intervals were counted and divided by the prescale-corrected luminosity of each run. Plotted as a function of run number, this ratio should be relatively uniform and is shown for central and forward \pp\ triggers and forward \pPb\ trigger in Fig.~\ref{fig:jetspectraperformance}. The large luminosity uncertainty during the \pPb\ data taking contributed to the statistical fluctuations seen this ratio for the forward jet trigger.

\begin{figure}[h]
	\centerline{
		\begin{tabular}{ccc}
			\includegraphics[width=0.33\textwidth]{output/output_pp_data/h_nJetsRun_HLT_j30_L1TE5_18_Ystar1_0_35_Pt1_45.pdf} & 
			\includegraphics[width=0.33\textwidth]{output/output_pp_data/h_nJetsRun_HLT_j35_320eta490_L1TE10_40_Ystar1_27_35_Pt1_45.pdf} & 
			\includegraphics[width=0.33\textwidth]{output/output_pPb_data/h_nJetsRun_HLT_j15_ion_n320eta490_L1MBTS_1_1_27_Ystar1_18_28_Pt1_35.pdf}  \\
		\end{tabular}
	}
	\caption{ Number of jets in some \ptone\ and \ystarone\ interval divided by prescale-corrected luminosity for each run. Central \pp\ trigger (left), forward \pp\ trigger (center), and forward \pPb\ trigger (right). }	
	\label{fig:jetspectraperformance}
\end{figure} 

\subsection{Disabled HEC in \pPb\ Data-taking}
\label{sec:hecdisabled}

During the 2016 \pPb\ data-taking period, part of the HEC in the lead going direction was disabled in the pseudorapidity and azimuthal intervals $-3.2<\eta<-1.3$ and $-\pi<\phi<-\pi/2$, respectively. Reconstructed dijets where the sub-leading jet area overlaps with the disabled HEC region are excluded from the analysis in \pPb\ data and MC samples. Plots of jet multiplicity in $\eta \times \phi$  space for the \pPb\ data, MC signal, and MC with data overlay samples for the lowest jet \pt\ interval $25<\pt<35$ GeV are shown in Fig.~\ref{fig:hecproblem}. In the signal MC simulation, which does not include any data overlay,  there appears to be a small cavity in the region covered by the HEC. This is also seen in the \pPb\ data. However, in the MC simulation with data overlay, this region is not disabled. To account for the jet radius \RFour\, the excluded region is increased to not include jets with jet axes in $-3.6<\eta<-0.9$ in pseudorapidity, and $-\pi<\phi<(-\pi/2 + 0.4)$ and $(\pi-0.4)<\phi<\pi$ in azimuth. This is detector inefficiency is corrected by a procedure that will be described in a later section. 


\begin{figure}
	\centerline{
		\begin{tabular}{ccc}
			\includegraphics[width=0.33\textwidth]{output/output_pPb_data/h_etaPhiMap_All.pdf} & 
			\includegraphics[width=0.33\textwidth]{output/output_pPb_mc_pythia8/h_etaPhiMap_signal_All.pdf} & 
			\includegraphics[width=0.33\textwidth]{output/output_pPb_mc_pythia8/h_etaPhiMap_All.pdf}  \\
		\end{tabular}
	}
	\caption{ Maps of $\phi$ vs $\eta$ shown for lowest \pt\ interval $25<\pt<35$ GeV for the \pPb\ data (left), \pPb\ MC with only signal included (centeR), and \pPb\ MC with data overlay (right). A depletion is seen in the data for the region covered by the HEC detector in the lead going direction (negative $\eta$), and a minor cavity is seen in the signal MC in the same region. No apparent effect is seen in the MC with data overlay. The red box indicates the HEC region which was turned off. Due to the jet radius \RFour\, the excluded region is increased, and is indicated by the black box. }	
	\label{fig:hecproblem}
\end{figure} 

\FloatBarrier