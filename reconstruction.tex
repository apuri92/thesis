% !TEX root = thesis-ex.tex

Jets are reconstructed using a heavy ion reconstruction procedure developed for previous jet measurements in Pb+Pb and \pPb\ collisions~\cite{ATLAS:2014cpa,Aaboud:2017tke}. The jet reconstruction is first run in four-momentum recombination mode, on $\Delta \eta \times \Delta \phi = 0.1\times 0.1$ calorimeter towers with the \antikt\ algorithm~\cite{Cacciari:2008qp} with $R=0.4$. Energies in the towers are obtained by summing the energies of calorimeter cells at the electromagnetic energy scale within the tower boundaries. Then, an iterative procedure is used to estimate the layer and $\eta$-dependent underlying event (UE) transverse energy density, while excluding the regions populated by jets. The UE transverse energy is subtracted from each calorimeter tower and the four-momentum of the jet is updated accordingly. Jets which do not overlap with the region included in the UE background subtraction also have a small correction applied on the order of a few percent. Then, a jet $\eta-$ and \pT-dependent  correction factor derived from the simulation samples is applied to correct for the calorimeter response. These factors are derived by the ATLAS Jet \Et\ Miss (JetEtMiss) group and are standard corrections used in all analyses. An additional data driven correction based on \textit{in situ} studies of the momentum balance of jets recoiling against photons, $Z$ bosons, and jets in other regions of the calorimeter is also applied ~\cite{Aad:2011he,Run2jetpubnote}.

Jets are selected in the transverse momentum range of $28<\pT<90$ GeV and a center-of-mass rapidity of $-4.0<\ystar<4.0$. This is the largest symmetric overlap between the two colliding systems for which most forward jets can be reconstructed using the FCal with full coverage for \RFour\ jets. All reconstructed jets are required to  have a \pT\ such that the jet trigger efficiency is greater than 99\%. As a result, no trigger efficiency correction is applied. 

The MC samples are used to evaluate the jet reconstruction performance and to correct the measured distributions for detector effects. This is done independently for both \pp\ and \pPb\ collisions. In the MC samples, the generator level jets are reconstructed from primary particles\footnote{Primary particles are defined as particles with a mean lifetime $\tau>0.3\times 10^{-10}$~s, excluding muons and neutrinos, which are weakly interacting and do not leave significant energy deposits in the calorimeters.} with the \antikt\ algorithm with radius \RFour. Using the pseudorapidity and azimuthal angles \etatruth, \phitruth, \etareco, and \phireco\ of the generated and reconstructed jets, respectively, generator level jets are matched to reconstructed jets by requiring $\Delta R <0.2$, where $\Deta = |\etareco - \etatruth|$, and $\Dphi = |\phireco - \phitruth|$.

The efficiency of reconstructing jets in \pp\ and \pPb\ collisions is evaluated using the \pythiaeight\ MC samples by determining the probability of finding a reconstructed jet associated with a generator level jet. The jet reconstruction efficiencies are shown in in Fig.~\ref{fig:jetrecoeff} for \pp\ and \pPb\ MC samples in different \ystar\ and \pT\ regions. The jet reconstruction efficiency is greater than 99\% for jets with $\pT>30$~GeV over the selected \ystar\ range $-4.0<\ystar<4.0$ and drops to 95\% at a jet $\pt = 28$~GeV. The variation of the jet reconstruction efficiency with \ystar\ is due to jets having a higher total energy for a given transverse energy as compared to more central regions.
  
\begin{figure}
	\centerline{
		\includegraphics[width=0.55\textwidth]{output/output_pp_mc_pythia8/h_eff_All.pdf}
		\includegraphics[width=0.55\textwidth]{output/output_pPb_mc_pythia8/h_eff_All.pdf}  
	}
	\caption{Jet reconstruction efficiency evaluated in the \pp\ (left) and \pPb\ (right) \pythiaeight\ MC samples.}
	\label{fig:jetrecoeff}
\end{figure}

\FloatBarrier

The ratios of transverse momenta of generated and reconstructed jets, \pttruth\ and \ptreco respectively, determine the relevant jet energy scale (JES) $\ptreco/\pttruth$, and jet energy resolution (JER) $\sigma(\ptreco/\pttruth)$, which characterize the jet energy reconstruction performance.  The JES and JER are plotted as a function of \pttruth, in intervals of generated jet pseudorapidity $\eta^{\mathrm{truth}}$ in Fig.~\ref{fig:jesjerpp},~\ref{fig:jesjerpPb} for \pp\ and \pPb\ MC samples, respectively. The means and standard deviations of the $\ptreco/\pttruth$ distributions, along with their errors are extracted from fits of the distributions to Gaussian function The JES shows a very small dependence on $\eta^{\mathrm{truth}}$, with a maximum deviation of $\pm 3\%$ from unity at \pttruth~=~30~GeV and a minimum of $-3\%$ deviation from unity at \pttruth~=~50~GeV. The JES decreases with $p_{T}^{\mathrm{truth}}$, and with decreasing $\eta$. 

\begin{figure}
	\centerline{
		\includegraphics[width=0.55\textwidth]{output/output_pp_mc_pythia8/h_recoTruthRpt_eta_mean.pdf} 
		\includegraphics[width=0.55\textwidth]{output/output_pp_mc_pythia8/h_recoTruthRpt_eta_sigma.pdf}
	}
	\caption{JES (left) and JER (right) evaluated in \pp\ MC samples and plotted as a function of \pttruth. }
	\label{fig:jesjerpp}
\end{figure}

\begin{figure}
	\centerline{
		\includegraphics[width=0.55\textwidth]{output/output_pPb_mc_pythia8/h_recoTruthRpt_eta_mean.pdf} 
		\includegraphics[width=0.55\textwidth]{output/output_pPb_mc_pythia8/h_recoTruthRpt_eta_sigma.pdf} }
	\caption{JES (left) and JER (right) evaluated in \pPb\ MC samples and plotted as a function of \pttruth. }
	\label{fig:jesjerpPb}
\end{figure}

The mean angular distance $\langle\Deta\rangle$ and jet angular resolution (JAR) for pseudorapidity $\sigma(\Deta)$ between truth and reconstructed jets $\Deta = \eta_{\mathrm{reco}} - \eta_{\mathrm{truth}}$ is plotted in Fig.s~\ref{fig:detapp},~\ref{fig:detapPb} for the \pp\ and \pPb\ MC samples respectively. Similarly, mean angular distance $\langle\Dphi\rangle$ and azimuthal JAR $\sigma(\Dphi)$ between truth and reconstructed jets $\Dphi = \phi_{\mathrm{reco}} - \phi_{\mathrm{truth}}$ is plotted in Fig.~\ref{fig:dphipp} and \ref{fig:dphipPb} in \pp\ and \pPb\ MC samples respectively. Similar to the procedure used for extracting the JER and JES, means and standard deviations are extracted from fits with a Gaussian function. For both pseudorapidity and azimuth, $\langle\Deta\rangle$ and $\langle\Dphi\rangle$ are consistent with zero in the \pp\ MC sample. In the \pPb\ MC sample, $\langle\Dphi\rangle$ is consistent with zero but there is a shift of less than $~0.01$ in $\langle\Deta\rangle$ from the underlying event contribution. This is a result of the UE pulling the reconstructed jet in the lead going direction, however it is a negligible effect which is less than $1/10$ of the tower size. The angular resolution $\sigma(\Deta)$ and $\sigma(\Dphi)$ decreases as a function \pttruth\ as expected.

\begin{figure}
	\centerline{
		\includegraphics[width=0.55\textwidth]{output/output_pp_mc_pythia8/h_recoTruthDeta_eta_mean.pdf} 
		\includegraphics[width=0.55\textwidth]{output/output_pp_mc_pythia8/h_recoTruthDeta_eta_sigma.pdf}
	}
	\caption{The mean angular distance \Deta\ (left) and resolution $\sigma(\Deta)$ (right) between truth and reconstructed jets evaluated in \pp\ MC samples and plotted as a function of \pttruth. }
	\label{fig:detapp}
\end{figure}

\begin{figure}
	\centerline{
		\includegraphics[width=0.55\textwidth]{output/output_pPb_mc_pythia8/h_recoTruthDeta_eta_mean.pdf}
		\includegraphics[width=0.55\textwidth]{output/output_pPb_mc_pythia8/h_recoTruthDeta_eta_sigma.pdf}
	}
	\caption{The mean angular distance \Deta\ (left) and resolution $\sigma(\Deta)$ (right) between truth and reconstructed jets evaluated in \pPb\ MC samples and plotted as a function of \pttruth. }
	\label{fig:detapPb}
\end{figure}

\begin{figure}
	\centerline{
		\includegraphics[width=0.55\textwidth]{output/output_pp_mc_pythia8/h_recoTruthDphi_eta_mean.pdf} 
		\includegraphics[width=0.55\textwidth]{output/output_pp_mc_pythia8/h_recoTruthDphi_eta_sigma.pdf}
	}
	\caption{The mean angular distance \Dphi\ (left) and resolution $\sigma(\Dphi)$ (right) between truth and reconstructed jets evaluated in \pp\ MC samples and plotted as a function of \pttruth. }
	\label{fig:dphipp}
\end{figure}

\begin{figure}[h]
	\centerline{
		\includegraphics[width=0.55\textwidth]{output/output_pPb_mc_pythia8/h_recoTruthDphi_eta_mean.pdf}
		\includegraphics[width=0.55\textwidth]{output/output_pPb_mc_pythia8/h_recoTruthDphi_eta_sigma.pdf}
	}
	\caption{ The mean angular distance \Dphi\ (left) and resolution $\sigma(\Dphi)$ (right) between truth and reconstructed jets evaluated in \pPb\ MC samples and plotted as a function of \pttruth. }
	\label{fig:dphipPb}
\end{figure}


\subsubsection{Performance Study of \pPb\ MC}
The wrongly configured HEC condition in the \pPb\ MC sample with data overlay raised questions about other possible discrepancies in detector conditions. One way check the reliably of the MC reconstruction conditions is to use tracks reconstructed in the inner detector tracker, which is very precise, and compare the results against jets. This is done by studying the comparison of  \rtrk\ distributions as a function of jet \pt\ in data and MC. \rtrk\ is defined as:

\begin{equation}
\rtrk\ = \frac{\sum p_{\mathrm{T}}^{\mathrm{trk_{i}}}}{\ptjet}
\end{equation}

where $\sum p_{\mathrm{T}}^{\mathrm{trk_{i}}}$ is the sum of transverse momenta of all tracks that fall within a reconstructed jets area. If the ratio of \rtrk\ between data and MC samples is consistent with unity, the test acts as a data-driven check that the MC conditions are consistent with those during the data-taking. This ratio is shown in Figure~\ref{fig:rtrk} for two proton going direction ranges of pseudorapidity in a region of the detector where the tracker can be used. The figures show the ratio of \rtrk\ between data and MC samples for the \pPb\ MC sample with data overlay, as well as the \pPb\ signal sample alone. The results in the central part of the barrel $-1.8<\eta<0$ show good closure. The results in the extended barrel $-2.5<\eta<-1.8$ have high statistical fluctuations, but are consistent with unity at lower \pt. The jet radios of \RFour\ near the edge of the tracker $\eta=-2.5$ also introduces uncertainties as not all of the tracks in the jet pass through the tracker. This test still shows that the conditions in the \pPb\ data and MC samples are consistent.

\begin{figure}[h]
	\centerline{
		\begin{tabular}{cc}
			\includegraphics[width=0.5\textwidth]{output/output_pPb_mc_pythia8/h_rtrk_1.pdf} & 
			\includegraphics[width=0.5\textwidth]{output/output_pPb_mc_pythia8/h_rtrk_2.pdf}  \\
		\end{tabular}
	}
	\caption{Ratios of \rtrk\ between data and MC with data overlay (black) and MC signal sample only (red).}	
	\label{fig:rtrk}
\end{figure} 


\FloatBarrier
