% !TEX encoding = UTF-8 Unicode
% !TEX root = thesis-ex.tex

This chapter gives a detailed outline for the analysis of azimuthal correlations in \pp\ and \pPb\ data taken with the ATLS detector. First, in Section~\ref{sec:data}, an overview of the size and type of data and simulation samples used in the analysis is given. Next, in Section~\ref{sec:event}, the rules for event selection in these respective data and MC samples are discussed. This is including but not limited to simple phase-space cuts or trigger requirements in data. Since jets are the observables used in this analysis, a detailed overview of the jet reconstruction is given in Section~\ref{sec:reconstruction}. As with all analysis done in ATLAS, the proper performance of the detector must be verified before the beginning of the physics measurement. Any irregularities that are identified must later be corrected for in order to have a proper physics measurement. Detector performance is evaluated using MC samples and is later used as input into any known systematics that should be taken into account for a precise physics measurement. Next, in Section~\ref{sec:analysis}, the main analysis procedure is described. This section goes step-by-step through all parts of the analysis, explaining why things were done, and backs up every part with respective plots. Systematic uncertainties, which are very important and a large part of the analysis are presented in Section~\ref{sec:systematics}. Finally, everything is put together and the results and discussion of the measurements are presented in Section~\ref{sec:results}. A summary of these analysis steps, with their respective sections are below:

\begin{itemize}
	\item Data sets - Section~\ref{sec:data}
	\item Trigger and Event Selection - Section~\ref{sec:event}
	\item Jet Selection and Reconstruction Performance - Section~\ref{sec:reconstruction}
	\item Analysis Procedure - Section~\ref{sec:analysis}
	\item Systematic Uncertainties- Section~\ref{sec:systematics}
	\item Results - Section~\ref{sec:results} 
\end{itemize}

\newpage
