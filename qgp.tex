% !TEX root = thesis-ex.tex
The quark-gluon plasma is a state of matter that comprises of free partons and is formed in extreme conditions of temperature and pressure \cite{SHURYAK198071}.
Its study is motivated by the fact that is the only way to access the dynamics of partons that are otherwise confined within hadrons. Moreover, its thermodynamic properties are
of particular interest since it filled the early universe a few microseconds after the Big Bang \cite{PhysRevLett.34.1353}.
The QGP also forms the core of neutron-stars \cite{Linde_1979} and the recent detection of gravitational waves from a neutron-star merger \cite{PhysRevLett.119.161101} has opened new avenues 
of investigation \cite{Han:2018mtj, PhysRevD.99.023009, PhysRevLett.122.061101}. These studies have the potential to provide information into the nuclear equation of state since the dynamics of the 
merger are sensitive to the behavior of extremely dense nuclear matter \cite{PhysRevD.86.063001}. The increase in temperatures and density during the merger results in different 
pre- and post-merger signals of gravitational-waves that suggest a signature of a first-order hadron-quark phase transition at extreme densities \cite{PhysRevLett.122.061102}. Colliders 
like RHIC and the LHC on the other hand probe regions that have comparitively low baryon densities. Lattice QCD calculations in these regions show that the transition between a hadronic gas and the QGP occurs 
at a temperature of approximately 160 MeV and corresponds to an energy density of 0.5 GeV/fm$^3$ \cite{Borsanyi:2010bp}. This is a smooth crossover that spans a 20--30 MeV temperature range, and can 
be seen in the QCD phase diagram shown in Figure~\ref{fig:qcd_phase}. This phase diagram shows the transition between free quarks and gluons within the QGP and the confined quarks and gluons 
within hadrons, as a function of temperature $T$ and baryon chemical potential $\mu$.
% matter and had   at $\mu = 0$ is applicable \cite{Aoki:2006we}.. 
\begin{figure}[htbp]
\begin{center}
\includegraphics[width=0.45\textwidth]{figures/theory/qcd_phase}
\caption{The QCD phase diagram of nuclear matter as a function of temperature $T$ and baryon chemical potential $\mu$. The n-$\star$ denotes a neutron star. Figure from from Ref.~\cite{Kronfeld:2012uk}. }
\label{fig:qcd_phase}
\end{center}
\end{figure}

When formed in a heavy ion collision, this state of matter exists for 1-10 fm/c, depending on the collision energy \cite{doi:10.1146/annurev.nucl.46.1.71}. Thermal photons from the QGP reveal that 
it reaches temperatures of 300--600 MeV in central collisions at 200 GeV \cite{PhysRevLett.104.132301} and 2.76 TeV \cite{2016235}, showing very little collision energy dependence. As the QGP 
cools via expansion, its temperature drops below the critical temperature of QCD phase transitions and it forms a hadron gas. This process, referred to as a chemical freeze-out, occurs at about
160 MeV \cite{Fodor_2004, ADAMS2005102, PhysRevC.93.024917}. The hadrons formed in this stage continue to interact with each other, but have energies below the threshold for inelastic 
particle production, resulting mainly in modifications to their momentum spectra. This continues till the medium cools further and reaches what is called a thermal freeze-out at 100--150 MeV \cite{PhysRevC.69.024904, PhysRevC.72.014908, PhysRevC.75.024910, PhysRevC.88.044910}.

%These measurements paint a picture of the QGP being formed early in the heavy ion collision. It has a non-uniform energy density and temperature determined by the colliding nuclei and collision energy. The QGP then cools and expands as described by relativistic hydrodynamics, and as its temperature falls below 160 MeV, it experiences a crossover phase transition and hadronizes. This system continues to cool and expand, until at 95 GeV there is a thermal freeze-out. 




%The QGP can be further characterized by comparing quarkonia production in heavy ion and \pp\ collisions. 

%Supposing that the interactions between quarks and gluons are negligible, their energy density can be written in terms of the temperature $T$ and quark chemical potential $\mu$ as:
%
%\begin{align}
%\varepsilon_g &= \frac{16 \pi^2}{30} T^4 \\
%\varepsilon_q+\varepsilon_{\bar{q}} &= 12 \left( \frac{7\pi^2}{120} T^4 + \frac{1}{4}\mu^2 T^2 + \frac{1}{8\pi^2} \mu^4 \right)
%\end{align}
%Then, the thermodynamical quantities of pressure $P$, entropy density $s$ and baryon number density $\rho_B$ are given by:
%
%\begin{align}
%P = \frac{1}{3} \varepsilon, \qquad s = \left(\frac{\partial P}{\partial T}\right)_\mu, \qquad \rho_B = \frac{1}{3} \left( \frac{\partial P}{\partial \mu} \right)_T
%\end{align}
%
%Using the framework of the MIT Bag Model and assuming the QGP as an ideal gas with a bag constant $\mathcal{B}$ that parameterizes the vacuum pressure \cite{Muller1993, Yagi:2005yb}, the equation of state of the QGP can be written as
%
%\begin{align}
%\varepsilon &= \varepsilon_g(T, \mu) + \varepsilon_q(T, \mu) + \varepsilon_{\qbar} (T, \mu) + \mathcal{B} \\
%P &= P_g(T, \mu) + P_q(T, \mu) + P_{\qbar} (T, \mu) - \mathcal{B}
%\end{align}
%
%This model considers quarks and gluons to move freely inside a ``bag'' and postulates that a deconfined medium can be formed by compressing the bags together. Assuming a baryon free case with $\mu = 0$ and idealizing hadronic matter as a gas of non-interacting massless pions:
%
%\begin{align}
%\varepsilon_\pi = \frac{3\pi^2}{30} T^4, \qquad P_\pi = \frac{1}{3} \varepsilon_\pi
%\end{align}
%%%%%%%%%%%%%%%%%%%%%%%%%%%%

The QGP was initially thought to be a weakly coupled parton gas because of asymptotic freedom from QCD \cite{PhysRevLett.34.1353}. The highly energetic collisions such as those at the LHC would 
imply a weak interaction between the partons that make up the plasma. This would result in rare scatterings between the constituents of the gas, washing out any spatial anisotropies based 
on ``'lumpy''-ness of the colliding nuclei or the collision geometry. On the other hand, a strong coupling within the QGP would result in the pressure gradients in the medium being driven by 
hydrodynamics and spatial anisotropies would be transformed to momentum anisotropies in the particles produced as shown in Figure~\ref{fig:overlap} \cite{Busza:2018rrf}. 
In this picture, the non-uniform structure of the colliding nuclei would cause a momentum anisotropy \cite{Ster:1999ib} that would be further enhanced when looking at collisions that are less central and do not 
have perfect overlap between the colliding nuclei \cite{Poskanzer:1999ea, Pinkenburg:1999ya}. These observations were seen in azimuthal correlation measurements implying that the medium is indeed strongly 
coupled \cite{Aaboud:2018ves, PhysRevLett.91.182301, Sirunyan:2017fts, PhysRevLett.116.132302}. 

\begin{figure}[htbp]
\begin{center}
\includegraphics[width=0.85\textwidth]{figures/theory/overlap}
\caption{Schematic diagrams of the initial overlap region (left) and the final spatial anisotropy generated (right). Taken from \cite{RevModPhys.90.025005}.}
\label{fig:overlap}
\end{center}
\end{figure}

%At the peak energy density of the collision, the system cannot be described at the level of hadrons, and has to be described in terms of quarks and gluons. The initial anisotropic energy density being reflected in the azimuthal variation of particle production implies a strongly coupled medium that expands hydrodynamically, with a faster expansion in the direction of larger gradients and hence resulting a momentum anisotropy. 

A Fourier Transform of the angular distribution of charged hadrons in the collision debris can quantify these momentum anisotropies and give the anisotropic flow coefficients $v_n$, defined as \cite{Poskanzer:1998yz}:

\begin{align}
\frac{d\bar{N}}{d\phi} = \frac{N}{2\pi} \left( 1 + 2 \sum_{n=1}^{\infty} v_{n} \cos(n(\phi-\Psi_n)) \right)
\end{align}

where $\phi$ is the angle in the transverse plane, $\Psi_n$ are the event plane angles, and $N$ is the average number of particles per event. Some of these coefficients are shown in 
Figure~\ref{fig:flow_coeff}. The measured anisotropies can be used to constrain the specific viscosity given by the ratio of viscosity to entropy density, $\eta / s$, and have shown that 
the QGP has a $\eta / s$ of near the theoretical minimum of $1/4\pi$ \cite{Heinz:2013th}. The azimuthal correlations that are a result of flow also provide information about the relativistic 
hydrodynamic nature of the medium. 


\begin{figure}[htbp]
\begin{center}
\includegraphics[width=0.65\textwidth]{figures/theory/flow_coefficients}
\caption{Comparison of a hydrodynamic model from \cite{Niemi:2015qia} to anisotropy measurements by ALICE \cite{ALICE:2011ab} for different parameterizations of $\eta / s $ and for different $v_n$, {\it{n}} = 2, 3, 4 from top to bottom, as a function of collision centrality.}
\label{fig:flow_coeff}
\end{center}
\end{figure}

The Bjorken energy density of the QGP can be derived using \cite{PhysRevD.27.140}:

\begin{align}
\varepsilon \geq \frac{d\Et/d\eta}{\tau_0 \pi R^2} = \frac{3}{2} \langle \Et/N \rangle \frac{d\Nch/d\eta}{\tau_0 \pi R^2}
\end{align}

where $d\Nch/d\eta$ is the number of charged particles produced per unity pseudorapidity, $d\Et/d\eta$ is the transverse energy per unit pseudorapidity, $\tau_0$ is the thermalization time, $R$ is the nuclear radius, and $\Et/N \approx 1$ GeV is the transverse energy per emitted particle. As shown in Figure~\ref{fig:energyDensity}, the energy density at the LHC was measured to be approximately 15 $\mathrm{GeV} / \mathrm{fm}^3$, much higher than the values measured at RHIC \cite{Adcox:2004mh, Krajcz_r_2011}.




\begin{figure}[htbp]
\begin{center}
\includegraphics[width=0.65\textwidth]{figures/theory/energyDensity}
\caption{$d\Nch/d\eta$ per colliding nucleon pair as a function of collision energy in \pp\ and nucleus-nucleus collisions \cite{Muller:2012zq}. }
\label{fig:energyDensity}
\end{center}
\end{figure}






















