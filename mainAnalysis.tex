% !TEX encoding = UTF-8 Unicode
% !TEX root = thesis-ex.tex

\section{Overview}

Heavy ion collisions at collider energies are performed in order to produce and study QCD matter at high temperature, the quark-gluon plasma (QGP).
 
Measurements of jets in such collisions are powerful tools to determine the properties of this matter by measuring the modification of jet production and fragmentation after the jets have traversed the hot QCD matter.
The rates of jet production~\cite{Abelev:2013kqa,Aad:2014bxa,Khachatryan:2016odn} and their correlations~\cite{Aad:2010bu} are observed to be modified in a centrality dependent manner in \PbPb\ collisions at 2.76~TeV and 5.02~TeV\cite{ATLAS:2017wvp}.
In addition, the longitudinal momentum distribution of charged particles within jets measured is observed to be modified as well~\cite{Aad:2014wha,Chatrchyan:2014ava, Aaboud:2017bzv,PbPb5TeVIntNote}.
Significant modifications of the jet fragmentation are observed, including an excess of particles with transverse momentum (\pT) less than about 4~GeV.
The excess is observed to increase with both centrality and the transverse momentum of the jet (\pTjet).
However, these measurements are insensitive to the angular distribution of charged particles within the jet and particles outside of the jet cone.
This information is crucial to understanding how the jet interacts with the hot QCD matter created in \pbpb\ collisions at the LHC.
The measurement of the particle distributions at large angles outside the jet cone provides information about the flow of transverse momentum lost by the jet due to the jet quenching phenomena.
The angular distribution of the soft particles around the jet is a key feature of models of the interaction of the jets with the hot QCD matter~\cite{Blaizot:2014ula,Brewer:2017fqy} and the response of the QCD matter to the propagation of the jet~\cite{Tachibana:2017syd,Yan:2017rku}.

In this note, a measurement of the angular and transverse momentum distributions of tracks around the jet axis is presented.
The distributions are presented as a function of centrality and \pTjet.
In order to quantify the effects due to the presence of the hot QCD matter, the same quantities are measured in \pp\ collisions at the same collision energy to provide a baseline of unmodified jets.

In this analysis, we extend previous ATLAS fragmentation studies by measuring of the angular distributions of charged particles around the jet axis, including those outside the jet cone.
Previous measurements from CMS~\cite{Khachatryan:2016tfj,CMSPASHIN16020} have studied similar correlations.
This measurement extends those by including the \pTjet\ dependence of the correlations and by utilizing unfolding such that detector effects are removed from the final results.






Ultra-relativistic nuclear collisions at the Large Hadron Collider (LHC) produce hot, dense matter called the quark-gluon plasma, QGP (see Refs.~\cite{Roland:2014jsa,Busza:2018rrf} for recent reviews).
Jets from hard-scattering processes in these collisions traverse and interact with the QGP, losing energy via a process called jet-quenching.
The rates and characteristics of these jets in heavy-ion collisions can be compared to the same quantities in \pp\ collisions, where we do not expect the production of QGP.
This comparison can provide information on the properties of the QGP and how it interacts with partons from the hard scatter.

Jets with large transverse momenta in central lead-lead (\pbpb) collisions at the LHC are measured at approximately half the rates in \pp\ collisions when the nuclear overlap function of \pbpb\ collisions is taken into account~\cite{Abelev:2013kqa,Aad:2014bxa,Adam:2015ewa,Khachatryan:2016jfl, 2019108}.
Similarly, back-to-back dijet~\cite{Aad:2010bu,Chatrchyan:2011sx,Aaboud:2017eww} and photon-jet pairs~\cite{Chatrchyan:2012gt,Aaboud:2018anc} are observed to have less balanced transverse momenta in \pbpb\ collisions compared to \pp\ collisions.
These observations suggest that some of the energy from the hard-scattered parton may be transferred outside of the jet through its interaction with the QGP medium.
 
Complementary measurements look at how the structure of jets is different between \pbpb\ and \pp\ collisions.
Jet shape measurements in the \pp\ and \pbpb\ collision systems have shown a broadening of the jets due to the QGP~\cite{Aad:2011sc, Acharya:2018uvf, Chatrchyan:2012mec, Chatrchyan:2013kwa}.
Additionally, measurements of jet fragmentation functions at the LHC show an excess, in PbPb collisions, of low and high momentum particles with a depletion of intermediate momentum particles inside the jet compared to pp collisions~\cite{Aad:2014wha,Chatrchyan:2014ava,Aaboud:2017bzv,Aaboud:2018hpb}.
Particles carrying a large fraction of the jet momentum are generally closely aligned with the jet axis, whereas low momentum particles are observed to have a much broader angular distribution extending outside the jet~\cite{Chatrchyan:2011sx,Khachatryan:2015lha,Khachatryan:2016tfj,Sirunyan:2018jqr}.
These observations suggest that the energy lost via jet-quenching is being transferred to soft particles around the jet axis via soft gluon emission~\cite{Vitev:2008rz,Ovanesyan:2011xy,Blaizot:2014ula,Qin:2015srf,Escobedo:2016jbm,Casalderrey-Solana:2016jvj,Tachibana:2017syd}.
Measurements of yields of these particles as a function of transverse momentum and angular distance between the particle and the jet axis have a potential to provide further insight into on the structure of jets in the QGP, as well as provide information on how the medium is affected by the presence of the jet.


This paper presents charged-particle \pt\ distributions around the jet axis that have been corrected for detector effects.
The measured yields are defined as:

\begin{align*}
\Dptr = \frac{1}{N_{\mathrm{jet}}} \frac{1}{A} \frac{\mathrm{d} n_{\mathrm{ch}} (\pt, r)}{\mathrm{d} \pt},
%%%
%D(\pt,\ptjet) = \frac{1}{N_{\mathrm{jet}}} ~ \frac{1}{\epsilon(\pttrk)} ~ \frac{\mathrm{d} N_{\mathrm{ch}}}{\mathrm{d} \pt}~(\ptjet).
%%%
\end{align*}
where $r = \sqrt{\Delta \eta^2 + \Delta \phi^2}$ \footnote{ATLAS uses a right-handed coordinate system with its origin at the nominal interaction point (IP) in the centre of the detector, and the $z$-axis along the beam pipe.
The $x$-axis points from the IP to the centre of the LHC ring, and the $y$-axis points upward.
Cylindrical coordinates $(r,\phi)$ are used in the transverse plane, $\phi$ being the azimuthal angle around the $z$-axis.
The pseudorapidity is defined in terms of the polar angle $\theta$ as $\eta=-\ln\tan(\theta/2)$.
The rapidity is defined as $y = 0.5\text{ln}[(E + p_z)/(E-p_z)]$ where $E$ and $p_z$ are the energy and $z$-component of the momentum along the beam direction respectively.
Transverse momentum and transverse energy are defined as $\pt = p \sin\theta$ and $\Et = E \sin\theta$, respectively.
The angular distance between two objects with relative differences $\Delta \eta$ and $\Delta \phi$ in pseudorapidity and azimuth respectively is given by $\sqrt{(\Delta \eta )^2 + (\Delta \phi)^2}$.} 
is the angular distance from the jet axis and $N_{\mathrm{jet}}$ is the number of jets in consideration.
$A = \pi (r_{\mathrm{max}}^2 - r_{\mathrm{min}}^2) $ is the area of an annulus around the jet axis with its inner and outer radii $r_{\mathrm{min}}$ and $r_{\mathrm{max}}$ respectively and $n_{\mathrm{ch}}(\pt, r)$ is the number of charged particles with a given \pt\ within the annulus.
The ratios of the charged-particle yields measured in \pbpb\ and \pp\ collisions,

\begin{align*}
   \RDptr = \frac{\Dptr_\mathrm{Pb+Pb}}{\Dptr_{\pp}},
\end{align*}
quantify the modifications of the yields due to the QGP medium.
Furthermore, the differences between the \Dptr\ distributions in \pbpb\ and \pp\ collisions, 

\begin{align*}
   \Delta \Dptr = \Dptr_\mathrm{Pb+Pb} - \Dptr_{pp},
\end{align*}
allow for measuring the absolute differences in charged-particle yields between the two collision systems.



%The analysis is done using 0.49~nb$^{-1}$ of \pbpb\ collisions and 
%25~pb$^{-1}$ of \pp\ collisions at center-of-mass energy of 5.02~\TeV\ collected in 2015 by ATLAS.

%The \pbpb\ collisions are divided into the following centrality intervals: 0--10\%, 10--20\%, 20--30\%, 30--40\%, 40--60\%, 60--80\%.
%It uses jets reconstructed with the \antikt\ algorithm \cite{Cacciari:2008qp} using a radius parameter of \RFour, restricted to the rapidity interval of $|\yjet| <$~1.7 and having transverse momenta \ptjet\ in the 126--316 GeV range.
%Charged particles associated with these jets are restricted to $|\eta| < 2.5$ and have a transverse momenta of $\pt > 1$ GeV.
%The measurement is done in annuli at increasing distances from the jet axis.
%These annuli have their inner and outer radius $r_{\textrm{min}}$ and $r_{\textrm{max}}$ and take the following values: 0.0, 0.05, 0.1, 0.15, 0.2, 0.25, 0.3, 0.4, 0.5, 0.6, 0.7, 0.8.





This section discusses the main analysis undertaken by the author.
This chapter is divided into the sections detailed below:

\begin{itemize}
\item \ref{sec:trkjet_corr_measurement} defines and discusses the quantity measured,
%\item \ref{sec:used_data} discusses the datasets used
\item \ref{sec:event_selection} describes the various event selection criteria
\item \ref{sec:cuts_corrections} describes the cuts and corrections applied to the measured quantities
\item \ref{sec:systematic} discusses the systematic uncertainties
\item \ref{sec:results} describes the results
\item \ref{sec:discussion} goes into a discussion of the results.

\end{itemize}

\section{Definition of Measured Quantities}
\label{sec:trkjet_corr_measurement}
% !TEX encoding = UTF-8 Unicode
% !TEX root = thesis-ex.tex

The main quantity of interest here is the charged particle \pt\ distribution in and around the jet as illustrated in Fig.~\ref{Fig:dpt_def}.
The measured quantity is defined as:
  \begin{equation}
  \Dptr = \frac{1}{N_{\mathrm{jet}}} \frac{1}{\mathrm{A}} \frac{\mathrm{d} n_{\mathrm{ch}} (\pt, r)}{\mathrm{d} \pt},
\end{equation}

where $N_{\mathrm{jet}}$ is the number of jets in consideration, $A = \pi (r_{\mathrm{max}}^2 - r_{\mathrm{min}}^2) $ is the area of an annulus around the jet with its inner and outer radii $r_{\mathrm{min}}$ and $r_{\mathrm{max}}$.
The angular distance from the jet axis is given by $r = \sqrt{\Delta \eta^2 + \Delta \phi^2}$\footnote{$\Delta \eta$ and $\Delta \phi$ are the distances between the jet axis and the charged particle position in pseudorapidity and azimuth}, and $n_{\mathrm{ch}}(\pt, r)$ is the number of charged particles with a given \pt\ within the annulus.
The measurement is performed for the following successive intervals in $r$ around the jet, forming the annuli with inner and outer radii $r_{\textrm{min}}$ and $r_{\textrm{max}}$: 0.0, 0.05, 0.1, 0.15, 0.2, 0.25, 0.3, 0.4, 0.5, 0.6, 0.7, 0.8


\begin{figure}
\centerline{
\includegraphics[width=0.55\textwidth]{figures/main/general/fragScheme_Shape.pdf} }
\caption{Illustration of the tracks in and around the jet.
}
\label{Fig:dpt_def}
\end{figure}

These distributions are of interest because they indicate how the energy of the jet is lost both in and outside the jet in \pbpb\ collisions.
Similar measurements have been made by CMS~\cite{CMSPASHIN16020, Chatrchyan:2014ava}, and ATLAS~\cite{PhysRevC.98.024908, Aaboud:2017bzv}.

The entire analysis flow of this measurement, along with the various cuts and corrections (discussed in Sec.~\ref{sec:cuts_corrections}) is shown in Fig:\ref{Fig:analysis_flow} and briefly described in the following paragraph.

\begin{figure}
\centerline{
\includegraphics[width=20.cm]{figures/main/general/Shape_analyses_flow.pdf}}
\caption{The diagram presents various corrections and cuts that are applied during the analysis.}
\label{Fig:analysis_flow}
\end{figure}

First, the measured charged particle yield, $\text{d}n^{\text{meas}}_{\text{ch}}/\text{d}\pTch$, within an annulus with radii $r_{\text{min}}$ and $r_{\text{max}}$ is evaluated as:
\begin{equation}
\frac{\text{d}n^{\text{meas}}_{\text{ch}}}{\text{d}\pTch} = \frac{1}{\epsilon(\pttrk, \etatrk)} \frac{\Delta N_{\text{ch}} (\pTch, r)}{\Delta \pTch}
\end{equation}

where $\Delta N_{\text{ch}} (\pTch, r)$ is the number of charged particles in a given \pTch\ range that passed the jet and track selection criteria, $r = (r_{\text{min}} + r_{\text{max}}) / 2$, and $\epsilon(\pttrk, \etatrk)$ is the charged particles reconstruction efficiency correction, applied on a track-by-track basis.
In \pbpb\ collisions, the measured distributions are affected by charged particles from the underlying event, and thus need to be subtracted out (see Sec.~\ref{sec:cuts_corrections} for details):

\begin{equation}
\frac{\text{d}n^{\text{sub}}_{\text{ch}}}{\text{d}\pTch} = \frac{\text{d}n^{\text{meas}}_{\text{ch}}}{\text{d}\pTch} - \frac{\text{d}n^{\text{UE}}_{\text{ch}}}{\text{d}\pTch}
\end{equation}

The final \Dptr\ distributions are then evaluated after unfolding and normalizing with respect to the unfolded number of jets, $N_{\text{jet}}^{\text{unfolded}}$, as well as the area $A$ of the annulus at given distance $r$ :
\begin{equation}
\Dptr = \frac{1}{N_{\text{jet}}^{\text{unfolded}}} \frac{1}{\text{A}} \frac{\text{d}n^{\text{unfolded}}_{\text{ch}}}{\text{d}\pTch} \quad \quad \text{where } A = \pi (r_{\text{max}}^2 - r_{\text{min}}^2)
\end{equation}

The unfolding procedure is a combination of a two-dimensional Bayesian unfolding method in \ptjet\ and \pttrk, one-dimensional Bayesian unfolding method to correct jet spectra for the normalization and a one-dimensional bin-by-bin correction for the jet and track position resolution.


The analysis is performed differentially in \ptjet, and centrality, with the jet \pt\ bin size growing logarithmically with \ptjet\ to ensure good statistics in the full range of the measurement.
This scheme was also used in other ATLAS jet measurements~\cite{ATLAS276FFConf}.


In order to quantify the differences between charged particle spectra in \pbpb\ and \pp\  collisions, the ratios of the charged particle spectra in \pbpb\ collisions to those in \pp\ collisions are also reported:
\begin{equation}
   R_{\Dptr} \equiv \frac{\Dptr_{\pbpb}}{\Dptr{\pp}}
\end{equation}




\section{Input Data}
\label{sec:used_data}
% !TEX encoding = UTF-8 Unicode
% !TEX root = thesis-ex.tex

The \PbPb\ and \pp\ data used in this analysis were recorded in 2015.
The data samples consisted of 25~pb$^{-1}$ of $\sqrts=5.02$ TeV \pp\ and 0.49~nb$^{-1}$ of $\sqrtsnn =5.02$ TeV \pbpb\ data.

%%L1: 10MHz to 100kHz
%%HLT: 100kHz to 1.5 kHz

Events in both the \pp\ and \pbpb\ samples were selected by the ATLAS Trigger system discussed in Chapter~\ref{sec:setup}.
The general scheme is to identify events using the Level 1 (L1) triggers, and pass them as ``seeds'' to the High Level Trigger (HLT).
In \pbpb, the selection was based on the L1 Total Energy trigger, \texttt{L1\_TE50} that identified events with at least 50 GeV in the calorimeter system.
These events were passed to the HLT, where the \texttt{HLT\_j75\_ion\_L1TE50} used an online jet reconstruction algorithm to select on jets above \mbox{75 GeV}.
In \pp, the event selection was done using a L1 jet trigger, \texttt{L1\_j20}, that used a simple sliding window algorithm to find jet candidates with a $\ptjet > 20$ GeV.
These were then used as seeds to the HLT, where the \texttt{HLT\_j85} trigger further selection on jets with $\ptjet > 85$ GeV.
The performance of the jet triggers in 2015 is described in Refs.~\cite{HITMF, Aaboud:2016leb} and the trigger efficiency is shown in Figure~\ref{fig:trigger_selections}.
This analysis then further selected jets with $\ptjet > 100$ GeV, thus ensuring a fully efficient trigger selection.

In addition to the jet triggered samples described above, a Minimum Bias \pbpb\ data sample was also recorded.
This was triggered based on a logical OR of the total energy trigger with a threshold of 50~\GeV\ and the ZDC coincidence trigger was used as part of the MC overlay procedure

%These were triggered using a logical OR of two triggers: 1) total energy Level-1 trigger selecting more central collisions; 2) ZDC coincidence trigger at Level-1 and a veto on the total energy trigger, with the additional requirement of least one track in the HLT, selecting peripheral collisions.


\begin{figure}
\centerline{
\begin{tabular}{cc}
\includegraphics[width=0.45\textwidth]{figures/main/general/Eff_pp_5TeV_central.pdf} &
\includegraphics[width=0.45\textwidth]{figures/main/general/trigger_eff_PbPb_CentInclusive.pdf}
\end{tabular}}
\caption{Jet trigger efficiencies for (left) \pp\ and (right) 0--80\% central \pbpb\ collisions at 5.02 TeV for R=0.4 offline jets.
The broader turn-on of the jet trigger in \pbpb\ compared to \pp\ collisions is caused by significant differences between the HI jet trigger reconstruction algorithm used at the time of the data taking and the current version of the offline reconstruction software.
Figure from Ref.~\cite{Sickles:2235420} }
\label{fig:trigger_selections}
\end{figure}


%\begin{figure}
%\begin{subfigure}{.5\textwidth}
%\includegraphics[width=1\textwidth]{figures/main/general/Eff_pp_5TeV_central.pdf}
%\caption{.}
%\label{fig:Trigger_pp5}
%\end{subfigure}
%\begin{subfigure}{.5\textwidth}
%\includegraphics[width=1\textwidth]{figures/main/general/trigger_eff_PbPb_CentInclusive.pdf}
%\caption{}
%\label{fig:Trigger_PbPb}
%\end{subfigure}
%\label{fig:trigger_selections}
%\caption{Jet trigger efficiencies for (left) \pp\ and (right) 0--80\% central \pbpb\ collisions at 5.02 TeV for R=0.4 offline jets.
%The broader turn-on of the jet trigger in \pbpb\ compared to \pp\ collisions is caused by significant differences between the HI jet trigger reconstruction algorithm used at the time of the data taking and the current version of the offline reconstruction software.
%Figure from Ref.~\cite{Sickles:2235420} }
%\end{figure}

In both samples, events were required to have a reconstructed vertex within 150~mm of the nominal IP along the beam axis.
The pileup was negligible in the \pbpb\ while the \pp\ data was collected in low pileup mode, where the average number of interactions per bunch crossing in \pp\ collisions ranged from 0.6 to 1.3.
Only events taken during stable beam conditions and satisfying detector and data-quality requirements that include the detector subsystems being in nominal operating conditions were considered.
The total number of \pp\ and \pbpb\ events entering the analysis, along with the with rejection power of various event quality cuts is shown in Figure~\ref{Fig:EventCounts}.
Some of these events are rejected by multiple cuts. ``Rejection by centrality'' indicates the number of events outside the 0-80\% centrality bin.

\begin{figure}
\centerline{
\begin{tabular}{cc}
\includegraphics[width=0.45\textwidth]{figures/main/general/EventAccept_pp.pdf} & 
\includegraphics[width=0.45\textwidth]{figures/main/general/EventAccept_PbPb.pdf}
\end{tabular}}
\caption{The number of 2015 \pp\  (left) and \PbPb\ (right) events used and rejected by various event quality cuts.}
\label{Fig:EventCounts}
\end{figure}



The centrality intervals used in this analysis were defined according to successive percentiles of the \ETfcal\ distribution obtained in minimum bias (MB) collisions, ordered from the most central (highest \ETfcal) to the most peripheral (lowest \ETfcal) collisions: 0--10\%, 10--20\%, 20--30\%, 30--40\%, 40--60\%, 60--80\%.

The \pp\ Monte Carlo (MC) used a set of $1.8\times10^7$ 5.02 TeV hard-scattering dijet \pp\ events generated with \powheg{}+\pythiaeight\ \cite{Nason:2004rx,Sjostrand:2014zea} using the A14 tune of parameters \cite{ATLAS2014021} and the NNPDF23LO PDF set \cite{Ball:2012cx}.
The \pbpb\ MC was generated by overlaying the additional sample of MB \pbpb\ data events on a separate set of $1.8\times10^7$ 5.02 TeV hard-scattering dijet \pp\ events generated with the same tune and PDFs as the \pp\ MC.
This ``MC overlay'' sample was reweighted on an event-by-event basis such that it had the same centrality distribution as the jet triggered sample.
Another sample of MB \pbpb\ events was generated using HIJING (version 1.38b) \cite{Wang:1991hta} and was only used to evaluate the track reconstruction performance.
The detector response in all MC samples was simulated using \textsc{Geant4} \cite{Agostinelli:2002hh,Aad:2010ah}.
These MC samples were used to evaluate the performance of the detector and analysis procedure and correct the measured distributions for detector effects.


%The event fraction as a function of run number for both the hard probes stream and the minimum bias overlay stream in \pbpb\ is shown in Figure~\ref{fig:evnt_fraction}
%
%\begin{figure}[h]
%\centering
%\includegraphics[width=0.5\textwidth]{figures/main/general/EventPercentages_c0.pdf}
%\caption{Event fraction as a function of runs for Hard Probes and the Minimum Bias Overlay Streams in \pbpb\ collisions.}
%\label{fig:evnt_fraction}
%\end{figure}

The time dependence of the underlying event (a core part of this measurement) was tested by dividing the data and MC into three data taking periods with approximately equal number of events in each period.
The underlying event determined for each period compared to the nominal underlying event evaluated for the entire dataset is shown in Figure~\ref{fig:weighted_runs}, and it can be seen that it is stable throughout the data taking period.

 \begin{figure}[h]
\centering
\includegraphics[width=0.75\textwidth]{figures/main/general/weightedRuns.pdf}
\caption{Stability of the underlying event for three different periods of the data taking.
The different curves indicate the ratio of the underlying event in each period of data taking to the underlying event determined in the entire dataset.}
\label{fig:weighted_runs}
\end{figure}







\section{Event Selection }
\label{sec:event_selection}
% !TEX encoding = UTF-8 Unicode
% !TEX root = thesis-ex.tex

The standard ATLAS event quality requirements were applied for the event selection both for the \pp\ and \PbPb\ event selection.
\begin{itemize}
\item All the sub-detector systems were required to be fully functional: all the data were required to pass the official good run list:
 \\ $\texttt{\scriptsize data15\_5TeV.periodAllYear\_DetStatus-v75-repro20-01\_DQDefects-00-02-02\_PHYS\_HeavyIonP\_All\_Good.xml}$ (2015 \pp) 
 \\ $\texttt{\scriptsize data15\_5TeV.periodVdM\_DetStatus-v75-repro20-01\_DQDefects-00-02-02\_PHYS\_HeavyIonP\_All\_Good.xml}$ (2015, VdM \pp)
 \\ $\texttt{\scriptsize data15\_hi.periodAllYear\_DetStatus-v75-repro20-01\_DQDefects-00-02-02\_PHYS\_HeavyIonP\_All\_Good.xml } $ (2015 \pbpb).

\item All events are required to have a good reconstructed primary vertex.
\item The primary vertex must be within 150~mm from the center of ATLAS detector, as a fiducial tracking region.
 
\item Additional event cleaning to remove additional detector imperfections as described here~\cite{2015EventCleaning} is used. 
\item In \PbPb\ collisions the pileup contribution is removed using the  $\texttt{HIAnalysisTools}$ \cite{HIAnalysisTools}. 
\end{itemize}


Figures~\ref{Fig:EventCounts} presents the total number of \pp\ and \pbpb\ events, respectively, entering the analysis together with rejection power of various event quality cuts. A slightly higher fraction of empty events without primary vertex is observed in pp collisions. Some of these events are rejected by multiple cuts. ``Rejection by centrality'' indicates the number of event in HP stream that is outside the 0-80\% centrality bin.

\begin{figure}
\centerline{
\begin{tabular}{cc}
\includegraphics[width=0.45\textwidth]{figures/main/general/EventAccept_pp.pdf} & 
\includegraphics[width=0.45\textwidth]{figures/main/general/EventAccept_PbPb.pdf}
\end{tabular}}
\caption{
The number of 2015 \pp\  (left) and \PbPb\ (right) events used and rejected by various event quality cuts. }
\label{Fig:EventCounts}
\end{figure}


\subsection{Centrality Selection}
\label{sec:cent}

The centrality of the collision is a degree of the overlap of two colliding nuclei that can be quantified by the impact parameter that is the distance between the centers of the two nuclei. If they collide head on the collision is central, if they just graze each other we speak about peripheral collisions. We cannot measure the impact parameter to determine the centrality, but we can measure the overall event activity in the collision, characterized e.g. by the sum of \Et\ measured in FCal calorimeters on both sites. Central collisions have large \Et\ deposits in the FCal, peripheral have small \Et\ deposits.

In this analysis, The \ETfcal\ distribution is divided into percentiles of the total inelastic cross section for \PbPb\ collisions. The first percentile, $0-10\%$, represents the $10\%$ of collisions with the largest event activity, smallest impact parameter. The last percentile, $90-100\%$, represents the $10\%$ of collisions where there is the smallest event activity and largest impact parameter. 
Seven centrality classes have been used: 0-10\%, 10-20\%, 20-30\%, 30-40\%, 40-60\%, 60-80\%. 
The most peripheral collisions 80-100\%, are excluded due to  the small number of jets.
The centrality selections are documented in Ref.~\cite{ref:centrality}. The \PbPb\ MC is re-weighted in the way that it has the same centrality distribution as the jet triggered data sample.

\clearpage


\section{Jet Reconstruction}
\label{sec:reconstruction}
% !TEX root = thesis-ex.tex

\label{Sec:JetRec}
For the measurement presented here, we use the jets reconstructed in the calorimeter 
using the \antikt\ algorithm \cite{Cacciari:2008gp} with \RFour.
The underlying event (UE) contribution to jets is subtracted on 
an event by event basis at the cell level. The details on the jet reconstruction 
procedure and performance in heavy ion collisions have been described in 
\cite{ATLAS-COM-PHYS-2011-1733}, here we will only shortly summarize the main 
features of the heavy ion jet reconstruction.

In order to reconstruct jets in heavy ion collisions, a large background from 
the UE has to be subtracted from each jet. 
The UE subtraction procedure is done in several iterative steps. 
First an estimate of the UE average transverse energy density, $\rho_i(\eta)$, 
is evaluated for each calorimeter layer $i$ in intervals of $\eta$ of width 
$\Delta \eta = 0.1$ using all cells in each calorimeter layer, within a given 
$\eta$ interval excluding those within $\Delta R < 0.4$ of ``seed'' jets. In the first 
subtraction step, the ``seed'' jets are defined to be jets reconstructed using the 
\antikt\ algorithm with \RTwo\ jets which have at 
least one tower  (a tower is a 0.1x0.1 region of the calorimeter and the energy
associated with it is the sum of the energies from all contributing calorimeter layers
in that region)
with $\Et > 3$~GeV and which have a ratio of the maximum to 
the mean tower associated with the jet of at least 4. 
  The UE-subtracted cell energies  were calculated according to:
\begin{equation}
\label{eqn:UE}
E_{\mathrm{T},i}(\eta, \phi)^{\mathrm{sub}} = E_{\mathrm{T},i}(\eta, \phi) - A_i \times \rho_i(\eta) 
\end{equation}
where $E_{\mathrm{T},i}$, $\eta$, $\phi$,  and $A_i$ represent the $\Et$, $\eta$, 
$\phi$, and area of the cell in the layer $i$. The $\rho_i(\eta)$ is the energy density per unit area in the layer $i$. The kinematics for \RTwo\ jets 
generated in this first subtraction step were calculated via a four-vector sum 
of all (assumed massless) cells contained within the jets using the \Et\ values 
obtained from Eq.~\ref{eqn:UE}.

The second subtraction step starts with the definition of a new set of 
seeds using a list of \RTwo\ calorimeter jets from the first 
subtraction step, each with $\Et > 4$~GeV. Using this new set of 
seeds, a new estimate of the UE, $\rho'_i(\eta)$, was calculated excluding 
cells within $\Delta R < 0.4$ of the new ``seed'' jets, where $\Delta R = \sqrt{ 
(\eta_{\mathrm{cell}} - \eta_{\mathrm{jet}})^2 + (\phi_{\mathrm{cell}} - \phi_{\mathrm{jet}})^2}$.


The jet energy scale calibration is based on the numerical inversion method and provides calibration constants for all jet collections used in this study~\cite{CalibrationTwiki}. The final jet energy calibration using in-situ studies is applied in the offline analysis and it is described in Sec~\ref{Sec:JetSelection}.   



The jet reconstruction performance in 5.02 TeV \pp\ collisions was evaluated using corresponding MC samples with a full detector simulation. The kinematics of the truth jets are reconstructed from primary particles\footnote{Primary particles are defined as having a mean lifetime of $\tau > 0.3 \times 10^{-10}$ s, and are produced directly in \pp\ interactions or from decays of particles with shorter lifetimes} with the \antikt\ algorithm with radius parameter $R = 0.4$. The jet reconstruction efficiency, JES (in this case evaluated as $\langle (\ETreco)\rangle/\ETtrue$), and JER for \pp\ collisions is shown in Fig.~\ref{Fig:Performancepp5} for \RFour\ jet.
For \pbpb\ collisions the JES is shown in Fig.~\ref{Fig:PerformancepbpbJES} and the JER is shown
in Figure~\ref{Fig:PerformancepbpbJER}.  Further studies of the jet performance in the 2015 \pbpb\
data are found in Ref.~\cite{Aad:2014bxa}. Figures~\ref{Fig:PerformancepbpbJPReta0p4}-\ref{Fig:PerformancepbpbJPRphi0p4} present the jet angular resolution in $\eta$ and $\phi$ as a function of jet \pt\ evaluated in six centrality classes. The angular resolution is improving with the increasing jet \pT\ and decreasing collision centrality. The angular resolution is found to be significantly better for smaller jets as expected since the smaller jets are less affected by the presence of the UE. 

\begin{figure}
\centerline{
\begin{tabular}{cc}
\includegraphics[width=7cm]{figures/main/figures_general/Eff_pp5.pdf} &
\includegraphics[width=7.3cm]{figures/main/figures_general/JES_pp5.pdf} \\
\includegraphics[width=7.3cm]{figures/main/figures_general/JER_pp5.pdf} 
\end{tabular}}
\caption{
Top panels: Jet reconstruction efficiency in 5.02 TeV \pp\ collisions (left) as a function of truth jet \pT\ and different $\eta$ bins. Jet energy scale (JES) in 5.02 TeV \pp\ collisions (right) as a function of truth jet \pT\ and different $\eta$ bins. Bottom panels: Jet energy resolution (JER) in 5.02 \pp\ collisions as a function of truth jet \pT\ and different $\eta$ bins.
}
\label{Fig:Performancepp5}
\end{figure}

\begin{figure}
   \centering
   \includegraphics[width = 0.75\textwidth]{figures/main/figures_general/PbPb_JES_pT_eta2p8.pdf}
   \caption{ JES in \pbpb\ collisions for eight centrality selections.  Plot is from Ref.~\cite{Aad:2014bxa}.}
   \label{Fig:PerformancepbpbJES}
\end{figure}

\begin{figure}
   \centering
   \includegraphics[width = 0.75\textwidth]{figures/main/figures_general/PbPb_JER_pT_eta2p8.pdf}
   \caption{ JER in \pbpb\ collisions for eight centrality selections.  Plot is from Ref.~\cite{Aad:2014bxa}. The points are fit to the standard function that describes the calorimetric resolution.}
   \label{Fig:PerformancepbpbJER}
\end{figure}


\begin{figure}
   \centering
   \includegraphics[width = 0.75\textwidth]{figures/main/figures_general/jet_res_eta_r04.pdf}
   \caption{ Jet angular resolution in $\eta$ for $R=0.4$ jets in \pbpb\ collisions as a function of jet \pT\ for six centrality selections.}
   \label{Fig:PerformancepbpbJPReta0p4}
\end{figure}

\begin{figure}
   \centering
   \includegraphics[width = 0.75\textwidth]{figures/main/figures_general/jet_res_phi_r04.pdf}
   \caption{ Jet angular resolution in $\phi$ for $R=0.4$ jets in \pbpb\ collisions as a function of jet \pT\ for six centrality selections.}
   \label{Fig:PerformancepbpbJPRphi0p4}
\end{figure}



\section{Basic Cuts and Corrections}
\label{sec:cuts_corrections}
\subsection{Overview}

In both the \pp\ and \pPb\ MC and data samples, two highest \pt\ jets are used to study azimuthal angular correlations. This measurement uses jets with a transverse momentum from 28~GeV to 90~GeV, in a \ystar\ range from -4.0 to 4.0. The final observables in this analysis are widths of di-jet \Dphi\ distributions and conditional yields. The widths are sensitive to broadening between the leading and sub-leading jets and the yields show the number of di-jets, given a leading jet in each \pT\ and \ystar\ kinematic region. 

The binning of this measurement is summarized in  Table~\ref{tab:binning} and is composed of different combinations of \ystarone, \ystartwo, \ptone, and \pttwo, where (\ystarone, \ptone) is the position and transverse energy of the leading jet, and (\ystartwo, \pttwo) the position and transverse energy of the sub-leading jet. Since the measurement aims to probe low-x partons, only the interval $2.7<\ystarone<4.0$, which is the proton going direction in \pPb\ is used. In the 2016 \pPb\ Transverse momentum binning was chosen on the edges of \pt\ intervals used for different triggers in \pp. 

Leading jet \ptone\ spectra are estimated in different \ystarone\ bins, unfolded, and used as a normalization of \Dphi\ distributions. Di-jet azimuthal angular correlation distributions are evaluated as a function of \Dphi\ in combinations of \ystarone, \ystartwo, \ptone, and \pttwo\ bins, unfolded, and normalized by the leading jet \pt\ spectra. The \Dphi\ distributions are fitted to extract the widths, which do not depend on the overall normalization. Conditional yields are obtained by integrating the \Dphi\ distributions over their full range so the correct normalization by number of leading jets is important. 

\begin{table}
	\centering
	\begin{tabular}{|| c | c | c || } 
		\hline
		\ptone Bins [GeV] & \pttwo Bins [GeV] & \ystartwo Bins \\ 
		\hline
		$28<\ptone<35$   & $28<\pttwo<35$  & $2.7<\ystarjet<4.0$ \\ 
		$35<\ptone<45$   & $35<\pttwo<45$  & $1.8<\ystarjet<2.7$ \\ 
		$45<\ptone<90$   & $45<\pttwo<90$  & $0.0<\ystarjet<1.8$ \\
						 & 				   & $-1.8<\ystarjet<0.0$ \\
						 &				   & $-4.0<\ystarjet<-1.8$ \\
		\hline
	\end{tabular}
	\caption{\label{tab:binning} Transverse momentum and \ystar\ binning for leading and sub-leading jets. For the leading jet, only the $2.7<\ystarone<4.0$ bin is used. }
\end{table}

To account for detector affects, the distributions in data have to be unfolded using MC information. The unfolding method used is the bin-by-bin unfolding which relies on MC information about the relationship between any truth and reconstructed quantity. This type of unfolding is sensitive to differences in the shapes of data and MC distributions and requires a re-weighting of the MC before unfolding factors can be evaluated. 

\subsection{Unfolding Procedure}
\label{sec:unfolding}
Due to effects of bin migration from JER and position resolution, it is necessary to perform an unfolding to account for these effects. Bayesian unfolding was first attempted, but the sensitivity to statistic fluctuations did not give good convergence. As a result, the bin-by-bin unfolding is the method used throughout the analysis. With this procedure, migration along multiple \ystar\ and \pT\ bins can be accounted for, more information can be found in Appendix~\ref{sec:appendixA}. Pairs of truth and reconstructed jets are used to fill the respective distributions and response matrices. The diagonal elements of these matrices represent pairs of truth and reconstructed jets agree in momentum and position intervals of the measurement. The response matrix is always a multidimensional object with twice the number of dimensions used in the phase space of the measurement. In \Dphi\ bins with index $i$, the correction factors $C_{i}$ are defined as  

\begin{eqnarray}
C_{i} = \frac{T_{i}}{R_{i}}
\label{eqn:factors}
\end{eqnarray}

where $T_{i}$ and $R_{i}$ are the number of truth and reconstructed di-jets, respectively.  Due to the fact that $T_{i}$ and $R_{i}$ are partially correlated, the resulting errors on the correction factors are defined as

\begin{eqnarray}
\delta C_{i}^{2} = \frac{T_{i}^{2}}{R_{i}^{3}}\bigg(1-\frac{M_{ii}^{2}}{T_{i}R_{i}}\bigg)
\label{eqn:factorserrors}
\end{eqnarray}

where $M_{ii}$ are the diagonal elements of the response matrix. These errors take into account the correlation between the truth and reconstructed quantities.

The bin-by-bin unfolding procedure is sensitive to the shapes of the distributions from which the correction factors are derived. This method works when the shape of the data distribution matches the shape of the MC distributions. Since both the spectra and \Dphi\ distributions are unfolded with correction factors, both the MC spectra and MC \Dphi\ distributions must first be re-weighted. The weights are estimated as ratios of distributions of $\mathrm{Data/MC_{Reco}}$. The value of the weight for a given truth and reconstructed jet pair is obtained from the truth jet kinematics. This procedure is done for all jet measurements and is motivated by the need to re-weight the prior (truth) distribution. Further, re-weighting using reconstructed kinematics could introduce inefficiency to the response matrix. In the following procedure, jet \pt\ spectra weights are derived first. Then \Dphi\ weights are derived with the spectra weight applied. With this intermediate re-weighting in jet \pt\ spectra, it is found that the \Dphi\ weights are invariant in \pT, allowing extrapolation into underflow and overflow bins in \pT, and reducing statistical fluctuations. Final \Dphi\ weights are derived only as a function of \Dphi\ in bins of \ystar, removing the \pT\ dependence. The product of spectra weights and the \Dphi\ weights is applied to the final MC distributions when deriving the correction factors.

From the re-weighted MC truth and reconstructed distributions, correction factors are derived and applied to data both for the spectra and \Dphi\ distributions. The unfolded \Dphi\ data distributions are scaled by the unfolded leading jet \pt\ spectra information, and fitted to the exponentially modified Gaussian function. 


\subsection{Jet Spectra}

Jets in \pp\ and \pPb\ data are required to have a trigger fired, and any jet(s) are required to be in the trigger's pseudorapidity range and transverse momentum interval where the trigger efficiency is above $99\%$. The jets are entered with prescale weights given by the ATLAS Lumi-Calc for each trigger and run. For the $2.7<\ystarone<4.0$ rapidity range, the contribution of different triggers to the final spectra is shown for \pp\ data in Figure~\ref{fig:ppspectrawithtrig}. The leading jet \pt\ spectra for \pp\ data are presented in different forward \ystar\ bins in Figure~\ref{fig:ppspectra} and for \pPb\ data in Figure~\ref{fig:pPbspectra}. In \pPb\ data, only one trigger with no pre-scale is used, thus, unlike the \pp\ spectra, where there are many trigger contributions, the final spectra is composed entirely of one trigger. The \pT\ binning is consistent with what is shown in Table~\ref{tab:binning} because these spectra will eventually be used for normalization of \Dphi\ distributions.

\begin{figure}
	\centering
	\includegraphics[width=0.65\textwidth]{output/output_pp_data/ystar_spect_All.pdf} 
	\caption{ Single-jet \pt\ spectra for jets in \pp\ data in bins of \ystar. }	
	\label{fig:ppspectra}
\end{figure}

\begin{figure}
	\centering
	\includegraphics[width=0.65\textwidth]{output/output_pp_data/ystar_spect_fine_40_Ystar1_27.pdf} 
	\caption{ Individual triggers, and resulting jet \pT\ spectra for \pp\ data for the $2.7<\ystarone<4.0$ rapidity range. }	
	\label{fig:ppspectrawithtrig}
\end{figure}

\begin{figure}	\centering
	\includegraphics[width=0.65\textwidth]{output/output_pPb_data/ystar_spect_All.pdf} 
	\caption{ Single-jet \pt\ spectra for jets in \pPb\ data in bins of pseudorapidity. }	
	\label{fig:pPbspectra}
\end{figure}

In MC, jet \pt\ spectra are filled separately for each cross setction weighted (JZx) sample, and then combined using the cross section weights and filtering efficiencies. Reconstructed and truth leading jet \pt\ spectra for the \pp\ MC are shown in Figure~\ref{fig:ppmcrecospectra} and for the \pPb\ MC in Figure~\ref{fig:pPbmcrecospectra}. 

\begin{figure}
	\centerline{
		\begin{tabular}{cc}
			\includegraphics[width=0.45\textwidth]{output/output_pp_mc_pythia8/ystar_spect_reco_All.pdf} & 
			\includegraphics[width=0.45\textwidth]{output/output_pp_mc_pythia8/ystar_spect_truth_All.pdf}  \\
		\end{tabular}
	}
	\caption{ Reconstructed  (left) and truth (right) level leading jet \pt\ spectra in \pp\ MC in bins of \ystar.}	
	\label{fig:ppmcrecospectra}
\end{figure}

\begin{figure}
	\centerline{
		\begin{tabular}{cc}
			\includegraphics[width=0.45\textwidth]{output/output_pPb_mc_pythia8/ystar_spect_reco_All.pdf} & 
			\includegraphics[width=0.45\textwidth]{output/output_pPb_mc_pythia8/ystar_spect_truth_All.pdf} \\
		\end{tabular}
	}
	\caption{ Reconstructed  (left) and truth (right) level leading jet \pt\ spectra in \pPb\ MC in bins of \ystar.} \label{fig:pPbmcrecospectra}
\end{figure}

\FloatBarrier
\subsection{Jet Spectra Re-weighting}
The leading jet \pt\ spectra weights in both the \pp\ and \pPb\ MCs are derived as the ratio of $Data/MC_{Reco}$ leading jet \pt\ spectra. Jet spectra with fine \pT\ binning are used to have better sensitivity to the shape. The data and MC leading jet \pt\ spectra with fine \pT\ binning are shown for \pp\ in Figure~\ref{fig:ppspectfine}, and for \pPb\ in Figure~\ref{fig:pPbspectfine}. The weights are derived by first scaling the Data and MC spectra to a common integral and then taking their quotient in bins of \ystar. The spectra weights are smoothed to avoid introducing statistical fluctuations. The smoothed \pp\ and \pPb\ leading jet \pt\ spectra weights as a function of \ptone\ are shown in Figure~\ref{fig:spectweights}.

\begin{figure}[ht]
	\centerline{
		\begin{tabular}{cc}
			\includegraphics[width=0.5\textwidth]{output/output_pp_data/ystar_spect_fine_All.pdf} &
			\includegraphics[width=0.5\textwidth]{output/output_pp_mc_pythia8/ystar_spect_fine_reco_All.pdf} \\
		\end{tabular}
	}
	\caption{Leading jet \pt\ spectra in fine bins if \pT\ for \pp\ data (left) and MC (right). }
	\label{fig:ppspectfine}
\end{figure}


\begin{figure}[ht]
	\centerline{
		\begin{tabular}{cc}
			\includegraphics[width=0.5\textwidth]{output/output_pPb_data/ystar_spect_fine_All.pdf} &
			\includegraphics[width=0.5\textwidth]{output/output_pPb_mc_pythia8/ystar_spect_fine_reco_All.pdf} \\
		\end{tabular}
	}
	\caption{Leading jet \pt\ spectra in fine bins if \pT\ for \pPb\ data (left) and MC (right). }
	\label{fig:pPbspectfine}
\end{figure}

\begin{figure}[ht]
	\centerline{
		\begin{tabular}{cc}
			\includegraphics[width=0.5\textwidth]{output/output_pp_mc_pythia8/h_spect_weights_All.pdf} &
			\includegraphics[width=0.5\textwidth]{output/output_pPb_mc_pythia8/h_spect_weights_All.pdf} \\
		\end{tabular}
	}
	\caption{Leading jet \pt\ spectra weights for \pp\ (left) and \pPb\ (right). Only the $2.7<\ystarone<4.0$ bin is used in the analysis but the other \ystarone\ bins are shown in \pp\ for comparison.}
	\label{fig:spectweights}
\end{figure}

The shape of the re-weighted reconstructed level MC jet spectra should match the shape of the reconstructed level jet spectra from data. To check this, reconstructed jet spectra from data are compared to reconstructed jet spectra before and after re-weighting in MC. The ratio of data to re-weighted MC is consistent with unity for \pp\ and \pPb\ reconstructed jet spectra as shown in Figure~\ref{fig:spectwithwithoutweight}.

\begin{figure}[ht]
	\centerline{
		\begin{tabular}{cc}
			\includegraphics[width=0.5\textwidth]{output/output_pp_data/hSpectMC_40_Ystar1_27.pdf} &
			\includegraphics[width=0.5\textwidth]{output/output_pPb_data/hSpectMC_40_Ystar1_27.pdf} \\
		\end{tabular}
	}
	\caption{Reconstructed level data (black) and re-weighted (red) and default (blue) reconstructed jet spectra from MC, with ratios. The ratio of re-weighted MC to data is consistent with unity for \pp\ (left) and \pPb\ (right). Shown for  $2.7<\ystarone<4.0$.}
	\label{fig:spectwithwithoutweight}
\end{figure}

Jet spectra are not re-weighted in \ystar\ because the effect from the JAR is much smaller than from JER and additionally, wide bins in rapidity are used. Response matrices for \pp\ and \pPb\ MC showing migration in \ystar\ are shown in Figure~\ref{fig:ystarrespmat}. There is very minor migration, with a purity of over 99\% indicating no change in the shape of the distribution as a function of \ystar.

\begin{figure}[ht]
	\centerline{
		\begin{tabular}{cc}
			\includegraphics[width=0.5\textwidth]{output/output_pp_mc_pythia8/h_yStarRespMat_28_Pt_35.pdf} &
			\includegraphics[width=0.5\textwidth]{output/output_pPb_mc_pythia8/h_yStarRespMat_28_Pt_35.pdf} \\
		\end{tabular}
	}
	\caption{Response matrices for \ystar, shown for \pp\ (left) and \pPb\ MCs. High purity indicates very minor effect on the shape of the distribution. Shown for the $28<\pt<35$ GeV interval.}
	\label{fig:ystarrespmat}
\end{figure}

\FloatBarrier
\subsection{Jet Spectra Unfolding}
To unfold the leading jet \pT\ spectra, the unfolding procedure described in~\ref{sec:unfolding} is used with correction factors obtained from the ratio the truth to reconstructed leading jet \pt\ spectra. The response matrix describes the bin migration between \pttruth\ and \ptreco. The \pp\ reconstructed and truth jet \pt\ spectra, with the response matrix and resulting correction factors are shown in Figure~\ref{fig:ppspectCFrespmat}. Similarly, the \pPb\ reconstructed and truth jet \pt\ spectra, with the response matrix and resulting correction factors are shown in Figure~\ref{fig:pPbspectCFrespmat}. The correction factors and ratios of unfolded to reconstructed MC are shown as a check that the unfolding procedure is working correctly, not as a check of closure.

\begin{figure}[ht]
	\centerline{
		\begin{tabular}{c}
			\includegraphics[width=0.6\textwidth]{output/output_pp_mc_pythia8/h_ystar_spect_unfolded_All_MUT_40_Ystar1_27.pdf} \\
			\includegraphics[width=0.6\textwidth]{output/output_pp_mc_pythia8/h_ystar_spect_respMat_All_40_Ystar1_27.pdf} \\
		\end{tabular}
	}
	\caption{ \pp\ MC reconstructed and truth jet \pt\ spectra distributions (top plot), the resulting correction factors (middle plot) and the \pT\ response matrix (bottom plot). }
	\label{fig:ppspectCFrespmat}
\end{figure}

\begin{figure}[ht]
	\centerline{
		\begin{tabular}{cc}
			\includegraphics[width=0.6\textwidth]{output/output_pPb_mc_pythia8/h_ystar_spect_unfolded_All_MUT_40_Ystar1_27.pdf} \\
			\includegraphics[width=0.6\textwidth]{output/output_pPb_mc_pythia8/h_ystar_spect_respMat_All_40_Ystar1_27.pdf} \\
		\end{tabular}
	}
	\caption{ \pPb\ MC reconstructed and truth jet \pt\ spectra distributions with correction factors (top plot), and the \pT\ response matrix (bottom plot). }
	\label{fig:pPbspectCFrespmat}
\end{figure}

\FloatBarrier
\subsection{Di-Jet Azimuthal Angular Distributions}
Distributions of the azimuthal angular correlations |\Dphi| of two jets are constructed from the leading and sub-leading jet kinematics. In \pp\ and \pPb\ data, a trigger is required, and the leading jet is required to be in the trigger's pseudorapidity and transverse momentum range. In the di-jet system there is a combinatoric contribution which can come from split jets or multi-parton scattering in both \pp\ and \pPb, as well as hard scattering \pPb. This is corrected for by fitting to a constant in the range $0<|\Dphi|<1$, and subtracting the result on the full range $0<|\Dphi|<\pi$. This is done at the reconstructed and truth levels in the same manner.	 The \Dphi\ distributions are then normalized by the leading jet \pt\ spectra counts, fitted to measure the widths, and integrated to measure the yields.

\subsection{ Re-weighting \Dphi\ Distributions }
The weights for \Dphi\ distributions in both \pp\ and \pPb\ MCs are derived as the ratios of Data to MC \Dphi\ distributions. This way, the \pT\ dependence of the \Dphi\ distributions can be eliminated and only residual differences in shapes of \Dphi\ distributions between Data and MC need to be accounted for. The \pp\ MC \Dphi\ weights in all combinations of \ptone\ and \pttwo\ and increasing bins in \ystartwo\ are shown in Figure~\ref{fig:ppIndividualDphiWeights} as a function of \Dphi. In such fine binning the weights have very high statistical fluctuations but they are invariant in \pT, so they can be combined and smoothed to form weights only only depending on \ystartwo, as shown in Figure~\ref{fig:ppAllDphiWeights}. The \pPb\ \Dphi\ weights are evaluated with the same method. The \pPb\ MC \Dphi\ weights in all combinations of \ptone\ and \pttwo\ in increasing bins in \ystartwo\ are shown in Figure~\ref{fig:pPbIndividualDphiWeights}, and the combined and smoothed weights are shown in Figure~\ref{fig:pPbAllDphiWeights}, all as a function of \Dphi.

\begin{figure}[ht]
	\centerline{
		\begin{tabular}{cc}
			\includegraphics[width=0.5\textwidth]{output/output_pp_mc_pythia8/cw_40_Ystar1_27_40_Ystar2_27.pdf} &
			\includegraphics[width=0.5\textwidth]{output/output_pp_mc_pythia8/cw_40_Ystar1_27_27_Ystar2_18.pdf} \\
		\end{tabular}
	}
	\caption{ \pp\ MC \Dphi\ weights shown for increasing bins of \ystartwo\ and all possible combinations of \ptone\ and \pttwo. Weights have high statistical fluctuations but are invariant in \pT. }
	\label{fig:ppIndividualDphiWeights}
\end{figure}

\begin{figure}[ht]
	\centerline{
		\begin{tabular}{cc}
			\includegraphics[width=0.5\textwidth]{output/output_pPb_mc_pythia8/cw_40_Ystar1_27_40_Ystar2_27.pdf} &
			\includegraphics[width=0.5\textwidth]{output/output_pPb_mc_pythia8/cw_40_Ystar1_27_27_Ystar2_18.pdf} \\
		\end{tabular}
	}
	\caption{ \pPb\ MC \Dphi\ weights shown for increasing bins of \ystartwo and all possible combinations of \ptone\ and \pttwo. Weights have high statistical fluctuations but are invariant in \pT. }
	\label{fig:pPbIndividualDphiWeights}
\end{figure}

\begin{figure}[ht]
	\centerline{
		\begin{tabular}{c}
			\includegraphics[width=0.75\textwidth]{output/output_pp_mc_pythia8/h_dPhi_weights_All.pdf}\\
		\end{tabular}
	}
	\caption{ \pp\ MC \Dphi\ weights for combined \pT\ bins, now shown only in bins of \ystartwo.  }
	\label{fig:ppAllDphiWeights}
\end{figure}

\begin{figure}[ht]
	\centerline{
		\begin{tabular}{c}
			\includegraphics[width=0.75\textwidth]{output/output_pPb_mc_pythia8/h_dPhi_weights_All.pdf}\\
		\end{tabular}
	}
	\caption{ \pPb\ MC \Dphi\ weights for combined \pT\ bins, now shown only in bins of \ystartwo.  }
	\label{fig:pPbAllDphiWeights}
\end{figure}

\FloatBarrier

To properly use re-weighting in the unfolding procedure, the re-weighted reconstructed MC and data distributions should have a similar shape. There is not expected to be a complete match between Data and re-weighted MC because the re-weighting is done as a function of truth kinematics. Comparisons of the re-weighted and default MC distributions to the data are shown in Figure~\ref{fig:ppweightscomp} for \pp\ and Figure~\ref{fig:pPbweightscomp} for \pPb. The ratio of the data to re-weighted MC is constant in \Dphi, indicating a consistent shape. The ratio is fitted in the same range as \Dphi\ distributions ($2.5<\Dphi<\pi$) to a constant, and in order to test fit quality, probability distributions of the fit results are shown for \pp\ and \pPb\ in Figure~\ref{fig:weightscompfitsflat}. The probability distributions are flat indicating a good fit to constant. 

\begin{figure}[ht]
	\centerline{
		\begin{tabular}{ccc}
			\includegraphics[width=0.33\textwidth]{output/output_pp_data/hMC_dPhi_40_Ystar1_27_28_Pt1_35_28_Pt2_35_40_Ystar2_27.pdf} &			\includegraphics[width=0.33\textwidth]{output/output_pp_data/hMC_dPhi_40_Ystar1_27_35_Pt1_45_28_Pt2_35_40_Ystar2_27.pdf} &
			\includegraphics[width=0.33\textwidth]{output/output_pp_data/hMC_dPhi_40_Ystar1_27_45_Pt1_90_45_Pt2_90_18_Ystar2_0.pdf} \\
		\end{tabular}
	}
	\caption{ \Dphi\ distributions for \pp\ data and MC. For MC, both re-weighted and default reconstructed distributions ares shown. The re-weighting makes the shapes flat in \Dphi\ as indicated by the constant ratio.}
	\label{fig:ppweightscomp}
\end{figure}

\begin{figure}[ht]
	\centerline{
		\begin{tabular}{ccc}
			\includegraphics[width=0.33\textwidth]{output/output_pPb_data/hMC_dPhi_40_Ystar1_27_28_Pt1_35_28_Pt2_35_40_Ystar2_27.pdf} &
			\includegraphics[width=0.33\textwidth]{output/output_pPb_data/hMC_dPhi_40_Ystar1_27_35_Pt1_45_28_Pt2_35_40_Ystar2_27.pdf} &
			\includegraphics[width=0.33\textwidth]{output/output_pPb_data/hMC_dPhi_40_Ystar1_27_45_Pt1_90_45_Pt2_90_18_Ystar2_0.pdf} \\
		\end{tabular}
	}
	\caption{ \Dphi\ distributions for \pPb\ data and MC. For MC, both re-weighted and default reconstructed distributions ares shown. The re-weighting makes the shapes flat in \Dphi\ as indicated by the constant ratio.}
	\label{fig:pPbweightscomp}
\end{figure}

\begin{figure}[ht]
	\centerline{
		\begin{tabular}{cc}
			\includegraphics[width=0.45\textwidth]{output/output_pp_data/h_probWeights_pp.pdf} &			\includegraphics[width=0.45\textwidth]{output/output_pPb_data/h_probWeights_pPb.pdf} \\
		\end{tabular}
	}
	\caption{ Probability distribution for constant fits to ratio of re-weighted reco MC to data \Dphi\ distributions. Shown for \pp\ (left) and \pPb\ (right) MCs. }
	\label{fig:weightscompfitsflat}
\end{figure}


\subsection{ Fitting of \Dphi\ Distributions } \label{sec:fitting}

The unfolded jet \pT\ spectra and $\mathrm{d}N_{1,2}(\Dphi)/\mathrm{d}\Dphi$ are further used to evaluate \conetwo\ distributions both in \pp\ and \pPb\ collisions. The \conetwo\ distributions are then fitted by an a double-exponential distribution smeared by a Gaussian function.  This fit function is obtained from a convolution of a double-exponential and a Gaussian:

\begin{eqnarray}
f(x) = \int_{-\infty}^{\infty}d\delta\frac{e^{-\delta^{2}/2\sigma^{2}}}{\sqrt{8\pi\sigma^{2}\tau^{2}}}e^{-|x-\delta|/\tau}.
\end{eqnarray}

Expanding the convolution of the Gaussian and double exponential functions, the resulting formula used in the analysis is:

\begin{eqnarray}
f(x) = A\frac{e^{\sigma^2/2\tau^2}}{2\tau}\bigg(\frac{1}{2}e^{\frac{x}{\tau}}Erfc\bigg(\frac{1}{\sqrt{2}}\bigg[\frac{x}{\sigma}+\frac{\sigma}{\tau}\bigg]\bigg)+e^{\frac{-x}{\tau}}\bigg[1-\frac{1}{2}Erfc\bigg(\frac{1}{\sqrt{2}}\bigg[\frac{x}{\sigma}-\frac{\sigma}{\tau}\bigg]\bigg)\bigg]\bigg)
\end{eqnarray} 

where $\tau$ is the inverse slope of the exponential component, $\sigma$ the width of the Gaussian distribution, and $A$ is the overall scaling factor. The widths of \conetwo\ distributions are calculated as 

\begin{eqnarray}
RMS(\conetwo) =  \sqrt{2\tau^2 + \sigma^{2}}.
\end{eqnarray}
%The fit function is not able to describe the \Dphi\ distributions in their full range, especially at large \Dphi\ away from $\pi$ due to multi-jets contributions which are not of interest to this analysis. 
where \conetwo\ is fitted in the interval $2.5<\Dphi<\pi$, similarly to the phase-space used in a previous di-jet measurement~\cite{Chatrchyan:2014hqa}.

%Di-jet azimuthal angular correlation distributions are fitted to an exponentially modified gaussion function. This fit function is obtained from a convolution of an exponential and a Gaussian, shown in Equation~\ref{eqn:conv}.

%Expanding the convolution of the Gaussian and exponential functions, the resulting formula used in the analysis is shown in Equation~\ref{eqn:fit}. 

%In the formula, the exponential component is $\tau$, the Gaussian component is $\sigma$, and $A$ is the overall multiplicative scaling factor.


%The fit function is not able to describe the \Dphi\ distributions in their full range, especially at large \Dphi\ away from $\pi$ due to multi-jets contributions which are not of interest to this analysis. The fit range is chosen from $2.5<\Dphi<\pi$, similar to the phase-space used in a previous di-jet transverse momentum balance measurement~\cite{Chatrchyan:2014hqa}. The resulting width, which is defined as $RMS =  \sqrt{2\tau^2 + \sigma^{2}}$, is then extracted and plotted in bins of \ystarone, \ystartwo, \ptone, and \pttwo.      

\FloatBarrier
\subsection{ Unfolding \Dphi\ Distributions }
When filling the truth and reconstructed distributions in either \pp\ or \pPb, the leading jet weights shown in Figure~\ref{fig:spectweights}, in addition to the \pT\ invariant \Dphi\ weights shown in Figures~\ref{fig:ppAllDphiWeights} and ~\ref{fig:pPbAllDphiWeights} for \pp\ and \pPb\ are applied as product. Using the re-weighted truth and reconstructed \Dphi\ distributions, along with the respective re-weighted response matrices, new correction factors are then derived using the bin-by-bin procedure described earlier. \Dphi\ distributions for truth, reconstructed, and unfolded \pp\ MC in two different bins of \ptone\ are shown in Figure~\ref{fig:ppUnfoldingMC}, along with the correction factors and respective response matrices. Similarly, two different \Dphi\ distributions for truth, reco, and unfolded \pPb\ MC distributions in two different bins of \ptone\ are shown in Figure~\ref{fig:pPbUnfoldingMC}, along with the correction factors and respective response matrices. 
All the \Dphi\ distributions from truth MC, unfolded reconstructed MC, and data, along with correction factors are shown in Appendix~\ref{sec:appendixB}.

\begin{figure}[ht]
	\centerline{
		\begin{tabular}{cc}
			\includegraphics[width=0.5\textwidth]{output/output_pp_mc_pythia8/h_dPhi_unfolded_All_MUT_40_Ystar1_27_28_Pt1_35_28_Pt2_35_40_Ystar2_27.pdf} &
			\includegraphics[width=0.5\textwidth]{output/output_pp_mc_pythia8/h_dPhi_unfolded_All_MUT_40_Ystar1_27_35_Pt1_45_28_Pt2_35_40_Ystar2_27.pdf} \\
			\includegraphics[width=0.5\textwidth]{output/output_pp_mc_pythia8/h_dPhi_respMat_All_40_Ystar1_27_28_Pt1_35_28_Pt2_35_40_Ystar2_27.pdf} &
			\includegraphics[width=0.5\textwidth]{output/output_pp_mc_pythia8/h_dPhi_respMat_All_40_Ystar1_27_35_Pt1_45_28_Pt2_35_40_Ystar2_27.pdf} \\
		\end{tabular}
	}
	\caption{ \pp\ MC truth, reconstructed, and unfolded \Dphi\ distributions for two different bins of \ptone, with correction factors (top row) and respective response matrices (bottom row). }
	\label{fig:ppUnfoldingMC}
\end{figure}

\begin{figure}[ht]
	\centerline{
		\begin{tabular}{ccc}
			\includegraphics[width=0.5\textwidth]{output/output_pPb_mc_pythia8/h_dPhi_unfolded_All_MUT_40_Ystar1_27_28_Pt1_35_28_Pt2_35_40_Ystar2_27.pdf} &
			\includegraphics[width=0.5\textwidth]{output/output_pPb_mc_pythia8/h_dPhi_unfolded_All_MUT_40_Ystar1_27_35_Pt1_45_28_Pt2_35_40_Ystar2_27.pdf} \\
			\includegraphics[width=0.5\textwidth]{output/output_pPb_mc_pythia8/h_dPhi_respMat_All_40_Ystar1_27_28_Pt1_35_28_Pt2_35_40_Ystar2_27.pdf} &
			\includegraphics[width=0.5\textwidth]{output/output_pPb_mc_pythia8/h_dPhi_respMat_All_40_Ystar1_27_35_Pt1_45_28_Pt2_35_40_Ystar2_27.pdf} \\
		\end{tabular}
	}
	\caption{ \pPb\ MC truth, reconstructed, and unfolded \Dphi\ distributions for two different bins of \ptone, with correction factors (top row) and respective response matrices (bottom row). }
	\label{fig:pPbUnfoldingMC}
\end{figure} 

\FloatBarrier
\subsection{MC Closure Test}
As a check, the MC reconstructed results are unfolded using the derived correction factors. The comparison of the \pp\ MC truth and unfolded widths, and the respective ratios are shown in Figure~\ref{fig:ppwidthsTruthUF} in bins of \ptone\ and \pttwo. The ratios between unfolded and truth results are consistent with unity within statistical uncertainties indicating there is good closure between the unfolded and truth results.  Similarly, comparison of the \pPb\ MC truth and unfolded widths, and the respective ratios are shown in Figure~\ref{fig:pPbwidthsTruthUF} in bins of \ptone\ and \pttwo. The ratios between unfolded and truth results are consistent with unity within statistical uncertainties indicating there is good closure between the unfolded and truth results.  

The comparison of the \pp\ MC truth and unfolded yields, and the respective ratios are shown in Figure~\ref{fig:ppyieldsTruthUF} in bins of \ptone\ and \pttwo. The ratios between unfolded and truth results are consistent with unity within statistical uncertainties indicating there is good closure between the unfolded and truth results.  Similarly, comparison of the \pPb\ MC truth and unfolded yields, and the respective ratios are shown in Figure~\ref{fig:pPbyieldsTruthUF} in bins of \ptone\ and \pttwo. The ratios between unfolded and truth results are consistent with unity within statistical uncertainties indicating there is good closure between the unfolded and truth results.  

\begin{figure}[ht]
	\centerline{
		\begin{tabular}{ccc}
			\includegraphics[width=0.33\textwidth]{output/All/pp_mc_pythia8_0/h_dPhi_width_40_Ystar1_27_28_Pt1_35.pdf} &
			\includegraphics[width=0.33\textwidth]{output/All/pp_mc_pythia8_0/h_dPhi_width_40_Ystar1_27_35_Pt1_45.pdf} &
			\includegraphics[width=0.33\textwidth]{output/All/pp_mc_pythia8_0/h_dPhi_width_40_Ystar1_27_45_Pt1_90.pdf} \\
		\end{tabular}
	}
	\caption{ Comparison of widths from \Dphi\ fits between unfolded and truth results for the \pp\ MC. Ratios are consistent with unity, indicating good unfolding closure. }
	\label{fig:ppwidthsTruthUF}
\end{figure}


\begin{figure}[ht]
	\centerline{
		\begin{tabular}{ccc}
			\includegraphics[width=0.33\textwidth]{output/All/pPb_mc_pythia8_0/h_dPhi_width_40_Ystar1_27_28_Pt1_35.pdf} &
			\includegraphics[width=0.33\textwidth]{output/All/pPb_mc_pythia8_0/h_dPhi_width_40_Ystar1_27_35_Pt1_45.pdf} &
			\includegraphics[width=0.33\textwidth]{output/All/pPb_mc_pythia8_0/h_dPhi_width_40_Ystar1_27_45_Pt1_90.pdf} \\
		\end{tabular}
	}
	\caption{ Comparison of widths from \Dphi\ fits between unfolded and truth results for the \pPb\ MC. Ratios are consistent with unity, indicating good unfolding closure. }
	\label{fig:pPbwidthsTruthUF}
\end{figure}

\begin{figure}[ht]
	\centerline{
		\begin{tabular}{ccc}
			\includegraphics[width=0.33\textwidth]{output/All/pp_mc_pythia8_0/h_dPhi_yield_40_Ystar1_27_28_Pt1_35.pdf} &
			\includegraphics[width=0.33\textwidth]{output/All/pp_mc_pythia8_0/h_dPhi_yield_40_Ystar1_27_35_Pt1_45.pdf} &
			\includegraphics[width=0.33\textwidth]{output/All/pp_mc_pythia8_0/h_dPhi_yield_40_Ystar1_27_45_Pt1_90.pdf} \\
		\end{tabular}
	}
	\caption{ Comparison of yields from \Dphi\ distributions between unfolded and truth results for the \pp\ MC. Ratios are consistent with unity, indicating good unfolding closure. }
	\label{fig:ppyieldsTruthUF}
\end{figure}

\begin{figure}[ht]
	\centerline{
		\begin{tabular}{ccc}
			\includegraphics[width=0.33\textwidth]{output/All/pPb_mc_pythia8_0/h_dPhi_yield_40_Ystar1_27_28_Pt1_35.pdf} &
			\includegraphics[width=0.33\textwidth]{output/All/pPb_mc_pythia8_0/h_dPhi_yield_40_Ystar1_27_35_Pt1_45.pdf} &
			\includegraphics[width=0.33\textwidth]{output/All/pPb_mc_pythia8_0/h_dPhi_yield_40_Ystar1_27_45_Pt1_90.pdf} \\
		\end{tabular}
	}
	\caption{ Comparison of yields from \Dphi\ distributions between unfolded and truth results for the \pPb\ MC. Ratios are consistent with unity, indicating good unfolding closure. }
	\label{fig:pPbyieldsTruthUF}
\end{figure}


As an additional closure test, the jet \pT\ spectra and \Dphi\ correction factors derived from the \pythiaeight\ MC were applied to reconstructed jets from the \herwig\ MC. A comparison of unfolded to truth \conetwo\ and \ionetwo\ fromthe \pp\ \herwig\ are shown in Figure~\ref{fig:herwigpythiaclosure}. For \pPb\ there is no additional MC so this test was only done on the \pp\ MC. Ratios of unfolded to truth distributions indicate good closure. From Tables~\ref{tab:mcsamplespp},~\ref{tab:mcsamplesppherwig} it is clear that the statistics in the \pp\ \herwig\ MC is roughly 50\% of the \pp\ \pythiaeight\ MC, and the resulting fluctuations can be taken as statistical. 


\begin{figure}[ht]
	\centerline{
		\begin{tabular}{ccc}
			\includegraphics[width=0.33\textwidth]{output/All/pp_mc_herwig_0/h_dPhi_width_40_Ystar1_27_35_Pt1_45.pdf} &
			\includegraphics[width=0.33\textwidth]{output/All/pp_mc_herwig_0/h_dPhi_yield_40_Ystar1_27_35_Pt1_45.pdf} \\
		\end{tabular}
	}
	\caption{ Comparison of \conetwo\ (left) and \ionetwo\ (right) between unfolded and truth results for the \pp\ \herwig\ MC. Unfolding is done using correction factors derived from the \pythiaeight\ MC. Ratios are consistent with unity, indicating good unfolding closure. }
	\label{fig:herwigpythiaclosure}
\end{figure}

\FloatBarrier

\section{Systematic Uncertainties}
\label{sec:systematic}
% !TEX encoding = UTF-8 Unicode
% !TEX root = thesis-ex.tex

This section gives an overview of the sources of systematic uncertainties on the \pp\ and \pbpb\ charged particle spectra associated with jet.
These include:

\begin{itemize}
\item Jet energy scale
\item Jet energy resolution
\item Tracking selections
%\item Truth track definition
%\item Detector material description in simulation
%\item Tracking in dense environments
%\item Fake track subtraction
%\item Track momentum
\item Unfolding
\item Underlying event contribution
\item MC non-closure
\end{itemize}

The systematic uncertainties are evaluated separately for \Dptr\ distributions and for their ratios as a function of jet \pT\ for \pp\ and \pbpb\ collisions.
For each systematic variation, the entire analysis procedure is repeated to ensure that the jets are treated in a consistent manner throughout the analysis.
The positive relative shift was used to calculate the upper bound of the systematic uncertainty, whereas the negative relative shift was used to calculate the lower bound.
All uncertainties except the unfolding and the MC non-closure are assumed to be correlated and are evaluated by comparing the \Rdptr\ distributions for the various systematic variations to the nominal \Rdptr\ distribution.
For uncorrelated systematic uncertainties, the uncertainty on the \RDptr\ distribution is evaluated by adding the uncertainties on the \pp\ and \pbpb\ \Dptr\ distributions in quadrature.
The total systematic uncertainties on the \Rdptr\ distributions for a selection of track \pt\ ranges (1.0--1.6 \GeV, 2.5--4.0 \GeV, 6.3--10 \GeV) in jets with \pt\ in the 126--158 \GeV\ range are shown in Figures~\ref{fig:rdptr_sys_uncert1} and \ref{fig:rdptr_sys_uncert2}. 
% Figure~\ref{fig:rdptr_sys_uncert1}-\ref{fig:rdptr_sys_uncert2}.
%The systematic uncertainties for other jet \pT\ intervals as show in appendix \ref{sec:appendixA}.



\begin{figure}
\centering
\begin{subfigure}[b]{\textwidth}
    \centering
    \includegraphics[page=1, width=\textwidth]{figures/main/systematics/Summary_ChPS_dR_sys_PbPb_error}
    \caption{}
    \label{fig:rdptr_sys_uncert1a}
\end{subfigure} \\
\begin{subfigure}[b]{\textwidth}
    \centering
    \includegraphics[page=3, width=\textwidth]{figures/main/systematics/Summary_ChPS_dR_sys_PbPb_error}
    \caption{}
    \label{fig:rdptr_sys_uncert1b}
\end{subfigure}\hfill
   \caption{A summary of the systematic uncertainties on \RDptr\ distributions for different track \mbox{$1.0 < \pt < 1.6$ GeV} (top) and \mbox{$2.5 < \pt < 4.0$ GeV} (bottom), for jets with \pt\ 126--158 \GeV, as a function of \rvar\ for different centrality bins.
Different panels are different centrality bins.
The total systematic uncertainty and its individual contributions are shown.}
\label{fig:rdptr_sys_uncert1}
\end{figure}


\begin{figure}
\centering
\begin{subfigure}[b]{\textwidth}
    \centering
    \includegraphics[page=5, width=\textwidth]{figures/main/systematics/Summary_ChPS_dR_sys_PbPb_error}
    \caption{}
    \label{fig:rdptr_sys_uncert2a}
\end{subfigure} \\
\begin{subfigure}[b]{\textwidth}
    \centering
    \includegraphics[page=6, width=\textwidth]{figures/main/systematics/Summary_ChPS_dR_sys_PbPb_error}
    \caption{}
    \label{fig:rdptr_sys_uncert2b}
\end{subfigure}\hfill
   \caption{A summary of the systematic uncertainties on \RDptr\ distributions for different track \mbox{$6.3 < \pt < 10.0$ GeV} (top) and \mbox{$10.0 < \pt < 25.1$ GeV} (bottom), for jets with \pt\ 126--158 \GeV, as a function of \rvar\ for different centrality bins.
Different panels are different centrality bins.
The total systematic uncertainty and its individual contributions are shown.}
\label{fig:rdptr_sys_uncert2}
\end{figure}

\subsection{Jet energy scale uncertainty}

The uncertainty on the JES for heavy ion jets has two parts.
The first is taken from \pp\ JES uncertainties for jets in \pp\ collisions while the second is specific to the heavy ion jets.
For the \pp\ part we use the strongly reduced set of 4 nuisance parameters using Scenario 1 as described in Ref.~\cite{JESuncertaintytwiki}.
Nuisance parameters that are not applicable for HI jet collections (pileup, b-jets, flavor and MC non closure) are removed or replaced (flavor uncertainties).
The heavy ion specific components are from the cross calibration~\cite{cc2015} and the jet flavor uncertainties at 5.02~TeV~\cite{2015392}.
For each component of the variation the response matrices are regenerated with the shifted \ptjet:

\begin{equation}
\pT^{\star,\mathrm{reco}} = \pT^{\mathrm{reco}} (1\pm U^{\mathrm{JES}}(\pT , \eta)).
\end{equation}
The data is then re-unfolded with these response matrices and the variation in the fragmentation functions is taken as the systematic uncertainty.

The centrality dependent uncertainty on the JES was evaluated by shifting the jet \pt\ of all measured jets up and down by shift between 0\% and 0.5\%.
The magnitude of the shift depends on the centrality in the way that the uncertainty on the jet \pt\ is 0.5\% in 1\% most central collisions and than linearly decreases to 0\% in 60\% peripheral bin.
The size of the shift reflects the uncertainty on the JES evaluated as using the $r-$track study where the sum of \pT\ of the tracks associated to a reconstructed jet is compared to the reconstructed jet \pT\ in ratio that is than compared between PbPb data and MC~\cite{HIjesnote,Aad:2014bxa}.



\subsection{Jet energy resolution}
To account for systematic uncertainties due to disagreement between the jet energy resolution in data and MC, the unfolding procedure was repeated with a modified response matrix.
The matrix was generated by repeating the MC study with modifications to the $\Delta \pt$ for each matched truth-reconstructed jet pair.
The procedure to generate modified migration matrices follows the standard procedure applied in \pp\ jet measurements and is used for both the \pp\ and \pbpb\ collisions.
The $\texttt{JetEnergyResolutionProvider}$ tool~\cite{JERUncertaintyProviderRun2} was used to retrieve uncertainty on the fractional resolution, $\sigma^{\mathrm{syst}}_{\mathrm{JER}}$ as a function of jet $\pt$ and $\eta$.
An additional HI jet specific uncertainty from the cross calibration of the HI jet collections~\cite{cc2015} is applied to jets in both \pp\ and \pbpb\ collisions.
The full JER uncertainty on 2015 \pp\ data is shown also in Ref.~\cite{Aad:1696485}
The jet $\pt^{\mathrm{reco}}$ was then smeared by

\begin{align}
\pt^{\star, \mathrm{reco}} = \pt^{\mathrm{reco}}\times \mathcal{N}(1,\sigma^{\mathrm{eff}}_{\mathrm{JER}})\,,
\end{align}
where $\mathcal{N}(1,\sigma^{\mathrm{eff}}_{\mathrm{JER}})$ is the normal distribution with the effective resolution $\sigma^{\mathrm{eff}}_{\mathrm{JER}}=\sqrt{(\sigma_{\mathrm{JER}} + \sigma^{\mathrm{syst}}_{\mathrm{JER}})^{2} - \sigma_{\mathrm{JER}}^{2}}$.

%%%%%%%%%%%%%%%%%
%The systematic uncertainties on the \Dptr\ distributions decreases with decreasing \pt\ and increasing jet \pT.
%The typical systematic uncertainty originating from JER changes varies from 10\% to 1\% depending on the jet \ET\ and $z$.
%%%%%%%%%%%%%%%%%


\subsection{Tracking selections}
\paragraph{Track selection}
This uncertainty was estimated by tightening the tracking cuts by adding the cuts on the significance of $d_0$ and $z_0$ as described in the Section~\ref{sec:trackselection}. 
The entire analysis is redone with these track selections (including re-deriving the tracking efficiencies and the $\eta-\phi$ maps for the UE estimation) and the difference from the nominal analysis is taken as the systematic uncertainty.

\paragraph{Truth track definition}  
This uncertainty quantifies the robustness of the matching of reconstructed to truth particles.
The uncertainty is taken as a difference in the final results obtained with  $\mcprob > 0.3$ and results obtained with $ \mcprob > 0.5$.
This systematic included a re-derivation of the $\eta-\phi$ maps for UE estimation.
%The change in tracking efficiency for these two selections is negligible.

\paragraph{Detector material description in simulation}
The uncertainty on the inner detector material varies with \pttrk\ and \etatrk\ from 0.5\% to 2.0\%~\cite{ref:tracktwiki} on the efficiency correction.
This systematic also included a re-derivation of the $\eta-\phi$ maps for UE estimation.

\paragraph{Tracking in dense environments}
There is a 0.4\% uncertainty on the efficiency due to tracking in dense environments (the core of the jet)~\cite{ref:tracktwiki}.
This systematic also included a re-derivation of the $\eta-\phi$ maps for UE estimation.

\paragraph{Fake rate and secondaries}
The uncertainty on the rate of fake tracks and secondaries is taken to be 30\% independent of \pttrk\ and \etatrk~\cite{ref:tracktwiki, Nachman:2259091}.
This uncertainty is conservatively symmetrized.

\paragraph{Uncertainty on the track momentum}
To account for a possible misalignment in \pp\ and \PbPb\ data, the reconstructed \pT\ of each track (corrected first as described in section~\ref{Sec:Trackmomentumcorrection}) was changed according to~\cite{TrackingRec}:

\begin{equation}
\pt \rightarrow \pt \times (1 + q \times \pt \delta_{sagitta}(\eta, \phi))^{-1},
\end{equation}
where $q$ is charge of the track and $\delta_{sagitta}(\eta, \phi)$ is uncertainty on the track curvature.
The uncertainty derived for 5.02~TeV \pp\ and \PbPb\ data is included in InDetTrackSystematicsTools-00-00-19.
Due to statistical origin of the uncertainty the resulting systematic uncertainty is symmetrized.
This systematic also included a re-derivation of the $\eta-\phi$ maps for UE estimation.

%%%%%%%%%%%%%%%%%
%The resulting systematic uncertainty is $<<1$\% for low and intermediate $z$ and \pT\ and reaches up to 4\% at high $z$.
%As the source of the shift is present both in \pp\ and \PbPb\ it does partially cancel in the ratios. 
%%%%%%%%%%%%%%%%%


\subsection{Systematic uncertainty due to unfolding}
The systematic uncertainty associated with the unfolding is connected with the sensitivity of the unfolding procedure to the choice of the input distributions.
The systematic is evaluated by generating response matrices from the MC distributions without the reweighting factor that is used to match the jet spectrum and \Dptr\ distributions in data, and then unfolding the data using these response matrices.
This has minimal effect on track \pt\ because of the good track momentum resolution in the kinematic region of interest.
The uncertainty is evaluated by comparing the nominal result with the un-reweighed result, and is considered to be uncorrelated between \pbpb\ and \pp.


\subsection{Systematic uncertainty due to the UE event subtraction}
The systematic uncertainty associated with the estimation of the UE has two main components: one is the statistical uncertainty on the $\eta-\phi$ maps used in the map method (described in section~\ref{sec:map_method}), and the other is the comparison of the map method to the alternative cone method (discussed in section~\ref{sec:cone_method}.
More details on the cone method can be found in Ref.~\cite{PhysRevC.98.024908}.
The contributions of both components to the underlying event uncertainty can be seen in Figure~\ref{fig:UE_sys_contrib}, with the uncertainty from the map statistic dominating in central collisions.
The uncertainty on the underlying event convolutes with the signal to background ratio to produce the uncertainty on the charged particle spectra.

\begin{figure}
\centering
\includegraphics[page=1,width=1.\textwidth]{figures/main/systematics/Summary_UE_RDpT_dR_sys_error}
\caption{Size of the individual contributions to the underlying event systematic uncertainty as a function for \rvar\ for 0-10\% \pbpb\ collisions, in 126-158 GeV jets, 1-1.6 GeV tracks.}
\label{fig:UE_sys_contrib}
\end{figure}


\paragraph{Uncertainty from map statistic:} 
The $\eta-\phi$ maps used in the estimation of the underlying event are sparsely populated for high track \pt\ and high \ptjet, and are susceptible to statistical fluctuations.
To take this into account, 100 pseudo-experiments are conducted to re-estimate the set of maps, with a bin-by-bin gaussian variation where the mean and standard deviation were taken to be the bin content and bin error from the nominal set of maps.
The distribution of the relative difference between each estimation of the shifted underlying event and and the nominal value is fit to a gaussian.
The width of this gaussian is taken to be the systematic uncertainty.
This uncertainty is symmetrized to be conservative.
A few examples of the distribution of normalized relative differences can be seen in Figure~\ref{fig:gaus_diff}.
The size of the systematic from this can be seen in Fig.\ref{fig:mapstat_corr}.


\begin{figure}
\begin{subfigure}{0.5\textwidth}
\centering
\includegraphics[width=1\textwidth]{figures/main/systematics/map_stat_gaus}
\caption{}
\label{fig:gaus_diff}
\end{subfigure}
\begin{subfigure}{0.5\textwidth}
\centering
\includegraphics[width=1\textwidth]{figures/main/systematics/map_stat_size}
\caption{}
\label{fig:mapstat_corr}
\end{subfigure}
\caption{(Left) An example of the relative difference between the nominal and shifted values of the UE, fit to a gaussian. The width is taken as the systematic uncertainty.
Wider distributions larger statistical uncertainty on the bin content in the $\eta-\phi$ map used to estimate the UE.
(Right) Size of the systematic uncertainty from the map statistic component, as a function for \pttrk\ and \ptjet\ for 0-10\% \pbpb\ collisions, $0.15 < r < 0.20$ away from the jet axis.}
\end{figure}

%\begin{figure}[h]
%\centering
%\includegraphics[width=0.75\textwidth]{figures/main/systematics/map_stat_size}
%\caption{Size of the systematic uncertainty from the map statistic component, as a function for \pttrk\ and \ptjet\ for 0-10\% \pbpb\ collisions, $0.15 < r < 0.20$ away from the jet axis.}
%\label{fig:mapstat_corr}
% \end{figure}


\paragraph{Uncertainty from cone method: } The difference between the UE from the two methods is discussed in section \ref{sec:cone_method} and is shown in Figure~\ref{fig:conemethod_mapmethod}.
The effect of the different UE estimation methods on the charged particle spectra is seen in Fig.\ref{fig:conemethod_chps_comparison}.
This uncertainty is conservatively symmetrized.
While the absolute size of the uncertainty on the UE is typically small, the small signal-to-background ratio makes this the dominant systematic uncertainty in central collisions for lowest \pT\ tracks and large \rvar.

\begin{figure}
\centerline{\includegraphics[page=2,width=1.\textwidth]{figures/main/systematics/ChPS_UE_Comparison}}
\caption{Ratio of the charged particle spectra as determined using two different UE estimation methods as a function for \rvar\ for 0-10\% \pbpb\ collisions in 126-158 GeV jets and 1-1.6 GeV tracks.
Deviations from unity are a combination of the difference between the two methods and the signal to background ratio.
The largest differences between the spectra are seen at large \rvar, where the signal to background is the smallest.
Points are offset along the x-axis for ease of viewing.}
\label{fig:conemethod_chps_comparison}
\end{figure}




\subsection{MC non-closure}
To make sure that all the sources of systematic uncertainties were covered, the systematic uncertainty from the non closure in the MC was also evaluated.
It was calculated using the technical closure (done using non-reweighed response matrices) between the fully corrected and reconstructed charged particle distributions in MC to the charged particle distributions evaluated at the truth level.
This uncertainty can be considered a measure of unknowns in the analysis, but it also includes fluctuations due to the finite statistics in the MC which are used to evaluate it (especially in high \pttrk\ regions of the analysis.
The non-closure can be seen in Figure~\ref{fig:pbpbclosure}.
The systematic uncertainty is taken to be uncorrelated between \pbpb\ and \pp 

\begin{figure}
\centerline{\includegraphics[page=1,width=1.\textwidth]{figures/main/systematics/ChPS_final_dR_PbPb_MC.pdf}}
\caption{Size of the non-closure as a function for \rvar\ for 0-10\% \pbpb\ collisions, in 126-158 GeV jets for different \pttrk\ ranges.
Points in the bottom panel are offset along the x-axis for ease of viewing.}
\label{fig:pbpbclosure}
\end{figure}



\subsection{Correlations between the systematic uncertainties in \pbpb\ and \pp\ collisions}
Due to the common analysis and reconstruction procedure, and detector conditions, the systematic uncertainties are correlated between the \pp\ and \pbpb\ collisions in most cases.
Table~\ref{tab:systematics} summarizes correlations between \pp\ and \PbPb\ and also point-to-point correlations of individual distributions.
The unfolding uncertainty is uncorrelated between the two systems because it comes from the sensitivity of the unfolding to the starting MC distribution.
In \pbpb\ collisions where the fragmentation is modified by the presence of the QGP, this sensitivity could be different than in \pp\ collisions where the fragmentation functions are quite similar to those in \pythiaeight~\cite{201865}.
The impact of the modification of the fragmentation process in \PbPb\ compared to \pp\ and MC simulations is account for in the HI specific data-driven and centrality dependent uncertainty on the JES.

\begin{table}[h]
\centering
\begin{tabular}{ | m{3cm} | m{3cm} | m{3cm} | m{3cm} |}
\hline
\textbf{Uncertainty} & \textbf{\pp\ and \PbPb\ correlated} & \textbf{Point-to-point correlated} & \textbf{One/two sided or symmetrized} \\ \hline
JES (\pp) & yes & yes & two sided \\ \hline
JES (HI) & no & yes & two sided \\ \hline
JER & yes & yes & symmetrized \\ \hline
Track selection & yes & yes & one sided \\ \hline
\mcprob & yes & yes & one sided \\ \hline
Material & yes & yes & one sided \\ \hline
Dense environment & yes & yes & one sided \\ \hline
Fake rate & yes & yes & symmetrized \\ \hline
Track momentum & yes & no & two sided \\ \hline
Unfolding & no & yes & symmetrized \\ \hline
UE subtraction & no & yes & symmetrized \\ \hline
MC non-closure & no & no & symmetrized \\ \hline
\end{tabular}
\caption{Summary of correlation of different systematic uncertainties.}
\label{tab:systematics}
\end{table}

In the case where the systematic uncertainties are correlated, we evaluate \Rdptr\ ratios using the systematic variation from the nominal distributions in both \pp\ and \pbpb.
The variation in the ratio is used as the systematic uncertainty.
The variations in the ratios are summed in quadrature to get the total systematic uncertainty on the ratio.




\section{Results}
\label{sec:results}
% !TEX root = thesis-ex.tex

The \Dptr\ distributions are studied as a function of \ptjet\ for \pp\ data and \PbPb\ collisions with different centralities.
The interplay between the hot and dense matter and the parton shower is explored by evaluating the ratios and differences between \Dptr\ distributions in \pbpb\ and \pp\ collisions, as well as some integrated quantities.



%%%%%%%    DPtr distributions    %%%%%%%
\subsection{\Dptr\ distributions}
\label{sec:dptr}
The \Dptr\ distributions evaluated in \pp\ and \pbpb\ collisions for $126 < \ptjet < 158$ GeV are shown in Figure~\ref{fig:dptr}.
The distributions exhibit a difference in shape between \PbPb\ and \pp\ collisions, with the \pbpb\ distributions being broader at low \pt\ (\pt < 4 GeV) and narrower at high \pt\ (\pt > 4 GeV) in \mbox{0--10\%} central collisions.
This modification is centrality dependent and is smaller for peripheral \pbpb\ collisions.

\begin{figure}[h]
\centerline{
\begin{tabular}{ccc}
\includegraphics[width=0.36\textwidth]{figures/main/results/DpT_dR_jet7_cent0} &
\includegraphics[width=0.36\textwidth]{figures/main/results/DpT_dR_jet7_cent1} &
\includegraphics[width=0.36\textwidth]{figures/main/results/DpT_dR_jet7_cent2} \\
\includegraphics[width=0.36\textwidth]{figures/main/results/DpT_dR_jet7_cent3} &
\includegraphics[width=0.36\textwidth]{figures/main/results/DpT_dR_jet7_cent4} &
\includegraphics[width=0.36\textwidth]{figures/main/results/DpT_dR_jet7_cent5} \\
\end{tabular}
}
\caption{The \Dptr\ distributions in \pp\ (open symbols) and \pbpb\ (closed symbols) as a function of angular distance $r$ for \ptjet\ of 126 to 158~\GeV.
The colors represent different track \pt\ ranges, and each panel is a different centrality selection.
The vertical bars on the data points indicate statistical uncertainties while the shaded boxes indicate systematic uncertainties.
The widths of the boxes are not indicative of the bin size and the points are shifted horizontally for better visibility.
The distributions for $\pt > 6.3$ GeV are restricted to smaller \rvar\ values as discussed in Section~\ref{sec:analysis}.}
\label{fig:dptr}
\end{figure}



%%%%%%%    RDptr distributions    %%%%%%%
\subsection{\RDptr\ distributions}
\label{sec:rdptr}
In order to quantify the differences seen in Figure~\ref{fig:dptr}, ratios of the \Dptr\ distributions in \pbpb\ collisions to those measured in \pp\ collisions for $126 < \ptjet < 158$ GeV and $200 < \ptjet < 251$ GeV jets are presented in Figure~\ref{fig:rdptr}.
They are shown as a function of $r$ for different \pt\ and centrality selections.
In 0--10\% central collisions, \RDptr\ is greater than unity for $\rvar < 0.8$ for charged particles with \pT less than 4.0~\GeV\ in both jet selections.
For these particles, the enhancement of yields in \pbpb\ collisions compared to those in  \pp\ collisions grows with increasing \rvar\ up to approximately \mbox{$\rvar  = 0.3$}, with \RDptr\ reaching up to two for 1.0~$< \pt <$~2.5~\GeV.
The value of \RDptr\ is approximately constant for \rvar\ in the interval \mbox{0.3--0.6} and decreases for \mbox{$\rvar > 0.6$}.
For charged particles with $\pt > 4.0$ \GeV, \RDptr\ shows a depletion outside the jet core for $r > 0.05$.
The magnitude of this depletion increases with increasing \rvar\ up to $r = 0.3$ and is approximately constant thereafter.
For 30--40\% mid-central collisions, the enhancement of particles with $\pt < 4.0$~\GeV\ is similar to that in the most central collisions, however the depletion of particles with $\pt > 4.0$~\GeV\ is not as strong.
For 60--80\% peripheral collisions, \RDptr\ has no significant \rvar\ dependence and the values of \RDptr\ are within approximately 50\% of unity.

The observed behavior inside the jet cone, $r < 0.4$, agrees with the measurement of the inclusive jet fragmentation functions~\cite{Aaboud:2017eww, Aaboud:2017bzv, PhysRevC.98.024908}, where yields of fragments with $\pt < 4$ GeV are observed to be enhanced and yields of charged particles with intermediate \pT\ are suppressed in \PbPb\ collisions compared to those in \pp\ collisions.
%The variation of \RDptr\ with centrality, \ptjet, and charged-particle \pt\ is further discussed.
Calculations done in Ref.~\cite{Tachibana:2017syd} show that the medium response to the jet compensates the energy that is lost by the jet in \pbpb\ collisions even up to $r = 1.0$ from the jet axis.
The plateauing and slight decrease seen in Figure~\ref{fig:rdptr} for the \RDptr\ distributions in central \pbpb\ collisions beyond $r = 0.6$ from the jet axis suggests that the medium response to the jet is smaller than predicted for $r > 0.6$.


\begin{figure}[h]
\centerline{
\begin{tabular}{ccc}
\includegraphics[width=0.36\textwidth]{figures/main/results/RDpT_dR_jet7_cent0} &
\includegraphics[width=0.36\textwidth]{figures/main/results/RDpT_dR_jet7_cent3} &
\includegraphics[width=0.36\textwidth]{figures/main/results/RDpT_dR_jet7_cent5} \\
\includegraphics[width=0.36\textwidth]{figures/main/results/RDpT_dR_jet9_cent0} &
\includegraphics[width=0.36\textwidth]{figures/main/results/RDpT_dR_jet9_cent3} &
\includegraphics[width=0.36\textwidth]{figures/main/results/RDpT_dR_jet9_cent5} \\
\end{tabular}
}
\caption{Ratios of \Dptr\ distributions in \PbPb\ and \pp\ collisions as a function of angular distance $r$ for \ptjet\ of 126 to 158~\GeV\ (top) and of 200 to 251~\GeV\ (bottom) for seven \pt\ selections.
Different centrality selections are shown: 0--10\% (left), 30--40\% (middle), 60--80\% (right).
The vertical bars on the data points indicate statistical uncertainties while the shaded boxes indicate systematic uncertainties.
The widths of the boxes are not indicative of the bin size and the points are shifted horizontally for better visibility.}
\label{fig:rdptr}
\end{figure}

% This observation is in agreement with the previous measurement of jet fragmentation functions \cite{Chatrchyan:2014ava, Sirunyan:2018jqr, Aaboud:2017bzv, } and may indicate the dependence of the response of the hot dense matter to the momentum of a jet passing through it.
%\FloatBarrier

% !TEX root = trkjet.tex


%%%%%%%    Centrality-RDptr distributions    %%%%%%%
The centrality dependence of \RDptr\ for two charged-particle \pt\ intervals: 1.6--2.5~\GeV\ and \mbox{6.3--10.0~\GeV}, and two different \ptjet\ ranges: 126--158~\GeV\ and 200--251~\GeV, is presented in Figure~\ref{fig:centdep}.
For both \ptjet\ selections and  1.6--2.5~\GeV\ charged particles, the magnitude of the excess increases for more central events and for \rvar\ for $\rvar < 0.3$.
The magnitude of the excess is approximately a factor of two in the most central collisions for $\rvar >$~0.3.
A continuous centrality dependent suppression of  yields of charged particles with $6.3 < \pt < 10.0$ GeV is observed.
The magnitude of the modification decreases for more peripheral collisions in both \pt\ intervals and \ptjet\ selections.

\begin{figure}[ht]
\centerline{
\begin{tabular}{cc}
\includegraphics[width=0.36\textwidth]{figures/main/results/RDpT_dR_trk3_trk6_jet7} & 
\includegraphics[width=0.36\textwidth]{figures/main/results/RDpT_dR_trk3_trk6_jet9} \\
\end{tabular}}
\caption{The \RDptr\ distributions for \ptjet\ of 126--158~\GeV\ (left) and 200--251~\GeV\ (right) as a function of angular distance $r$ for two \pt\ selections, 1.6--2.5~\GeV\ (closed symbols) and 6.3--10.0~\GeV\ (open symbols), and six centrality intervals.
The vertical bars on the data points indicate statistical uncertainties while the shaded boxes indicate systematic uncertainties.
The widths of the boxes are not indicative of the bin size and the points are shifted horizontally for better visibility.}
\label{fig:centdep}
\end{figure}

%%%%%%%    trkpt-RDptr distributions    %%%%%%%
%In Figure~\ref{fig:rdptr}, it was shown that for central and mid-central collisions, there is an enhancement of charged particles with $\pt <$~4.0~\GeV\ and a suppression of charged particles with $\pt >$~4.0~\GeV.
Figure~\ref{fig:pttrkdep} shows the \pt\ dependence for selections in \rvar\  for 126--158 GeV and 200--251~\GeV\ jets in the following centrality intervals: 0--10\%, 30--40\% and 60--80\%.
Interestingly, there is no significant suppression of the yields in \pbpb\ collisions for $\rvar < 0.05$ at all measured \pt.
For larger \rvar\ values the yields are enhanced for charged particles with $\pt <$~4~\GeV\ and suppressed for higher \pt\ charged particles in both the 0--10\% and 30--40\% centrality selections and both \ptjet\  ranges presented here.
The magnitude of the enhancement increases for decreasing \pt\ below 4 GeV while the suppression is enhanced with increasing \pt\ for 4--10 GeV, after which it is approximately constant.
At fixed \pt\ the magnitude of the deviation from unity is largest for $0.3< \rvar < 0.4$ and $0.5< \rvar < 0.6$.
In the 60--80\% peripheral collisions, the same trend remains true (but with smaller magnitude modifications) for \mbox{$126 < \ptjet < 158$ GeV}; for the higher \ptjet\ selection the larger uncertainties do not allow a clear conclusion to be drawn for peripheral collisions.

The enhancement of charged particles in the kinematic region of \mbox{$\pt < 4$ GeV} has two common explanations.
First, gluon radiation from the hard scattered parton as it propagates through the QGP would lead to extra soft particles \cite{Chien:2015vja, Kang:2017frl}.
Second, the interactions of a jet with the QGP and its hydrodynamic response could induce a wake that manifests itself as an enhancement of low \pt\ particles \cite{Tachibana:2017syd}.

The observed modification at \mbox{$\pt > 4$ GeV} can be explained on the basis of the larger expected energy loss of gluon-initiated jets, resulting in a relative enhancement of quark jets in \pbpb\ collisions compared to \pp\ collisions at a given \ptjet\ value~\cite{PhysRevC.98.024908, Spousta:2015fca}.
Since gluon jets have a broader distribution of particle transverse momentum with respect to the jet direction compared to quark-initiated jets \cite{OPAL:1995ab}, such an effect could describe the narrowing of the particle distribution around the jet direction for particles with $\pt >$~4.0~\GeV\ that is observed here, though no calculations of this are available.


\begin{figure}[h]
\centerline{
\begin{tabular}{ccc}
\includegraphics[width=0.36\textwidth]{figures/main/results/RDpT_trkpt_jet7_cent0} &
\includegraphics[width=0.36\textwidth]{figures/main/results/RDpT_trkpt_jet7_cent3} &
\includegraphics[width=0.36\textwidth]{figures/main/results/RDpT_trkpt_jet7_cent5} \\
\includegraphics[width=0.36\textwidth]{figures/main/results/RDpT_trkpt_jet9_cent0} &
\includegraphics[width=0.36\textwidth]{figures/main/results/RDpT_trkpt_jet9_cent3} &
\includegraphics[width=0.36\textwidth]{figures/main/results/RDpT_trkpt_jet9_cent5} \\
\end{tabular}}
\caption{\RDptr\ as a function of \pt\ for  0--10\% (left), 30--40\% (middle), and 60--80\% (right) \PbPb\ collisions in two different \ptjet\ selections: 126--158~\GeV\ (top) and 200--251~\GeV\ (bottom).
The different colors indicate different angular distances from the jet axis.
The vertical bars on the data points indicate statistical uncertainties while the shaded boxes indicate systematic uncertainties.
The widths of the boxes are not indicative of the bin size and the points are shifted horizontally for better visibility.}
\label{fig:pttrkdep}
\end{figure}


%%%%%%%    Jet pT-RDptr distributions    %%%%%%%
The \RDptr\ distributions for low and high \pt\ particles in the different \ptjet\ selections are directly overlaid in Figure~\ref{fig:ptjetdep}.
These distributions are for the 0--10\% most central collisions, and show a hint of enhancement in \RDptr\ with increasing \ptjet\  for $r < 0.25$ for low  \pt\ charged particles.
No significant \ptjet\ dependence is seen at larger \rvar\ values, or for high-\pt\ charged particles at any \rvar.
This \ptjet\ dependence is further explored by defining an integral over the low \pt\ excess and is discussed in Section~\ref{sec:discussion_int}.

\begin{figure}[ht]
\centerline{
\includegraphics[width=0.36\textwidth]{figures/main/results/RDpT_dR_trk3_trk6_cent0}}
\caption{\RDptr\ as a function of \rvar\ for 0--10\% collisions for charged particles with 1.6~$< \pt <$~2.5~\GeV\ (closed symbols) and 6.3~$< \pt <$10.0~\GeV\ (open symbols) for different \ptjet\ selections.
The vertical bars on the data points indicate statistical uncertainties while the shaded boxes indicate systematic uncertainties.
The widths of the boxes are not indicative of the bin size and the points are shifted horizontally for better visibility.}
\label{fig:ptjetdep}
\end{figure}



%%%%%%%    Delta DPtr distributions    %%%%%%%
\subsection{\DeltaDptr\ distributions}
\label{sec:delta_dptr}
In addition to the ratios of the \Dptr\ distributions, differences between the unfolded charged-particle yields are also evaluated as \DeltaDptr\ to quantify the modification in terms of the particle density.

These differences are presented as a function of $r$ for different \pt\ selections in 0--10\% central collisions in Figure~\ref{fig:deltadptr}.
These distributions show an excess in the charged-particle yield density for \pbpb\ collisions compared to \pp\ collisions for charged particles with $\pt <4.0$ GeV.
This ranges from 0.5 to 4 particles per unit area per GeV for 1 \GeV\ charged particles in 126--158~\GeV\ jets for 0--10\% central \pbpb\ collisions and increases with increasing \ptjet.
The largest excess for charged particles with $\pt <$~4.0~\GeV\ is within the jet cone.
For large \rvar\ values, the difference decreases, but remains positive.
A depletion for higher \pt\ particles of approximately 0.5 particles per unit area per GeV is seen for 126--158~\GeV\ jets in 0--10\% central \pbpb\ collisions.
The magnitude of this depletion increases for higher \ptjet.
There is a minimum in the \DeltaDptr\ distributions of charged particles with \mbox{$ 4.0 < \pt <  25.1$}~\GeV\ at $0.05 < \rvar < 0.10$ that is seen at many \ptjet\ ranges under investigation.
The magnitudes of the excesses and deficits discussed here are dependent on the selected charged-particle \pt.

\begin{figure}
\centerline{
\begin{tabular}{cc}
\includegraphics[width=0.36\textwidth]{figures/main/results/DeltaDpT_dR_jet7_cent0} &
\includegraphics[width=0.36\textwidth]{figures/main/results/DeltaDpT_dR_jet8_cent0} \\
\includegraphics[width=0.36\textwidth]{figures/main/results/DeltaDpT_dR_jet9_cent0} &
\includegraphics[width=0.36\textwidth]{figures/main/results/DeltaDpT_dR_jet10_cent0} \\
\end{tabular} }
\caption{\DeltaDptr\ as a function of \rvar\ in central collisions for all \pt\ ranges in four \ptjet\ selections: 126--158~\GeV, 158--200~\GeV, 200--251~\GeV, and 251--316~\GeV.
The vertical bars on the data points indicate statistical uncertainties while the shaded boxes indicate systematic uncertainties.
The widths of the boxes are not indicative of the bin size and the points are shifted horizontally for better visibility.}
\label{fig:deltadptr}
\end{figure}





%%%%%%%%%%%%%
\subsection{\pt\ integrated distributions}
\label{sec:discussion_int}
Motivated by similar studies of the enhancement of soft fragments in jet fragmentation functions in \pbpb\ compared to \pp\ collisions from Ref.~\cite{PhysRevC.98.024908}, the unfolded \Dptr\ distributions are integrated for charged particles with \pt\ < 4 GeV to construct the quantities $\Theta(\rvar)$ and $P(\rvar)$ defined as:

\begin{align*}
\Theta(\rvar) &= \int_{1 \text{ GeV}}^{4 \text{ GeV}} \Dptr  \fd \pt \\
P(\rvar) &= \int_0^r \int_{1 \text{ GeV}}^{4 \text{ GeV}} D(\pt, r') \fd \pt \fd r'
\end{align*}
The $\Theta(\rvar)$ values are integrated over the charged-particle \pt\ interval of 1--4~\GeV\ to provide a summary look at the \pt\ region of enhancement discussed above.
The $P(\rvar)$ values further add a running integral over \rvar\ and provide information about the jet shape.
Both of these quantities are compared between the \pp\ and \pbpb\ systems to give the following distributions:

\begin{align*}
\Delta_{\Theta(\rvar)} &= \Theta(\rvar)_{\mathrm{Pb+Pb}} - \Theta(\rvar)_{pp} \\
R_{\Theta(\rvar)} &= \frac{\Theta(\rvar)_{\mathrm{Pb+Pb}}}{\Theta(\rvar)_{\mathrm{pp}}} \\
R_{P(\rvar)} &= \frac{P(\rvar)_{\mathrm{Pb+Pb}}}{P(\rvar)_{pp}}
\end{align*}
These integrated quantities are intended to provide some summary information about the location with respect to the jet axis, magnitude, and \ptjet\ dependence of the low-\pt\ charged-particle excess discussed above.
The ratio quantities are useful for comparisons to other \pbpb\ measurements; $\Delta_{\Theta(\rvar)}$ is very similar to $\DeltaDptr$, however it is integrated over charged-particle \pt\ in the 1--4~\GeV\ interval \cite{PhysRevC.98.024908}.

Figure~\ref{fig:deltaPdeltaT} shows the \DeltaTheta\ distributions as a function of \rvar\ in centrality intervals: 0--10\%, 30--40\%, 60--80\%.
In the most central collisions, a significant \ptjet\ dependence to \DeltaTheta\ is observed; for $\rvar <$~0.4 (particles within the jet cone) \DeltaTheta\ increases with increasing \ptjet.
The value of \DeltaTheta\ decreases in more peripheral collisions and the \ptjet\ dependence is no longer significant.

%%%
%Now, the \ptjet\ dependence to the excess in charged-particle density can be seen clearly; 
%in the most central collisions
%there is an increase in \DeltaTheta\ with increasing \ptjet, but in the mid-central and peripheral collisions this is no longer
%observed within the uncertainties.
%%%%

\begin{figure}
\centerline{
\begin{tabular}{ccc}
\includegraphics[width=0.36\textwidth]{figures/main/results/DeltaDpT_lowpt_integ_cent0} &
\includegraphics[width=0.36\textwidth]{figures/main/results/DeltaDpT_lowpt_integ_cent3} &
\includegraphics[width=0.36\textwidth]{figures/main/results/DeltaDpT_lowpt_integ_cent5} \\
\end{tabular} }
\caption{\DeltaTheta\ as a function of \rvar\ for charged particles with \pt\ < 4 GeV  in four \ptjet\ selections: 126--158~\GeV, 158--200~\GeV, 200--251~\GeV, and 251--316~\GeV and three centrality selections: 0--10\% (left), 30--40\% (middle) and 60--80\% (right).
The vertical bars on the data points indicate statistical uncertainties while the shaded boxes indicate systematic uncertainties.
The widths of the boxes are not indicative of the bin size and the points are shifted horizontally for better visibility.}
\label{fig:deltaPdeltaT}
\end{figure}


\begin{figure}
\centerline{
\begin{tabular}{ccc}
\includegraphics[width=0.36\textwidth]{figures/main/results/RDpT_lowpt_integ_cent0} &
\includegraphics[width=0.36\textwidth]{figures/main/results/RDpT_lowpt_integ_cent3} &
\includegraphics[width=0.36\textwidth]{figures/main/results/RDpT_lowpt_integ_cent5} \\
\includegraphics[width=0.36\textwidth]{figures/main/results/RDpT_jetshape_cent0} &
\includegraphics[width=0.36\textwidth]{figures/main/results/RDpT_jetshape_cent3} &
\includegraphics[width=0.36\textwidth]{figures/main/results/RDpT_jetshape_cent5} \\
\end{tabular} }
\caption{\RTheta\ (top) and \RP\ (bottom) as a function of \rvar\ for charged particles with $\pt < 4$ GeV ranges in four \ptjet\ selections: 126--158~\GeV, 158--200~\GeV, 200--251~\GeV, and 251--316~\GeV\ and three centrality selections: 0--10\% (left), 30--40\% (middle) and 60--80\% (rights).
The vertical bars on the data points indicate statistical uncertainties while the shaded boxes indicate systematic uncertainties.
The widths of the boxes are not indicative of the bin size and the points are shifted horizontally for better visibility.}
\label{fig:RPRT}
\end{figure}


Figure~\ref{fig:RPRT} shows \RTheta\ and \RP\ for the following centrality intervals: 0--10\%, 30--40\% and 60--80\%.
The \RTheta\ distributions of the most central collisions show a maximum for $\rvar \sim 0.4$ and a flattening or a decrease for larger \rvar.
However, since \RTheta\ remains at or above unity for the full range of \rvar\ values presented, \RP\ shows no suppression with increasing \rvar\ over the entire measured range.
In more peripheral collisions the magnitude of the excess is reduced and the trends in \RTheta\ are less clear, however the slow increase of \RP\ is clearly seen for the 30--40\% central collisions.
The flattening of the \RP\ distributions at large distances demonstrates what while wider jets have a softer fragmentation and contain more particles with less \pt\ in \pbpb\ compared to \pp\ collisions \cite{Chesler:2015nqz, Hulcher:2017cpt}, this effect flattens out for jets with radius larger than 0.6.


%%%%
%These measurements show that the excess of particles with $\pt <$~4.0~\GeV\ observed in~\cite{PhysRevC.98.024908} extends
%outside the \RFour\ jet cone.
%The measured dependence of \RDptr\ suggests that the energy lost by jets through the jet quenching process is being transferred to particles with $\pt <$~4.0~\GeV\ at larger radial distances from the jet axis.
%This is qualitatively consistent with theoretical calculations \mbox{\cite{Blaizot:2014ula}}.
%Additionally, these observations are in agreement with the previous measurement of jet fragmentation functions \cite{Chatrchyan:2014ava, Sirunyan:2018jqr, Aaboud:2017bzv, PhysRevC.98.024908} and may indicate the dependence of the response of the hot dense matter to the momentum of a jet passing through it.
%%%%%

\FloatBarrier



\section{Discussion}
\label{sec:discussion}
% !TEX encoding = UTF-8 Unicode
% !TEX root = thesis-ex.tex

This section further discusses results from the previous section.

%%cent dependence
\subsection{\RDptr\ distributions}
Here the centrality, \ptjet\ and the charged-particle \pt\ depdendence of the \RDptr\ distributions introduced in 
Section~\ref{sec:results} are discussed.

\begin{figure}[ht]
\centerline{
         \begin{tabular}{cc}
%            \includegraphics[width=0.5\textwidth]{figures/main/results/RDpT_dR_trk2_trk6_jet7.pdf} & 
%            \includegraphics[width=0.5\textwidth]{figures/main/results/RDpT_dR_trk2_trk6_jet9.pdf} \\
            \includegraphics[width=0.5\textwidth]{figures/main/results/RDpT_dR_trk3_trk6_jet7.pdf} & 
            \includegraphics[width=0.5\textwidth]{figures/main/results/RDpT_dR_trk3_trk6_jet9.pdf} \\
      \end{tabular}
      }
   \caption{The \RDptr\ distributions for \ptjet\ of 126--158~\GeV\ and 200--251~\GeV\ as a function of angular distance $r$ for two \pt\ selections, 1.6--2.5~\GeV\ (closed symbols) and 6.3--10.0~\GeV\ (open symbols), and six centrality intervals. The vertical bars on the data points indicate statistical uncertainties while the shaded boxes indicate systematic uncertainties. The widths of the boxes are not indicative of the bin size and the points are shifted horizontally for better visibility.}
\label{fig:centdep}
\end{figure}
%%%%%%%%%%%%%

The centrality dependence of \RDptr\ for two charged-particle \pt\ intervals: 1.6--2.5~\GeV\ and \mbox{6.3--10.0~\GeV}, and two different \ptjet\ ranges: 126--158~\GeV\ and 200--251~\GeV, is presented in Figure~\ref{fig:centdep}. 
For both \ptjet\ selections and  1.6--2.5~\GeV\ charged particles, the magnitude of the excess increases
with increasing collision centrality and \rvar\ for $\rvar < 0.3$.  The magnitude of the excess is
approximately a factor of two in the most central collisions for $\rvar >$~0.3.
A continuous centrality dependent suppression of  yields of charged-particles with $6.3 < \pt < 10.0$ GeV is observed.
%With the same \ptjet\ selections and 
%charged-particles with 6.3~$ < \pt < $~10.0~\GeV a clear ordering in centrality is observed with
%the most central collisions exhibiting the smallest \RDptr\ values.
The magnitude of the modifications decreases with decreasing collision centrality for both \pt\ 
intervals and \ptjet\ selections.

\begin{figure}[ht]
\centerline{
\includegraphics[width=0.8\textwidth]{figures/main/results/RDpT_dR_trk3_trk6_cent0.pdf} 
}
\caption{\RDptr\ as a function of \rvar\ for 0--10\% collisions for charged particles with 1.0~$< \pt <$~1.6~\GeV\
(closed symbols) and 6.3~$< \pt <$10.0~\GeV\ (open symbols) for different \ptjet\ selections. The vertical bars on the data points indicate statistical uncertainties while the shaded boxes indicate systematic uncertainties. The widths of the boxes are not indicative of the bin size and the points are shifted horizontally for better visibility.}
\label{fig:ptjetdep}
\end{figure}
%%%%%%%%%%%%%


%In order to directly explore the \ptjet\ dependence of \RDptr\, the values are overlaid for all four
%\ptjet\ selections measured here in Figure~\ref{fig:ptjetdep} for the 0--10\% most central collisions 
%and the same two charged-particle \pt\ selections as in Figure~\ref{fig:centdep}.
The \ptjet\ dependence of the \RDptr\ values is directly explored by overlaying 
\ptjet\ selections in Figure~\ref{fig:ptjetdep}. These distributions are for the 0--10\% 
most central collisions, for the same two charged-particle \pt\ selections shown in Figure~\ref{fig:centdep}. 
  A trend of increasing \RDptr\ with increasing \ptjet\ is observed for $r < 0.25$ for low 
\pt\ charged particles; at larger \rvar\ values there is no significant dependence of \RDptr\ on \ptjet. 
Furthermore, for the higher-\pt\ charged particles, no significant dependence on \ptjet\ is observed. 


\begin{figure}
\centering{
\begin{tabular}{cc}
	 \includegraphics[width=0.5\textwidth]{figures/main/results/RDpT_trkpt_jet7_cent0} &
	 \includegraphics[width=0.5\textwidth]{figures/main/results/RDpT_trkpt_jet9_cent0} \\
	 \includegraphics[width=0.5\textwidth]{figures/main/results/RDpT_trkpt_jet7_cent3} &
	 \includegraphics[width=0.5\textwidth]{figures/main/results/RDpT_trkpt_jet9_cent3} \\
	 \includegraphics[width=0.5\textwidth]{figures/main/results/RDpT_trkpt_jet7_cent5} &
	 \includegraphics[width=0.5\textwidth]{figures/main/results/RDpT_trkpt_jet9_cent5} \\
\end{tabular} }
   \caption{\RDptr\ as a function of \pt\ in  0--10\% (top), 30--40\% (middle), and 60--80\% (bottom) \PbPb\ collisions to \pp\ collisions for two different \ptjet\ selections: 126--158~\GeV\ (left) and 200--251~\GeV\ (right). The different colors indicate different angular distances from the jet axis. The vertical bars on the data points indicate statistical uncertainties while the shaded boxes indicate systematic uncertainties. The widths of the boxes are not indicative of the bin size and the points are shifted horizontally for better visibility.}
      \label{fig:pttrkdep}
\end{figure}


%%pt track dependence
In Figure~\ref{fig:rdptr}, it was shown that for central and mid-central collisions, there is an enhancement of
charged particles with $\pt <$~4.0~\GeV\ and a suppression of charged particles with $\pt >$~4.0~\GeV.  In
Figure~\ref{fig:pttrkdep} 
the \pt\ dependance for selections in \rvar\ is directly investigated for 0--10\%, 30--40\% and 60--80\% central 
collisions for 126--158 and 200--251~\GeV\ jets.
Interestingly, at all measured \pt, there is no significant suppression of the yields in \pbpb\ collisions
for $\rvar < 0.05$.  For larger \rvar\ values the yields are enhanced for charged-particles with $\pt <$~4~\GeV\ and 
suppressed for higher \pt\ charged-particles in both the 0--10\% and 30--40\% centrality selections and both \ptjet\ 
ranges presented here.  The magnitude of the enhancement increases for decreasing \pt\ at low \pt and increases
with increasing \pt at high \pt, until about 10~\GeV, after which the suppression remains approximately constant.
At fixed \pt\ the magnitude of the deviation from unity is largest for 0.3$< \rvar <$~0.4 and 0.5$< \rvar <$~0.6.
In the 60--80\% central collisions, the same trend remains true (but with smaller magnitude 
modifications) for \mbox{$126 < \ptjet < 158$ GeV}; for the higher \ptjet\ selection the larger uncertainties 
do not allow a clear conclusion to be drawn for peripheral collisions.

One possible explanation of the modification of the 
jet fragmentation in this kinematic range~\cite{PhysRevC.98.024908} is the larger expected energy loss
of gluon-initiated jets leading to a relative enhancement of quark jets in \pbpb\ collisions compared
to \pp\ collisions at a given \ptjet\ value~\cite{Spousta:2015fca}. Since gluon jets have a broader distribution of particle transverse momentum with respect to the jet direction compared to quark-initiated jets \cite{OPAL:1995ab}
 , such an effect could potentially describe the narrowing of particle distribution around the jet direction for particles with $\pt >$~4.0~\GeV\
observed here, though no calculations of this are available.

%\begin{figure}
%\centering{
%\begin{tabular}{cc}
%	 \includegraphics[width=0.5\textwidth]{figures/main/results/RDpT_trkpt_jet7_dR0} &
%	 \includegraphics[width=0.5\textwidth]{figures/main/results/RDpT_trkpt_jet9_dR0} \\
%	 \includegraphics[width=0.5\textwidth]{figures/main/results/RDpT_trkpt_jet9_dR0} &
%	 \includegraphics[width=0.5\textwidth]{figures/main/results/RDpT_trkpt_jet10_dR0} \\
%	 \includegraphics[width=0.5\textwidth]{figures/main/results/RDpT_trkpt_jet7_dR3} &
%	 \includegraphics[width=0.5\textwidth]{figures/main/results/RDpT_trkpt_jet9_dR3} \\
%	 \includegraphics[width=0.5\textwidth]{figures/main/results/RDpT_trkpt_jet7_dR10} &
%	 \includegraphics[width=0.5\textwidth]{figures/main/results/RDpT_trkpt_jet9_dR10} \\
%\end{tabular} }
%   \caption{\RDptr\ for central \pbpb\ collisions as a function of \pt\ for different jet selections. The different colors represent different centrality bins. The vertical bars on the data points indicate statistical uncertainties while the shaded boxes indicate systematic uncertainties. The widths of the boxes are not indicative of the bin size and the points are shifted horizontally for better visibility.}
%      \label{fig:rdptr_trk_cent}
%\end{figure}
%%%%%%%%%%%%%


\FloatBarrier


\subsection{Differences of \Dptr\ distributions}
In addition to the ratios of the \Dptr\ distributions, differences between the charged particle yields are also evaluated to quantify the modification in terms of the particle density. These are given as:

\begin{align}
\DeltaDptr = \Dptr_{\mathrm{Pb+Pb}} - \Dptr_{pp}
\end{align}

These differences are presented as a function of $r$ for different \pt\ selections in 0--10\% central collisions in Figure~\ref{fig:deltadptr}. 
These distributions show an excess  in the charged-particle yield density for \pbpb\ collisions compared to \pp\ collisions for charged particles with $\pt <4.0$ GeV. This excess ranges from 0.5 to 4 particles per unit area at 1 \GeV\ in 126--158~\GeV\ jets for 0--10\% central \pbpb\ collisions and increases with increasing \ptjet. 
The largest excesses for charged particles with $\pt <$~4.0~\GeV\ is within the jet cone.  For large \rvar\ values, the
density decreases, but remains positive.
A depletion for higher \pt\ particles of approximately 0.5 particles per unit area is seen for 126--158~\GeV\ jets in 0--10\% central \pbpb\ collisions. The magnitude of this depletion increases for higher \ptjet. 
There is a minimum in the \DeltaDptr\ distributions of charged 
particles with \mbox{$ 4.0 < \pt <  25.1$}~\GeV\ at $0.05 < \rvar < 0.10$ that is seen at many \ptjet\ ranges under investigation.
The magnitudes of the excesses and deficits discussed here are dependent on the sizes of the charged-particle \pt\ selections
chosen.  In order to remove that dependence, Section~\ref{sec:discussion_int} provides similar quantities in which a
wider charged-particle \pt\ range is integrated over.
%For particles with 25.1~$< \pt <$~63.1~\GeV, the \DeltaDptr\ distribution is consistent with unity over the entire measured range of \rvar\ and \ptjet.

\begin{figure}
\centering{
\begin{tabular}{cc}
	 \includegraphics[width=0.5\textwidth]{figures/main/results/DeltaDpT_dR_jet7_cent0} &
	 \includegraphics[width=0.5\textwidth]{figures/main/results/DeltaDpT_dR_jet8_cent0} \\
	 \includegraphics[width=0.5\textwidth]{figures/main/results/DeltaDpT_dR_jet9_cent0} &
	 \includegraphics[width=0.5\textwidth]{figures/main/results/DeltaDpT_dR_jet10_cent0} \\
\end{tabular} }
   \caption{\DeltaDptr\ as a function of \rvar\ in central collisions for all \pt\ ranges in four \ptjet\ selections: 126--158~\GeV, 158--200~\GeV, 200--251~\GeV, and 251--316~\GeV. The vertical bars on the data points indicate statistical uncertainties while the shaded boxes indicate systematic uncertainties. The widths of the boxes are not indicative of the bin size and the points are shifted horizontally for better visibility. }
      \label{fig:deltadptr}
\end{figure}
%%%%%%%%%%%%%
\FloatBarrier

\subsection{\pt\ integrated distributions}
\label{sec:discussion_int}
Motivated by similar studies of the enhancement of soft fragments in 
jet fragmentation functions in \pbpb\ compared to \pp\ collisions from \cite{PhysRevC.98.024908}, the \Dptr\ distributions can be integrated for charged particles with \pt\ < 4 GeV to construct the quantities $\Theta(\rvar)$ and $P(\rvar)$ defined as:

\begin{align}
   \Theta(\rvar) &= \int_1^{4} \Dptr  \fd \pt \\
   P(\rvar) &= \int_0^r \int_1^{4} \Dptr \fd \pt \fd r'
\end{align}

The $\Theta(\rvar)$ values are integrated between 1.0--4.0~\GeV\ charged particles to provide a summary look at
the \pt\ region of enhancement discussed above.  The $P(\rvar)$ values further add a running integral over \rvar\
and provide information about the jet shape.
Both of these quantities can be compared between the \pp\ and \pbpb\ systems to give the following distributions:

\begin{align}
   \Delta_{\Theta(\rvar)} &= \Theta(\rvar)_{\mathrm{Pb+Pb}} - \Theta(\rvar)_{pp} \\
   R_{\Theta(\rvar)} &= \frac{\Theta(\rvar)_{\mathrm{Pb+Pb}}}{\Theta(\rvar)_{\mathrm{pp}}} \\
   R_{P(\rvar)} &= \frac{P(\rvar)_{\mathrm{Pb+Pb}}}{P(\rvar)_{pp}}
\end{align}

(the quantity $\Delta_{P(\rvar)}$ can also be analogously defined, but is omitted from the present discussion).
These aggregate quantities are intended to provide some summary information about the location with respect to the 
jet axis, magnitude, and \ptjet\ dependence of the low-\pt\ charged-particle excess discussed above.
The ratio quantities are useful for comparisons to other \pbpb\ measurements; $\Delta_{\Theta(\rvar)}$ is very similar 
to $\DeltaDptr$, however it is integrated over charged-particle \pt\ from 1.0--4.0~\GeV.

Figure~\ref{fig:deltaPdeltaT} shows the \DeltaTheta\ distributions as a function of \rvar\ for 0--10\%, 30--40\%,
and 60--80\% central collisions. 
In the most central collisions, a significant \ptjet\ dependence to \DeltaTheta\ is observed; for $\rvar <$~0.4 (particles
within the jet cone) \DeltaTheta\ increases with increasing \ptjet.
The value of \DeltaTheta\ decreases in more peripheral collisions and the \ptjet\ dependence is no longer significant.

%Now, the \ptjet\ dependence to the excess in charged-particle density can be seen clearly; 
%in the most central collisions
%there is an increase in \DeltaTheta\ with increasing \ptjet, but in the mid-central and peripheral collisions this is no longer
%observed within the uncertainties.

\begin{figure}
\centering{
\begin{tabular}{ccc}
	 \includegraphics[width=0.3\textwidth]{figures/main/results/DeltaDpT_lowpt_integ_cent0.pdf} &
   %	 \includegraphics[width=0.5\textwidth]{figures/main/results/DeltaDpT_jetshape_cent0.pdf} \\
	 \includegraphics[width=0.3\textwidth]{figures/main/results/DeltaDpT_lowpt_integ_cent3.pdf} &
%	 \includegraphics[width=0.5\textwidth]{figures/main/results/DeltaDpT_jetshape_cent3.pdf} \\
	 \includegraphics[width=0.3\textwidth]{figures/main/results/DeltaDpT_lowpt_integ_cent5.pdf} \\
%	 \includegraphics[width=0.5\textwidth]{figures/main/results/DeltaDpT_jetshape_cent5.pdf} \\
\end{tabular} }
   \caption{\DeltaTheta\ as a function of \rvar\ for charged-particles with \pt\ < 4 GeV ranges in 
   four \ptjet\ selections: 126--158~\GeV, 158--200~\GeV, 200--251~\GeV, and 251--316~\GeV and three centrality 
   selections: 0--10\% (left), 30--40\% (middle) and 60--80\% (right). 
   The vertical bars on the data points indicate statistical uncertainties while the shaded boxes indicate systematic uncertainties. The widths of the boxes are not indicative of the bin size and the points are shifted horizontally for better visibility. }
      \label{fig:deltaPdeltaT}
\end{figure}


\begin{figure}
\centering{
\begin{tabular}{cc}
	 \includegraphics[width=0.5\textwidth]{figures/main/results/RDpT_lowpt_integ_cent0.pdf} &
	 \includegraphics[width=0.5\textwidth]{figures/main/results/RDpT_jetshape_cent0.pdf} \\
	 \includegraphics[width=0.5\textwidth]{figures/main/results/RDpT_lowpt_integ_cent3.pdf} &
	 \includegraphics[width=0.5\textwidth]{figures/main/results/RDpT_jetshape_cent3.pdf} \\
	 \includegraphics[width=0.5\textwidth]{figures/main/results/RDpT_lowpt_integ_cent5.pdf} &
	 \includegraphics[width=0.5\textwidth]{figures/main/results/RDpT_jetshape_cent5.pdf} \\
\end{tabular} }
   \caption{\RTheta\ (left) and \RP\ (right) as a function of \rvar\ in central collisions for charged-particles with \pt\ < 4 GeV ranges in four \ptjet\ selections: 126--158~\GeV, 158--200~\GeV, 200--251~\GeV, and 251--316~\GeV and three centrality selections: 0--10\% (top), 30--40\% (middle) and 60--80\% (bottom). The vertical bars on the data points indicate statistical uncertainties while the shaded boxes indicate systematic uncertainties. The widths of the boxes are not indicative of the bin size and the points are shifted horizontally for better visibility. }
      \label{fig:RPRT}
\end{figure}


Figure~\ref{fig:RPRT} shows \RTheta\ and \RP\ for 0--10\%, 30--40\% and 60--80\% central collisions.  The \RTheta\ 
distributions of the most central collisions show a maximum for $\rvar \sim 0.4$ and decrease for larger \rvar.
However, since \RTheta\ remains at or above unity for the full range of \rvar\ values presented \RP\ continues
to slowly increase with increasing \rvar\ over the full measured range.  In more peripheral collisions,
the magnitude of the excess is reduced and the trends in \RTheta\ are less clear, however the slow increase
of \RP\ is clearly seen for the 30--40\% central collisions.


%These measurements show that the excess of particles with $\pt <$~4.0~\GeV\ observed in~\cite{PhysRevC.98.024908} extends
%outside the \RFour\ jet cone.
%The measured dependence of \RDptr\ suggests that the energy lost by jets through the jet quenching process is being transferred to particles with $\pt <$~4.0~\GeV\ at larger radial distances from the jet axis. 
%This is qualitatively consistent with theoretical calculations \mbox{\cite{Blaizot:2014ula}}.
%Additionally, these observations are in agreement with the previous measurement of jet fragmentation functions \cite{Chatrchyan:2014ava, Sirunyan:2018jqr, Aaboud:2017bzv, PhysRevC.98.024908} and may indicate the dependence of the response of the hot dense matter to the momentum of a jet passing through it. 


\FloatBarrier


%\appendix
%\section{Appendix}
%\label{sec:appendixA}
%% !TEX root = thesis-ex.tex

\begin{figure}
	\centering
	\includegraphics[width=1.0\textwidth]{figures/c.pdf} 
	\caption{ $\Delta\phi$ distributions for truth and reco (left). Response Matrix $M_{ij}$ (center). Correction factors with errors (right). }	
	\label{fig:plots}
\end{figure}

In Figure~\ref{fig:plots}, we have truth and reconstructed $\Delta\phi$ distributions on the left-most plot, the response matrix $M_{ij}$ where $\Delta\phi_{Reco}$ is along the x-axis, along the j-index, and $\Delta\phi_{Truth}$ is along the y-axis, along the i-index, and resulting correction factors with errors on the right-most. 

Define $T_{i}$ as the total number of entries in the $i^{th}$ bin of the Truth distribution (blue points on left plot), and $R_{i}$ as the total number of entries in the $i^{th}$ bin of the Reconstructed distribution (red points on left plot). 

In terms of the response matrix, $R_{j}$ is

\begin{eqnarray} \label{eq:rj}
R_j = \sum_{i}^{}M_{ij} = M_{jj} + \sum_{i\neq j}^{}M_{ij} 
\end{eqnarray}

The last part is just the diagonal element plus the off-diagonal vertical elements of the $i^{th}$ bin (on the x-axis).

Similarly, in terms of the response matrix, $T_{i}$ is

\begin{eqnarray} \label{eq:ti}
T_i = \sum_{j}^{}M_{ij} = M_{ii} + \sum_{j\neq i}^{}M_{ij} 
\end{eqnarray}

For some bin $i^{th}$ reconstructed bin,

\begin{eqnarray} \label{eq:leavearrive}
R_{i} = T_{i} - N_{Leaving} + N_{Arriving} = T_{i} - \sum_{k\neq i}^{}M_{ik} + \sum_{j\neq i}^{}M_{ji}
\end{eqnarray}

We can express the number leaving and number arriving in terms of off-diagonal row or column elements of $M_{ij}$, or in terms of  $T_{i}$, $R_{i}$, and diagonal elements of  $M_{ij}$.

\begin{eqnarray} \label{eq:leavearrivediagonal}
N_{Leaving} = T_{i} - M_{ii} \\
N_{Arriving} = R_{i} - M_{ii} 
\end{eqnarray}

Now, $T_{i}$ is taken as a constant. This means that reconstructed distribution can be different time to time, but the truth distribution stays the same. In the language of a toy MC, this is equivalent to generating one Truth distribution, and smearing it many different times, each time (or for each new "experiment") getting new results.

When $T_{i}$ is taken as constant, the bin migration of leaving and arriving is different. The distribution of $N_{Leaving}$ is binomial, while $N_{Arriving}$ is Poisson. If $T_{i}$ is fixed, there is only a certain number of entries that can leave, while the number that arrives depends on, and is a mix of the entries leaving neighboring bins. 

In a toy MC \footnote{A Toy MC with a randomly generated exponential was generated for the truth distribution 5000 times, with smearing from the ATLAS MC response matrix applied to the reconstructed distribution. The experiment was then repeated 10,000 times to get some good statistics on correction factors, their errors, bin migration, etc.}, for the case where the truth distribution was generated one time, but smearing applied to the reconstructed (case with "fixed" $T_{i}$), it is clear from Figure~\ref{fig:mig_same} that the migration where entries are leaving is narrower than where the arrive. In the same toy MC, when for every experiment a new truth distribution was used, it is evident that the migration to and from is the same.

\begin{figure}
	\centering
	\includegraphics[width=1.0\textwidth]{figures/c3_same.pdf} 
	\caption{ For the case where for every experiment a the same generated truth distribution but differently smeared reconstructed distribution, histogram of migration between $\Delta\phi$ bins (x and y axes) for entries arriving (right) and entries leaving (left).Migration where entries leave has a binomial (narrower) distribution, while entries arriving is Poisson. }	
	\label{fig:mig_same}
\end{figure}

\begin{figure}
	\centering
	\includegraphics[width=1.0\textwidth]{figures/c3_diff.pdf} 
	\caption{ For the case where for every experiment a new truth distribution is generated and the reconstructed is smeared from that, histogram of migration between $\Delta\phi$ bins (x and y axes) for entries arriving (right) and entries leaving (left). Both migrations have Poisson distributions. }	
	\label{fig:mig_diff}
\end{figure}

Correction factors $C_{i}$, which relate $T_{i}$ and $R_{i}$ are

\begin{eqnarray} \label{eq:cfactors}
C_{i} = \frac{T_{i}}{R_{i}}
\end{eqnarray} 

and their respective errors $\sigma_{C_{i}}$ are

\begin{eqnarray} \label{eq:cfactorerrors}
\sigma_{C_{i}}^{2} = \frac{C_{i}^{2}}{R_{i}^{2}}\sigma_{R_{i}}^{2}
\end{eqnarray}	

Now since $T_{i}$ is constant, and the entries leaving a $T_{i}$ bin follow binomial statistics, while entries arriving are Poisson, we continue from Equation~\ref{eq:leavearrive}. The error in $R_{i}$ is 

\begin{eqnarray}
\sigma_{R_{i}}^{2} = \sigma_{N_{Leave}}^{2} + \sigma_{N_{Arrive}}^{2} \\ 
\sigma_{R_{i}}^{2} = T_{i}\frac{T_{i} - M_{ii}}{T_{i}}\big(1 - \frac{T_{i} - M_{ii}}{T_{i}}\big)+(R_{i}-M_{ii}) \\
\sigma_{R_{i}}^{2} = T_{i} + R_{i} - 2M_{ii} - \frac{(T_{i} - M_{ii})^{2}}{T_{i}}
\end{eqnarray}	

From this, plugging into Equation~\ref{eq:cfactorerrors}, the error on the correction factor is

\begin{eqnarray} \label{eq:cfactorerror}
\sigma_{C_{i}}^{2} = \frac{T_{i}^{2}}{R_{i}^{4}}\Big(T_{i} + R_{i} -2M_{ii} - \frac{(T_{i} - M_{ii})^{2}}{T_{i}}\Big) \\
\sigma_{C_{i}}^{2} = \frac{T_{i}^{2}}{R_{i}^{3}}\Big(1 - \frac{M_{ii}^{2}}{T_{i}R_{i}}\Big).
\end{eqnarray}	

%\clearpage
