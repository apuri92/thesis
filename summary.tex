% !TEX encoding = UTF-8 Unicode
% !TEX root = thesis-ex.tex

This thesis presents a measurement of the yields of charged particles, \Dptr, inside and around \RFour\ \antikt\ jets with $|\yjet| <$1.7 up to a distance of $r = 0.8$ from the jet axis.
The yields are measured in intervals of \ptjet\ from 126 to 316~\GeV\ in \PbPb\ and \pp\ collisions at 5.02~\TeV\ as a function of charged-particle \pt\ and the angular distance \rvar\ between the jet axis and charged particle.

The results show a broadening of the \Dptr\ distribution for low \pt\ particles inside the jet in central \pbpb\ collisions compared to those in \pp\ collisions while for higher \pt\ particles angular distributions are narrower in \pbpb\ collisions compared to \pp\ collisions.
These modifications are centrality dependent and decrease for more peripheral collisions.
The \RDptr\ distributions for charged particles with $\pt <$~4~\GeV\ are above unity and grow with increasing angular separation up to $r \sim0.3$, showing weak to no dependence on $r$ in the interval 0.3~$< \rvar <$~0.6 followed with a small decrease in the enhancement for 0.6~$< \rvar <$~0.8.
For charged particles with $\pt >$~4~\GeV, a suppression in \RDptr\ is observed, and the distributions decrease with increasing \rvar\ for 0.05 $ < \rvar < $~0.3, with no \rvar\ dependence for $r>0.3$.
For all charged-particle \pt\ values, the \RDptr\ values are greater than or equal to unity for $\rvar <$~0.05.
Between $0.1 < r < 0.25$, a statistically significant trend of increasing \RDptr\ with increasing \ptjet\ is observed for low-\pt\ particles.
No significant \ptjet\ dependence is seen for particles  with $\pt >$~4~\GeV.

While there have been a variety of measurements and models that describe the jet shape, this measurement is the first to describe both the radial and momentum dependence of charged particles inside and outside the jet cone. In particular, observations made in this thesis can help constrain models of jet energy loss that distinguish the modifications of jet due to the presence of the plasma from the response of the medium to the jet.

At the time of writing this thesis, preliminary results from this analysis have been shown at both the Hard Probes 2018 Conference in Aix-le-Bains, France, and the Quark Matter Conference 2018 in Venice, Italy. The full analysis is currently in the process of approval from the ATLAS Collaboration and will be published in Physical Review C.
