The \pPb\ data used in this analysis were recorded in 2016 and the samples used are shown in Table~\ref{tab:datasamples} in the appendix. The LHC was configured with a 4 TeV proton beam and a 1.57~TeV per nucleon Pb beam  producing collisions with \sqrtsnn~=~5.02~TeV and a rapidity shift of the  nucleon-nucleon center-of-mass frame $\Delta y=-0.465$ relative to the lab frame. The data collected had one beam configuration with the Pb beam traveling to the positive pseudorapidity direction and the proton beam to the negative pseudorapidity direction. To be consistent with previous \pPb\ physics measurements~\cite{ATLAS:2014cpa,Aaboud:2017tke}, the positive center-of-mass rapidity direction, $\ystar>0$ is chosen as the proton beam direction. The physical detector is described in terms of $\eta$ and is consistent with conditions used during data-taking while the center-of-mass rapidity \ystar\ is the physics quantity in which results are presented. The integrated luminosity of the 2016 \pPb\ data taken is 360~\mubarn$^{-1}$. The \pp\ data used in this measurement was recorded in 2015 with the LHC configured to collide two equal energy proton going beams at a center-of-mass energy of \sqrts~=~5.02~TeV. These \pp\ and \pPb\ data samples are shown in Table~\ref{tab:datasamples} in the appendix. The instantaneous luminosity conditions provided by the LHC during \pPb\ data taking resulted in an average number of interactions per bunch crossing of 0.03. During \pp\ data taking, the average number of interactions per bunch crossing varied from 0.6 to 1.3. 

The performance for measuring azimuthal angular correlations and conditional yields in both the 2015  \pp\ and 2016 \pPb\ data samples is evaluated with a 5.02~TeV \pp\ MC sample simulated using \pythia\ 8.212~\cite{Sjostrand:2014zea}. Hard scattering \pp\ events with the A14~\cite{ATLAS2014021} tune and the next-to-next order NNPDF23LO PDF set~\cite{Ball:2012cx} are used. The detector response is then simulated using GEANT4~\cite{Agostinelli:2002hh,Aad:2010ah}. The \pp\ samples used for this analysis contain approximately 12 million events, and are listed with their respective number of events in the top Table~\ref{tab:mcsamples} in the appendix. Corresponding \pPb\ MC samples are obtained by overlaying minimum-bias \pPb\ data events recorded during the 2016 data-taking period with simulated 5.02~TeV \pp\ events generated with the same MC tune as for the \pp\ MC sample but with a rapidity shift equivalent to that in the \pPb\ collisions. Detector response is also modeled using GEANT4. Due to the forward rapidity filtering, approximately 3 million events were used in the \pPb\ MC samples. These samples are listed in the middle of Table~\ref{tab:mcsamples} in the appendix, along with their respective number of events. Additionally, approximately 5 million events of the 5.02~TeV \pp\ \herwig~\cite{Bahr:2008pv} MC simulation are used to compare with the \pp\ \pythiaeight\ performance to determine the uncertainties on position resolution. The samples used in the \herwig\ MC, with their respective number of events are listed in the bottom of Table~\ref{tab:mcsamples} in the appendix.

\begin{table}[h]
	\centering
	\begin{tabular}{|| c | c | c | c || } 
		\hline
		$JZ$N & \RFour\ \pttruth\ [GeV] & $\sigma$ [nb] $\times\ \epsilon$ (\pp)  & $\sigma$ [nb] $\times\ \epsilon$ (\pPb) \\ 
		\hline
		1 & $20-60$ & $8.15\times 10^{7} \times 2.83\times 10^{-3}$ & $6.79\times 10^{7} \times 3.85\times 10^{-4}$ \\
		2 & $60-160$ & $6.40\times 10^{5} \times 4.28\times 10^{-3}$ & $8.96\times 10^{5} \times 2.53\times 10^{-3}$ \\
		\hline	
	\end{tabular}
	\caption{ Summary of \pt\ ranges, cross-section weights $\sigma$, and filtering efficiencies $\epsilon$ in $JZ$N slices for \pp\ and \pPb\ MC samples.  }
	\label{tab:mcweights}
\end{table}

The MC samples used in this analysis are split into so called cross-section weighted slices. This is done in order for different analysis to be able to the \pt\ regions of phase space that they are interested in for their measurement. Some measurements require high \pt\ jets and some require the lower end of the spectra. The slices are numbered $JZ$N, where N is an integer indicating the \pt\ interval covered by that sample. Each slice has a cross section weight $\sigma_{i}$ and a filtering efficiency $\epsilon_{i}$ which represents the generator level filtering that was implemented to select the  appropriate \pt\ of jets for each $JZ$ sample. This analysis uses the $JZ1$ and $JZ2$ cross section weighted slices. Their respective cross section weights and filtering efficiencies are summarized in Table~\ref{tab:mcweights}. Transverse momentum intervals for each $JZ$ slice are consistent between \pp\ and \pPb\ MC samples, but filtering efficeincies and cross section weights are different. If a wide interval of jet \pt\ is used in an analysis, covering the ranges of multiple $JZ$ slices, a cross section re-weighting must be implemented when combining slices in order to guarantee a smooth jet \pt\ spectra. If an observable $\omega$ in some bin is a counted quantity, the prescription for combined counts over all cross section weighted slices $i$ with cross section weights $\sigma_{i}$ and filtering efficiencies $\epsilon_{i}$ is:

\begin{equation}
	\omega = \frac{\sum_{}^{}\omega_{i}\sigma_{i}\epsilon_{i}}{\sum_{}^{} \sigma_{i}\epsilon_{i}}.
\end{equation}

If an observable $\omega$ is a result of a calculation, the prescription for getting a final cross section weighted value also depends on the number of entries $n_{i}$ in each bin of the observable and the total number of events $N_{i}^{ev}$ in each $JZ$ slice: 


\begin{equation}
\omega = \frac{\sum_{}^{}\omega_{i}\sigma_{i}\epsilon_{i}\frac{n_{i}}{N_{i}^{ev}}}{\sum_{}^{} \sigma_{i}\epsilon_{i}\frac{n_{i}}{N_{i}^{ev}} }.
\end{equation}


\FloatBarrier

