% !TEX encoding = UTF-8 Unicode
%%
%% This is file `thesis-ex.tex',
%% generated with the docstrip utility.
%%
%% The original source files were:
%%
%% uiucthesis2009.dtx  (with options: `example')
%% 
\newcommand*{\ATLASLATEXPATH}{latex/}
%\documentclass[PAPER, atlasdraft=true, texlive=2016, UKenglish, coverpage, PAPER]{\ATLASLATEXPATH atlasdoc}
\RequirePackage{lineno}
\linenumbers
\documentclass[fullpage, UKenglish]{uiucthesis2009}
\usepackage{amsmath}
\usepackage{amssymb}
\usepackage{graphicx}
\usepackage{pdfpages}
\usepackage{amsmath}
\usepackage{slashed}
\usepackage{placeins}
\usepackage{tabularx}
\usepackage[utf8]{inputenc}

\usepackage[backend=bibtex]{\ATLASLATEXPATH atlaspackage}
\usepackage{\ATLASLATEXPATH atlasbiblatex}


\usepackage{hyperref}
\usepackage{booktabs}
\usepackage{subcaption}
%\usepackage[applemac]{inputenc}
%\captionsetup{compatibility=false}
% !TEX encoding = UTF-8 Unicode
% !TEX root = thesis-ex.tex
\renewcommand{\labelitemii}{$\circ$}
\newcommand{\epm}{$e^+ e^-$}
\newcommand{\qqbar}{$q \bar{q}$}
\newcommand{\sigmainel}{\sigma_{\mathrm{inel}}^{\mathrm{NN}}}
\newcommand{\DeltaP}{$\Delta_P(\rvar)$}
\newcommand{\DeltaTheta}{$\Delta_{\Theta(\rvar)}$}
\newcommand{\RP}{$R_{P(\rvar)}$}
\newcommand{\RTheta}{$R_{\Theta(\rvar)}$}
\newcommand{\DeltaDptr}{\mbox{$\Delta\Dptr$}}
\newcommand{\alphas}{$\alpha_s$}
\newcommand{\qbar}{\bar{q}}


\newcommand{\EM}{\mbox{\text{EM}}}
\newcommand{\HI}{\mbox{\text{HI}}}

\newcommand{\ptEM}{\mbox{$p_{\mathrm{T}}^{\mathrm{EM}}$}}
\newcommand{\ptHI}{\mbox{$p_{\mathrm{T}}^{\mathrm{HI}}$}}
\newcommand{\REM}{\mbox{$R_{\mathrm{EM}}$}}
\newcommand{\RHI}{\mbox{$R_{\mathrm{HI}}$}}
\newcommand{\sEM}{\mbox{$s_{\mathrm{EM}}$}}
\newcommand{\sHI}{\mbox{$s_{\mathrm{HI}}$}}
\newcommand{\deltaEM}{\mbox{$\Delta_{\mathrm{EM}}$}}
\newcommand{\deltaHI}{\mbox{$\Delta_{\mathrm{HI}}$}}




\newcommand{\pttrkreco}{\mbox{$p_{{\mathrm{T}}}^{\mathrm{trk,reco}}$}}
\newcommand{\Dptrmeas}{\mbox{$D^{\mathrm{meas}}(\pT,r)$}}
\newcommand{\Dptrsub}{\mbox{$D^{\mathrm{sub}}(\pT,r)$}}
\newcommand{\Rdptr}{\mbox{$R_{D( \pT, r)}$}}


\newcommand{\phijet}{\mbox{$\phi^{\mathrm{jet}}$}}
\newcommand{\NchUECone}{\mbox{$N_{\mathrm{ch}}^{\mathrm{UE\ cone}}$}}


\newcommand{\radlen}{\mbox{$X_{0}$}}
\newcommand{\intlen}{\mbox{$\lambda_{I}$}}
\newcommand{\rphi}{\mbox{$R - \phi$}}


\newcommand\blfootnote[1]{%
  \begingroup
  \renewcommand\thefootnote{}\footnote{#1}%
  \addtocounter{footnote}{-1}%
  \endgroup
}


\newcommand{\AtlasCopyrightFooter}{%
  \parbox[b]{\linewidth}{%
    \rmfamily\mdseries\fontsize{10}{12}\selectfont
    \copyright\ 2018 \ CERN for the benefit of the ATLAS Collaboration.\newline
    Reproduction of this article or parts of it is allowed as specified in the CC-BY-4.0 license.
  }\par
}

\newcommand{\pb}{pb$^{-1}$}
\newcommand{\nb}{nb$^{-1}$}

\newcommand{\pttrktruth}{\mbox{$p_{\mathrm{T}}^{\mathrm{trk, Truth}}$}}
\newcommand{\etatrktruth}{\mbox{$\eta^{\mathrm{trk}}_{\mathrm{truth}}$}}


\newcommand{\Dptr}{\mbox{$D( \pT, r)$}}
\newcommand{\RDptr}{\mbox{$R_{D( p_{\mathrm{T}}, r)}$}}
\newcommand{\nucnuc}{\mbox{A+A}}
\newcommand{\AuAu}{\mbox{Au+Au}}
\newcommand{\PbPb}{\mbox{Pb+Pb}}
\newcommand{\pbpb}{\mbox{Pb+Pb}}
\newcommand{\pp}{\mbox{$pp$}}
\newcommand{\dR}{\mbox{$\Delta R$}}

\newcommand{\pTmin}{\mbox{$p_{\mathrm{T,min}}$}}
\newcommand{\pTmax}{\mbox{$p_{\mathrm{T,max}}$}}
\newcommand{\dpT}{\mbox{$\mathrm{d}p_{\mathrm{T}}$}}
\newcommand{\NBJ}{\mbox{$R_{\Delta R}$}}
\newcommand{\Etmiss}{\mbox{$E_{\mathrm{T}^{\mathrm{miss}}}$}}
\newcommand{\Et}{\mbox{$E_{\mathrm{T}}$}}
\newcommand{\pt}{\mbox{$p_{\mathrm{T}}$}}
\newcommand{\pT}{\mbox{$p_{\mathrm{T}}$}}
\newcommand{\RNBJ}{\mbox{$\rho_{R_{\Delta R}}$}}
\newcommand{\ETtest}{\mbox{$E_{\mathrm{T}}^\mathrm{test}$}}
\newcommand{\ETnbj}{\mbox{$E_{\mathrm{T}}^\mathrm{nbr}$}}
\newcommand{\ETcombi}{\mbox{$E_{\mathrm{T}}^\mathrm{comb}$}}
\newcommand{\ETmerged}{\mbox{$E_{\mathrm{T}}^\mathrm{merged}$}}

\newcommand{\ANpart}{\mbox{$\langle N_{\mathrm{part}}\rangle$}}
\newcommand{\Ncoll}{\mbox{$N_{\mathrm{coll}}$}}
\newcommand{\Nevt}{\mbox{$N_{\mathrm{evt}}$}}
\newcommand{\Ncone}{\mbox{$N_{\mathrm{cone}}$}}
\newcommand{\Njetcent}{\mbox{$N_{\mathrm{jet}}^{\mathrm{cent}}$}}
\newcommand{\Njet}{\mbox{$N_{\mathrm{jet}}$}}
\newcommand{\Nch}{\mbox{$N_{\mathrm{ch}}$}}
\newcommand{\NchUE}{\mbox{$N_{\mathrm{ch}}^{\mathrm{UE}}$}}
\newcommand{\nchUE}{\mbox{$n_{\mathrm{ch}}^{\mathrm{UE}}$}}
\newcommand{\nchmeas}{\mbox{$n_{\mathrm{ch}}^{\mathrm{meas}}$}}
\newcommand{\nchsub}{\mbox{$n_{\mathrm{ch}}^{\mathrm{sub}}$}}
\newcommand{\nchunf}{\mbox{$n_{\mathrm{ch}}^{\mathrm{unfolded}}$}}
\newcommand{\nch}{\mbox{$n_{\mathrm{ch}}$}}
\newcommand{\Npart}{\mbox{$N_{\mathrm{part}}$}}
\newcommand{\Ntptrt}{N_{\mathrm{2p}}}
\newcommand{\Ntptr}{\mbox{$N_{\mathrm{2p}}$}}
\newcommand{\Nraw}{\mbox{$N^{\mathrm{raw}}$}}
\newcommand{\sqrtsnn}{\mbox{$\sqrt{s_{_\text{NN}}}$}}
%\newcommand*{\sqn}{\ensuremath{\sqrt{s_{_\text{NN}}}}\xspace}
\newcommand{\sqrts}{\mbox{$\sqrt{s}$}}
\newcommand{\centrm}{\mathrm{cent}}
\newcommand{\pythia}{{\textsc PYTHIA}}
\newcommand{\pythiasix}{{\textsc Pythia}6}
\newcommand{\pythiaeight}{\textsc{Pythia}8}
\newcommand{\herwig}{{\textsc Herwig++}}
\newcommand{\powheg}{\textsc{Powheg}}
%\newcommand{\text}{\textit{}}


\newcommand{\RAA}{\mbox{$R_{\rm AA}$}}
\newcommand{\RFive}{\mbox{$R = 0.5$}}
\newcommand{\RFour}{\mbox{$R = 0.4$}}
\newcommand{\RThree}{\mbox{$R = 0.3$}}
\newcommand{\RTwo}{\mbox{$R= 0.2$}}
\newcommand{\Rcp}{\mbox{$R_{\rm CP}$}}
\newcommand{\Rpc}{\mbox{$R_{\rm PC}$}}
\newcommand{\Sk}{\mbox{$S(k)$}}
\newcommand{\TAA}{\mbox{$T_{\mathrm{AA}}$}}

\newcommand{\antikt}{\mbox{anti-\kt}}
\newcommand{\avgpttrue}{\mbox{$\langle p_{\mathrm{T}}^{\mathrm{truth}}\rangle$}}
\newcommand{\centup}{^{\mathrm{cent}}}
\newcommand{\deffrel}{\mbox{$\delta \varepsilon/\varepsilon$}}
\newcommand{\ETfcal}{\mbox{$\Sigma E_{\mathrm{T}}^{\mathrm{FCal}}$}}
\newcommand{\eff}{\mbox{$\varepsilon$}}
\newcommand{\ETtruth}{\mbox{$E_{\mathrm{T}}^{\mathrm{truth}}$}}
\newcommand{\fs}{\mbox{${f_{\mathrm{S}}}$}}
\newcommand{\gjet}{\mbox{$\gamma$-jet}}
\newcommand{\invnb}{\mbox{${\rm nb^{-1}}$}}
\newcommand{\kt}{\mbox{$k_{t}$}}
\newcommand{\kTsq}{\mbox{$k_{\rm T}^2$}}
\newcommand{\kT}{\mbox{$k_{\rm T}$}}
\newcommand{\sumet}{\mbox{$\Sigma E_{\mathrm{T}}$}}
\newcommand{\ETrec}{\mbox{$E_{\mathrm{T}}^{\mathrm{rec}}$}}
\newcommand{\ETtbyf}{\mbox{$E_{\mathrm{T}}^{3\times 4}$}}
\newcommand{\ETsbys}{\mbox{$E_{\mathrm{T}}^{7\times 7}$}}
\newcommand{\Dphi}{\mbox{$\Delta \phi$}}
\newcommand{\Deta}{\mbox{$\Delta \eta$}}
\newcommand{\DpT}{\mbox{$\Delta \pT$}}
\newcommand{\Dz}{\mbox{$D(z)$}}
\newcommand{\Dzmeas}{\mbox{$D^{\mathrm{meas}}(z)$}}
\newcommand{\Dzsub}{\mbox{$D^{\mathrm{sub}}(z)$}}
\newcommand{\Dpt}{\mbox{$D(\pT)$}}
\newcommand{\Dptmeas}{\mbox{$D^{\mathrm{meas}}(\pT)$}}
\newcommand{\Dptsub}{\mbox{$D^{\mathrm{sub}}(\pttrk)$}}
\newcommand{\Dptratio}{\mbox{$D(\pT)|_{\mathrm{cent}}/D(\pT)|_{\mathrm{60-80}}$}}
\newcommand{\Dzratio}{\mbox{$D(z)|_{\mathrm{cent}}/D(z)|_{\mathrm{60-80}}$}}
\newcommand{\Delpt}{\mbox{$\Delta \pT$}}
\newcommand{\DRtrk}{\mbox{$\Delta R_{\mathrm{trk}}$}}
\newcommand{\effpteta}{\mbox{$\varepsilon(\pT, \eta)$}}
\newcommand{\effpt}{\mbox{$\varepsilon(\pT)$}}
\newcommand{\phat}{\mbox{$\hat{p}_{\mathrm{T}}$}}
\newcommand{\pthatmin}{\mbox{$\hat{p}^{\mathrm{min}}_{\mathrm{T}}$}}
\newcommand{\pthat}{\mbox{$\hat{p}_{\mathrm{T}}$}}
\newcommand{\etajet}{\mbox{$\eta^{\mathrm{jet}}$}}
\newcommand{\pTjet}{\mbox{$p_{{\mathrm{T}}}$}}
\newcommand{\pTjetcorr}{\mbox{$p_{{\mathrm{T}}}^{\mathrm{corr}}$}}
\newcommand{\pTch}{\mbox{$p_{{\mathrm{T}}}^{\mathrm{ch}}$}}
\newcommand{\pTtrk}{\mbox{$p_{{\mathrm{T}}}^{\mathrm{trk}}$}}
\newcommand{\pttrk}{\mbox{$p_{\mathrm{T}}^{\mathrm{ch}}$}}
%\newcommand{\pttrk}{\mbox{$p_{\mathrm{T}}^{\mathrm{trk}}$}}
%\newcommand{\pTtrk}{\mbox{$p_{{\mathrm{T}}}^{\mathrm{trk}}$}}
\newcommand{\pTrec}{\mbox{$p_{{\mathrm{T}}}^{\mathrm{rec}}$}}
\newcommand{\pTtrue}{\mbox{$p_{\mathrm{T}}^{\mathrm{truth}}$}}
\newcommand{\etatrue}{\mbox{$\eta^{\mathrm{truth}}$}}
\newcommand{\ztrue}{\mbox{$z^{\mathrm{truth}}$}}
\newcommand{\zrec}{\mbox{$z^{\mathrm{rec}}$}}
\newcommand{\ptvjet}{\mbox{$\displaystyle {\vec{p}_{\mathrm{T}}}^{\, \mathrm{jet}}$}}
\newcommand{\ptvchg}{\mbox{$\displaystyle {\vec{p}_{\mathrm{T}}}^{\, \mathrm{chg}}$}}
\newcommand{\ntrue}{\mbox{$N^{\mathrm{truth}}$}}
\newcommand{\nmatch}{\mbox{$N^{\mathrm{match}}$}}
\newcommand{\rcpcorr}{\mbox{$R_{\mathrm{CP}}$}}
\newcommand{\rcpraw}{\mbox{$R_{\mathrm{CP}}^{\mathrm{meas}}$}}
\newcommand{\unfdn}{_{\mathrm{unf}}}
\newcommand{\xini}{\mbox{$x_{\mathrm{ini}}$}}
\newcommand{\vtjet}{\mbox{$v_2^{\mathrm{jet}}$}}
\newcommand{\vtjetmeas}{\mbox{${v_2^{\mathrm{jet}}|_{\mathrm{meas}}}$}}
\newcommand{\vtwo}{\mbox{$v_2$}}
\newcommand{\Rdz}{\mbox{$R_{D(z)}$}}
\newcommand{\Rdpt}{\mbox{$R_{D(\pT)}$}}
\newcommand{\Rdzsub}{\mbox{$R_{D(z)}^{\mathrm{sub}}$}}
\newcommand{\Rdptsub}{\mbox{$R_{D(\pT)}^{\mathrm{sub}}$}}
\newcommand{\Psit}{\mbox{$\Psi_2$}}

\newcommand{\Psires}{\mbox{$\mathrm{Res}\{\Psi_2\}$}}
\newcommand{\Rpsi}{\mbox{$R_{\Delta \phi}$}}
\newcommand{\diff}{\mathrm{d}}
\newcommand{\dpsi}{\mbox{$\Delta\phi$}}
\newcommand{\dNdpTdpsi}{\mbox{$\diff^2\Njet/\diff\pt\diff\dpsi$}}
\newcommand{\dNdpTdpsiRaw}{\mbox{$\dfrac{\diff^2N_{\mathrm{jet}}^{\mathrm{raw}}}{\diff\pt\diff\dpsi}$}}
\newcommand{\dNdpTdpsiCorr}{\mbox{$\dfrac{\diff^2N_{\mathrm{jet}}^{\mathrm{corr}}}{\diff\pt\diff\dpsi}$}}

\newcommand{\ETjet}{\mbox{$\ET^{\mathrm{jet}}$}}
\newcommand{\pbarp}{\mbox{$p+\bar{p}$}}
\newcommand{\pPb}{\mbox{$p$+Pb}}
%\newcommand{\pt}{\mbox{$p_T$}}
\newcommand{\ptjet}{\mbox{$p_{\mathrm{T}}^{\mathrm{jet}}$}}
%\newcommand{\pt}{\mbox{$p_{\mathrm{T}}$}}
%\newcommand{\etatrk}{\mbox{$\eta^{\mathrm{trk}}$}}
\newcommand{\etatrk}{\mbox{$\eta^{\mathrm{ch}}$}}
\newcommand{\ptpart}{\mbox{$p_{\mathrm{T}}^{\mathrm{part}}$}}
\newcommand{\pttruth}{\mbox{$p_{\mathrm{T}}^{\mathrm{reco}}$}}
\newcommand{\ptreco}{\mbox{$p_{\mathrm{T}}^{\mathrm{truth}}$}}
%\newcommand{\etapart}{\mbox{$\eta^{part}$}}
\newcommand{\ystar}{\mbox{$y^{*}_{\mathrm{jet}}$}}
\newcommand{\yjet}{\mbox{$y^{\mathrm{jet}}$}}
\newcommand{\Lres}{\mbox{$L_{\mathrm{res}}$}}
\newcommand{\Rres}{\mbox{$R_{\mathrm{res}}$}}
\newcommand{\dndeta}{\mbox{$1/\Nevt \, d\Nch/d\eta$}}
\newcommand{\RpPb}{\mbox{$R_{p\mathrm{Pb}}$}}
\newcommand{\Jt}{\mbox{$j_{\mathrm{T}}$}}
\newcommand{\jt}{\mbox{$j_{\mathrm{T}}$}}
\newcommand{\xt}{\mbox{$x_{\mathrm{T}}$}}
\newcommand{\ETtrue}{\mbox{$E_{\mathrm{T}}^{\mathrm{truth}}$}}
\newcommand{\ETreco}{\mbox{$E_{\mathrm{T}}^{\mathrm{reco}}$}}
\newcommand{\RSix}{\mbox{$R= 0.6$}}
\newcommand{\EffJet}{\mbox{$\varepsilon_{\mathrm{jet}}$}}
\newcommand{\Aj}{\mbox{$A_J$}}
\newcommand{\dphi}{$\Delta \phi$}
\newcommand{\p}{\partial}
\newcommand{\rvar}{\mbox{$r$}}
\renewcommand{\_}{{\tt \char`\_}}  % works properly only in \tt mode (!)
\newcommand{\z}{\mbox{$z$}}
\newcommand{\zunfolded}{\mbox{$z_{\mathrm{unfolded}}$}}
\newcommand{\zreco}{\mbox{$z_{\mathrm{reco}}$}}
\newcommand{\ztruth}{\mbox{$z_{\mathrm{truth}}$}}
\newcommand{\Rdzmeas}{\mbox{$R_{D(z)}^{\mathrm{meas}}$}}
\newcommand{\Rdptmeas}{\mbox{$R_{D(\pT)}^{\mathrm{meas}}$}}
\newcommand{\ptjettruth}{\mbox{$p_{{\mathrm{T}}}^{\mathrm{jet,truth}}$}}
\newcommand{\ptjetreco}{\mbox{$p_{{\mathrm{T}}}^{\mathrm{jet,reco}}$}}
\newcommand{\ptjetunfolded}{\mbox{$p_{{\mathrm{T}}}^{\mathrm{jet,unfolded}}$}}
\newcommand{\fd}{\mathrm{d}}
\newcommand{\cnchUE}{\mbox{$\tilde{n}_{\mathrm{ch}}^{\mathrm{UE+fake}}$}}

%\newcommand{\Dzunf}{\mbox{$D^{\mathrm{unfolded}}(z)$}}
%\newcommand{\Dptunf}{\mbox{$D^{\mathrm{unfolded}}(\pT)$}}
\newcommand{\Dzunf}{\mbox{$\frac{\diff\Nch}{\diff z}$}}
\newcommand{\Dptunf}{\mbox{$\frac{\diff\Nch}{\diff \pt}$}}

\newcommand{\GeV}{\mbox{GeV}}
\newcommand{\TeV}{\mbox{TeV}}
\newcommand{\dsigma}{\mbox{$\delta \sigma$}}
\newcommand{\hi}{\mathrm{HI}}
\newcommand{\emt}{\text{EMTopo}}
\newcommand{\tru}{\mathrm{Truth}}
\newcommand{\insitu}{\textit{in situ}}





\graphicspath{{logos/}{figures/}}
\addbibresource{thesis-ex.bib}
%\addbibresource{bibtex/bib/ATLAS.bib}
%\addbibresource{bibtex/bib/CMS.bib}
%\addbibresource{bibtex/bib/ConfNotes.bib}
%\addbibresource{bibtex/bib/PubNotes.bib}


\includeonly{   % Of course this list allows many more file
%  intro,       % should also work with files in different paths
%  chapter1,
   thesisintro.tex,
%  chapter3
}

\begin{document}

\title{Measurement of angular and momentum distributions of charged particles within and around jets in P\MakeLowercase{b}+P\MakeLowercase{b} and \MakeLowercase{$pp$} collisions at $\sqrt{s_{\mathrm{NN}}}=$~5.02~T\MakeLowercase{e}V with ATLAS at the LHC}
\author{Akshat Puri}
\department{Physics}
\schools{B.Sc., State University of New York At Stony Brook, 2014}
\phdthesis
\advisor{Anne Marie Sickles}
\degreeyear{2019}
\committee{Professor Matthias Grosse Perdekamp, Chair\\Professor Anne Marie Sickles, Advisor\\ Professor Lance Cooper \\ Professor Bryce Gadaway}
\maketitle

\frontmatter

%% Create an abstract that can also be used for the ProQuest abstract.
%% Note that ProQuest truncates their abstracts at 350 words.
\setcounter{page}{2}
\begin{abstract}

Heavy ion collision experiments have been centered around studying the extreme state of matter formed in such collisions, the quark-gluon plasma.
There have been efforts to measure and characterize this state of matter for almost two decades, first at the Relativistic Heavy Ion Collider and subsequently at the Large Hadron Collider.
While there have been different approaches to study it, correlated particle showers called jets have found a special place as a probe of the QGP.
Arising from highly energetic collisions between partons, jets are formed early enough in heavy ion collisions that they experience the formation of the QGP and its evolution.
They are modified from what they would be in a vacuum, and studying these modifications can give insight into the properties of the QGP as well as the strong interaction.

Jet measurements can focus on a variety of observables like yields, momenta, or fragmentation patterns, each with its own limitations and advantages.
This thesis presents a measurement of the angular distribution of charged particles around the jet axis as measured by the ATLAS detector for \pbpb\ and \pp\ collisions with a center of mass energy of $\sqrt{s_{\mathrm{NN}}}=$~5.02~TeV.
Comparing the measurement in the two  systems shows that charged particles carrying a transverse momenta \pt\ of less than 4 GeV have a broader distribution in \pbpb\ collisions, while those with $\pt > 4$ GeV have a narrower distribution.
Furthermore, there is an enhancement for particles with $\pt < 4$ GeV in \pbpb\ collisions, with the enhancement increasing up to 2 for \mbox{$\rvar < 0.3$} from the jet axis, and remaining constant for \mbox{$0.3 < \rvar < 0.6$}.
Charged particles with $\pt\ > 4$ \GeV\ show a small enhancement in the jet core for $\rvar < 0.05$, and are increasingly suppressed up to 0.5 for \mbox{$\rvar < 0.3$}.
This depletion remains constant for \mbox{$0.3 < \rvar < 0.6$}.

\end{abstract}


%The Large Hadron Collider (LHC) at the European Center for Nuclear Research (CERN) has been at the forefront of research in the field of heavy ion physics, and has provided unprecedented access to study the Quark Gluon Plasma (QGP). It was built with the purpose of accelerating subatomic particles close to the speed of light and colliding them to study their underlying structure. Detectors around the LHC ring, the biggest of which are  ATLAS (A Toroidal LHC ApparatuS), CMS (Compact Muon Solenoid), ALICE (A Large Ion-Collider Experiment), and LHCb (LHC-Beauty), study these collisions and use the debris as a playground to verify and expand the ``Standard Model'' of particle physics. This thesis will focus on measurements of collisions involving heavy ions as measured by the ATLAS detector.

%Relativistic heavy ion collisions such as those at the LHC provide insight into the interactions between quarks and gluons. These fundamental building blocks of all matter interact via the strong force, the theoretical framework of which is described by Quantum Chromodynamics (QCD). This theory dictates that quarks and gluons are confined, i.e. locked together to form composite particles and cannot exist independently, making their study extremely difficult. Relativistic heavy ion collisions provide an extreme environment where nuclear matter can "melt" and form a deconfined medium that consists of free quarks and gluons. This state of matter, called the Quark Gluon Plasma (QGP) is what existed a few microseconds after the Big Bang, and is what eventually cooled and expanded to form the existing universe. It

%The quark-gluon plasma (see Refs.~\cite{Roland:2014jsa,Busza:2018rrf} for recent reviews) can be probed by jets, sprays of particles that come from hard scattering processes between the nucleons involved in the collision. These jets are produced early in the collision and interact with the QGP as they make their way to the detector. Studying the rates and characteristics of these jets in \pbpb\ collisions, and comparing them to similar quantities in \pp\ collisions can provide information on the properties of the QGP. In particular, studying the  fragmentation pattern of these jets and how the energy is distributed around the jet axis can provide more information on the jet structure and put constraints on the medium response to the jet.

%This thesis is split into 4 main chapters. Chapter~\ref{sec:theory} briefly describes the general theoretical background on QCD, heavy ion collisions, QGP, and jets, giving context to the measurements discussed in this thesis. Chapter~\ref{sec:jetMeasurements} will briefly discuss major jet measurements done by the ATLAS Heavy Ion Group. Chapter~\ref{sec:setup} gives an overview of the LHC and the ATLAS detector. Chapter~\ref{sec:qualification} will describe the work undertaken to become a member of the ATLAS Collaboration, and Chapter~\ref{sec:mainanalysis} will provide a detailed description of the measurement to determine the angular distributions of charged particles in \pbpb\ and \pp\ collisions. 





% Create a dedication in italics with no heading, centered vertically
% on the page.

\begin{dedication}
For my Mother, Father, and Brother
\end{dedication}

% Create an Acknowledgements page, many departments require you to
% include funding support in this.
% \chapter*{Acknowledgments}

% The thesis format requires the Table of Contents to come
% before any other major sections, all of these sections after
% the Table of Contents must be listed therein (i.e., use \chapter,
% not \chapter*).  Common sections to have between the Table of
% Contents and the main text are:
%
% List of Tables
% List of Figures
% List Symbols and/or Abbreviations
% etc.


%\chapter*{Acknowledgments}
%% !TEX encoding = UTF-8 Unicode
% !TEX root = thesis-ex.tex

This thesis would not have been possible without a large number of people.
First and foremost, I would like to thank my advisor, Professor Anne Sickles.
Her encouragement and support helped me grow not only as a researcher but also as a person.
She gave me numerous opportunities to present my research at a variety of conferences and workshops, and trusted me with a variety of responsibilities including deciding where we would go for group lunch.
Our discussions that ranged from Physics to food made my time in graduate school extremely impactful and enjoyable. 

I would like to extend a big thank you to Martin Rybar, the post doctoral researcher in our group.
His unwavering support not only in my research, but also as a friend was invaluable.
Whether it was discussing how the ATLAS detector works or playing Red Alert and getting a beer after work, he was a huge positive presence in graduate school.

Other people I wish to thank include
\begin{itemize}
\item My research group members Michael Phipps, Yakov Kulinich, Sebastian Tapia, Virginia Bailey, Tim Rinn, Anabel Romero and Xiaoning Wang for sitting through numerous meetings and practice talks and for providing feedback on my work.

\item My doctoral committee members Matthias Grosse Perdekamp, Bryce Gadway, Lance Cooper, and Aida X El-Khadra.

\item The Heavy Ion Jet Group, in particular Martin Spousta, Aaron Angerami, Dennis Perepelitsa, Peter Steinberg and Brian Cole for a variety of discussions that broadened my physics knowledge.

\item The ATLAS Collaboration and the large team of people including Iwona Grabowska-Bold, Benjamin Nachman, Mario Martinez, Thomas Le Compte, Filippo Ceradini, Matteo Bauce, Gideon Bella, Maxwell Charles Baugh, Evgenii Baldin, Abi Soffer, Yuya Kano, Yann Coadou, Jiangyong Jia, Guillaume Unal and David Stoker, who read numerous drafts of my paper and provided crucial feedback.
\end{itemize}


Outside of my research, I would like to extend gratitude towards my ballroom dance team.
I joined the team when I first started in graduate school, and it very quickly became the center of my social life.
I would like to especially thank my dance partners Anna Kalinowski and Bridget Regan for the dances that let me get away from work, and for their patience when I was not available for practice because of my research.
I would also be remiss if I did not mention my coaches over the years: Kato Lindholm, Alex Tecza, Kirsti Roslender and Peter Minkov.

I would also like to acknowledge the love and support from the friends I have made over the past five years, including but certainly not limited to Parul Agrawal, John Hadley, David Meldgin, Varun Badrinath, Anna Vardanyan, Katie Bolan, Timothy Chin and Yana Garmash.

And last but certainly not least, I have to thank my parents and my brother. None of this would have been possible without them cheering me on and supporting me all my life. Their unconditional support and their pride in my work was humbling, and this is for them. \\ \\ \\ 



\epigraph{If I have seen further it is by standing on the shoulders of Giants}{\textit{Sir Isaac Newton}}

%
%\begin{dedication}
%To Anna Glazatova
%\end{dedication}
%
\tableofcontents
%% \listoftables
%% \listoffigures
%
%%% Create a List of Abbreviations. The left column
%%% is 1 inch wide and left-justified
%\chapter{List of Abbreviations}
%\begin{symbollist*}
\item [LHC] Large Hadron Collider
\item [ATLAS] A Toroidal LHC Apparatus
\item [CERN] European Organization for Nuclear Research
\item [LS] Long Shutdown
\item [RF] Radio Frequency
\item [LINAC] Linear Accelerator
\item [ALICE] A Large Ion Collider Experiment
\item [CMS] Compact Muon Solenoid
\item [LHCb] LHC Beauty
\item [IP] Interaction Point
\item [ID] Inner Detector
\item [EM] Electromagnetic
\item [LAr] Liquid Argon
\item [EMB] EM Barrel
\item [EMEC] EM End-Cap
\item [FCal] Forward Calorimeter
\item [HEC] Hadronic End-Cap
\item [TileCal] Tile Calorimeter
\item [L1,L2] Level 1,2 Triggers
\item [HLT] High Level Trigger
\item [EF] Event Filtering
\item [QCD] Quantum Chromo-Dynamics
\item [IR] Infrared
\item [pQCD] Perturbative QCD
\item [QGP] Quark Gluon Plasma
\item [DIS] Deep Inelastic Scattering
\item [$ep$] Electron-Proton Collisions
\item [$p+\mathrm{Au}$] Proton-Gold Collisions
\item [\pp] Proton-Proton Collisions
\item [\pPb] Proton-Lead Collisions
\item [HERA] Hadron-Electron Ring Accelerator
\item [RHIC] Relativistic Heavy Ion Collider
\item [RMS] Root-mean-square, synonym for standard deviation
\item [MC] Monte-Carlo
\item [GRL] Good Runs List
\item [UE] Underlying Event
\item [JER] Jet Energy Resolution
\item [JES] Jet Energy Scale
\item [JAR] Jet Angular Resolution
\end{symbollist*}

%
%%% Create a List of Symbols. The left column
%%% is 0.7 inch wide and centered
%\chapter{List of Symbols}
%\begin{symbollist*}
\item[$z$] Cartesian coordinate defined to be in the direction of the LHC beam direction.
\item[$x-y$] Cartesian coordinates defined to be perpendicular to the LHC beam direction.
\item[$(E, p_{x}, p_{y}, p_{z})$] Four-vector describing energy and Cartesian momentum components of an object.
\item[$p_{tot} = \sqrt{p_{x}^{2} + p_{y}^{2} + p_{z}^{2}}$] Total momentum of an object.
\item[$\phi=arcsin(p_{z}/\pt)$] Azimuthal angle in the $x-y$ plane perpendicular to the beam direction.
\item[$\theta = arccos(p_{z}/p_{tot})$] Polar angle with respect to the beam direction ($z$-axis).
\item[$\pt=\sqrt{p_{x}^{2}+p_{y}^{2}}$] Transverse momentum defined in the $x-y$ plane.
\item[$\eta=-\ln\tan(\theta/2)$] Pseudorapidity, defined in terms of the polar angle $\theta$.
\item[$y=\frac{1}{2}ln(\frac{E+p_{z}}{E-p_{z}})$] Rapidity, defined in terms of energy $E$.
\item[$\Delta R \equiv \sqrt{(\Delta\eta)^{2} + (\Delta\phi)^{2}}$] Measure of angular distance.
\item[$x$] Parton longitudinal momentum fraction of a nucleon.
\item[$\Dphi=\phi_{1}-\phi_{2}$] Difference in azimuthal angle $\phi$ between two objects.
\item[\conetwo] Azimuthal correlation between two jets normalized by the number of leading jets.
\item[\wonetwo] Width of the \conetwo\ distribution. Defined as the RMS or standard deviaton of a fit to \conetwo.
\item[\ionetwo] Conditional yield, extracted as the integral of a \conetwo\ distributon. \
\item[\cppb] Ratio of \wonetwo\ between \pp\ and \pPb\ collisions.
\item[\ippb] Ratio of \ionetwo\ between \pp\ and \pPb\ collisions.
\end{symbollist*}


\mainmatter
%-------------------------------------------------------------------------------

%\chapter{Introduction}
%\label{sec:intro}
%% !TEX encoding = UTF-8 Unicode
% !TEX root = thesis-ex.tex


The Large Hadron Collider (LHC) at the European Center for Nuclear Research (CERN), is one of the worlds most expensive and complicated machines. It was built with the purpose of accelerating subatomic particles to close to the speed of light and colliding them to study their underlying structure. Detectors around the LHC ring, the biggest of which are  ATLAS (A Toroidal LHC ApparatuS), CMS (Compact Muon Solenoid), ALICE (A Large Ion-Collider Experiment), and LHCb (LHC-Beauty), study these collisions and use the debris as a playground to verify and expand the "Standard Model" of particle physics. This thesis will focus on measurements of collisions involving heavy ions as measured by the ATLAS detector.

Relativistic heavy ion collisions such as those at the LHC provide insight into the interactions between quarks and gluons. These fundamental building blocks of all matter interact via the strong force, the theoretical framework of which is described by Quantum Chromodynamics (QCD). This theory dictates that quarks and gluons are confined, i.e. locked together to form composite particles and cannot exist independently, making their study extremely difficult. Relativistic heavy ion collisions provide an extreme environment where nuclear matter can "melt" and form a deconfined medium that consists of free quarks and gluons. This state of matter, called the Quark Gluon Plasma (QGP) is what existed a few microseconds after the Big Bang, and is what eventually cooled and expanded to form the existing universe. It

The quark-gluon plasma (see Refs.~\cite{Roland:2014jsa,Busza:2018rrf} for recent reviews) can be probed by jets, sprays of particles that come from hard scattering processes between the nucleons involved in the collision. These jets are produced early in the collision and interact with the QGP as they make their way to the detector. Studying the rates and characteristics of these jets in \pbpb\ collisions, and comparing them to similar quantities in \pp\ collisions can provide information on the properties of the QGP. In particular, studying the  fragmentation pattern of these jets and how the energy is distributed around the jet axis can provide more information on the jet structure and put constraints on the medium response to the jet.

This thesis is split into 4 main chapters. Chapter~\ref{sec:theory} briefly describes the general theoretical background on QCD, heavy ion collisions, QGP, and jets, giving context to the measurements discussed in this thesis. Chapter~\ref{sec:jetMeasurements} will briefly discuss major jet measurements done by the ATLAS Heavy Ion Group. Chapter~\ref{sec:setup} gives an overview of the LHC and the ATLAS detector. Chapter~\ref{sec:qualification} will describe the work undertaken to become a member of the ATLAS Collaboration, and Chapter~\ref{sec:mainanalysis} will provide a detailed description of the measurement to determine the angular distributions of charged particles in \pbpb\ and \pp\ collisions. 





%
%Jets with large transverse momenta are observed to be produced in central lead-lead (\pbpb) collisions at the LHC at a rate that is reduced by a factor of two with respect to the expectation from these cross sections measured in \pp\ interactions, re-scaled by the nuclear overlap function of \pbpb\ collisions~\cite{Abelev:2013kqa,Aad:2014bxa,Khachatryan:2016jfl}. 
%%The rates of jet production are observed to be reduced by  approximately a factor of two in lead-lead~(\PbPb) collisions at LHC energies compared to  expectations from the jet production cross sections measured in \pp\ interactions scaled by the nuclear overlap function of \PbPb\ collisions~\cite{Abelev:2013kqa,Aad:2014bxa,Khachatryan:2016jfl}. 
%%This reduction is termed ``jet-quenching'' and is due to the interaction of
%%constituents of the parton shower with the QGP.  
%Similarly, back-to-back dijet~\cite{Aad:2010bu,Chatrchyan:2011sx,Aaboud:2017eww} 
%and photon-jet pairs~\cite{Chatrchyan:2012gt} are observed to have
%unbalanced transverse momenta in \pbpb\ collisions compared to \pp\ collisions.
%These observations suggest that some of the energy from the hard-scattered parton is
%transferred outside of the jet through its interaction with the QGP.  
%
%Also of interest are measurements sensitive to the distributions of particles
%within the jet.  Measurements of the jet shape~\cite{Chatrchyan:2013kwa} and  the longitudinal fragmentation functions~\cite{Aad:2014wha,Chatrchyan:2014ava,Aaboud:2017bzv} were performed in 2.76~\TeV\ \pbpb\
%collisions.
%These measurements show an excess of both low and high momentum particles inside the jet compared to \pp\ collisions.
%Particles carrying a large fraction of the jet momentum are generally closely
%aligned with the jet axis, whereas low momentum particles can have a much broader
%angular distribution extending outside the jet \cite{Khachatryan:2016tfj,Sirunyan:2018jqr}. 
%Fragmentation function measurements have shown that particles with transverse momentum, \pT,
%less than 4~\GeV\
%are enhanced in \pbpb\ collisions compared to \pp\ collisions~\cite{Aaboud:2017bzv}.
%These observations suggest that the energy lost by jets through the jet-quenching process is being transferred to soft particles within and around the jet~\cite{Qin:2015srf,Blaizot:2014ula}. Measurements of yields of these particles as a function of transverse momentum and
%distance between the particle and the jet axis have a potential to constrain
%the models of jet energy loss processes in \pbpb\ collisions.
%
%This note presents a measurement of charged particle \pt\ distributions inside and around jets. The measured yields are defined as\footnote{ATLAS uses a right-handed coordinate system with its origin at the nominal interaction point (IP) in the center of the detector, and the $z$-axis along the beam pipe. The $x$-axis points from the IP to the center of the LHC ring, and the $y$-axis points upward. Cylindrical coordinates $(r, \phi)$ are used the transverse plane, $\phi$ being the azimuthal angle around the $z$-axis. The pseudorapidity is defined in terms of the polar angle $\theta$ as $\eta = - \text{ln} \tan (\theta/2)$. Transverse momentum and transverse energy are defined as $\pt = p \sin\theta$ and $\Et = E \sin\theta$, respectively. $\Delta R = \sqrt{(\Delta \eta )^2 + (\Delta \phi)^2}$ gives the angular distance between two objects with relative differences $\Delta \eta$ and $\Delta \phi$ in pseudorapidity and azimuth respectively.}:
%  \begin{equation}
%\Dptr = \frac{1}{N_{\mathrm{jet}}} \frac{1}{2\pi r  } \frac{\fd^{2} n_{\mathrm{ch}} (r)}{\fd r \fd \pt},
%%D(\pt,\ptjet) = \frac{1}{N_{\mathrm{jet}}} ~ \frac{1}{\epsilon(\pttrk)} ~ \frac{\mathrm{d} N_{\mathrm{ch}}}{\mathrm{d} \pt}~(\ptjet).
%\end{equation}
%where $N_{\mathrm{jet}}$ is the total number of jets; $2\pi r \text{d}r$ is the area of the annulus at a given distance $r$ from the jet axis, where $r = \sqrt{\Delta \eta^2 + \Delta \phi^2}$ ($\Delta \eta$ and $\Delta \phi$ are the relative differences between the charged particle and the jet axis, in pseudorapidity and azimuth respectively) and $\fd r$ is the width of the annulus; $n_{\mathrm{ch}}(r)$ is the number of charged particles within a given annulus. The ratios of the charged-particle yields measured in \pbpb\ and \pp\ collisions,
%\begin{equation}
%   \RDptr = \frac{\Dptr_\mathrm{Pb+Pb}}{\Dptr_{pp}}
%   \label{eq:rdptr}
%\end{equation}
%%are evaluated to quantify the modifications in \pbpb\ collisions compared to the measurement in \pp\ collisions.
%allow evaluating the differences between the two yields. 
%
%The \RDptr\ distributions are measured using 0.49~nb$^{-1}$ of \pbpb\ collisions and 
%25~pb$^{-1}$ of \pp\ collisions at center-of-mass energy of 5.02~\TeV\ collected in 2015 by ATLAS.
%Jets are reconstructed with the \antikt\ algorithm~\cite{Cacciari:2008gp} using a radius parameter \RFour\ over a rapidity range of $|\yjet| <$~1.3. The measurement is presented for jets with transverse momenta (\ptjet) in the 126 to 316~\GeV\ range, for charged particles with $\pT>1.6$~\GeV\ and eight successive intervals of angular distance $r$ with the following edges: 0.0, 0.05, 0.1, 0.15, 0.2, 0.25, 0.3, 0.4, 0.5, and 0.6.
%
%


%%%The fundamental properties of the matter surrounding us have always been of great interest to humankind. The word atom dates back to ancient Greece, and the electron, a fundamental particle that plays an important role in everyday life was discovered just 125 years ago by J.J Thompson. In recent years, technology has allowed us to probe microscopic distances and study matter at an unprecedented level. To this day, many new breakthroughs in the understanding of microscopic and macroscopic properties of matter have been made.
%%%
%%%The LHC, a particle collider in CERN, Switzerland, is currently the worlds most powerful machine for probing the properties of known matter and carrying out searches for new forms of matter. It has contributed to the recent discovery of the Higgs boson and to an improved understanding of physics at high energies. The ATLAS detector is one of the largest instruments that measures collisions at the LHC and is the product of thousands of collaborators from hundreds of institutions from around the world. The author of this thesis is a member of the ATLAS collaboration, and had the privilege to use this wonderful machine to conduct the study which will be presented in this thesis.
%%%
%%%One of the fundamental building blocks of matter surrounding us is the proton, which like the electron, is a well known particle to most readers. The properties and structure of the proton have attracted a lot of attention over the years. While many of its macroscopic properties such as its mass, charge, and lifetime are known to a precise degree, there remain many unanswered questions about its microscopic properties. This dissertation will present a measurement probing into one of these unanswered questions - the behavior of subatomic particles called $partons$ at different energy regimes inside of the proton. More specifically, the measurement will focus on studying a parton called the $gluon$, which is a particle that binds together partons called $quarks$. These quarks and gluons, and the interactions between them, are currently described by a globally recognized model called the Standard Model. The system of there quarks, held together by three gluons, describes the simplest picture of the gluon. We will look at a more complex picture of the proton, where present measurements are not able to explain the observation that there is an unrealistically large (tending to infinity) amount of gluons seen in the proton at shorter timescales. This unphysical process has to stop at some point, and this is described by a phenomenon called $saturation$.
%%%
%%%This dissertation is split into four chapters. Chapter~\ref{sec:setup} describes the experimental apparatus used throughout this measurement. Chapter~\ref{sec:intro} gives a theoretical background that should help the reader understand the measurement that will be presented in this thesis. Chapter~\ref{sec:qualification} presents a brief overview of the qualification work completed as a requirement for becoming a member of the ATLAS collaboration. Finally, Chapter~\ref{sec:mainanalysis} presents a detailed outline of the measurement along with its results.
%%%
%%%In addition to carrying out this analysis into the structure of the proton. The author of this dissertation contributed to the commissioning of a large area drift chamber for the COMPASS experiment at CERN. The contributions included parts procurement, assembly, testing, and data acquisition for the detector. The author also contributed to the simulation work, assembly, and data taking at beam tests for new ATLAS zero degree calorimeter (ZDC) prototype. 
%%%
%%%I hope that you learn from, and enjoy reading this dissertation. Thank you.


\chapter{Introduction}
\label{sec:theory}
% !TEX root = thesis-ex.tex
This section shall discuss the theoretical background necessary to understand jet measurements. It will discuss the fundamentals of quantum chromodynamics (QCD), the heavy ion collision system and the quark gluon plasma that is formed, and finally jets and jet energy loss. 

\section{Quantum Chromodynamics}
\label{sec:qcd}
% !TEX root = thesis-ex.tex
Quantum Chromodynamics is a gauge theory with SU(3) symmetry that describes the dynamics of the strong interactions between quarks and gluons.
It is part of the Standard Model \cite{Gaillard:1998ui}, the building blocks of which are shown in Figure~\ref{fig:sm_particles}.

%The Standard Model (SM) \cite{Gaillard:1998ui} describes the interactions between elementary particles that are listed in Figure~\ref{fig:sm_particles}.It is one the most successful theories in physics and describes three of the four fundamental forces of nature.These are the strong interaction, the weak interaction, and the electromagnetic interaction.A quantum theory for gravity is not part of the SM.

\begin{figure}[htbp]
\begin{center}
\includegraphics[width=0.5\textwidth]{figures/theory/SM}
\caption{The elementary particles of the standard model.
Figure from Ref.~\cite{SMpict}.}
\label{fig:sm_particles}
\end{center}
\end{figure}

%Within the SM, the dynamics of the strong interactions involving quarks and gluons are described by Quantum Chromodynamics (QCD), 

Quarks are fermions with a spin of $1/2$, and carry a fractional electric charge as well as a color charge.
They all have mass and come in six flavors: up, down, strange, charm, top, bottom.
The lightest quarks (u and d) combine and form stable particles, while the heavier quarks can only be produced in energetic environments and decay rapidly.
Gluons are gauge bosons (force carriers) with a spin of $1$, and are what hold quarks together.
The dynamics of the quarks and gluons, collectively referred to as partons, are described by the QCD Lagrangian given by \cite{Beringer:1481544}:

\begin{align}
\mathcal{{L}}_{\mathrm{QCD}} = \sum_q \bar{\psi}_{q,a} (i \gamma^\mu \partial_\mu \delta_{ab} - g_s \gamma^\mu t_{ab}^C \mathcal{A}_\mu^C - m_q \delta_{ab}) \psi_{q,b} - \frac{1}{4} F_{\mu\nu}^A F^{A \mu\nu}
\end{align}
where $\psi_{q,a}$ and $\psi_{q,b}$ are quark-field spinors for a quarks with flavor $q$, mass $m_q$, and color $a$ and $b$ respectively, with the values for $a$ and $b$ ranging  from 1 to 3 (for the three colors).
The $\mathcal{A_\mu^C}$ corresponds to the gluon field with $C$ taking values from 1 through 8 (for the 8 types of gluons).
The $t_{ab}^C$ corresponds to the Gell-Mann matrices that are the generators of the SU(3) group, and dictate the rotation of the quarks color in SU(3) space when it interacts with a gluon.
The coupling constant is encoded within $g_s$, which is defined by $g_s \equiv \sqrt{4 \pi \alpha_s}$.
The field tensor $F_{\mu\nu}^A$ can be written in terms of the structure constants of the SU(3) group $f_{ABC}$, and is given by:

\begin{align}
F_{\mu\nu}^A = \partial_\mu \mathcal{A}_\nu^A - \partial_\nu \mathcal{A}_\mu^A - g_s f_{ABC} \mathcal{A}^B \mathcal{A}^C
\end{align}
While many parallels can be drawn between Quantum Electrodynamics (QED, the theory that describes photons and electrons) and QCD, the main difference between the two comes from the gluon-gluon interactions allowed in QCD, making it non-Abelian.
These interactions can be summarized as shown in Figure~\ref{fig:qcd_diagrams}.

\begin{figure}[htbp]
\begin{center}
\includegraphics[width=0.6\textwidth]{figures/theory/qcd_diagrams}
\caption{The allowed vertices in QCD.
The vertices involving three or four gluons are unique to QCD and do not have a QED analog.}
\label{fig:qcd_diagrams}
\end{center}
\end{figure}

A core feature of QCD is that the coupling constant \alphas\ has an energy dependence shown in Figure~\ref{fig:running_coupling}.
This dependence can be expressed in terms of the $\beta$ function as

\begin{align}
Q^2 \frac{\partial \alpha_s (Q^2)}{\partial Q^2} = \beta(\alpha_s (Q^2))
\end{align}
where $Q$ is the momentum transfer in the particle reaction
\footnote{The momentum transfer $Q$ is the amount of momentum transferred in a scattering process.}.
The beta function can be expressed using perturbative QCD (pQCD) as:

\begin{align}
\beta( \alpha_s ) = - (b_0 \alpha_s^2 + b_1 \alpha_s^3 + b_2 \alpha_s^4...)
\end{align}
where the coefficients $b_i$ depend on the number of colors and flavors.
This running coupling constant is small and asymptotically tends to zero at large energy scales (or at small distances) and is large at small energy scales (large distances).
This running coupling phenomenon leads to two key behaviors: asymptotic freedom and color confinement.

\begin{figure}[htbp]
\begin{center}
\includegraphics[width=0.5\textwidth]{figures/theory/running_coupling}
\caption{The running coupling constant $\alpha_s$ as a function of the momentum transfer $Q$.
Figure from Ref.~\cite{Beringer:1481544}.}
\label{fig:running_coupling}
\end{center}
\end{figure}



\subparagraph{Asymptotic Freedom: }
At high energy scales (small distances), the QCD coupling constant $\alpha_s$ is small and tends to zero, implying a free particle behavior of quarks and gluons \cite{PhysRevLett.30.1343, PhysRevD.8.3633}.
This has been observed by a variety of deep inelastic scattering (DIS) experiments \cite{Deur:2014vea, Kim:1998kia, Altarelli:1996nm, RevModPhys.63.597, Kataev:2001kk, Alekhin:2012ig, Alekhin:2013nua, Blumlein:2006be, Aaron:2007xx, Chekanov:2007pa, Chekanov:2008af, Abramowicz:2010cka, Abramowicz:2010ke, Aaron:2009vs}.
These scattering experiments shown in Figure~\ref{fig:dis_schematic}, probe the interior of a nucleon using highly energetic leptons like electrons.
The electron scatters off of the target proton, producing a lepton and a hadron shower.
First done by MIT-SLAC \cite{PhysRevLett.23.930, PhysRevLett.23.935}, these DIS experiments showed the weak $Q^2$ dependence on the inelastic scattering cross-sections, as well as Bjorken scaling \cite{PhysRev.179.1547}, where the proton structure functions are independent of the momentum transfer.
These experiments revealed the point-like constituents of the proton and paved the road to an asymptotically free theory.

\begin{figure}[htbp]
\begin{center}
\includegraphics[width=0.35\textwidth]{figures/theory/DIS}
\caption{Schematic of the deep inelastic scattering experiment.}
\label{fig:dis_schematic}
\end{center}
\end{figure}


\subparagraph{Color Confinement}
The opposite end of the running coupling constant phenomenon is color confinement.
Proved to be a consequence of asymptotic freedom in Ref.~\cite{Nishijima1996}, this property of QCD described in Ref.~\cite{PhysRevD.10.2445} forbids the direct observation of free quarks and gluons, allowing only for composite particles that are color singlets.
While there have been numerous efforts to understand the source of this phenomenon like in Refs.~\cite{BUCHMULLER1982479, KOGUT1976199, PhysRevD.31.2910, PhysRevD.57.2603, PhysRevD.62.114503, RevModPhys.55.775, PhysRevLett.90.102001}, these are based on numerical calculations.
An analytic proof of color confinement still escapes description and in fact, is one of the Millennium Problems \cite{MillenniumProb}.


%\subsection{Parton Distribution Functions}
%Deep ineslatic scattering experiments showed the internal point-like structure of a proton that can be described in terms of probability density functions called parton distribution functions.
%These are written in terms of the longitudinal momentum fraction carried by a constitutent proton, $x$, given as:
%
%\begin{align}
%x = \frac{Q^2}{2 P \dot q} = \frac{Q^2}{2 M \nu}
%\end{align}
%where $\nu$ is the energy of the photon shown in Figure~\ref{fig:dis_schematic}. This can be determined by 




%An analytic proof of the origin of color confinement is described in Ref.\cite{Gao:2018xsg}.
%==========================================================
%
%%QCD allows for predicting a variety of observables in particle reactions involving quarks and gluons in terms of the coupling constant $\alpha_s$.
%QCD amplitudes can be calculated by using the matrix elements for the basic diagrams shown in Figure~\ref{fig:qcd_diagrams} along with the quark and gluon propagators.Perturbative calculations done with the assumption that the coupling constant \alphas $ < 1 $ allow for 
%Another major feature 
% quark and gluon propagators along with teh corre
%
%
%

%The classical QCD Lagrangian is given by \cite{Chyla:2004zz}
%
%\begin{align}
%\mathcal{{L}}_{\mathrm{QCD}} = \sum_q \bar{\psi}_{q,a} (i \gamma^\mu \partial_\mu \delta_{ab} - g_s \gamma^\mu t_{ab}^C \mathcal{A}_\mu^C - m_q \delta_{ab}) \psi_{q,b} - \frac{1}{4} F_{\mu\nu}^A F^{A \mu\nu}
%\end{align}
%
%%\begin{align}
%%\mathcal{{L}}_{\mathrm{QCD}} &= \mathcal{L}_g + \mathcal{L}_q \\
%%&=  -\frac{1}{4} F_{\mu\nu} F^{\mu\nu} + \bar{\Psi}(i  \slashed{\partial} - m_q) \Psi + g \bar{\Psi} \gamma_\mu T \Psi A^\mu
%%\end{align}
%
%\begin{align}
%\mathcal{{L}}_{\mathrm{QCD}} =  -\frac{1}{4} F_{\mu\nu} F^{\mu\nu} + \bar{\Psi}(i  \slashed{\partial} - m_q) \Psi + g \bar{\Psi} \gamma_\mu T \Psi A^\mu
%\end{align}
%
%\begin{align}
%F^{\mu\nu}_a (x) \equiv \frac{\partial A_a^\nu (x)}{\partial x_\mu} - \frac{\partial A_a^\mu (x)}{\partial x_\nu}
%\end{align}
%
%%This Lagrangian is can be thought of as the sum of the quark and gluon terms $\mathcal{L}_g$ and $\mathcal{L}_q$, where
%%\begin{align}
%%\mathcal{L}_g &= - \frac{1}{4}\frac{1}{4} F_{\mu\nu} F^{\mu\nu} 
%%\end{align}
%
%
%where $Q$ is the momentum transfer in the particle reaction.This running coupling constant is small and asymptotically tends to zero at large energy scales (or at small distances) and is large at small energy scales (large distances).This can be seen in Figure \ref{fig:running_coupling}, which shows the $\alpha_s$ dependence on $Q$.This running coupling leads to two key behaviors discussed below.
%
%
%
%


\section{Heavy Ion Collisions and the Quark Gluon Plasma}
\label{sec:HICollisions}
% !TEX root = thesis-ex.tex

%The quark gluon plasma is a strongly coupled medium \cite{} that is produced in a heavy ion collision. This section will briefly describe the heavy ion environment it is formed in and subsequently the medium itself. 
Heavy ion collisions were suggested in Reference \cite{SHURYAK198071} as a tool to study the Quark Gluon Plasma. They provide access to the otherwise confined partons, and give insight into the QCD phase diagram and the transition between the QGP and hadronic matter. This section will briefly discuss a heavy ion collision and the properties of the medium that is formed in such a collision.  

\subsection{Heavy Ion Collisions}
In a heavy ion collision, the colliding nuclei are Lorentz contracted discs. In the case of a \pbpb\ collision, the nuclei have been accelerated to energies where the relativistic $\gamma$ factor is between 100 and 2500 for beam rapidities of $y = 5.3$ and 8.5. Each nucleus contains many colored quarks and antiquarks, with three more quarks than anti-quarks per nucleon, with the $q\bar{q}$ popping in and out of the vacuum due to quantum fluctuations. These $q\bar{q}$ pairs are sources of transverse color fields and the corresponding force carriers, the gluons.

When these pancake like discs collide, their color fields interact and there is a color charge exchange, producing longitudinal color fields that fill the space between the receding discs. While the maximum energy density in the process occurs just at the collision, the energy density 1 fm/c after the collision is 12 $\mathrm{GeV} / \mathrm{fm}^3$, much higher than the 500 $\mathrm{MeV} / \mathrm{fm}^3$ in a typical hadron. Lattice QCD calculations in thermodynamics show that at these energies, the partons produced in the collision cannot be treated as a collection of distinct hadrons. In fact, these partons are strongly coupled to each other and form a medium called the Quark Gluon Plasma (QGP) \cite{???}. 

%In a heavy ion collision, the experimenter can only tune the size of the colliding nuclei, and the energy that they are being collided at. There is no experimental control over the impact parameter or the structure functions that dictate the momentum distribution of nucleons within the nucleus. These have to be determined event by

% most of the partons are participate in soft interactions that do not involve large transverse momentum transfer, and are hence scattered only at small angles. A small fraction of the colliding partons however do undergo hard perturbative interactions and lead to particles with large transverse momenta.
%These subsequently decaying to $q\bar{q}$ pairs. 
%The QGP can be described by relativistic hydrodynamics, and has a viscosity to entropy ratio that is almost at the theoretical minimum of of $\eta / S = 1/4\pi$ \cite{5,6, check126}. 

After the collision the energy density between the receding nuclei starts to decrease as the QGP cools and expands. This process, seen in Figure~\ref{fig:qgp_form}, continues till the energy density drops to below that within a hadron and the fluid ``hadronizes''. These individual hadrons briefly scatter off of each other before they freely fly towards the detector (freeze-out).

%Once formed, the QGP flows hydrodynamically, with the initial pressure driving the expansion and the subsequent cooling. 
% It is to be noted that there is QGP continuously formed in the wake of the nuclei since the partons produced at large rapidities are highly relativistic and 

\begin{figure}[htbp]
\begin{center}
\includegraphics[width=0.85\textwidth]{figures/theory/qgp_formation}
\caption{(left) Space-time diagram for a heavy ion collision. The color is indicative of the temperature of the QGP formed. (right) Snapshots of a heavy ion collision at $\sqrtsnn = 2.76$ TeV at different times. The Lorentz contracted nuclei are in blue while the QGP is in red. Figures from References \cite{7, 8}.  }
\label{fig:qgp_form}
\end{center}
\end{figure}

While Figure~\ref{fig:qgp_form} shows snapshots of a head on (central) collision between two large nuclei, it is possible to have collisions where the impact parameter is larger and hence the overlap region is smaller. These collisions, called peripheral collisions, qualitatively undergo the same process described above, with the size and shape of the QGP being different.

Basic parameters of a heavy ion collision such as the number of participants \Npart and number of binary collisions \Ncoll can be determined using the Glauber Monte Carlo simulations \cite{doi:10.1146/annurev.nucl.57.090506.123020}. This technique  considers multiple scatterings of nucleons in nuclear targets by modeling the nucleus as a set of uncorrelated nucleons sampled from measured density distributions. Two nuclei are arranged with a random impact parameter and projected onto the $x-y$ plane as shown in Figure~\ref{fig:glauber}, with interaction probabilities being applied by using the relative distance between nucleon centroids as a proxy for the measured inelastic nucleon-nucleon cross section. 


\begin{figure}[htbp]
\begin{center}
\includegraphics[width=0.85\textwidth]{figures/theory/glauberMC}
\caption{A Glauber Monte Carlo event for $Au+Au$ at \sqrtsnn = 200 geV with impact parameter of 6 fm viewed in the (left) transverse plane and (right) along the beam axis. Darker circles represent the participating nucleons. Taken from \cite{doi:10.1146/annurev.nucl.57.090506.123020}. }
\label{fig:qgp_form}
\end{center}
\end{figure}


%In these collisions, the QGP formed is more lenticular in the transverse direction.
% Of course, the colliding nuclei are not perfectly smooth objects and are made of individual nucleons giving them a non-uniform structure. This results in the energy density in the overlap region being non-uniform with any variations giving rise to pressure gradients that cause azimuthal anisotropies in the momentum distribution of the produced particles. 

%The heavy ion collision system is an extraordinarily useful laboratory to study QCD because it gives access to the otherwise confined partons and provides for a way to study the phase transition between the QGP and ordinary hadronic matter. It is also able to replicate the conditions in the early universe, just after the Big Bang \cite{23, 24}, when it was too hot for hadrons to exist in the form that they do now. 

%We can differentiate different nucleons in the collision as per the following:
%\subparagraph{$\mathrm{N}_{\mathrm{part}}$: } This is the number nucleons that have collided with at least one other nucleon, and can be said to have participated in the heavy ion collision.
%\subparagraph{$\mathrm{N}_{\mathrm{coll}}$: } This is the number of binary collisions that take place between the nucleons of the colliding nuclei. It is typically much larger than \Npart.
%\subparagraph{$\mathrm{N}_{\mathrm{spec}}$: } This is the number nucleons that do not encounter any nucleon from the other nucleus and are just spectators to the collision. 

%The properties of the QGP can be determined by azimuthal correlation measurements \cite{5, 6, 90}, while how it interacts with a high energy parton can be determined by jet studies \cite{91, 92, 69, etc}. 


%%%%%%%%%%%%%%%%%%%%%%%%%%%%%%%%%%%%%%%%%%%%%%%%%%%%%%%%%%%%%%%%%%%%%%%%%

\subsection{The Quark Gluon Plasma}
\label{sec:qgp}
Quarks and gluons are deconfined at extremely high energy and density conditions and form a state called the Quark Gluon Plasma \cite{SHURYAK198071}. These conditions are met in high energy heavy ion collisions. 
The Quark Gluon Plasma has to be described in terms of its constituent quarks and gluons as opposed to the hadrons. This transition between confinement within hadrons and being free within the QGP occurs at very high temperatures and pressures. This can be seen in the QCD phase diagram shown in Figure~\ref{fig:qcd_phase}. 

\begin{figure}[htbp]
\begin{center}
\includegraphics[width=0.85\textwidth]{figures/theory/qcd_phase}
\caption{The QCD phase diagram of nuclear matter. Figure from from Reference~\cite{PhysRevD.72.034004}. }
\label{fig:qgp_form}
\end{center}
\end{figure}

This state of matter exists above $\lambda_{\mathrm{QCD}} = 200$ MeV, the fundamental energy scale in QCD, and is believed to have filled the early universe a few microseconds after the Big Bang \cite{23, 24} and might be present in the cores of extremely compact objects like neutron stars. 

The MIT Bag Model can be used to describe the QGP as a simple ideal gas with a bag constant $B$ that parameterizes the vacuum pressure \cite{Muller1993, Yagi:2005yb}.

The QGP was initially thought to be a weakly coupled parton gas. This was based on asymptotic freedom from QCD; the highly energetic collisions such as those at the LHC would imply a weak interaction between the quarks and gluons that make up the plasma. This would result in rare scatterings between the constituents of the gas and wash out any spatial anisotropies based on the collision geometry. On the other hand, if the QGP is assumed to be strongly coupled, the pressure gradients in the medium would be driven by hydrodynamics and transform spatial anisotropies to momentum anisotropies in the particles produced as shown in Figure~\ref{fig:overlap}. In this picture, the non-uniform structure of the colliding nuclei would cause a momentum anisotropy that would be further enhanced when looking at collisions that are less central and do not have perfect overlap between the colliding nuclei \cite{116, 117, 118, 63}. Azimuthal correlation measurements \cite{Aad2014, PhysRevLett.87.182301, PhysRevLett.91.182301, PhysRevLett.98.242302,PhysRevC.89.044906,PhysRevLett.116.132302} indicate momentum anisotropy in the collision, implying that the medium is strongly coupled. 


\begin{figure}[htbp]
\begin{center}
\includegraphics[width=0.85\textwidth]{figures/theory/overlap}
\caption{Schematic diagrams of the initial overlap region (left) and the final spatial anisotropy generated (right). Taken from \cite{RevModPhys.90.025005}.}
\label{fig:overlap}
\end{center}
\end{figure}

%At the peak energy density of the collision, the system cannot be described at the level of hadrons, and has to be described in terms of quarks and gluons. The initial anisotropic energy density being reflected in the azimuthal variation of particle production implies a strongly coupled medium that expands hydrodynamically, with a faster expansion in the direction of larger gradients and hence resulting a momentum anisotropy. 

A Fourier Transform of the angular distribution of charged hadrons in the collision debris can quantify these momentum anisotropies and give the anisotropic flow coefficients $v_n$, defined as \cite{115}:

\begin{align}
\frac{d\bar{N}}{d\phi} = \frac{\bar{N}}{2\pi} \left( 1 + 2 \sum_{n=1}^{\inf} v_{n} \cos(n(\phi-\bar{\Psi}_n)) \right)
\end{align}

where $\phi$ is the angle in the transverse plane, $\bar{\Psi}_n$ are the event plane angles, and $\bar{N}$ is the average number of particles per event. Some of these coefficients are shown in Figure~\ref{fig:flow_coeff}.


\begin{figure}[htbp]
\begin{center}
\includegraphics[width=0.65\textwidth]{figures/theory/flow_coefficients}
\caption{Comparison of a hydrodynamic model from \cite{107} to the anisotropy measurements by ALICE \cite{108} for different parameterizations of the $\eta/s$ and for different $v_n (n = 2, 3, 4)$ from top to bottom as a function of collision centrality.  -- see ATLAS measurement from \cite{109}.}
\label{fig:flow_coeff}
\end{center}
\end{figure}


Thermal photons from the QGP reveal that it reaches temperatures of 300--600 MeV in central collisions at 200 GeV \cite{PhysRevLett.104.132301} and 2.76 TeV \cite{2016235}, showing very little collision energy dependence. Further, the chemical freeze-out temperature was found to be 160 MeV via  measurements of ratios of final state hadrons \cite{Fodor_2004,ADAMS2005102, PhysRevC.93.024917} with the thermal freeze-out being 100--150 MeV \cite{PhysRevC.69.024904, PhysRevC.72.014908, PhysRevC.75.024910, PhysRevC.88.044910}.



\section{Jets and Jet Quenching}
\label{sec:jets}
% !TEX root = thesis-ex.tex


Hard scatterings in particle collisions result in the production of highly energetic partons that evolve, decay, and eventually form conical sprays of particles called jets. Jet production is well understood in the context of perturbative QCD \cite{PhysRevLett.39.1436}, and can be shown as Figure~\ref{fig:feynman_jet}. It can be described in terms of the parton distribution functions, scattering cross sections, and final state fragmentation functions as shown below:

\begin{figure}[htbp]
\begin{center}
\includegraphics[width=0.55\textwidth]{figures/theory/feynman_jet}
\caption{Jet production from the process $pp \rightarrow hX$, factorizing in terms of the parton distribution functions, scattering cross sections, and jet fragmentation functions. \cite{Qin:2015srf}}
\label{fig:feynman_jet}
\end{center}
\end{figure}

\begin{align}
d \sigma_{pp \rightarrow hX} \approx & \sum_{abjd} \int dx_a \int dx_b \int dz_j f_{a/p} (x_a, \mu_f) \times f_{b/p} (x_b, \mu_f) \\
& \times d\sigma_{ab\rightarrow jd} (\mu_f, \mu_F, \mu_R) \\
& \times D_{j \rightarrow h} (z_j, \mu_f)
\label{eq:jetCS}
\end{align}

where $x_a = p_a/P_A, x_b = p_b / P_b$ are the initial momentum fractions carried by the interacting partons, $z_j = p_h / p_j$ is the momentum fraction carried by the final observed hadron. $f_{a/p} (x_a, \mu_f)$ and $f_{b/p} (x_b, \mu_f)$ are the two parton distribution functions (PDFs), $d\sigma_{ab\rightarrow jd} (\mu_f, \mu_F, \mu_R)$ is the differential cross section for parton scattering and $D_{j\rightarrow }(z_j,\mu_F)$ is the fragmentation function (FFs) for parton $j$ to hadron $h$. $\mu_f$ and $\mu_F$ are the factorization scales and $\mu_R$ is the renormalization scale, and are typically taken to be the same hard scale $Q$. The PDFs characterize the initial state and represent the probability of finding a parton with momentum fraction $x$ (shown in Figure~\ref{fig:bjorkenX}) in the initial hadron, while the FFs describe the probability of fragmenting to a hadron $h$ with given kinematic properties. Both the PDFs and FFs are universal and evolve via the Dokshitzer-Gribov-Lipatov-Altarelli-Parisi (DGLAP) equations \cite{ALTARELLI1977298, Gribov:1972ri, Dokshitzer:1977sg}. 

\begin{figure}[htbp]
\begin{center}
\includegraphics[width=0.55\textwidth]{figures/theory/bjorkenX}
\caption{The next to leading order (NLO) PDFs at (left) $Q^2 = 10 \mathrm{GeV}^2$ and (right) $Q^2 = 10^4 \mathrm{GeV}^2$. The band is the associated one-sigma (68\%) confidence level uncertainty. Taken from \cite{Martin2009}}
\label{fig:bjorkenX}
\end{center}
\end{figure}

Equation~\ref{eq:jetCS} is written at leading order LO and includes contributions from $2\rightarrow2$ cross sections, LO resummed PDFs and FFs, and single loop expression for the strong coupling $\alphas$. At next to leading order (NLO), the contributions from real $2\rigtharrow3$ and virtual $2\rigtharrow2$ processes, as well as the double loop expression for \alphas are included. These calculations describe the inclusive  and  pQCD, NLO calculations hae Next to leading order (NL) calculations that include 
Figure~\ref{fig:incljetCS} shows the inclusive jet cross section as measured by ATLAS in \sqrts = 13 TeV \pp\ collisions. 


\begin{figure}[htbp]
\begin{center}
\includegraphics[width=0.55\textwidth]{figures/theory/inclJetCS}
\caption{The inclusive jet cross section as a function of \pt\ and $|y|$ as measured by ATLAS. The data are compared to NLO pQCD calculations. Taken from \cite{Aaboud:2017wsi}}
\label{fig:bjorkenX}
\end{center}
\end{figure}




In heavy ion collisions, jets must traverse the quark gluon plasma. This can result in the jet losing energy and forward momentum \cite{2012176, ATLAS:2017wvp}, while also picking up momentum transverse to the parton direction. Jets can also deposit energy in the medium, creating a wake \cite{Khachatryan2016}. 


In a heavy ion collision where the QGP is formed, the hard scattering interactions between the partons strongly interact with the QGP due to their color charge and are modified and lose energy via collisions with the medium constituents, or gluon bremsstrahlung. 

\clearpage

\chapter{Major Jet Measurements}
\label{sec:jetMeasurements}
% !TEX encoding = UTF-8 Unicode
% !TEX root = thesis-ex.tex
This chapter shall discuss some important experimental jet measurements that motivate the study of the main analysis in this thesis. These include the study of the jet yields, dijet asymmetry, and jet fragmentation. 


\section{Dijet Balance: $\mathrm{x}_{J}$}
\label{sec:xj}
This section will discuss the dijet balance for $R = 0.4$ jets as measured by ATLAS detector for \pbpb\ collisions at \sqrtsnn = 2.76 TeV \cite{Aaboud:2017eww}. The dijet imbalance can be expressed in terms of $x_J$ defined as

\begin{align}
x_J =  \frac{\pt_2}{\pt_1}
\end{align}

where $\pt_2$ and $\pt_1$ are the transverse momenta of the two highest-\pt\ jets in the event respectively. The minimum $\pt_2$ considered is 25 GeV and the pair of jets are separated by $|\Delta\phi| > 7\pi/8$. The dijet yields normalized by the number of jets and determined as $1/N_\mathrm{jets} dN/dx_J$ are presented as a function of $x_J$ for different centrality intervals, as well as different ranges for $\pt_1$. The measured distributions are further unfolded to remove detector resolution effects and allow comparison to theoretical models.

Figure~\ref{fig:xJ} shows the $x_J$ distribution for dijet pairs in \pp\ and \pbpb\ collisions in two different centrality bins and two $\pt_1$ ranges. It can be seen that the dijet yields in \pp\ are peaked at unity and become narrower for larger $\pt_1$ ranges. This reflects the fact that the effects of jet quenching are minimal and the higher-\pt\ jets are better balanced. The dijet yields in peripheral \pbpb\ collisions are similar to the distributions from the \pp\ data, showing that the effects of quenching are smaller. On the other hand, dijet yields in central \pbpb\ collisions are significantly broadened, reflecting the maximal  of jet quenching. This is consistent with the picture of the individual jets in the dijet pair traversing different lengths in the QGP and hence losing different amounts of energy. In fact, the distribution for \pbpb\ data is peaked at $x_J = 0.5$, implying a loss of 50\% of the jet \pt.

\begin{figure}[htbp]
\begin{center}
\includegraphics[width=0.55\textwidth]{figures/jetMeasurements/xJ}
\caption{The $1/N_\mathrm{jets} dN/dx_J$ distributions for $R=0.4$ jets as a function of $x_J$ for \pp\ (blue) and \pbpb\ (red) collisions. The different panels are for (top) central and (bottom) peripheral collisions in (left) $100 < \pt_1 < 126$ GeV and (right) $\pt_1 > 200 $ GeV. The \pp\ data is the same in all panels. The statistical uncertainties are indicated by the bars while the boxes indicate the systematic uncertainties. Figures taken from \cite{Aaboud:2017eww}}
\label{fig:xJ}
\end{center}
\end{figure}

Further measurements of $R = 0.3$ jets are shown in Figure~\ref{fig:xJ_R03}. These distributions are significantly flatter than the ones for $R=0.4$ jets, an observation that is consistent with the expectation that the transverse momenta correlation between the dijet pair is weaker for jets with smaller radii due to radiation that is outside the nominal jet cone.

\begin{figure}[htbp]
\begin{center}
\includegraphics[width=0.55\textwidth]{figures/jetMeasurements/xJ_R03}
\caption{The $1/N_\mathrm{jets} dN/dx_J$ distributions for $R=0.3$ jets as a function of $x_J$ in \pp\ and central \pbpb\ collisions. The different panels are for different, $\pt_1$ ranges (top left to bottom right) central and (bottom) peripheral collisions. The \pbpb\ data is in red circles while the \pp\ data is in blue diamonds and is the same in all panels. The statistical uncertainties are indicated by the bars while the boxes indicate the systematic uncertainties. Figures taken from \cite{Aaboud:2017eww}}
\label{fig:xJ_R03}
\end{center}
\end{figure}


\section{Modification of jet yields: $\mathrm{R}_{AA}$}
This section discusses the measurement of the inclusive jet \RAA\ as measured by the ATLAS detector for $R=0.4$ jets in $\sqrtsnn=5.02$ TeV \pbpb\ collisions \cite{2019108}.

While a measurement that compares the jets in a dijet system to each other as discussed in Section~\ref{sec:xj} can provide valuable information about how jets lose energy, it has the following limitation: If both jets lose equal amounts of energy, the dijet yield will still be peaked at unity and no new information will be obtained. Thus, it is useful to compare the jet yields directly between the \pp\ and \pbpb\ systems and construct the jet \RAA\ observable. This is defined as:

\begin{align}
\RAA  = \dfrac{\dfrac{1}{N_{\rm evt}} \left. \dfrac{d^2 N_{\rm jet}}{d\pt dy} \right|_{\rm cent}}{ \langle T_{\rm AA} \rangle \left. \dfrac{d^2\sigma_{\rm jet}}{d\pt dy} \right|_{\rm pp}}
\end{align}

where \TAA is the nuclear thickness function and accounts for the geometric enhancement between \pp\ and \pbpb\ as discussed in Section~\ref{sec:HICollisions} and \cite{doi:10.1146/annurev.nucl.57.090506.123020}. 

This measurement was conducted for jets in the 40--1000 GeV range in different rapidity and centrality intervals. The jet yields in \pp\ and \pbpb\ collisions are shown in Figure~\ref{fig:jet_yields} . The \pbpb\ jet yields are scaled by the thickness function and are shown for 8 centrality intervals. Figure~\ref{fig:raa} shows the measured inclusive jet \RAA\ as a function of jet \pt\ for different centrality bins and jet rapidity $|y| < 2.8$. It can be seen that the most central collisions show a clear suppression with an $\RAA \approx 0.45$ at jet $\pt\ 100$ GeV. The \RAA\ value slowly evolves with jet \pt\ and rises to 0.6 at jet $\pt = 800$ GeV. This modification becomes smaller for more peripheral collisions. The smooth centrality dependence can be more clearly seen in Figure~\ref{fig:raa_centDep}, where \RAA\ is shown as a function of \ANpart\ for jets the 100--126 GeV and 200--251 GeV ranges. The magnitude of the suppression is also seen to depend on jet \pt\ for $\ANpart \geq 50$. 


\begin{figure}[htbp]
\begin{center}
\includegraphics[width=0.85\textwidth]{figures/jetMeasurements/jetYields}
\caption{(Left) The inclusive jet cross section in \pp\ collisions as a function of jet \pt\ in different $|y|$ intervals scaled by successive powers of $10^2$ for visibility. (Right) Per event inclusive jet yield in \pbpb\ collisions normalized by $\langle \TAA \rangle$ as a function of jet \pt\ in different centrality intervals scaled by successive powers of $10^2$ for visibility. The solid lines represent the cross section from \pp\ data at the same rapidity interval scaled by the same $10^2$ factor.  Figure taken from \cite{2019108}}
\label{fig:jet_yields}
\end{center}
\end{figure}


\begin{figure}[htbp]
\begin{center}
\includegraphics[width=0.55\textwidth]{figures/jetMeasurements/raa}
\caption{The \RAA\ distributions as a function of jet \pt\ for different centrality bins and jet rapidity $|y| < 2.8$. The error bars represent statistical uncertainties while the the shaded boxes represent systematic uncertainties. Figure taken from \cite{2019108}}
\label{fig:raa}
\end{center}
\end{figure}

\begin{figure}[htbp]
\begin{center}
\includegraphics[width=0.55\textwidth]{figures/jetMeasurements/raa_centDep}
\caption{The \RAA\ distributions as a function of jet \pt\ for different centrality bins and jet rapidity $|y| < 2.8$. The error bars represent statistical uncertainties while the the shaded boxes represent systematic uncertainties. Figure taken from \cite{2019108}}
\label{fig:raa_centDep}
\end{center}
\end{figure}









\section{Jet Fragmentation}




\clearpage

%\chapter{Experimental Setup}
%\label{sec:setup}
%% !TEX root = thesis-ex.tex

\section{LHC}
The Large Hadron Collider (LHC)~\cite{Evans:2008zzb} was built by the European Organization for Nuclear Research (CERN) and is located on the France-Switzerland border outside of Geneva. The LHC is designed to collide beams of protons at a center of mass energy up to \sqrts=14 TeV and beams of lead ions at a center-of-mass energy per nucleon up to \sqrtsnn=8.16 TeV. It is the largest of many accelerators that constitute the the CERN accelerator complex, pictured in Fig.~\ref{fig:cern}. 

\begin{figure}[ht]
	\centering
	\includegraphics[width=0.60\textwidth]{figures/cern.jpg} %
	\caption{The accelerator complex at CERN. ATLAS can be seen inside the SPS on the LHC ring. Figure taken from Ref.~\cite{cernlhc}}
	\label{fig:cern}%
\end{figure}

During the LHC's first operational data taking run, referred to as Run 1 (2009-2013), the first collisions with stable beams were observed between protons and protons (\pp), as well as protons with lead ions (\pPb) at center of mass energies of \sqrts=8 TeV and \sqrtsnn=2.76 TeV, respectively. Center of mass energies for \pPb\ collisions were subsequently increased to \sqrtsnn=5.02 TeV in 2013. After an extended technical shutdown for upgrades following Run 1, the LHC was restarted for run Run 2, during which \pp\ and \pPb\ collisions with stable beams were observed at center-of-mass energies of \sqrts=13 TeV and \sqrtsnn=8.16 TeV, respectively. 

The LHC is located in a tunnel at depths of 50 to 175 m underground. Originally, this tunnel was built for the Large Electron-Proton Collider (LEP), an electron-proton collider that was operation from 1989-2000. In the LHC, particle packets in high vacuum beam pipes going in opposite directions are accelerated by 8 radio frequency cavities (RF) which deliver voltages up to 2 MV at an oscillator frequency of 400 MHz. Each 26.7 km ring consists of eight arched sections with 616 dipole super-conducting magnets per beam, which supply fields of up to 8.33 Tesla. An additional 196 beam focusing quadropole magnets per beam serve to narrow the beam and increase luminosity. To supply such high magnetic fields, LHC magnets use super-fluid helium and operate at temperatures down to 1.9 K while the RF cavities operate at temperatures down to 4.5 K.

Any proton or lead ion entering the LHC must go through the complex chain of accelerators shown in  Fig.~\ref{fig:cern}. In order to be accelerated and focused in the beams, the proton and lead ions are required to have a net positive charge. Thus, the hydrogen and lead atoms must be first stripped of the electrons in their atomic shells. Positively charged protons are obtained by stripping atoms of hydrogen gas from their electrons using an electric field. Positively charged lead ions are initially extracted from a source which provides partially stripped lead ions with an average around $\mathrm{Pb^{29+}}$. These ions then go through a series of pre-accelerators, seen at the bottom of Fig.~\ref{fig:cern}, starting with the Linear Accelerator 3 (LINAC3) where they are further stripped of electrons by passing through $3.0\ \mu$m  of carbon foil. Next, a mass spectrometer selects lead ions with an average $\mathrm{Pb^{29+}}$ to be fed into the Low Energy Ion Ring (LEIR). The protons, meanwhile, begin their journey at the Linear Accelerator 2 (LINAC2). Both protons and lead ions then enter the next phase of pre-accelerators which consist of the Proton Synchrotron (PS) and the Super Proton Synchrotron (SPS), where they continue to be  accelerated. The lead ions are completely stripped away of remaining electrons at the exit of the PS, where they pass through 0.8 mm aluminum foil. The final stage is at the exit of the SPS where the protons and lead ions enter the LHC for the final phase of acceleration before they are collided.
 
Beams in the LHC consist of 2808 bunches of protons or lead ions with bunch spacing down to 25 ns (7.5 m). A proton bunch contains approximately 1.15x$10^{11}$ protons while an ion bunch contains approximately 2.2x$10^{8}$ ions. These beams are brought to collide at four interaction points which can be seen along the circumference of the LHC in Fig.~\ref{fig:cern}. At these interaction points there are detectors present to analyze the collisions: A Large Ion Collider Experiment (ALICE), A Toroidal LHC Apparatus (ATLAS), Compact Muon Solenoid (CMS), and the Large Hadron Collider Beauty (LHCb).

\section{ATLAS Experiment}


\begin{figure}[ht]
	\centering
	\includegraphics[width=0.7\textwidth]{figures/atlas.pdf} %
	\caption{The ATLAS detector. Figure taken from Ref.~\cite{Aad:2008zzm}.}	
	\label{fig:atlas}%
\end{figure}

The ATLAS detector~\cite{Aad:2008zzm}, shown in Fig.~\ref{fig:atlas} is one of the two larger detectors on the LHC and is located at interaction point 1 (IP1) on the LHC ring\footnote{
	ATLAS uses a right-handed coordinate system with its origin at the nominal interaction point (IP) in the centre of the detector and the $z$-axis along the beam pipe. The $x$-axis points from the IP to the centre of the LHC ring, and the $y$ axis points upward. Cylindrical coordinates $(r,\phi)$ are used in the transverse plane, $\phi$ being the azimuthal angle around the beam pipe. The pseudorapidity is defined in terms of the polar angle $\theta$ as $\eta=-\ln\tan(\theta/2)$. Angular distance is measured in units of $\Delta R \equiv \sqrt{(\Delta\eta)^{2} + (\Delta\phi)^{2}}$. Rapidity is defined in terms of energy and momentum of a particle or jet as $y=\frac{1}{2}ln(\frac{E+p_{z}}{E-p_{z}})$. The rapidity with center-of-mass frame boost accounted for is denoted \ystar.} 
It is designed to perform measurements of Standard Model physics, including the search for the Higgs boson, and search for physics beyond the Standard Model. Although ATLAS is primarily a detector used to measure $pp$ collisions, it has also been used to study Heavy Ion physics with much higher nuclear collision energies and much larger particle multiplicities compared to \pp\ collision.

\begin{figure}
	\centering
	\includegraphics[width=0.7\textwidth]{figures/atlaspseudorap.png} %
	\caption{ ATLAS detector pseudorapidity coverage. All components cover $2\pi$ in azimuth. }	
	\label{fig:atlasrap}%
\end{figure}

The ATLAS detector consists of four main parts, or sub-detectors. The closest part to the interaction point is the Inner Detector (ID), which is placed close to the IP and is used to measure charged particle tracks. The ID is inside a 2 Tesla solenoidal magnetic field, which causes charged particles to curve, allowing their momentum to be measured. Outside of the ID are the electromagnetic (EM) and hadronic calorimeters. These give energy measurements and are the primary detectors for the analysis presented in this thesis. The fourth and outermost part is the muon spectrometer which is placed inside a toroidal field provided by eight toroid magnets. The muon system is the outermost part of the detector because due to their weakly interacting nature, muons are one of the only particles which pass through the calorimeters. All of the ATLAS sub-detectors have full $2\pi$ azimuthal coverage and different pseudorapidity coverages shown in  Fig.~\ref{fig:atlasrap}. A detailed description of the ATLAS detector and it's subsystems can be found in~\cite{Aad:2008zzm}. 

\subsection{ATLAS Trigger System}
\label{sec:trigger}

In order to select events during data-taking, a complex hardware and software system called the $trigger$ is required. It relies on many detector subsystems to flag events based on a set of rules that are defined prior to each run. A two-level trigger system was used to select the \pp\ and \pPb\ collisions analyzed for the measurement presented in this thesis. The first, the hardware-based trigger stage Level-1 (L1), is implemented with custom electronics. The second level is the software-based High Level Trigger (HLT). The HLT consists of the Level-2 (L2) trigger, followed by the event filter (EF). The ATLAS trigger was designed for a collision rate of 40 MHz, with the L1 trigger designed to reduce the rate to 75 kHz, and the HLT to perform a final reduction to about 200 Hz, which is the final even rate written to disk. A schematic of the ATLAS trigger and data acquisition systems can be seen in Fig.~\ref{fig:trigdaq}. Some triggers selecting minimum-bias (MB) events used the minimum-bias trigger scintillator detectors (MBTS). The MBTS detect charged particles over $2.1 < |\eta| < 3.9$ using two segmented counters placed at $z = \pm 3.6$~m. Each counter provides measurements of both the pulse heights and the arrival times of ionization energy deposits~\cite{Aad:2008zzm}.

\begin{figure}[t]
	\centerline{
		\includegraphics[width=0.52\textwidth]{figures/trig_daq.pdf} %
		\includegraphics[width=0.48\textwidth]{figures/trig_l1.pdf} %
	}
	\caption{ A schematic (left) of the ATLAS trigger and data acquisition systems, and the L1 hardware trigger (right). The total event rate of about 40 MHz is reduced by the L1 trigger to about 75 kHz, and further reduced to 200 Hz by the HLT (L2 + EF) trigger.Figure taken from Ref.~\cite{Aad:2008zzm}.}	
	\label{fig:trigdaq}
\end{figure}

Some triggers can be prescaled, meaning that not every event meeting the requirements of a particular trigger is saved to disk. If a trigger with prescale $c_{p}$ is saved $n$ times, this corresponds to $c_{p}n$ events passing through the HLT. The decision of what prescale to assign to a trigger is very complicated. Various physics analysis groups have different requirements, but unfortunately not all data from a run can be saved due to technical limitations. Depending on the physics goals of a particular run, the trigger menu, which assigns the triggers and their respective prescales, will change. The UIUC ATLAS group has been responsible for the trigger system operation in all of the heavy ion runs since 2015.


\subsection{Calorimetery}

\begin{figure}
	\centering
	\includegraphics[width=0.8\textwidth]{figures/calorimeters.pdf} %
	\caption{The ATLAS calorimeter system. Figure taken from Ref.~\cite{Aad:2008zzm}.}	
	\label{fig:calorimeters}
\end{figure}

The ATLAS calorimeter system ~\cite{Aad:2008zzm} is the main system used for the present analysis, a picture of this system is shown in Fig.~\ref{fig:calorimeters}. The calorimeters are of sampling and non-compensating nature with a pseudorapidity coverage of $|\eta|<4.9$. The non-compensating nature gives a different response on the EM and hadronic scales, and this is corrected in the calibration procedure. A sampling calorimeter is one where two distinctly different materials are chosen, one to produce a particle shower, and the other to measure the deposited energy. 

There are two different sampling technologies used in the ATLAS calorimeter system. One technology is where liquid argon (LAr) is interspaced with lead, which acts as the absorber material. This is used in all of the ATLAS EM systems - the electromagnetic barrel (EMB), electromagnetic end-cap (EMEC), forward calorimeter (FCal), as well as the hadronic end-cap (HEC). Shower development starts in the absorber, and due to moving electrons and ions from ionization in the active material (LAr), a signal can be read out from induced charge on copper electrodes. The LAr gap is subject to a high voltage electric field in order to direct the ionized electrons and ions to the electrodes in a predictable way. The second technology, used in the hadronic tile calorimeters (TileCal), uses absorber material interspaced with plastic scintillator. The readout is different from the LAr case since scintillation light converted by wavelength shifting fibers and transported to photomultipliers instead of reading induced charge from ionization in LAr.

\subsubsection{EM Calorimeters}
\begin{figure}[ht]
	\centerline{
		\includegraphics[width=0.7\textwidth]{figures/emmodules.pdf} 
	}
	\caption{Layouts of a barrel EM module (top), inner end-cap wheel (bottom left), and outer end-cap wheel (bottom right). Figure taken from Ref.~\cite{Aad:2008zzm}.  }
	\label{fig:emmodules}
\end{figure}

\begin{figure}[ht]
	\centerline{
		\includegraphics[width=0.7\textwidth]{figures/barrel_module.pdf} 
	}
	\caption{Sketch of a barrel EM module showing the different layers and their respective granularities. Radiation length ($X_{0}$) is the average distance an electron must travel through a given material to reduce its energy to $1/e$ of its initial energy. Trigger towers are sets of cells (strip or square) from which analog signals are summed for input to the L1 trigger. Figure taken from Ref.~\cite{Aad:2008zzm}.  }
	\label{fig:barrelemmodule}
\end{figure}

The ATLAS LAr electromagnetic calorimeter as chosen to have have an accordion geometry to minimize capacitance in the detecting elements. It is split into a barrel part covering $|\eta|<1.475$, and two end-caps covering $1.375<|\eta|<3.2$. The accordion design allows modules to have multiple layers in depth, with varying granularity ($\Deta \times \Dphi$). Layouts of segments from the barrel and end-cap EM calorimeters are shown in Fig.~\ref{fig:emmodules}. A detailed sketch of a barrel EM module and its constituent layers is shown in Fig.~\ref{fig:barrelemmodule}. All components are placed into cryostats at a temperature of approximately $86^{\circ}$ K~\cite{Bremer:449276}.  The design and size of the EM calorimeter provides a total thickness of at least 22 radiation lengths (\radlen). One \radlen\ represents the average distance an electron must travel through a material to reduce its energy to $1/e$ of its initial energy~\cite{Fabjan:2003aq}. The cumulative thickness of the calorimeter system can be seen as a function of pseudorapidity in Fig.~\ref{fig:radiationlengths}. All EM calorimeter systems were designed and tested to have an energy resolution of $\sigma(E_{T})/E_{T}=10\%/\sqrt{E_{T}}\bigoplus0.7\%$.

\begin{figure}[ht]
	\centerline{
		\includegraphics[width=0.49\textwidth]{figures/radiation_lengths_1.pdf} 
		\includegraphics[width=0.49\textwidth]{figures/radiation_lengths_2.pdf} 
	}
	\caption{Cumulative thickness, in units of radiation length \radlen\ and as a function of $|\eta|$, in front of (yellow distribution) and in the electromagnetic calorimeters. Shown separately are the amounts of radiation in the various layers of the barrel (left) and end-cap (right) EM calorimeters. Figure taken from Ref.~\cite{Aad:2008zzm}. }
	\label{fig:radiationlengths}
\end{figure}

A typical pulse in the LAr calorimeter originates from ionization electrons in the LAr gap. An electric field inside the gap collects the electrons and an ionization pulse is then read out and shaped. An ionization pulse is triangular in shape has a width of $\sim$450 ns~\cite{Nikiforou:2013nba}, as can be seen in Fig.~\ref{fig:larsignal}. The final pulse that is digitized has a width between 450 and 600 ns after shaping. This corresponds to roughly 18 to 24 LHC bunch crossings. During this time, there could be contributions from out-of-time events (pile-up), and various techniques such as optimal filtering~\cite{OliveiraDamazio:1630826} have been developed to minimize contributions from pile-up.

\begin{figure}[ht]
	\centerline{
		\includegraphics[width=0.5\textwidth]{figures/lar_signal.pdf} 
	}
	\caption{Amplitude versus time plot of a LAr calorimeter pulse before shaping (triangular). The shaped pulse is sampled every 25 ns, as indicated by the periodic points. The sampling frequency corresponds to the LHC bunch crossing frequency of 25 ns. Figure taken from Ref.~\cite{Aad:2008zzm}.}
	\label{fig:larsignal}
\end{figure}

\subsubsection{EM Barrel Calorimeter}
The EM barrel, covering $|\eta|<1.475$, consists of two half-barrels, each 3.2 meters long and weighing 57 tons. It has an inner and outer diameter of 2.8 m and 4.0 m, respectively. The calorimeter is comprised of three layers, with a thickness of at least 22 \radlen\, increasing to from 22 to 30 \radlen\ in the interval $0<|\eta|<0.8$, and from 24 to 33 \radlen\ in the interval $0.8<|\eta|<1.3$, as seen in Fig.~\ref{fig:radiationlengths}. In front of these three layers is a LAr presampler which is intended to recover energy lost to material in front of the EMCal. The granularity of the EM barrel calorimeter's first layer is $\Deta\times\Dphi=0.025\times0.025$ in order to be able to perform shower shape measurements and to distinguish pairs of $\gamma$ from $\pi^{0}$ decays with pairs of $\gamma$ from $H$ decay. The granularity of the presampler is $\Deta\times\Dphi=0.025\times0.1$. 


\subsubsection{EM End-cap Calorimeter}
The EM end-cap calorimeter, covering $1.375<|\eta|<3.2$, consists of two wheels on each side of the EM barrel calorimeter, each 63 cm thick, with a weight of 27 tons. Each wheel of the EM end-cap calorimeter consists of 32 identical azimuthal sectors. Similar to the EM barrel calorimeter, the barrel end-cap calorimeter consists of three layers. It has a total thickness of at least 24 \radlen\, increasing from 24 to 38 \radlen\ on the outer wheel ($1.475<|\eta|<2.5$), and from 26 to 36 \radlen\ on the inner wheel ($2.5<|\eta|<3.2$). Similar to the EM barrel calorimeter, the granularity of the first layer is $\Deta\times\Dphi=0.025\times0.025$ and the granularity of the presampler is $\Deta\times\Dphi=0.025\times0.1$.

\subsubsection{Hadronic Calorimeters}

\begin{figure}
	\centering
	\includegraphics[width=0.7\textwidth]{figures/interaction_lengths.pdf} %
	\caption{Cumulative thickness, units of interaction length (\intlen) as a function $\eta$, in front of the EM calorimeters, in the EM calorimeters themselves, in the hadronic calorimeters, and the total amount after all calorimeters. Figure taken from Ref.~\cite{Aad:2008zzm}.}	
	\label{fig:interactionlengths}
\end{figure}

The hadronic calorimeters surround the EM calorimeters and are designed to measure the energy deposited from hadrons and hadronic showers that passed through the EM calorimeters. Characteristic distance for hadronic calorimeters is described by the nuclear interaction length \intlen, which is the hadronic equivalent to a radiation length.  For the EM calorimeter system, \intlen\ is small, requiring hadronic calorimeters to have sufficiently larger thicknesses in order to fully contain hadronic showers. The hadronic calorimeter is composed of the Tile barrel calorimeter with a coverage $|\eta|<0.8$, the Tile extended barrel with a coverage $0.8<|\eta|<1.7$, and the HEC with a coverage $1.5<|\eta|<3.2$. Both Tile systems use steel as an absorber, with scintillator as the active material. The particle shower begins in the absorber, and scintillation light then gets transported through the wavelength shifting fiber into photomultiplier tubes where the signal is read out. The HEC is based on the same LAr technology used in the EM calorimeters, but uses copper, instead of lead, for the absorber material. Total interaction lengths of the ATLAS calorimeter system as a function of pseudorapidity are summarized in Fig.~\ref{fig:interactionlengths}. Both TileCal and HEC calorimeters have an energy resolution of $\sigma(E_{T})/E_{T}=50\%/\sqrt{E_{T}}\bigoplus3\%$. 

\subsubsection{Tile Barrel and Extended Barrel Calorimeters}
The Tile barrel and extended barrel calorimeters cover $|\eta|<0.8$ and $0.8<|\eta|<1.7$ respectively. The tile barrel calorimeter is 5.8 m long, the two tile extended barrels are each 2.6 m in length. Both the tile barrel and extended barrel calorimeters have an inner and outer diameter of 2.28 m and 4.25 m, respectively. They is composed of three layers with granularity of $\Deta\times\Dphi=0.1\times0.1$ for the first two layers, and the outermost layer with granularity $\Deta\times\Dphi=0.2\times0.1$. Each barrel consists of 64 modules roughly $\Dphi=0.1$ in size. A schematic showing a TileCal module is shown in Fig.~\ref{fig:tilecalmodule}.

\begin{figure}
	\centering
	\includegraphics[width=0.5\textwidth]{figures/tile_1.pdf} %
	\caption{ Schematic of a TileCal module, showing absorber material interspace with scintillator. Figure taken from Ref.~\cite{Aad:2008zzm}.}	
	\label{fig:tilecalmodule}
\end{figure}

\subsubsection{LAr Hadronic End-Cap Calorimeter}
The HEC calorimter is based the LAr technology used in the EM calorimter systems. The absorber material is copper, and the active material is LAr. The HEC covers a pseudorapidity region of $1.5<|\eta|<3.2$. The two barrels of the HEC each contain 32 modules symmetric in azimuth, with an outer radius of 2030 mm. The first two layers of the HEC have a granularity $\Deta\times\Dphi=0.1\times0.1$, while the last layer has a courser granularity of $\Deta\times\Dphi=0.2\times0.2$.

\subsubsection{Forward Calorimeter}

\begin{figure}
	\centering
	\includegraphics[width=0.75\textwidth]{figures/fcal_1.pdf} %
	\caption{ Diagram showing the three modules of the FCal. Shown in the $y-z$ plane, with the beam going in the $z$ direction. The FCal is the most forward calorimeter in ATLAS, covering a pseudorapidity interval $3.2<|\eta|<4.9$. Figure taken from Ref.~\cite{Aad:2008zzm}.}	
	\label{fig:fcalmodules}
\end{figure}

The forward calorimeter is an important sub-system in the present analysis due to its forward pseudorapidity coverage. The calorimeter is comprised of two halves located on either side of the ATLAS detector IP, surrounded by the HEC. It covers a pseudorapidity range of $3.2<|\eta|<4.9$, and has a granularity of $\Deta\times\Dphi=0.2\times0.2$. While the other EM calorimeter systems use an accordion design, the forward calorimeter has electrodes oriented parallel to the beamline (z-axis) which consist of thin tubes of copper with a gap for LAr that surround rods of absorber material. These tubes are located inside the same kind of absorber material. The LAr gap is thin, about 0.25 mm in the first module, in order to increase readout time and decrease noise from ion buildup.

\begin{figure}
	\centering
	\includegraphics[width=0.575\textwidth]{figures/fcal_2.pdf} %
	\includegraphics[width=0.375\textwidth]{figures/fcal_3.pdf} %
	\caption{ View of first FCal module (EM) as seen along the z-axis (left). Tubes of LAr inside absorber material. Shown is one Moliere radius $R_{M}$, which is the radius of a cylinder that would contain 90\% of the radiation inside a calorimeter. A schematic of the tungsten rods, enclosed in copper and a LAr gap, all surrounded by tungsten slugs (right). The design is used for the two hadronic FCal modules FCal2 and FCal3. Figures taken from Ref.~\cite{Aad:2008zzm}.}	
	\label{fig:fcalcloseup}
\end{figure}

Each FCal is composed of three modules, as shown in the $y-z$ plane in Fig.~\ref{fig:fcalmodules}. The first of three modules (FCal1) is the EM module and uses copper as the absorber. The last two hadronic modules (FCal2, FCal3) use tungsten as the absorber. FCal1 uses copper plates that are stacked one behind the other. These plates have 12,260 drilled holes to make space for the electrodes, which are rods made from absorber material coaxial to a thin surrounding LAr layer with precision, radiation-hard plastic fiber used for readout.  A schematic of first layer of the calorimeter as it appears in the $x-y$ plane, perpendicular to the beam direction, showing the tubes of LAr inside the absorber material, is shown in the left of Fig.~\ref{fig:fcalcloseup}. Signal is read out from ionized charges in the LAr that travel to electrodes which run parallel to the tubes. The hadronic modules FCal2 and FCal3 require large interaction lengths, which is why tungsten is chosen as the absorber material, rather than copper as in FCal1. The modules consist of two copper plates, 2.35 cm thick, that have many tungsten rods, coaxial to copper tubes with a LAr gap, enclosed in tungsten slugs, as shown in right of Fig.~\ref{fig:fcalcloseup}. These modules give a total of 10 \intlen\ interaction lengths. The FCal has an energy resolution of $\sigma(E_{T})/E_{T}=100\%/\sqrt{E_{T}}\bigoplus10\%$. 

\subsection{Solenoid Magnet}
The magnet system, shown in Figure~\ref{fig:magnet} has an overall dimension of 22 m in dameter and 26 m in length. It stores a total energy of 1.6 GJ and consists of a barrel solenid magnet, and toroidal magnets used by the muon system. The toroidal magnets are not used in the present analysis. The solenoid magnet, which is used by the inner detector tracker, provides a 2 T axial field which is supplied by a 7.73 kA current. NbTi is used as a conductor and is supercooled by a LAr cryostat temperatures down to 4.5 K. 


\begin{figure}
	\centerline{
		\includegraphics[width=0.52\textwidth]{figures/magnet.pdf} %
	}
	\caption{ The ATLAS magnet system. Shown is the cylindrical solenoid magnet, as well as the eight barrel toroid magnets used for muon detection. Figure taken from Ref.~\cite{Aad:2008zzm}.}	
	\label{fig:magnet}
\end{figure}

\subsection{Inner Detector}

The ATLAS Inner Detector (ID) is responsible for tracking, which is the precise determination of the position of charged particles. In an average collision there can be thousands of particles, which, in in the presence of a magnetic field, will curve. If their positions are well known and can be distinguished, the particles momentum can be calculated. The ID is designed to provide precision tracking for particles above a \pt\ threshold of 0.5 GeV, although some studies have had similar performance with particle \pt\ as low as 0.1 GeV. The ID is designed to have a transverse momentum resolution of $\sigma(\pt)/\pt=0.05\%/\sqrt{E_{T}}\bigoplus1\%$.  Tracking is a very important part of every high energy particle detector, and is usually placed closest to the interaction point of a detector. A cut-away and schematic of the ID is shown in Fig.~\ref{fig:id}. The detector sits inside the 2T magnetic field produced by the solenoid. The ID has a rapidity coverage of $|\eta|<2.5$ and has an outer radius of 1.15 m. There are there main subsystems that comprise the ID, listed outwards from the beam pipe: the pixel detectors, the semiconductor tracker (SCT), and the transitional radiation tracker (TRT).

\begin{figure}
	\centerline{
		\includegraphics[width=0.52\textwidth]{figures/id.pdf} %
		\includegraphics[width=0.52\textwidth]{figures/id_schematic.pdf} %
	}
	\caption{ Cut-away picture (left) and schematic (right) of the ATLAS Inner Detector. Figures taken from Ref.~\cite{Aad:2008zzm}.}	
	\label{fig:id}
\end{figure}

The pixel layer has the highest granularity out of the ATLAS tracking subsystems. There is a barrel layer and two end-cap layers, one on each side of the IP. The barrel detector has three concentric layers located 50.5mm, 88.5mm, and 122.5 mm radially away from the beam pipe. The end-caps also have three layers located 495mm, 580mm, and 650mm in the transverse direction on each side of the interaction point. All of the pixel subsystems have a granularity of $50x400$ $\mu\mathrm{m}^{2}$ and total approximately 80 million readout channels. The SCT has roughly 6.3 million channels and consists of four concentric barrel layers, and nine disks on each side of the IP. The accuracy of the barrel and end-cap regions is 17 $\mu$m in the $(R\phi)$ plane and 580 $\mu$m in the radial direction. The TRT, which is a drift tube (straw) detector, is the outermost tracking layer of the ID. It has a total of approximately 351,000 channels (one per straw) and an accuracy of 130 $\mu$m per straw tube. However, during HI running, the occupancy in the TRT is usually too large to use effectively.

\FloatBarrier

%\clearpage
%
%\chapter{ATLAS Qualification Task}
%\label{sec:qualification}
%In order to qualify as an author in the ATLAS collaboration a task must be completed as a contribution to the experiment. The assigned task was to study the  impact of service material from the recent Insertable B-Layer (IBL) upgrade on the transverse momentum reconstruction in the forward region. Throughout this task, a large portion of the software required for the thesis analysis was developed because this specific qualification work was chosen with the currently proposed analysis in mind. 

The IBL was installed during LS1 between 2013 and 2015. This new pixel detector was needed to achieve better vertex resolution during the higher luminosity Run 2. The service materials, which run out azimuthally from the beam-pipe in high pseudorapidity regions were found to have very high radiation lengths compared to the material previously there. This could have a negative impact on the forward calorimeter's performance.

In order to better understand the effect of the IBL services on forward physics, specifically forward jet measurements, the azimuthal dependence of $\Delta\phi$, and relative $p_{T}$ response in a forward-central dijet system, as well as azimuthal jet yields were looked at in 5.02~TeV $pp$ data and MC samples. Additionally, jet response and $\Delta\phi$ correlations between truth and reconstructed jets in MC were also studied.

\section{Event Selection and Cuts}
Data from the heavy ion 5.02~TeV $pp$ run in 2015 was used and selected by forward High Level Triggers.
The Monte-Carlo samples used were generated by PYTHIA~8 \cite{Sjostrand:2014zea}, with leading order PDFs, and simulated by GEANT4 \cite{Agostinelli:2002hh, Aad:2010ah}.

The relative azimuthal angular correlation, $\Delta\phi$, between forward and central, leading and sub-leading jets, was studied at as a function of the central jets' azimuthal angle $\phi$. A 20 GeV $p_{T}$ cut was placed on the central jet, and different cuts were placed on the forward jets. Three jet events were rejected if $40\%$ of the average $p_{T}$ of the two leading jets was less than the $p_{T}$ of the third jet. Initially, jets were required to be isolated such that if two jets fall within a cone of $R=1.0$, then one jet has to have at least twice the transverse momentum of the second jet. 

\section{Procedure}
Throughout this study, using the specified data and MC samples, jets were reconstructed using the anti-$k_{T}$ algorithm with a radius of $R=0.4$ \cite{Cacciari:2008qp}. Topological towers with a $\Delta\eta\times\Delta\phi = 0.1\times0.1$ were constructed from calorimeter information and used as input into the clustering jet reconstruction algorithm. The HI jet calibration was used along with standard event selection cuts. This is the same calibration used in the main thesis analysis and will be discussed in more detail in Chapter~\ref{sec:mainanalysis}.

Azimuthal jet yields in bins of pseudorapidity, azimuth, and transverse momentum were normalized to the mean in each bin to get the normalized azimuthal jet yield.

\begin{equation}
Normalized\ Yield(p_{T},\eta,\phi) = \frac{Yield(p_{T},\eta,\phi) - Mean\ Yield(p_{T},\eta)}{Mean\ Yield(p_{T},\eta)}
\end{equation}

Both the $\Delta\phi$ and relative $p_{T}$ distributions were filled in bins of $\eta$, $p_{T}^{Forward}$, and $\phi_{Central}$. For every $\phi_{Central}$ bin, the respective distribution was fitted to a Gaussian, as shown in Figure~\ref{fig:fitting}, with some $\Delta\phi$ fits as an example, to yield the final azimuthal distributions.

\begin{figure}
	\centering
	{\includegraphics[width=0.45\textwidth]{figures/qualification/dPhiFit1.pdf}}
	{\includegraphics[width=0.45\textwidth]{figures/qualification/dPhiFit2.pdf}} %
	\caption{Examples of Gaussian fitted $\Delta\phi$ distributions in various $\eta$, $p_{T}^{Forward}$, and $\phi_{Central}$ bins.}%
	\label{fig:fitting}%
\end{figure}

\section{Results}

For both IBL and non-IBL regions, one forward $p_{T}$ bin is selected and the normalized yield distributions are plotted for both data and MC in Figure~\ref{fig:mcdata}. The magnitude of the variation in the IBL and non-IBL regions is similar in the data, while in MC it is flat in the non-IBL region, but oscillates in the IBL region. In the data, however, there is even some modulation in the non-IBL region. 

Looking at the RMS of the projections onto the y-axis, in the non-IBL region ($3.6<\eta<3.8$) the data has an $RMS=0.20$ while the MC has an $RMS=0.098$, and the IBL region ($3.8<\eta<4.4$) the data has an $RMS=0.21$ and the MC an $RMS=0.16$. This shows that in the data, the two regions are not so different, but they are in the MC. This is due to the IBL services being described differently in the MC than what is actually seen in the data. 

\begin{figure}
	\centering
	\includegraphics[width=0.8\textwidth]{figures/qualification/mcData.pdf}
	\caption{Normalized azimuthal jet yields for the forward jet $p_{T}$ bin $25<p_{T}^{Forward}<45 GeV$ is shown in the IBL region ($3.6<\eta<3.8$), and non-IBL region ($3.8<\eta<4.4$), for both data and MC. Red lines indicate a $25\%$ deviation, and green lines indicate a $50\%$ deviation.}
	\label{fig:mcdata}
\end{figure}

There is some difference seen in normalized yields between IBL and non-IBL regions. It is also important to study the $\Delta\phi$ and relative $p_{T}$ response as functions of the central jet's azimuthal angle in data and MC. The $\Delta\phi$ between forward and central jets is shown in Figure~\ref{fig:mcdatadphi}. The distribution in the data exhibits a saw-tooth pattern which is not well understood, but there does not appear to be a major difference between the IBL and non-IBL regions overall. Relative $p_{T}$ response in the forward-central dijet system is shown in Figure~\ref{fig:mcdatarpt}. Jets were required to be back-to-back, $2.5<\Delta\phi<3.8$. As with the $\Delta\phi$ distributions, no strong difference is seen between IBL and non-IBL regions. 

\begin{figure}
	\centering
	\includegraphics[width=0.8\textwidth]{figures/qualification/mcDataDPhi.pdf}
	\caption{As a function of the central jet azimuthal angle, the $\Delta\phi$ distribution for the forward jet $p_{T}$ bin $25<p_{T}^{Forward}<45 GeV$ is shown in the IBL region ($3.6<\eta<3.8$), and non-IBL region ($3.8<\eta<4.4$), for both data and MC.}
	\label{fig:mcdatadphi}
\end{figure}

\begin{figure}
	\centering
	\includegraphics[width=0.8\textwidth]{figures/qualification/mcDataRPt.pdf}
	\caption{As a function of the central jet azimuthal angle, the relative $p_{T}$ distribution for the forward jet $p_{T}$ bin $25<p_{T}^{Forward}<45 GeV$ is shown in the IBL region ($3.6<\eta<3.8$), and non-IBL region ($3.8<\eta<4.4$), for both data and MC.}
	\label{fig:mcdatarpt}
\end{figure}

\section{Conclusion}

The IBL service material is found to have no significant impact on the relative $p_{T}$ response and $\Delta\phi$ azimuthal angular difference in the forward-central dijet system. This is important for the proposed thesis analysis because the forward calorimeter will be one of the most important detectors, and this study shows that the IBL material will not harm the current calibration or affect the important physical quantities. 

%\clearpage
%
%\chapter{Measurement of Angular Correlations Between Tracks And Jets}
%\label{sec:mainanalysis}
%% !TEX encoding = UTF-8 Unicode
% !TEX root = thesis-ex.tex

\section{Overview}

Measurements of jets in heavy ion collisions are powerful tools to determine the properties of the quark gluon plasma by measuring the modification of jet production and fragmentation after the jets have traversed the hot QCD matter.
As discussed in Section~\ref{sec:jetMeasurements}, jets with large transverse momenta in central lead-lead (\pbpb) collisions at the LHC are measured at approximately half the rates in \pp\ collisions when the nuclear overlap function of \pbpb\ collisions is taken into account~\cite{Abelev:2013kqa,Aad:2014bxa,Adam:2015ewa,Khachatryan:2016jfl, Khachatryan:2016odn, 2019108}.
Back-to-back dijet~\cite{Aad:2010bu,Chatrchyan:2011sx,Aaboud:2017eww} and photon-jet pairs~\cite{Chatrchyan:2012gt,2019167} are observed to have less balanced transverse momenta in \pbpb\ collisions compared to \pp\ collisions.
Jet shape measurements in the \pp\ and \pbpb\ collision systems have shown a broadening of the jets due to the QGP~\cite{Aad:2011sc, Acharya:2018uvf, Chatrchyan:2012mec, Chatrchyan:2013kwa}, and jet fragmentation functions in \pbpb\ collisions are modified with an excess for low and high momentum particles and depletion of intermediate momentum particles inside the jet compared to pp collisions~\cite{Aad:2014wha,Chatrchyan:2014ava,Aaboud:2017bzv,PhysRevC.98.024908}.
Particles carrying a large fraction of the jet momentum are generally closely aligned with the jet axis, whereas low momentum particles are observed to have a much broader angular distribution extending outside the jet~\cite{Chatrchyan:2011sx,Khachatryan2016,Khachatryan:2016tfj,Sirunyan:2018jqr}.
All these studies have suggested that the energy lost via jet-quenching is being transferred to soft particles around the jet axis via soft gluon emission~\cite{Vitev:2008rz,Ovanesyan:2011xy,Blaizot:2014ula,Qin:2015srf,Escobedo:2016jbm,Casalderrey-Solana:2016jvj,Tachibana:2017syd} and investigating the radial distribution of particles as a function of transverse momentum has the potential to provide further insight into the structure of jets in the QGP.
This can help provide information on not only how the jet is affected by the plasma, but also how the plasma is affected by the jet.

This thesis presents charged-particle \pt\ distributions around the jet axis as shown in Figure~\ref{fig:cartoon_dptr}. 
The measured yields are defined as:

\begin{figure}[htbp]
\centering
\includegraphics[width=0.7\textwidth]{figures/main/general/cartoon_dptr}
\caption{A schematic showing the distribution of tracks inside and outside a jet cone of radius $R$.}
\label{fig:cartoon_dptr}
\end{figure}


\begin{align}
\Dptr = \frac{1}{N_{\mathrm{jet}}} \frac{1}{A} \frac{\mathrm{d} n_{\mathrm{ch}} (\pt, r)}{\mathrm{d} \pt},
\end{align}
where $r = \sqrt{\Delta \eta^2 + \Delta \phi^2}$ is the angular distance from the jet axis and $N_{\mathrm{jet}}$ is the number of jets in consideration.
$A = \pi (r_{\mathrm{max}}^2 - r_{\mathrm{min}}^2) $ is the area of an annulus around the jet axis with its inner and outer radii $r_{\mathrm{min}}$ and $r_{\mathrm{max}}$ respectively and $n_{\mathrm{ch}}(\pt, r)$ is the number of charged particles with a given \pt\ within the annulus.
The boundaries of the annuli are given by: 0.0, 0.05, 0.1, 0.15, 0.2, 0.25, 0.3, 0.4, 0.5, 0.6, 0.7, 0.8.
The ratios of the charged-particle yields measured in \pbpb\ and \pp\ collisions,

\begin{align}
   \RDptr = \frac{\Dptr_\mathrm{Pb+Pb}}{\Dptr_{\pp}},
\end{align}
quantify the modifications of the yields due to the QGP medium.
Furthermore, the differences between the \Dptr\ distributions in \pbpb\ and \pp\ collisions, 

\begin{align}
   \Delta \Dptr = \Dptr_\mathrm{Pb+Pb} - \Dptr_{pp},
\end{align}
allow for measuring the absolute differences in charged-particle yields between the two collision systems.

The following sections describe all details of this analysis as follows:

\begin{itemize}
\item The datasets used for this analysis are discussed in Section~\ref{sec:used_data}
\item The different event selection criteria are discussed in \ref{sec:event_selection}
\item The cuts and corrections applied in the analysis procedure are discussed in  \ref{sec:cuts_corrections}
\item The various sources of systematic uncertainties and their sizes are discussed in \ref{sec:systematic}
\item The results of this analysis are described and interpreted in \ref{sec:results}
\end{itemize}


%\section{Definition of Measured Quantities}
%\label{sec:trkjet_corr_measurement}
%% !TEX encoding = UTF-8 Unicode
% !TEX root = thesis-ex.tex

The main quantity of interest here is the charged particle \pt\ distribution in and around the jet as illustrated in Fig.~\ref{Fig:dpt_def}.
The measured quantity is defined as:
  \begin{equation}
  \Dptr = \frac{1}{N_{\mathrm{jet}}} \frac{1}{\mathrm{A}} \frac{\mathrm{d} n_{\mathrm{ch}} (\pt, r)}{\mathrm{d} \pt},
\end{equation}

where $N_{\mathrm{jet}}$ is the number of jets in consideration, $A = \pi (r_{\mathrm{max}}^2 - r_{\mathrm{min}}^2) $ is the area of an annulus around the jet with its inner and outer radii $r_{\mathrm{min}}$ and $r_{\mathrm{max}}$.
The angular distance from the jet axis is given by $r = \sqrt{\Delta \eta^2 + \Delta \phi^2}$\footnote{$\Delta \eta$ and $\Delta \phi$ are the distances between the jet axis and the charged particle position in pseudorapidity and azimuth}, and $n_{\mathrm{ch}}(\pt, r)$ is the number of charged particles with a given \pt\ within the annulus.
The measurement is performed for the following successive intervals in $r$ around the jet, forming the annuli with inner and outer radii $r_{\textrm{min}}$ and $r_{\textrm{max}}$: 0.0, 0.05, 0.1, 0.15, 0.2, 0.25, 0.3, 0.4, 0.5, 0.6, 0.7, 0.8


\begin{figure}
\centerline{
\includegraphics[width=0.55\textwidth]{figures/main/general/fragScheme_Shape.pdf} }
\caption{Illustration of the tracks in and around the jet.
}
\label{Fig:dpt_def}
\end{figure}

These distributions are of interest because they indicate how the energy of the jet is lost both in and outside the jet in \pbpb\ collisions.
Similar measurements have been made by CMS~\cite{CMSPASHIN16020, Chatrchyan:2014ava}, and ATLAS~\cite{PhysRevC.98.024908, Aaboud:2017bzv}.

The entire analysis flow of this measurement, along with the various cuts and corrections (discussed in Sec.~\ref{sec:cuts_corrections}) is shown in Fig:\ref{Fig:analysis_flow} and briefly described in the following paragraph.

\begin{figure}
\centerline{
\includegraphics[width=20.cm]{figures/main/general/Shape_analyses_flow.pdf}}
\caption{The diagram presents various corrections and cuts that are applied during the analysis.}
\label{Fig:analysis_flow}
\end{figure}

First, the measured charged particle yield, $\text{d}n^{\text{meas}}_{\text{ch}}/\text{d}\pTch$, within an annulus with radii $r_{\text{min}}$ and $r_{\text{max}}$ is evaluated as:
\begin{equation}
\frac{\text{d}n^{\text{meas}}_{\text{ch}}}{\text{d}\pTch} = \frac{1}{\epsilon(\pttrk, \etatrk)} \frac{\Delta N_{\text{ch}} (\pTch, r)}{\Delta \pTch}
\end{equation}

where $\Delta N_{\text{ch}} (\pTch, r)$ is the number of charged particles in a given \pTch\ range that passed the jet and track selection criteria, $r = (r_{\text{min}} + r_{\text{max}}) / 2$, and $\epsilon(\pttrk, \etatrk)$ is the charged particles reconstruction efficiency correction, applied on a track-by-track basis.
In \pbpb\ collisions, the measured distributions are affected by charged particles from the underlying event, and thus need to be subtracted out (see Sec.~\ref{sec:cuts_corrections} for details):

\begin{equation}
\frac{\text{d}n^{\text{sub}}_{\text{ch}}}{\text{d}\pTch} = \frac{\text{d}n^{\text{meas}}_{\text{ch}}}{\text{d}\pTch} - \frac{\text{d}n^{\text{UE}}_{\text{ch}}}{\text{d}\pTch}
\end{equation}

The final \Dptr\ distributions are then evaluated after unfolding and normalizing with respect to the unfolded number of jets, $N_{\text{jet}}^{\text{unfolded}}$, as well as the area $A$ of the annulus at given distance $r$ :
\begin{equation}
\Dptr = \frac{1}{N_{\text{jet}}^{\text{unfolded}}} \frac{1}{\text{A}} \frac{\text{d}n^{\text{unfolded}}_{\text{ch}}}{\text{d}\pTch} \quad \quad \text{where } A = \pi (r_{\text{max}}^2 - r_{\text{min}}^2)
\end{equation}

The unfolding procedure is a combination of a two-dimensional Bayesian unfolding method in \ptjet\ and \pttrk, one-dimensional Bayesian unfolding method to correct jet spectra for the normalization and a one-dimensional bin-by-bin correction for the jet and track position resolution.


The analysis is performed differentially in \ptjet, and centrality, with the jet \pt\ bin size growing logarithmically with \ptjet\ to ensure good statistics in the full range of the measurement.
This scheme was also used in other ATLAS jet measurements~\cite{ATLAS276FFConf}.


In order to quantify the differences between charged particle spectra in \pbpb\ and \pp\  collisions, the ratios of the charged particle spectra in \pbpb\ collisions to those in \pp\ collisions are also reported:
\begin{equation}
   R_{\Dptr} \equiv \frac{\Dptr_{\pbpb}}{\Dptr{\pp}}
\end{equation}



%
%\section{Input Data}
%\label{sec:used_data}
%% !TEX encoding = UTF-8 Unicode
% !TEX root = thesis-ex.tex

The \PbPb\ and \pp\ data used in this analysis were recorded in 2015.
The data samples consisted of 25~pb$^{-1}$ of $\sqrts=5.02$ TeV \pp\ and 0.49~nb$^{-1}$ of $\sqrtsnn =5.02$ TeV \pbpb\ data.

%%L1: 10MHz to 100kHz
%%HLT: 100kHz to 1.5 kHz

Events in both the \pp\ and \pbpb\ samples were selected by the ATLAS Trigger system discussed in Chapter~\ref{sec:setup}.
The general scheme is to identify events using the Level 1 (L1) triggers, and pass them as ``seeds'' to the High Level Trigger (HLT).
In \pbpb, the selection was based on the L1 Total Energy trigger, \texttt{L1\_TE50} that identified events with at least 50 GeV in the calorimeter system.
These events were passed to the HLT, where the \texttt{HLT\_j75\_ion\_L1TE50} used an online jet reconstruction algorithm to select on jets above \mbox{75 GeV}.
In \pp, the event selection was done using a L1 jet trigger, \texttt{L1\_j20}, that used a simple sliding window algorithm to find jet candidates with a $\ptjet > 20$ GeV.
These were then used as seeds to the HLT, where the \texttt{HLT\_j85} trigger further selection on jets with $\ptjet > 85$ GeV.
The performance of the jet triggers in 2015 is described in Refs.~\cite{HITMF, Aaboud:2016leb} and the trigger efficiency is shown in Figure~\ref{fig:trigger_selections}.
This analysis then further selected jets with $\ptjet > 100$ GeV, thus ensuring a fully efficient trigger selection.

In addition to the jet triggered samples described above, a Minimum Bias \pbpb\ data sample was also recorded.
This was triggered based on a logical OR of the total energy trigger with a threshold of 50~\GeV\ and the ZDC coincidence trigger was used as part of the MC overlay procedure

%These were triggered using a logical OR of two triggers: 1) total energy Level-1 trigger selecting more central collisions; 2) ZDC coincidence trigger at Level-1 and a veto on the total energy trigger, with the additional requirement of least one track in the HLT, selecting peripheral collisions.


\begin{figure}
\centerline{
\begin{tabular}{cc}
\includegraphics[width=0.45\textwidth]{figures/main/general/Eff_pp_5TeV_central.pdf} &
\includegraphics[width=0.45\textwidth]{figures/main/general/trigger_eff_PbPb_CentInclusive.pdf}
\end{tabular}}
\caption{Jet trigger efficiencies for (left) \pp\ and (right) 0--80\% central \pbpb\ collisions at 5.02 TeV for R=0.4 offline jets.
The broader turn-on of the jet trigger in \pbpb\ compared to \pp\ collisions is caused by significant differences between the HI jet trigger reconstruction algorithm used at the time of the data taking and the current version of the offline reconstruction software.
Figure from Ref.~\cite{Sickles:2235420} }
\label{fig:trigger_selections}
\end{figure}


%\begin{figure}
%\begin{subfigure}{.5\textwidth}
%\includegraphics[width=1\textwidth]{figures/main/general/Eff_pp_5TeV_central.pdf}
%\caption{.}
%\label{fig:Trigger_pp5}
%\end{subfigure}
%\begin{subfigure}{.5\textwidth}
%\includegraphics[width=1\textwidth]{figures/main/general/trigger_eff_PbPb_CentInclusive.pdf}
%\caption{}
%\label{fig:Trigger_PbPb}
%\end{subfigure}
%\label{fig:trigger_selections}
%\caption{Jet trigger efficiencies for (left) \pp\ and (right) 0--80\% central \pbpb\ collisions at 5.02 TeV for R=0.4 offline jets.
%The broader turn-on of the jet trigger in \pbpb\ compared to \pp\ collisions is caused by significant differences between the HI jet trigger reconstruction algorithm used at the time of the data taking and the current version of the offline reconstruction software.
%Figure from Ref.~\cite{Sickles:2235420} }
%\end{figure}

In both samples, events were required to have a reconstructed vertex within 150~mm of the nominal IP along the beam axis.
The pileup was negligible in the \pbpb\ while the \pp\ data was collected in low pileup mode, where the average number of interactions per bunch crossing in \pp\ collisions ranged from 0.6 to 1.3.
Only events taken during stable beam conditions and satisfying detector and data-quality requirements that include the detector subsystems being in nominal operating conditions were considered.
The total number of \pp\ and \pbpb\ events entering the analysis, along with the with rejection power of various event quality cuts is shown in Figure~\ref{Fig:EventCounts}.
Some of these events are rejected by multiple cuts. ``Rejection by centrality'' indicates the number of events outside the 0-80\% centrality bin.

\begin{figure}
\centerline{
\begin{tabular}{cc}
\includegraphics[width=0.45\textwidth]{figures/main/general/EventAccept_pp.pdf} & 
\includegraphics[width=0.45\textwidth]{figures/main/general/EventAccept_PbPb.pdf}
\end{tabular}}
\caption{The number of 2015 \pp\  (left) and \PbPb\ (right) events used and rejected by various event quality cuts.}
\label{Fig:EventCounts}
\end{figure}



The centrality intervals used in this analysis were defined according to successive percentiles of the \ETfcal\ distribution obtained in minimum bias (MB) collisions, ordered from the most central (highest \ETfcal) to the most peripheral (lowest \ETfcal) collisions: 0--10\%, 10--20\%, 20--30\%, 30--40\%, 40--60\%, 60--80\%.

The \pp\ Monte Carlo (MC) used a set of $1.8\times10^7$ 5.02 TeV hard-scattering dijet \pp\ events generated with \powheg{}+\pythiaeight\ \cite{Nason:2004rx,Sjostrand:2014zea} using the A14 tune of parameters \cite{ATLAS2014021} and the NNPDF23LO PDF set \cite{Ball:2012cx}.
The \pbpb\ MC was generated by overlaying the additional sample of MB \pbpb\ data events on a separate set of $1.8\times10^7$ 5.02 TeV hard-scattering dijet \pp\ events generated with the same tune and PDFs as the \pp\ MC.
This ``MC overlay'' sample was reweighted on an event-by-event basis such that it had the same centrality distribution as the jet triggered sample.
Another sample of MB \pbpb\ events was generated using HIJING (version 1.38b) \cite{Wang:1991hta} and was only used to evaluate the track reconstruction performance.
The detector response in all MC samples was simulated using \textsc{Geant4} \cite{Agostinelli:2002hh,Aad:2010ah}.
These MC samples were used to evaluate the performance of the detector and analysis procedure and correct the measured distributions for detector effects.


%The event fraction as a function of run number for both the hard probes stream and the minimum bias overlay stream in \pbpb\ is shown in Figure~\ref{fig:evnt_fraction}
%
%\begin{figure}[h]
%\centering
%\includegraphics[width=0.5\textwidth]{figures/main/general/EventPercentages_c0.pdf}
%\caption{Event fraction as a function of runs for Hard Probes and the Minimum Bias Overlay Streams in \pbpb\ collisions.}
%\label{fig:evnt_fraction}
%\end{figure}

The time dependence of the underlying event (a core part of this measurement) was tested by dividing the data and MC into three data taking periods with approximately equal number of events in each period.
The underlying event determined for each period compared to the nominal underlying event evaluated for the entire dataset is shown in Figure~\ref{fig:weighted_runs}, and it can be seen that it is stable throughout the data taking period.

 \begin{figure}[h]
\centering
\includegraphics[width=0.75\textwidth]{figures/main/general/weightedRuns.pdf}
\caption{Stability of the underlying event for three different periods of the data taking.
The different curves indicate the ratio of the underlying event in each period of data taking to the underlying event determined in the entire dataset.}
\label{fig:weighted_runs}
\end{figure}






%
%\section{Event Selection }
%\label{sec:event_selection}
%% !TEX encoding = UTF-8 Unicode
% !TEX root = thesis-ex.tex

The standard ATLAS event quality requirements were applied for the event selection both for the \pp\ and \PbPb\ event selection.
\begin{itemize}
\item All the sub-detector systems were required to be fully functional: all the data were required to pass the official good run list:
 \\ $\texttt{\scriptsize data15\_5TeV.periodAllYear\_DetStatus-v75-repro20-01\_DQDefects-00-02-02\_PHYS\_HeavyIonP\_All\_Good.xml}$ (2015 \pp) 
 \\ $\texttt{\scriptsize data15\_5TeV.periodVdM\_DetStatus-v75-repro20-01\_DQDefects-00-02-02\_PHYS\_HeavyIonP\_All\_Good.xml}$ (2015, VdM \pp)
 \\ $\texttt{\scriptsize data15\_hi.periodAllYear\_DetStatus-v75-repro20-01\_DQDefects-00-02-02\_PHYS\_HeavyIonP\_All\_Good.xml } $ (2015 \pbpb).

\item All events are required to have a good reconstructed primary vertex.
\item The primary vertex must be within 150~mm from the center of ATLAS detector, as a fiducial tracking region.
 
\item Additional event cleaning to remove additional detector imperfections as described here~\cite{2015EventCleaning} is used. 
\item In \PbPb\ collisions the pileup contribution is removed using the  $\texttt{HIAnalysisTools}$ \cite{HIAnalysisTools}. 
\end{itemize}


Figures~\ref{Fig:EventCounts} presents the total number of \pp\ and \pbpb\ events, respectively, entering the analysis together with rejection power of various event quality cuts. A slightly higher fraction of empty events without primary vertex is observed in pp collisions. Some of these events are rejected by multiple cuts. ``Rejection by centrality'' indicates the number of event in HP stream that is outside the 0-80\% centrality bin.

\begin{figure}
\centerline{
\begin{tabular}{cc}
\includegraphics[width=0.45\textwidth]{figures/main/general/EventAccept_pp.pdf} & 
\includegraphics[width=0.45\textwidth]{figures/main/general/EventAccept_PbPb.pdf}
\end{tabular}}
\caption{
The number of 2015 \pp\  (left) and \PbPb\ (right) events used and rejected by various event quality cuts. }
\label{Fig:EventCounts}
\end{figure}


\subsection{Centrality Selection}
\label{sec:cent}

The centrality of the collision is a degree of the overlap of two colliding nuclei that can be quantified by the impact parameter that is the distance between the centers of the two nuclei. If they collide head on the collision is central, if they just graze each other we speak about peripheral collisions. We cannot measure the impact parameter to determine the centrality, but we can measure the overall event activity in the collision, characterized e.g. by the sum of \Et\ measured in FCal calorimeters on both sites. Central collisions have large \Et\ deposits in the FCal, peripheral have small \Et\ deposits.

In this analysis, The \ETfcal\ distribution is divided into percentiles of the total inelastic cross section for \PbPb\ collisions. The first percentile, $0-10\%$, represents the $10\%$ of collisions with the largest event activity, smallest impact parameter. The last percentile, $90-100\%$, represents the $10\%$ of collisions where there is the smallest event activity and largest impact parameter. 
Seven centrality classes have been used: 0-10\%, 10-20\%, 20-30\%, 30-40\%, 40-60\%, 60-80\%. 
The most peripheral collisions 80-100\%, are excluded due to  the small number of jets.
The centrality selections are documented in Ref.~\cite{ref:centrality}. The \PbPb\ MC is re-weighted in the way that it has the same centrality distribution as the jet triggered data sample.

\clearpage


%\section{Jet Reconstruction}
%\label{sec:reconstruction}
%% !TEX root = thesis-ex.tex

\label{Sec:JetRec}
For the measurement presented here, we use the jets reconstructed in the calorimeter 
using the \antikt\ algorithm \cite{Cacciari:2008gp} with \RFour.
The underlying event (UE) contribution to jets is subtracted on 
an event by event basis at the cell level. The details on the jet reconstruction 
procedure and performance in heavy ion collisions have been described in 
\cite{ATLAS-COM-PHYS-2011-1733}, here we will only shortly summarize the main 
features of the heavy ion jet reconstruction.

In order to reconstruct jets in heavy ion collisions, a large background from 
the UE has to be subtracted from each jet. 
The UE subtraction procedure is done in several iterative steps. 
First an estimate of the UE average transverse energy density, $\rho_i(\eta)$, 
is evaluated for each calorimeter layer $i$ in intervals of $\eta$ of width 
$\Delta \eta = 0.1$ using all cells in each calorimeter layer, within a given 
$\eta$ interval excluding those within $\Delta R < 0.4$ of ``seed'' jets. In the first 
subtraction step, the ``seed'' jets are defined to be jets reconstructed using the 
\antikt\ algorithm with \RTwo\ jets which have at 
least one tower  (a tower is a 0.1x0.1 region of the calorimeter and the energy
associated with it is the sum of the energies from all contributing calorimeter layers
in that region)
with $\Et > 3$~GeV and which have a ratio of the maximum to 
the mean tower associated with the jet of at least 4. 
  The UE-subtracted cell energies  were calculated according to:
\begin{equation}
\label{eqn:UE}
E_{\mathrm{T},i}(\eta, \phi)^{\mathrm{sub}} = E_{\mathrm{T},i}(\eta, \phi) - A_i \times \rho_i(\eta) 
\end{equation}
where $E_{\mathrm{T},i}$, $\eta$, $\phi$,  and $A_i$ represent the $\Et$, $\eta$, 
$\phi$, and area of the cell in the layer $i$. The $\rho_i(\eta)$ is the energy density per unit area in the layer $i$. The kinematics for \RTwo\ jets 
generated in this first subtraction step were calculated via a four-vector sum 
of all (assumed massless) cells contained within the jets using the \Et\ values 
obtained from Eq.~\ref{eqn:UE}.

The second subtraction step starts with the definition of a new set of 
seeds using a list of \RTwo\ calorimeter jets from the first 
subtraction step, each with $\Et > 4$~GeV. Using this new set of 
seeds, a new estimate of the UE, $\rho'_i(\eta)$, was calculated excluding 
cells within $\Delta R < 0.4$ of the new ``seed'' jets, where $\Delta R = \sqrt{ 
(\eta_{\mathrm{cell}} - \eta_{\mathrm{jet}})^2 + (\phi_{\mathrm{cell}} - \phi_{\mathrm{jet}})^2}$.


The jet energy scale calibration is based on the numerical inversion method and provides calibration constants for all jet collections used in this study~\cite{CalibrationTwiki}. The final jet energy calibration using in-situ studies is applied in the offline analysis and it is described in Sec~\ref{Sec:JetSelection}.   



The jet reconstruction performance in 5.02 TeV \pp\ collisions was evaluated using corresponding MC samples with a full detector simulation. The kinematics of the truth jets are reconstructed from primary particles\footnote{Primary particles are defined as having a mean lifetime of $\tau > 0.3 \times 10^{-10}$ s, and are produced directly in \pp\ interactions or from decays of particles with shorter lifetimes} with the \antikt\ algorithm with radius parameter $R = 0.4$. The jet reconstruction efficiency, JES (in this case evaluated as $\langle (\ETreco)\rangle/\ETtrue$), and JER for \pp\ collisions is shown in Fig.~\ref{Fig:Performancepp5} for \RFour\ jet.
For \pbpb\ collisions the JES is shown in Fig.~\ref{Fig:PerformancepbpbJES} and the JER is shown
in Figure~\ref{Fig:PerformancepbpbJER}.  Further studies of the jet performance in the 2015 \pbpb\
data are found in Ref.~\cite{Aad:2014bxa}. Figures~\ref{Fig:PerformancepbpbJPReta0p4}-\ref{Fig:PerformancepbpbJPRphi0p4} present the jet angular resolution in $\eta$ and $\phi$ as a function of jet \pt\ evaluated in six centrality classes. The angular resolution is improving with the increasing jet \pT\ and decreasing collision centrality. The angular resolution is found to be significantly better for smaller jets as expected since the smaller jets are less affected by the presence of the UE. 

\begin{figure}
\centerline{
\begin{tabular}{cc}
\includegraphics[width=7cm]{figures/main/figures_general/Eff_pp5.pdf} &
\includegraphics[width=7.3cm]{figures/main/figures_general/JES_pp5.pdf} \\
\includegraphics[width=7.3cm]{figures/main/figures_general/JER_pp5.pdf} 
\end{tabular}}
\caption{
Top panels: Jet reconstruction efficiency in 5.02 TeV \pp\ collisions (left) as a function of truth jet \pT\ and different $\eta$ bins. Jet energy scale (JES) in 5.02 TeV \pp\ collisions (right) as a function of truth jet \pT\ and different $\eta$ bins. Bottom panels: Jet energy resolution (JER) in 5.02 \pp\ collisions as a function of truth jet \pT\ and different $\eta$ bins.
}
\label{Fig:Performancepp5}
\end{figure}

\begin{figure}
   \centering
   \includegraphics[width = 0.75\textwidth]{figures/main/figures_general/PbPb_JES_pT_eta2p8.pdf}
   \caption{ JES in \pbpb\ collisions for eight centrality selections.  Plot is from Ref.~\cite{Aad:2014bxa}.}
   \label{Fig:PerformancepbpbJES}
\end{figure}

\begin{figure}
   \centering
   \includegraphics[width = 0.75\textwidth]{figures/main/figures_general/PbPb_JER_pT_eta2p8.pdf}
   \caption{ JER in \pbpb\ collisions for eight centrality selections.  Plot is from Ref.~\cite{Aad:2014bxa}. The points are fit to the standard function that describes the calorimetric resolution.}
   \label{Fig:PerformancepbpbJER}
\end{figure}


\begin{figure}
   \centering
   \includegraphics[width = 0.75\textwidth]{figures/main/figures_general/jet_res_eta_r04.pdf}
   \caption{ Jet angular resolution in $\eta$ for $R=0.4$ jets in \pbpb\ collisions as a function of jet \pT\ for six centrality selections.}
   \label{Fig:PerformancepbpbJPReta0p4}
\end{figure}

\begin{figure}
   \centering
   \includegraphics[width = 0.75\textwidth]{figures/main/figures_general/jet_res_phi_r04.pdf}
   \caption{ Jet angular resolution in $\phi$ for $R=0.4$ jets in \pbpb\ collisions as a function of jet \pT\ for six centrality selections.}
   \label{Fig:PerformancepbpbJPRphi0p4}
\end{figure}



\section{Basic Cuts and Corrections}
\label{sec:cuts_corrections}
\subsection{Overview}

In both the \pp\ and \pPb\ MC and data samples, two highest \pt\ jets are used to study azimuthal angular correlations. This measurement uses jets with a transverse momentum from 28~GeV to 90~GeV, in a \ystar\ range from -4.0 to 4.0. The final observables in this analysis are widths of di-jet \Dphi\ distributions and conditional yields. The widths are sensitive to broadening between the leading and sub-leading jets and the yields show the number of di-jets, given a leading jet in each \pT\ and \ystar\ kinematic region. 

The binning of this measurement is summarized in  Table~\ref{tab:binning} and is composed of different combinations of \ystarone, \ystartwo, \ptone, and \pttwo, where (\ystarone, \ptone) is the position and transverse energy of the leading jet, and (\ystartwo, \pttwo) the position and transverse energy of the sub-leading jet. Since the measurement aims to probe low-x partons, only the interval $2.7<\ystarone<4.0$, which is the proton going direction in \pPb\ is used. In the 2016 \pPb\ Transverse momentum binning was chosen on the edges of \pt\ intervals used for different triggers in \pp. 

Leading jet \ptone\ spectra are estimated in different \ystarone\ bins, unfolded, and used as a normalization of \Dphi\ distributions. Di-jet azimuthal angular correlation distributions are evaluated as a function of \Dphi\ in combinations of \ystarone, \ystartwo, \ptone, and \pttwo\ bins, unfolded, and normalized by the leading jet \pt\ spectra. The \Dphi\ distributions are fitted to extract the widths, which do not depend on the overall normalization. Conditional yields are obtained by integrating the \Dphi\ distributions over their full range so the correct normalization by number of leading jets is important. 

\begin{table}
	\centering
	\begin{tabular}{|| c | c | c || } 
		\hline
		\ptone Bins [GeV] & \pttwo Bins [GeV] & \ystartwo Bins \\ 
		\hline
		$28<\ptone<35$   & $28<\pttwo<35$  & $2.7<\ystarjet<4.0$ \\ 
		$35<\ptone<45$   & $35<\pttwo<45$  & $1.8<\ystarjet<2.7$ \\ 
		$45<\ptone<90$   & $45<\pttwo<90$  & $0.0<\ystarjet<1.8$ \\
						 & 				   & $-1.8<\ystarjet<0.0$ \\
						 &				   & $-4.0<\ystarjet<-1.8$ \\
		\hline
	\end{tabular}
	\caption{\label{tab:binning} Transverse momentum and \ystar\ binning for leading and sub-leading jets. For the leading jet, only the $2.7<\ystarone<4.0$ bin is used. }
\end{table}

To account for detector affects, the distributions in data have to be unfolded using MC information. The unfolding method used is the bin-by-bin unfolding which relies on MC information about the relationship between any truth and reconstructed quantity. This type of unfolding is sensitive to differences in the shapes of data and MC distributions and requires a re-weighting of the MC before unfolding factors can be evaluated. 

\subsection{Unfolding Procedure}
\label{sec:unfolding}
Due to effects of bin migration from JER and position resolution, it is necessary to perform an unfolding to account for these effects. Bayesian unfolding was first attempted, but the sensitivity to statistic fluctuations did not give good convergence. As a result, the bin-by-bin unfolding is the method used throughout the analysis. With this procedure, migration along multiple \ystar\ and \pT\ bins can be accounted for, more information can be found in Appendix~\ref{sec:appendixA}. Pairs of truth and reconstructed jets are used to fill the respective distributions and response matrices. The diagonal elements of these matrices represent pairs of truth and reconstructed jets agree in momentum and position intervals of the measurement. The response matrix is always a multidimensional object with twice the number of dimensions used in the phase space of the measurement. In \Dphi\ bins with index $i$, the correction factors $C_{i}$ are defined as  

\begin{eqnarray}
C_{i} = \frac{T_{i}}{R_{i}}
\label{eqn:factors}
\end{eqnarray}

where $T_{i}$ and $R_{i}$ are the number of truth and reconstructed di-jets, respectively.  Due to the fact that $T_{i}$ and $R_{i}$ are partially correlated, the resulting errors on the correction factors are defined as

\begin{eqnarray}
\delta C_{i}^{2} = \frac{T_{i}^{2}}{R_{i}^{3}}\bigg(1-\frac{M_{ii}^{2}}{T_{i}R_{i}}\bigg)
\label{eqn:factorserrors}
\end{eqnarray}

where $M_{ii}$ are the diagonal elements of the response matrix. These errors take into account the correlation between the truth and reconstructed quantities.

The bin-by-bin unfolding procedure is sensitive to the shapes of the distributions from which the correction factors are derived. This method works when the shape of the data distribution matches the shape of the MC distributions. Since both the spectra and \Dphi\ distributions are unfolded with correction factors, both the MC spectra and MC \Dphi\ distributions must first be re-weighted. The weights are estimated as ratios of distributions of $\mathrm{Data/MC_{Reco}}$. The value of the weight for a given truth and reconstructed jet pair is obtained from the truth jet kinematics. This procedure is done for all jet measurements and is motivated by the need to re-weight the prior (truth) distribution. Further, re-weighting using reconstructed kinematics could introduce inefficiency to the response matrix. In the following procedure, jet \pt\ spectra weights are derived first. Then \Dphi\ weights are derived with the spectra weight applied. With this intermediate re-weighting in jet \pt\ spectra, it is found that the \Dphi\ weights are invariant in \pT, allowing extrapolation into underflow and overflow bins in \pT, and reducing statistical fluctuations. Final \Dphi\ weights are derived only as a function of \Dphi\ in bins of \ystar, removing the \pT\ dependence. The product of spectra weights and the \Dphi\ weights is applied to the final MC distributions when deriving the correction factors.

From the re-weighted MC truth and reconstructed distributions, correction factors are derived and applied to data both for the spectra and \Dphi\ distributions. The unfolded \Dphi\ data distributions are scaled by the unfolded leading jet \pt\ spectra information, and fitted to the exponentially modified Gaussian function. 


\subsection{Jet Spectra}

Jets in \pp\ and \pPb\ data are required to have a trigger fired, and any jet(s) are required to be in the trigger's pseudorapidity range and transverse momentum interval where the trigger efficiency is above $99\%$. The jets are entered with prescale weights given by the ATLAS Lumi-Calc for each trigger and run. For the $2.7<\ystarone<4.0$ rapidity range, the contribution of different triggers to the final spectra is shown for \pp\ data in Figure~\ref{fig:ppspectrawithtrig}. The leading jet \pt\ spectra for \pp\ data are presented in different forward \ystar\ bins in Figure~\ref{fig:ppspectra} and for \pPb\ data in Figure~\ref{fig:pPbspectra}. In \pPb\ data, only one trigger with no pre-scale is used, thus, unlike the \pp\ spectra, where there are many trigger contributions, the final spectra is composed entirely of one trigger. The \pT\ binning is consistent with what is shown in Table~\ref{tab:binning} because these spectra will eventually be used for normalization of \Dphi\ distributions.

\begin{figure}
	\centering
	\includegraphics[width=0.65\textwidth]{output/output_pp_data/ystar_spect_All.pdf} 
	\caption{ Single-jet \pt\ spectra for jets in \pp\ data in bins of \ystar. }	
	\label{fig:ppspectra}
\end{figure}

\begin{figure}
	\centering
	\includegraphics[width=0.65\textwidth]{output/output_pp_data/ystar_spect_fine_40_Ystar1_27.pdf} 
	\caption{ Individual triggers, and resulting jet \pT\ spectra for \pp\ data for the $2.7<\ystarone<4.0$ rapidity range. }	
	\label{fig:ppspectrawithtrig}
\end{figure}

\begin{figure}	\centering
	\includegraphics[width=0.65\textwidth]{output/output_pPb_data/ystar_spect_All.pdf} 
	\caption{ Single-jet \pt\ spectra for jets in \pPb\ data in bins of pseudorapidity. }	
	\label{fig:pPbspectra}
\end{figure}

In MC, jet \pt\ spectra are filled separately for each cross setction weighted (JZx) sample, and then combined using the cross section weights and filtering efficiencies. Reconstructed and truth leading jet \pt\ spectra for the \pp\ MC are shown in Figure~\ref{fig:ppmcrecospectra} and for the \pPb\ MC in Figure~\ref{fig:pPbmcrecospectra}. 

\begin{figure}
	\centerline{
		\begin{tabular}{cc}
			\includegraphics[width=0.45\textwidth]{output/output_pp_mc_pythia8/ystar_spect_reco_All.pdf} & 
			\includegraphics[width=0.45\textwidth]{output/output_pp_mc_pythia8/ystar_spect_truth_All.pdf}  \\
		\end{tabular}
	}
	\caption{ Reconstructed  (left) and truth (right) level leading jet \pt\ spectra in \pp\ MC in bins of \ystar.}	
	\label{fig:ppmcrecospectra}
\end{figure}

\begin{figure}
	\centerline{
		\begin{tabular}{cc}
			\includegraphics[width=0.45\textwidth]{output/output_pPb_mc_pythia8/ystar_spect_reco_All.pdf} & 
			\includegraphics[width=0.45\textwidth]{output/output_pPb_mc_pythia8/ystar_spect_truth_All.pdf} \\
		\end{tabular}
	}
	\caption{ Reconstructed  (left) and truth (right) level leading jet \pt\ spectra in \pPb\ MC in bins of \ystar.} \label{fig:pPbmcrecospectra}
\end{figure}

\FloatBarrier
\subsection{Jet Spectra Re-weighting}
The leading jet \pt\ spectra weights in both the \pp\ and \pPb\ MCs are derived as the ratio of $Data/MC_{Reco}$ leading jet \pt\ spectra. Jet spectra with fine \pT\ binning are used to have better sensitivity to the shape. The data and MC leading jet \pt\ spectra with fine \pT\ binning are shown for \pp\ in Figure~\ref{fig:ppspectfine}, and for \pPb\ in Figure~\ref{fig:pPbspectfine}. The weights are derived by first scaling the Data and MC spectra to a common integral and then taking their quotient in bins of \ystar. The spectra weights are smoothed to avoid introducing statistical fluctuations. The smoothed \pp\ and \pPb\ leading jet \pt\ spectra weights as a function of \ptone\ are shown in Figure~\ref{fig:spectweights}.

\begin{figure}[ht]
	\centerline{
		\begin{tabular}{cc}
			\includegraphics[width=0.5\textwidth]{output/output_pp_data/ystar_spect_fine_All.pdf} &
			\includegraphics[width=0.5\textwidth]{output/output_pp_mc_pythia8/ystar_spect_fine_reco_All.pdf} \\
		\end{tabular}
	}
	\caption{Leading jet \pt\ spectra in fine bins if \pT\ for \pp\ data (left) and MC (right). }
	\label{fig:ppspectfine}
\end{figure}


\begin{figure}[ht]
	\centerline{
		\begin{tabular}{cc}
			\includegraphics[width=0.5\textwidth]{output/output_pPb_data/ystar_spect_fine_All.pdf} &
			\includegraphics[width=0.5\textwidth]{output/output_pPb_mc_pythia8/ystar_spect_fine_reco_All.pdf} \\
		\end{tabular}
	}
	\caption{Leading jet \pt\ spectra in fine bins if \pT\ for \pPb\ data (left) and MC (right). }
	\label{fig:pPbspectfine}
\end{figure}

\begin{figure}[ht]
	\centerline{
		\begin{tabular}{cc}
			\includegraphics[width=0.5\textwidth]{output/output_pp_mc_pythia8/h_spect_weights_All.pdf} &
			\includegraphics[width=0.5\textwidth]{output/output_pPb_mc_pythia8/h_spect_weights_All.pdf} \\
		\end{tabular}
	}
	\caption{Leading jet \pt\ spectra weights for \pp\ (left) and \pPb\ (right). Only the $2.7<\ystarone<4.0$ bin is used in the analysis but the other \ystarone\ bins are shown in \pp\ for comparison.}
	\label{fig:spectweights}
\end{figure}

The shape of the re-weighted reconstructed level MC jet spectra should match the shape of the reconstructed level jet spectra from data. To check this, reconstructed jet spectra from data are compared to reconstructed jet spectra before and after re-weighting in MC. The ratio of data to re-weighted MC is consistent with unity for \pp\ and \pPb\ reconstructed jet spectra as shown in Figure~\ref{fig:spectwithwithoutweight}.

\begin{figure}[ht]
	\centerline{
		\begin{tabular}{cc}
			\includegraphics[width=0.5\textwidth]{output/output_pp_data/hSpectMC_40_Ystar1_27.pdf} &
			\includegraphics[width=0.5\textwidth]{output/output_pPb_data/hSpectMC_40_Ystar1_27.pdf} \\
		\end{tabular}
	}
	\caption{Reconstructed level data (black) and re-weighted (red) and default (blue) reconstructed jet spectra from MC, with ratios. The ratio of re-weighted MC to data is consistent with unity for \pp\ (left) and \pPb\ (right). Shown for  $2.7<\ystarone<4.0$.}
	\label{fig:spectwithwithoutweight}
\end{figure}

Jet spectra are not re-weighted in \ystar\ because the effect from the JAR is much smaller than from JER and additionally, wide bins in rapidity are used. Response matrices for \pp\ and \pPb\ MC showing migration in \ystar\ are shown in Figure~\ref{fig:ystarrespmat}. There is very minor migration, with a purity of over 99\% indicating no change in the shape of the distribution as a function of \ystar.

\begin{figure}[ht]
	\centerline{
		\begin{tabular}{cc}
			\includegraphics[width=0.5\textwidth]{output/output_pp_mc_pythia8/h_yStarRespMat_28_Pt_35.pdf} &
			\includegraphics[width=0.5\textwidth]{output/output_pPb_mc_pythia8/h_yStarRespMat_28_Pt_35.pdf} \\
		\end{tabular}
	}
	\caption{Response matrices for \ystar, shown for \pp\ (left) and \pPb\ MCs. High purity indicates very minor effect on the shape of the distribution. Shown for the $28<\pt<35$ GeV interval.}
	\label{fig:ystarrespmat}
\end{figure}

\FloatBarrier
\subsection{Jet Spectra Unfolding}
To unfold the leading jet \pT\ spectra, the unfolding procedure described in~\ref{sec:unfolding} is used with correction factors obtained from the ratio the truth to reconstructed leading jet \pt\ spectra. The response matrix describes the bin migration between \pttruth\ and \ptreco. The \pp\ reconstructed and truth jet \pt\ spectra, with the response matrix and resulting correction factors are shown in Figure~\ref{fig:ppspectCFrespmat}. Similarly, the \pPb\ reconstructed and truth jet \pt\ spectra, with the response matrix and resulting correction factors are shown in Figure~\ref{fig:pPbspectCFrespmat}. The correction factors and ratios of unfolded to reconstructed MC are shown as a check that the unfolding procedure is working correctly, not as a check of closure.

\begin{figure}[ht]
	\centerline{
		\begin{tabular}{c}
			\includegraphics[width=0.6\textwidth]{output/output_pp_mc_pythia8/h_ystar_spect_unfolded_All_MUT_40_Ystar1_27.pdf} \\
			\includegraphics[width=0.6\textwidth]{output/output_pp_mc_pythia8/h_ystar_spect_respMat_All_40_Ystar1_27.pdf} \\
		\end{tabular}
	}
	\caption{ \pp\ MC reconstructed and truth jet \pt\ spectra distributions (top plot), the resulting correction factors (middle plot) and the \pT\ response matrix (bottom plot). }
	\label{fig:ppspectCFrespmat}
\end{figure}

\begin{figure}[ht]
	\centerline{
		\begin{tabular}{cc}
			\includegraphics[width=0.6\textwidth]{output/output_pPb_mc_pythia8/h_ystar_spect_unfolded_All_MUT_40_Ystar1_27.pdf} \\
			\includegraphics[width=0.6\textwidth]{output/output_pPb_mc_pythia8/h_ystar_spect_respMat_All_40_Ystar1_27.pdf} \\
		\end{tabular}
	}
	\caption{ \pPb\ MC reconstructed and truth jet \pt\ spectra distributions with correction factors (top plot), and the \pT\ response matrix (bottom plot). }
	\label{fig:pPbspectCFrespmat}
\end{figure}

\FloatBarrier
\subsection{Di-Jet Azimuthal Angular Distributions}
Distributions of the azimuthal angular correlations |\Dphi| of two jets are constructed from the leading and sub-leading jet kinematics. In \pp\ and \pPb\ data, a trigger is required, and the leading jet is required to be in the trigger's pseudorapidity and transverse momentum range. In the di-jet system there is a combinatoric contribution which can come from split jets or multi-parton scattering in both \pp\ and \pPb, as well as hard scattering \pPb. This is corrected for by fitting to a constant in the range $0<|\Dphi|<1$, and subtracting the result on the full range $0<|\Dphi|<\pi$. This is done at the reconstructed and truth levels in the same manner.	 The \Dphi\ distributions are then normalized by the leading jet \pt\ spectra counts, fitted to measure the widths, and integrated to measure the yields.

\subsection{ Re-weighting \Dphi\ Distributions }
The weights for \Dphi\ distributions in both \pp\ and \pPb\ MCs are derived as the ratios of Data to MC \Dphi\ distributions. This way, the \pT\ dependence of the \Dphi\ distributions can be eliminated and only residual differences in shapes of \Dphi\ distributions between Data and MC need to be accounted for. The \pp\ MC \Dphi\ weights in all combinations of \ptone\ and \pttwo\ and increasing bins in \ystartwo\ are shown in Figure~\ref{fig:ppIndividualDphiWeights} as a function of \Dphi. In such fine binning the weights have very high statistical fluctuations but they are invariant in \pT, so they can be combined and smoothed to form weights only only depending on \ystartwo, as shown in Figure~\ref{fig:ppAllDphiWeights}. The \pPb\ \Dphi\ weights are evaluated with the same method. The \pPb\ MC \Dphi\ weights in all combinations of \ptone\ and \pttwo\ in increasing bins in \ystartwo\ are shown in Figure~\ref{fig:pPbIndividualDphiWeights}, and the combined and smoothed weights are shown in Figure~\ref{fig:pPbAllDphiWeights}, all as a function of \Dphi.

\begin{figure}[ht]
	\centerline{
		\begin{tabular}{cc}
			\includegraphics[width=0.5\textwidth]{output/output_pp_mc_pythia8/cw_40_Ystar1_27_40_Ystar2_27.pdf} &
			\includegraphics[width=0.5\textwidth]{output/output_pp_mc_pythia8/cw_40_Ystar1_27_27_Ystar2_18.pdf} \\
		\end{tabular}
	}
	\caption{ \pp\ MC \Dphi\ weights shown for increasing bins of \ystartwo\ and all possible combinations of \ptone\ and \pttwo. Weights have high statistical fluctuations but are invariant in \pT. }
	\label{fig:ppIndividualDphiWeights}
\end{figure}

\begin{figure}[ht]
	\centerline{
		\begin{tabular}{cc}
			\includegraphics[width=0.5\textwidth]{output/output_pPb_mc_pythia8/cw_40_Ystar1_27_40_Ystar2_27.pdf} &
			\includegraphics[width=0.5\textwidth]{output/output_pPb_mc_pythia8/cw_40_Ystar1_27_27_Ystar2_18.pdf} \\
		\end{tabular}
	}
	\caption{ \pPb\ MC \Dphi\ weights shown for increasing bins of \ystartwo and all possible combinations of \ptone\ and \pttwo. Weights have high statistical fluctuations but are invariant in \pT. }
	\label{fig:pPbIndividualDphiWeights}
\end{figure}

\begin{figure}[ht]
	\centerline{
		\begin{tabular}{c}
			\includegraphics[width=0.75\textwidth]{output/output_pp_mc_pythia8/h_dPhi_weights_All.pdf}\\
		\end{tabular}
	}
	\caption{ \pp\ MC \Dphi\ weights for combined \pT\ bins, now shown only in bins of \ystartwo.  }
	\label{fig:ppAllDphiWeights}
\end{figure}

\begin{figure}[ht]
	\centerline{
		\begin{tabular}{c}
			\includegraphics[width=0.75\textwidth]{output/output_pPb_mc_pythia8/h_dPhi_weights_All.pdf}\\
		\end{tabular}
	}
	\caption{ \pPb\ MC \Dphi\ weights for combined \pT\ bins, now shown only in bins of \ystartwo.  }
	\label{fig:pPbAllDphiWeights}
\end{figure}

\FloatBarrier

To properly use re-weighting in the unfolding procedure, the re-weighted reconstructed MC and data distributions should have a similar shape. There is not expected to be a complete match between Data and re-weighted MC because the re-weighting is done as a function of truth kinematics. Comparisons of the re-weighted and default MC distributions to the data are shown in Figure~\ref{fig:ppweightscomp} for \pp\ and Figure~\ref{fig:pPbweightscomp} for \pPb. The ratio of the data to re-weighted MC is constant in \Dphi, indicating a consistent shape. The ratio is fitted in the same range as \Dphi\ distributions ($2.5<\Dphi<\pi$) to a constant, and in order to test fit quality, probability distributions of the fit results are shown for \pp\ and \pPb\ in Figure~\ref{fig:weightscompfitsflat}. The probability distributions are flat indicating a good fit to constant. 

\begin{figure}[ht]
	\centerline{
		\begin{tabular}{ccc}
			\includegraphics[width=0.33\textwidth]{output/output_pp_data/hMC_dPhi_40_Ystar1_27_28_Pt1_35_28_Pt2_35_40_Ystar2_27.pdf} &			\includegraphics[width=0.33\textwidth]{output/output_pp_data/hMC_dPhi_40_Ystar1_27_35_Pt1_45_28_Pt2_35_40_Ystar2_27.pdf} &
			\includegraphics[width=0.33\textwidth]{output/output_pp_data/hMC_dPhi_40_Ystar1_27_45_Pt1_90_45_Pt2_90_18_Ystar2_0.pdf} \\
		\end{tabular}
	}
	\caption{ \Dphi\ distributions for \pp\ data and MC. For MC, both re-weighted and default reconstructed distributions ares shown. The re-weighting makes the shapes flat in \Dphi\ as indicated by the constant ratio.}
	\label{fig:ppweightscomp}
\end{figure}

\begin{figure}[ht]
	\centerline{
		\begin{tabular}{ccc}
			\includegraphics[width=0.33\textwidth]{output/output_pPb_data/hMC_dPhi_40_Ystar1_27_28_Pt1_35_28_Pt2_35_40_Ystar2_27.pdf} &
			\includegraphics[width=0.33\textwidth]{output/output_pPb_data/hMC_dPhi_40_Ystar1_27_35_Pt1_45_28_Pt2_35_40_Ystar2_27.pdf} &
			\includegraphics[width=0.33\textwidth]{output/output_pPb_data/hMC_dPhi_40_Ystar1_27_45_Pt1_90_45_Pt2_90_18_Ystar2_0.pdf} \\
		\end{tabular}
	}
	\caption{ \Dphi\ distributions for \pPb\ data and MC. For MC, both re-weighted and default reconstructed distributions ares shown. The re-weighting makes the shapes flat in \Dphi\ as indicated by the constant ratio.}
	\label{fig:pPbweightscomp}
\end{figure}

\begin{figure}[ht]
	\centerline{
		\begin{tabular}{cc}
			\includegraphics[width=0.45\textwidth]{output/output_pp_data/h_probWeights_pp.pdf} &			\includegraphics[width=0.45\textwidth]{output/output_pPb_data/h_probWeights_pPb.pdf} \\
		\end{tabular}
	}
	\caption{ Probability distribution for constant fits to ratio of re-weighted reco MC to data \Dphi\ distributions. Shown for \pp\ (left) and \pPb\ (right) MCs. }
	\label{fig:weightscompfitsflat}
\end{figure}


\subsection{ Fitting of \Dphi\ Distributions } \label{sec:fitting}

The unfolded jet \pT\ spectra and $\mathrm{d}N_{1,2}(\Dphi)/\mathrm{d}\Dphi$ are further used to evaluate \conetwo\ distributions both in \pp\ and \pPb\ collisions. The \conetwo\ distributions are then fitted by an a double-exponential distribution smeared by a Gaussian function.  This fit function is obtained from a convolution of a double-exponential and a Gaussian:

\begin{eqnarray}
f(x) = \int_{-\infty}^{\infty}d\delta\frac{e^{-\delta^{2}/2\sigma^{2}}}{\sqrt{8\pi\sigma^{2}\tau^{2}}}e^{-|x-\delta|/\tau}.
\end{eqnarray}

Expanding the convolution of the Gaussian and double exponential functions, the resulting formula used in the analysis is:

\begin{eqnarray}
f(x) = A\frac{e^{\sigma^2/2\tau^2}}{2\tau}\bigg(\frac{1}{2}e^{\frac{x}{\tau}}Erfc\bigg(\frac{1}{\sqrt{2}}\bigg[\frac{x}{\sigma}+\frac{\sigma}{\tau}\bigg]\bigg)+e^{\frac{-x}{\tau}}\bigg[1-\frac{1}{2}Erfc\bigg(\frac{1}{\sqrt{2}}\bigg[\frac{x}{\sigma}-\frac{\sigma}{\tau}\bigg]\bigg)\bigg]\bigg)
\end{eqnarray} 

where $\tau$ is the inverse slope of the exponential component, $\sigma$ the width of the Gaussian distribution, and $A$ is the overall scaling factor. The widths of \conetwo\ distributions are calculated as 

\begin{eqnarray}
RMS(\conetwo) =  \sqrt{2\tau^2 + \sigma^{2}}.
\end{eqnarray}
%The fit function is not able to describe the \Dphi\ distributions in their full range, especially at large \Dphi\ away from $\pi$ due to multi-jets contributions which are not of interest to this analysis. 
where \conetwo\ is fitted in the interval $2.5<\Dphi<\pi$, similarly to the phase-space used in a previous di-jet measurement~\cite{Chatrchyan:2014hqa}.

%Di-jet azimuthal angular correlation distributions are fitted to an exponentially modified gaussion function. This fit function is obtained from a convolution of an exponential and a Gaussian, shown in Equation~\ref{eqn:conv}.

%Expanding the convolution of the Gaussian and exponential functions, the resulting formula used in the analysis is shown in Equation~\ref{eqn:fit}. 

%In the formula, the exponential component is $\tau$, the Gaussian component is $\sigma$, and $A$ is the overall multiplicative scaling factor.


%The fit function is not able to describe the \Dphi\ distributions in their full range, especially at large \Dphi\ away from $\pi$ due to multi-jets contributions which are not of interest to this analysis. The fit range is chosen from $2.5<\Dphi<\pi$, similar to the phase-space used in a previous di-jet transverse momentum balance measurement~\cite{Chatrchyan:2014hqa}. The resulting width, which is defined as $RMS =  \sqrt{2\tau^2 + \sigma^{2}}$, is then extracted and plotted in bins of \ystarone, \ystartwo, \ptone, and \pttwo.      

\FloatBarrier
\subsection{ Unfolding \Dphi\ Distributions }
When filling the truth and reconstructed distributions in either \pp\ or \pPb, the leading jet weights shown in Figure~\ref{fig:spectweights}, in addition to the \pT\ invariant \Dphi\ weights shown in Figures~\ref{fig:ppAllDphiWeights} and ~\ref{fig:pPbAllDphiWeights} for \pp\ and \pPb\ are applied as product. Using the re-weighted truth and reconstructed \Dphi\ distributions, along with the respective re-weighted response matrices, new correction factors are then derived using the bin-by-bin procedure described earlier. \Dphi\ distributions for truth, reconstructed, and unfolded \pp\ MC in two different bins of \ptone\ are shown in Figure~\ref{fig:ppUnfoldingMC}, along with the correction factors and respective response matrices. Similarly, two different \Dphi\ distributions for truth, reco, and unfolded \pPb\ MC distributions in two different bins of \ptone\ are shown in Figure~\ref{fig:pPbUnfoldingMC}, along with the correction factors and respective response matrices. 
All the \Dphi\ distributions from truth MC, unfolded reconstructed MC, and data, along with correction factors are shown in Appendix~\ref{sec:appendixB}.

\begin{figure}[ht]
	\centerline{
		\begin{tabular}{cc}
			\includegraphics[width=0.5\textwidth]{output/output_pp_mc_pythia8/h_dPhi_unfolded_All_MUT_40_Ystar1_27_28_Pt1_35_28_Pt2_35_40_Ystar2_27.pdf} &
			\includegraphics[width=0.5\textwidth]{output/output_pp_mc_pythia8/h_dPhi_unfolded_All_MUT_40_Ystar1_27_35_Pt1_45_28_Pt2_35_40_Ystar2_27.pdf} \\
			\includegraphics[width=0.5\textwidth]{output/output_pp_mc_pythia8/h_dPhi_respMat_All_40_Ystar1_27_28_Pt1_35_28_Pt2_35_40_Ystar2_27.pdf} &
			\includegraphics[width=0.5\textwidth]{output/output_pp_mc_pythia8/h_dPhi_respMat_All_40_Ystar1_27_35_Pt1_45_28_Pt2_35_40_Ystar2_27.pdf} \\
		\end{tabular}
	}
	\caption{ \pp\ MC truth, reconstructed, and unfolded \Dphi\ distributions for two different bins of \ptone, with correction factors (top row) and respective response matrices (bottom row). }
	\label{fig:ppUnfoldingMC}
\end{figure}

\begin{figure}[ht]
	\centerline{
		\begin{tabular}{ccc}
			\includegraphics[width=0.5\textwidth]{output/output_pPb_mc_pythia8/h_dPhi_unfolded_All_MUT_40_Ystar1_27_28_Pt1_35_28_Pt2_35_40_Ystar2_27.pdf} &
			\includegraphics[width=0.5\textwidth]{output/output_pPb_mc_pythia8/h_dPhi_unfolded_All_MUT_40_Ystar1_27_35_Pt1_45_28_Pt2_35_40_Ystar2_27.pdf} \\
			\includegraphics[width=0.5\textwidth]{output/output_pPb_mc_pythia8/h_dPhi_respMat_All_40_Ystar1_27_28_Pt1_35_28_Pt2_35_40_Ystar2_27.pdf} &
			\includegraphics[width=0.5\textwidth]{output/output_pPb_mc_pythia8/h_dPhi_respMat_All_40_Ystar1_27_35_Pt1_45_28_Pt2_35_40_Ystar2_27.pdf} \\
		\end{tabular}
	}
	\caption{ \pPb\ MC truth, reconstructed, and unfolded \Dphi\ distributions for two different bins of \ptone, with correction factors (top row) and respective response matrices (bottom row). }
	\label{fig:pPbUnfoldingMC}
\end{figure} 

\FloatBarrier
\subsection{MC Closure Test}
As a check, the MC reconstructed results are unfolded using the derived correction factors. The comparison of the \pp\ MC truth and unfolded widths, and the respective ratios are shown in Figure~\ref{fig:ppwidthsTruthUF} in bins of \ptone\ and \pttwo. The ratios between unfolded and truth results are consistent with unity within statistical uncertainties indicating there is good closure between the unfolded and truth results.  Similarly, comparison of the \pPb\ MC truth and unfolded widths, and the respective ratios are shown in Figure~\ref{fig:pPbwidthsTruthUF} in bins of \ptone\ and \pttwo. The ratios between unfolded and truth results are consistent with unity within statistical uncertainties indicating there is good closure between the unfolded and truth results.  

The comparison of the \pp\ MC truth and unfolded yields, and the respective ratios are shown in Figure~\ref{fig:ppyieldsTruthUF} in bins of \ptone\ and \pttwo. The ratios between unfolded and truth results are consistent with unity within statistical uncertainties indicating there is good closure between the unfolded and truth results.  Similarly, comparison of the \pPb\ MC truth and unfolded yields, and the respective ratios are shown in Figure~\ref{fig:pPbyieldsTruthUF} in bins of \ptone\ and \pttwo. The ratios between unfolded and truth results are consistent with unity within statistical uncertainties indicating there is good closure between the unfolded and truth results.  

\begin{figure}[ht]
	\centerline{
		\begin{tabular}{ccc}
			\includegraphics[width=0.33\textwidth]{output/All/pp_mc_pythia8_0/h_dPhi_width_40_Ystar1_27_28_Pt1_35.pdf} &
			\includegraphics[width=0.33\textwidth]{output/All/pp_mc_pythia8_0/h_dPhi_width_40_Ystar1_27_35_Pt1_45.pdf} &
			\includegraphics[width=0.33\textwidth]{output/All/pp_mc_pythia8_0/h_dPhi_width_40_Ystar1_27_45_Pt1_90.pdf} \\
		\end{tabular}
	}
	\caption{ Comparison of widths from \Dphi\ fits between unfolded and truth results for the \pp\ MC. Ratios are consistent with unity, indicating good unfolding closure. }
	\label{fig:ppwidthsTruthUF}
\end{figure}


\begin{figure}[ht]
	\centerline{
		\begin{tabular}{ccc}
			\includegraphics[width=0.33\textwidth]{output/All/pPb_mc_pythia8_0/h_dPhi_width_40_Ystar1_27_28_Pt1_35.pdf} &
			\includegraphics[width=0.33\textwidth]{output/All/pPb_mc_pythia8_0/h_dPhi_width_40_Ystar1_27_35_Pt1_45.pdf} &
			\includegraphics[width=0.33\textwidth]{output/All/pPb_mc_pythia8_0/h_dPhi_width_40_Ystar1_27_45_Pt1_90.pdf} \\
		\end{tabular}
	}
	\caption{ Comparison of widths from \Dphi\ fits between unfolded and truth results for the \pPb\ MC. Ratios are consistent with unity, indicating good unfolding closure. }
	\label{fig:pPbwidthsTruthUF}
\end{figure}

\begin{figure}[ht]
	\centerline{
		\begin{tabular}{ccc}
			\includegraphics[width=0.33\textwidth]{output/All/pp_mc_pythia8_0/h_dPhi_yield_40_Ystar1_27_28_Pt1_35.pdf} &
			\includegraphics[width=0.33\textwidth]{output/All/pp_mc_pythia8_0/h_dPhi_yield_40_Ystar1_27_35_Pt1_45.pdf} &
			\includegraphics[width=0.33\textwidth]{output/All/pp_mc_pythia8_0/h_dPhi_yield_40_Ystar1_27_45_Pt1_90.pdf} \\
		\end{tabular}
	}
	\caption{ Comparison of yields from \Dphi\ distributions between unfolded and truth results for the \pp\ MC. Ratios are consistent with unity, indicating good unfolding closure. }
	\label{fig:ppyieldsTruthUF}
\end{figure}

\begin{figure}[ht]
	\centerline{
		\begin{tabular}{ccc}
			\includegraphics[width=0.33\textwidth]{output/All/pPb_mc_pythia8_0/h_dPhi_yield_40_Ystar1_27_28_Pt1_35.pdf} &
			\includegraphics[width=0.33\textwidth]{output/All/pPb_mc_pythia8_0/h_dPhi_yield_40_Ystar1_27_35_Pt1_45.pdf} &
			\includegraphics[width=0.33\textwidth]{output/All/pPb_mc_pythia8_0/h_dPhi_yield_40_Ystar1_27_45_Pt1_90.pdf} \\
		\end{tabular}
	}
	\caption{ Comparison of yields from \Dphi\ distributions between unfolded and truth results for the \pPb\ MC. Ratios are consistent with unity, indicating good unfolding closure. }
	\label{fig:pPbyieldsTruthUF}
\end{figure}


As an additional closure test, the jet \pT\ spectra and \Dphi\ correction factors derived from the \pythiaeight\ MC were applied to reconstructed jets from the \herwig\ MC. A comparison of unfolded to truth \conetwo\ and \ionetwo\ fromthe \pp\ \herwig\ are shown in Figure~\ref{fig:herwigpythiaclosure}. For \pPb\ there is no additional MC so this test was only done on the \pp\ MC. Ratios of unfolded to truth distributions indicate good closure. From Tables~\ref{tab:mcsamplespp},~\ref{tab:mcsamplesppherwig} it is clear that the statistics in the \pp\ \herwig\ MC is roughly 50\% of the \pp\ \pythiaeight\ MC, and the resulting fluctuations can be taken as statistical. 


\begin{figure}[ht]
	\centerline{
		\begin{tabular}{ccc}
			\includegraphics[width=0.33\textwidth]{output/All/pp_mc_herwig_0/h_dPhi_width_40_Ystar1_27_35_Pt1_45.pdf} &
			\includegraphics[width=0.33\textwidth]{output/All/pp_mc_herwig_0/h_dPhi_yield_40_Ystar1_27_35_Pt1_45.pdf} \\
		\end{tabular}
	}
	\caption{ Comparison of \conetwo\ (left) and \ionetwo\ (right) between unfolded and truth results for the \pp\ \herwig\ MC. Unfolding is done using correction factors derived from the \pythiaeight\ MC. Ratios are consistent with unity, indicating good unfolding closure. }
	\label{fig:herwigpythiaclosure}
\end{figure}

\FloatBarrier

\section{Systematic Uncertainties}
\label{sec:systematic}
% !TEX encoding = UTF-8 Unicode
% !TEX root = thesis-ex.tex

This section gives an overview of the sources of systematic uncertainties on the \pp\ and \pbpb\ charged particle spectra associated with jet.
These include:

\begin{itemize}
\item Jet energy scale
\item Jet energy resolution
\item Tracking selections
%\item Truth track definition
%\item Detector material description in simulation
%\item Tracking in dense environments
%\item Fake track subtraction
%\item Track momentum
\item Unfolding
\item Underlying event contribution
\item MC non-closure
\end{itemize}

The systematic uncertainties are evaluated separately for \Dptr\ distributions and for their ratios as a function of jet \pT\ for \pp\ and \pbpb\ collisions.
For each systematic variation, the entire analysis procedure is repeated to ensure that the jets are treated in a consistent manner throughout the analysis.
The positive relative shift was used to calculate the upper bound of the systematic uncertainty, whereas the negative relative shift was used to calculate the lower bound.
All uncertainties except the unfolding and the MC non-closure are assumed to be correlated and are evaluated by comparing the \Rdptr\ distributions for the various systematic variations to the nominal \Rdptr\ distribution.
For uncorrelated systematic uncertainties, the uncertainty on the \RDptr\ distribution is evaluated by adding the uncertainties on the \pp\ and \pbpb\ \Dptr\ distributions in quadrature.
The total systematic uncertainties on the \Rdptr\ distributions for a selection of track \pt\ ranges (1.0--1.6 \GeV, 2.5--4.0 \GeV, 6.3--10 \GeV) in jets with \pt\ in the 126--158 \GeV\ range are shown in Figures~\ref{fig:rdptr_sys_uncert1} and \ref{fig:rdptr_sys_uncert2}. 
% Figure~\ref{fig:rdptr_sys_uncert1}-\ref{fig:rdptr_sys_uncert2}.
%The systematic uncertainties for other jet \pT\ intervals as show in appendix \ref{sec:appendixA}.



\begin{figure}
\centering
\begin{subfigure}[b]{\textwidth}
    \centering
    \includegraphics[page=1, width=\textwidth]{figures/main/systematics/Summary_ChPS_dR_sys_PbPb_error}
    \caption{}
    \label{fig:rdptr_sys_uncert1a}
\end{subfigure} \\
\begin{subfigure}[b]{\textwidth}
    \centering
    \includegraphics[page=3, width=\textwidth]{figures/main/systematics/Summary_ChPS_dR_sys_PbPb_error}
    \caption{}
    \label{fig:rdptr_sys_uncert1b}
\end{subfigure}\hfill
   \caption{A summary of the systematic uncertainties on \RDptr\ distributions for different track \mbox{$1.0 < \pt < 1.6$ GeV} (top) and \mbox{$2.5 < \pt < 4.0$ GeV} (bottom), for jets with \pt\ 126--158 \GeV, as a function of \rvar\ for different centrality bins.
Different panels are different centrality bins.
The total systematic uncertainty and its individual contributions are shown.}
\label{fig:rdptr_sys_uncert1}
\end{figure}


\begin{figure}
\centering
\begin{subfigure}[b]{\textwidth}
    \centering
    \includegraphics[page=5, width=\textwidth]{figures/main/systematics/Summary_ChPS_dR_sys_PbPb_error}
    \caption{}
    \label{fig:rdptr_sys_uncert2a}
\end{subfigure} \\
\begin{subfigure}[b]{\textwidth}
    \centering
    \includegraphics[page=6, width=\textwidth]{figures/main/systematics/Summary_ChPS_dR_sys_PbPb_error}
    \caption{}
    \label{fig:rdptr_sys_uncert2b}
\end{subfigure}\hfill
   \caption{A summary of the systematic uncertainties on \RDptr\ distributions for different track \mbox{$6.3 < \pt < 10.0$ GeV} (top) and \mbox{$10.0 < \pt < 25.1$ GeV} (bottom), for jets with \pt\ 126--158 \GeV, as a function of \rvar\ for different centrality bins.
Different panels are different centrality bins.
The total systematic uncertainty and its individual contributions are shown.}
\label{fig:rdptr_sys_uncert2}
\end{figure}

\subsection{Jet energy scale uncertainty}

The uncertainty on the JES for heavy ion jets has two parts.
The first is taken from \pp\ JES uncertainties for jets in \pp\ collisions while the second is specific to the heavy ion jets.
For the \pp\ part we use the strongly reduced set of 4 nuisance parameters using Scenario 1 as described in Ref.~\cite{JESuncertaintytwiki}.
Nuisance parameters that are not applicable for HI jet collections (pileup, b-jets, flavor and MC non closure) are removed or replaced (flavor uncertainties).
The heavy ion specific components are from the cross calibration~\cite{cc2015} and the jet flavor uncertainties at 5.02~TeV~\cite{2015392}.
For each component of the variation the response matrices are regenerated with the shifted \ptjet:

\begin{equation}
\pT^{\star,\mathrm{reco}} = \pT^{\mathrm{reco}} (1\pm U^{\mathrm{JES}}(\pT , \eta)).
\end{equation}
The data is then re-unfolded with these response matrices and the variation in the fragmentation functions is taken as the systematic uncertainty.

The centrality dependent uncertainty on the JES was evaluated by shifting the jet \pt\ of all measured jets up and down by shift between 0\% and 0.5\%.
The magnitude of the shift depends on the centrality in the way that the uncertainty on the jet \pt\ is 0.5\% in 1\% most central collisions and than linearly decreases to 0\% in 60\% peripheral bin.
The size of the shift reflects the uncertainty on the JES evaluated as using the $r-$track study where the sum of \pT\ of the tracks associated to a reconstructed jet is compared to the reconstructed jet \pT\ in ratio that is than compared between PbPb data and MC~\cite{HIjesnote,Aad:2014bxa}.



\subsection{Jet energy resolution}
To account for systematic uncertainties due to disagreement between the jet energy resolution in data and MC, the unfolding procedure was repeated with a modified response matrix.
The matrix was generated by repeating the MC study with modifications to the $\Delta \pt$ for each matched truth-reconstructed jet pair.
The procedure to generate modified migration matrices follows the standard procedure applied in \pp\ jet measurements and is used for both the \pp\ and \pbpb\ collisions.
The $\texttt{JetEnergyResolutionProvider}$ tool~\cite{JERUncertaintyProviderRun2} was used to retrieve uncertainty on the fractional resolution, $\sigma^{\mathrm{syst}}_{\mathrm{JER}}$ as a function of jet $\pt$ and $\eta$.
An additional HI jet specific uncertainty from the cross calibration of the HI jet collections~\cite{cc2015} is applied to jets in both \pp\ and \pbpb\ collisions.
The full JER uncertainty on 2015 \pp\ data is shown also in Ref.~\cite{Aad:1696485}
The jet $\pt^{\mathrm{reco}}$ was then smeared by

\begin{align}
\pt^{\star, \mathrm{reco}} = \pt^{\mathrm{reco}}\times \mathcal{N}(1,\sigma^{\mathrm{eff}}_{\mathrm{JER}})\,,
\end{align}
where $\mathcal{N}(1,\sigma^{\mathrm{eff}}_{\mathrm{JER}})$ is the normal distribution with the effective resolution $\sigma^{\mathrm{eff}}_{\mathrm{JER}}=\sqrt{(\sigma_{\mathrm{JER}} + \sigma^{\mathrm{syst}}_{\mathrm{JER}})^{2} - \sigma_{\mathrm{JER}}^{2}}$.

%%%%%%%%%%%%%%%%%
%The systematic uncertainties on the \Dptr\ distributions decreases with decreasing \pt\ and increasing jet \pT.
%The typical systematic uncertainty originating from JER changes varies from 10\% to 1\% depending on the jet \ET\ and $z$.
%%%%%%%%%%%%%%%%%


\subsection{Tracking selections}
\paragraph{Track selection}
This uncertainty was estimated by tightening the tracking cuts by adding the cuts on the significance of $d_0$ and $z_0$ as described in the Section~\ref{sec:trackselection}. 
The entire analysis is redone with these track selections (including re-deriving the tracking efficiencies and the $\eta-\phi$ maps for the UE estimation) and the difference from the nominal analysis is taken as the systematic uncertainty.

\paragraph{Truth track definition}  
This uncertainty quantifies the robustness of the matching of reconstructed to truth particles.
The uncertainty is taken as a difference in the final results obtained with  $\mcprob > 0.3$ and results obtained with $ \mcprob > 0.5$.
This systematic included a re-derivation of the $\eta-\phi$ maps for UE estimation.
%The change in tracking efficiency for these two selections is negligible.

\paragraph{Detector material description in simulation}
The uncertainty on the inner detector material varies with \pttrk\ and \etatrk\ from 0.5\% to 2.0\%~\cite{ref:tracktwiki} on the efficiency correction.
This systematic also included a re-derivation of the $\eta-\phi$ maps for UE estimation.

\paragraph{Tracking in dense environments}
There is a 0.4\% uncertainty on the efficiency due to tracking in dense environments (the core of the jet)~\cite{ref:tracktwiki}.
This systematic also included a re-derivation of the $\eta-\phi$ maps for UE estimation.

\paragraph{Fake rate and secondaries}
The uncertainty on the rate of fake tracks and secondaries is taken to be 30\% independent of \pttrk\ and \etatrk~\cite{ref:tracktwiki, Nachman:2259091}.
This uncertainty is conservatively symmetrized.

\paragraph{Uncertainty on the track momentum}
To account for a possible misalignment in \pp\ and \PbPb\ data, the reconstructed \pT\ of each track (corrected first as described in section~\ref{Sec:Trackmomentumcorrection}) was changed according to~\cite{TrackingRec}:

\begin{equation}
\pt \rightarrow \pt \times (1 + q \times \pt \delta_{sagitta}(\eta, \phi))^{-1},
\end{equation}
where $q$ is charge of the track and $\delta_{sagitta}(\eta, \phi)$ is uncertainty on the track curvature.
The uncertainty derived for 5.02~TeV \pp\ and \PbPb\ data is included in InDetTrackSystematicsTools-00-00-19.
Due to statistical origin of the uncertainty the resulting systematic uncertainty is symmetrized.
This systematic also included a re-derivation of the $\eta-\phi$ maps for UE estimation.

%%%%%%%%%%%%%%%%%
%The resulting systematic uncertainty is $<<1$\% for low and intermediate $z$ and \pT\ and reaches up to 4\% at high $z$.
%As the source of the shift is present both in \pp\ and \PbPb\ it does partially cancel in the ratios. 
%%%%%%%%%%%%%%%%%


\subsection{Systematic uncertainty due to unfolding}
The systematic uncertainty associated with the unfolding is connected with the sensitivity of the unfolding procedure to the choice of the input distributions.
The systematic is evaluated by generating response matrices from the MC distributions without the reweighting factor that is used to match the jet spectrum and \Dptr\ distributions in data, and then unfolding the data using these response matrices.
This has minimal effect on track \pt\ because of the good track momentum resolution in the kinematic region of interest.
The uncertainty is evaluated by comparing the nominal result with the un-reweighed result, and is considered to be uncorrelated between \pbpb\ and \pp.


\subsection{Systematic uncertainty due to the UE event subtraction}
The systematic uncertainty associated with the estimation of the UE has two main components: one is the statistical uncertainty on the $\eta-\phi$ maps used in the map method (described in section~\ref{sec:map_method}), and the other is the comparison of the map method to the alternative cone method (discussed in section~\ref{sec:cone_method}.
More details on the cone method can be found in Ref.~\cite{PhysRevC.98.024908}.
The contributions of both components to the underlying event uncertainty can be seen in Figure~\ref{fig:UE_sys_contrib}, with the uncertainty from the map statistic dominating in central collisions.
The uncertainty on the underlying event convolutes with the signal to background ratio to produce the uncertainty on the charged particle spectra.

\begin{figure}
\centering
\includegraphics[page=1,width=1.\textwidth]{figures/main/systematics/Summary_UE_RDpT_dR_sys_error}
\caption{Size of the individual contributions to the underlying event systematic uncertainty as a function for \rvar\ for 0-10\% \pbpb\ collisions, in 126-158 GeV jets, 1-1.6 GeV tracks.}
\label{fig:UE_sys_contrib}
\end{figure}


\paragraph{Uncertainty from map statistic:} 
The $\eta-\phi$ maps used in the estimation of the underlying event are sparsely populated for high track \pt\ and high \ptjet, and are susceptible to statistical fluctuations.
To take this into account, 100 pseudo-experiments are conducted to re-estimate the set of maps, with a bin-by-bin gaussian variation where the mean and standard deviation were taken to be the bin content and bin error from the nominal set of maps.
The distribution of the relative difference between each estimation of the shifted underlying event and and the nominal value is fit to a gaussian.
The width of this gaussian is taken to be the systematic uncertainty.
This uncertainty is symmetrized to be conservative.
A few examples of the distribution of normalized relative differences can be seen in Figure~\ref{fig:gaus_diff}.
The size of the systematic from this can be seen in Fig.\ref{fig:mapstat_corr}.


\begin{figure}
\begin{subfigure}{0.5\textwidth}
\centering
\includegraphics[width=1\textwidth]{figures/main/systematics/map_stat_gaus}
\caption{}
\label{fig:gaus_diff}
\end{subfigure}
\begin{subfigure}{0.5\textwidth}
\centering
\includegraphics[width=1\textwidth]{figures/main/systematics/map_stat_size}
\caption{}
\label{fig:mapstat_corr}
\end{subfigure}
\caption{(Left) An example of the relative difference between the nominal and shifted values of the UE, fit to a gaussian. The width is taken as the systematic uncertainty.
Wider distributions larger statistical uncertainty on the bin content in the $\eta-\phi$ map used to estimate the UE.
(Right) Size of the systematic uncertainty from the map statistic component, as a function for \pttrk\ and \ptjet\ for 0-10\% \pbpb\ collisions, $0.15 < r < 0.20$ away from the jet axis.}
\end{figure}

%\begin{figure}[h]
%\centering
%\includegraphics[width=0.75\textwidth]{figures/main/systematics/map_stat_size}
%\caption{Size of the systematic uncertainty from the map statistic component, as a function for \pttrk\ and \ptjet\ for 0-10\% \pbpb\ collisions, $0.15 < r < 0.20$ away from the jet axis.}
%\label{fig:mapstat_corr}
% \end{figure}


\paragraph{Uncertainty from cone method: } The difference between the UE from the two methods is discussed in section \ref{sec:cone_method} and is shown in Figure~\ref{fig:conemethod_mapmethod}.
The effect of the different UE estimation methods on the charged particle spectra is seen in Fig.\ref{fig:conemethod_chps_comparison}.
This uncertainty is conservatively symmetrized.
While the absolute size of the uncertainty on the UE is typically small, the small signal-to-background ratio makes this the dominant systematic uncertainty in central collisions for lowest \pT\ tracks and large \rvar.

\begin{figure}
\centerline{\includegraphics[page=2,width=1.\textwidth]{figures/main/systematics/ChPS_UE_Comparison}}
\caption{Ratio of the charged particle spectra as determined using two different UE estimation methods as a function for \rvar\ for 0-10\% \pbpb\ collisions in 126-158 GeV jets and 1-1.6 GeV tracks.
Deviations from unity are a combination of the difference between the two methods and the signal to background ratio.
The largest differences between the spectra are seen at large \rvar, where the signal to background is the smallest.
Points are offset along the x-axis for ease of viewing.}
\label{fig:conemethod_chps_comparison}
\end{figure}




\subsection{MC non-closure}
To make sure that all the sources of systematic uncertainties were covered, the systematic uncertainty from the non closure in the MC was also evaluated.
It was calculated using the technical closure (done using non-reweighed response matrices) between the fully corrected and reconstructed charged particle distributions in MC to the charged particle distributions evaluated at the truth level.
This uncertainty can be considered a measure of unknowns in the analysis, but it also includes fluctuations due to the finite statistics in the MC which are used to evaluate it (especially in high \pttrk\ regions of the analysis.
The non-closure can be seen in Figure~\ref{fig:pbpbclosure}.
The systematic uncertainty is taken to be uncorrelated between \pbpb\ and \pp 

\begin{figure}
\centerline{\includegraphics[page=1,width=1.\textwidth]{figures/main/systematics/ChPS_final_dR_PbPb_MC.pdf}}
\caption{Size of the non-closure as a function for \rvar\ for 0-10\% \pbpb\ collisions, in 126-158 GeV jets for different \pttrk\ ranges.
Points in the bottom panel are offset along the x-axis for ease of viewing.}
\label{fig:pbpbclosure}
\end{figure}



\subsection{Correlations between the systematic uncertainties in \pbpb\ and \pp\ collisions}
Due to the common analysis and reconstruction procedure, and detector conditions, the systematic uncertainties are correlated between the \pp\ and \pbpb\ collisions in most cases.
Table~\ref{tab:systematics} summarizes correlations between \pp\ and \PbPb\ and also point-to-point correlations of individual distributions.
The unfolding uncertainty is uncorrelated between the two systems because it comes from the sensitivity of the unfolding to the starting MC distribution.
In \pbpb\ collisions where the fragmentation is modified by the presence of the QGP, this sensitivity could be different than in \pp\ collisions where the fragmentation functions are quite similar to those in \pythiaeight~\cite{201865}.
The impact of the modification of the fragmentation process in \PbPb\ compared to \pp\ and MC simulations is account for in the HI specific data-driven and centrality dependent uncertainty on the JES.

\begin{table}[h]
\centering
\begin{tabular}{ | m{3cm} | m{3cm} | m{3cm} | m{3cm} |}
\hline
\textbf{Uncertainty} & \textbf{\pp\ and \PbPb\ correlated} & \textbf{Point-to-point correlated} & \textbf{One/two sided or symmetrized} \\ \hline
JES (\pp) & yes & yes & two sided \\ \hline
JES (HI) & no & yes & two sided \\ \hline
JER & yes & yes & symmetrized \\ \hline
Track selection & yes & yes & one sided \\ \hline
\mcprob & yes & yes & one sided \\ \hline
Material & yes & yes & one sided \\ \hline
Dense environment & yes & yes & one sided \\ \hline
Fake rate & yes & yes & symmetrized \\ \hline
Track momentum & yes & no & two sided \\ \hline
Unfolding & no & yes & symmetrized \\ \hline
UE subtraction & no & yes & symmetrized \\ \hline
MC non-closure & no & no & symmetrized \\ \hline
\end{tabular}
\caption{Summary of correlation of different systematic uncertainties.}
\label{tab:systematics}
\end{table}

In the case where the systematic uncertainties are correlated, we evaluate \Rdptr\ ratios using the systematic variation from the nominal distributions in both \pp\ and \pbpb.
The variation in the ratio is used as the systematic uncertainty.
The variations in the ratios are summed in quadrature to get the total systematic uncertainty on the ratio.




\section{Results}
\label{sec:results}
% !TEX root = thesis-ex.tex

The \Dptr\ distributions are studied as a function of \ptjet\ for \pp\ data and \PbPb\ collisions with different centralities.
The interplay between the hot and dense matter and the parton shower is explored by evaluating the ratios and differences between \Dptr\ distributions in \pbpb\ and \pp\ collisions, as well as some integrated quantities.



%%%%%%%    DPtr distributions    %%%%%%%
\subsection{\Dptr\ distributions}
\label{sec:dptr}
The \Dptr\ distributions evaluated in \pp\ and \pbpb\ collisions for $126 < \ptjet < 158$ GeV are shown in Figure~\ref{fig:dptr}.
The distributions exhibit a difference in shape between \PbPb\ and \pp\ collisions, with the \pbpb\ distributions being broader at low \pt\ (\pt < 4 GeV) and narrower at high \pt\ (\pt > 4 GeV) in \mbox{0--10\%} central collisions.
This modification is centrality dependent and is smaller for peripheral \pbpb\ collisions.

\begin{figure}[h]
\centerline{
\begin{tabular}{ccc}
\includegraphics[width=0.36\textwidth]{figures/main/results/DpT_dR_jet7_cent0} &
\includegraphics[width=0.36\textwidth]{figures/main/results/DpT_dR_jet7_cent1} &
\includegraphics[width=0.36\textwidth]{figures/main/results/DpT_dR_jet7_cent2} \\
\includegraphics[width=0.36\textwidth]{figures/main/results/DpT_dR_jet7_cent3} &
\includegraphics[width=0.36\textwidth]{figures/main/results/DpT_dR_jet7_cent4} &
\includegraphics[width=0.36\textwidth]{figures/main/results/DpT_dR_jet7_cent5} \\
\end{tabular}
}
\caption{The \Dptr\ distributions in \pp\ (open symbols) and \pbpb\ (closed symbols) as a function of angular distance $r$ for \ptjet\ of 126 to 158~\GeV.
The colors represent different track \pt\ ranges, and each panel is a different centrality selection.
The vertical bars on the data points indicate statistical uncertainties while the shaded boxes indicate systematic uncertainties.
The widths of the boxes are not indicative of the bin size and the points are shifted horizontally for better visibility.
The distributions for $\pt > 6.3$ GeV are restricted to smaller \rvar\ values as discussed in Section~\ref{sec:analysis}.}
\label{fig:dptr}
\end{figure}



%%%%%%%    RDptr distributions    %%%%%%%
\subsection{\RDptr\ distributions}
\label{sec:rdptr}
In order to quantify the differences seen in Figure~\ref{fig:dptr}, ratios of the \Dptr\ distributions in \pbpb\ collisions to those measured in \pp\ collisions for $126 < \ptjet < 158$ GeV and $200 < \ptjet < 251$ GeV jets are presented in Figure~\ref{fig:rdptr}.
They are shown as a function of $r$ for different \pt\ and centrality selections.
In 0--10\% central collisions, \RDptr\ is greater than unity for $\rvar < 0.8$ for charged particles with \pT less than 4.0~\GeV\ in both jet selections.
For these particles, the enhancement of yields in \pbpb\ collisions compared to those in  \pp\ collisions grows with increasing \rvar\ up to approximately \mbox{$\rvar  = 0.3$}, with \RDptr\ reaching up to two for 1.0~$< \pt <$~2.5~\GeV.
The value of \RDptr\ is approximately constant for \rvar\ in the interval \mbox{0.3--0.6} and decreases for \mbox{$\rvar > 0.6$}.
For charged particles with $\pt > 4.0$ \GeV, \RDptr\ shows a depletion outside the jet core for $r > 0.05$.
The magnitude of this depletion increases with increasing \rvar\ up to $r = 0.3$ and is approximately constant thereafter.
For 30--40\% mid-central collisions, the enhancement of particles with $\pt < 4.0$~\GeV\ is similar to that in the most central collisions, however the depletion of particles with $\pt > 4.0$~\GeV\ is not as strong.
For 60--80\% peripheral collisions, \RDptr\ has no significant \rvar\ dependence and the values of \RDptr\ are within approximately 50\% of unity.

The observed behavior inside the jet cone, $r < 0.4$, agrees with the measurement of the inclusive jet fragmentation functions~\cite{Aaboud:2017eww, Aaboud:2017bzv, PhysRevC.98.024908}, where yields of fragments with $\pt < 4$ GeV are observed to be enhanced and yields of charged particles with intermediate \pT\ are suppressed in \PbPb\ collisions compared to those in \pp\ collisions.
%The variation of \RDptr\ with centrality, \ptjet, and charged-particle \pt\ is further discussed.
Calculations done in Ref.~\cite{Tachibana:2017syd} show that the medium response to the jet compensates the energy that is lost by the jet in \pbpb\ collisions even up to $r = 1.0$ from the jet axis.
The plateauing and slight decrease seen in Figure~\ref{fig:rdptr} for the \RDptr\ distributions in central \pbpb\ collisions beyond $r = 0.6$ from the jet axis suggests that the medium response to the jet is smaller than predicted for $r > 0.6$.


\begin{figure}[h]
\centerline{
\begin{tabular}{ccc}
\includegraphics[width=0.36\textwidth]{figures/main/results/RDpT_dR_jet7_cent0} &
\includegraphics[width=0.36\textwidth]{figures/main/results/RDpT_dR_jet7_cent3} &
\includegraphics[width=0.36\textwidth]{figures/main/results/RDpT_dR_jet7_cent5} \\
\includegraphics[width=0.36\textwidth]{figures/main/results/RDpT_dR_jet9_cent0} &
\includegraphics[width=0.36\textwidth]{figures/main/results/RDpT_dR_jet9_cent3} &
\includegraphics[width=0.36\textwidth]{figures/main/results/RDpT_dR_jet9_cent5} \\
\end{tabular}
}
\caption{Ratios of \Dptr\ distributions in \PbPb\ and \pp\ collisions as a function of angular distance $r$ for \ptjet\ of 126 to 158~\GeV\ (top) and of 200 to 251~\GeV\ (bottom) for seven \pt\ selections.
Different centrality selections are shown: 0--10\% (left), 30--40\% (middle), 60--80\% (right).
The vertical bars on the data points indicate statistical uncertainties while the shaded boxes indicate systematic uncertainties.
The widths of the boxes are not indicative of the bin size and the points are shifted horizontally for better visibility.}
\label{fig:rdptr}
\end{figure}

% This observation is in agreement with the previous measurement of jet fragmentation functions \cite{Chatrchyan:2014ava, Sirunyan:2018jqr, Aaboud:2017bzv, } and may indicate the dependence of the response of the hot dense matter to the momentum of a jet passing through it.
%\FloatBarrier

% !TEX root = trkjet.tex


%%%%%%%    Centrality-RDptr distributions    %%%%%%%
The centrality dependence of \RDptr\ for two charged-particle \pt\ intervals: 1.6--2.5~\GeV\ and \mbox{6.3--10.0~\GeV}, and two different \ptjet\ ranges: 126--158~\GeV\ and 200--251~\GeV, is presented in Figure~\ref{fig:centdep}.
For both \ptjet\ selections and  1.6--2.5~\GeV\ charged particles, the magnitude of the excess increases for more central events and for \rvar\ for $\rvar < 0.3$.
The magnitude of the excess is approximately a factor of two in the most central collisions for $\rvar >$~0.3.
A continuous centrality dependent suppression of  yields of charged particles with $6.3 < \pt < 10.0$ GeV is observed.
The magnitude of the modification decreases for more peripheral collisions in both \pt\ intervals and \ptjet\ selections.

\begin{figure}[ht]
\centerline{
\begin{tabular}{cc}
\includegraphics[width=0.36\textwidth]{figures/main/results/RDpT_dR_trk3_trk6_jet7} & 
\includegraphics[width=0.36\textwidth]{figures/main/results/RDpT_dR_trk3_trk6_jet9} \\
\end{tabular}}
\caption{The \RDptr\ distributions for \ptjet\ of 126--158~\GeV\ (left) and 200--251~\GeV\ (right) as a function of angular distance $r$ for two \pt\ selections, 1.6--2.5~\GeV\ (closed symbols) and 6.3--10.0~\GeV\ (open symbols), and six centrality intervals.
The vertical bars on the data points indicate statistical uncertainties while the shaded boxes indicate systematic uncertainties.
The widths of the boxes are not indicative of the bin size and the points are shifted horizontally for better visibility.}
\label{fig:centdep}
\end{figure}

%%%%%%%    trkpt-RDptr distributions    %%%%%%%
%In Figure~\ref{fig:rdptr}, it was shown that for central and mid-central collisions, there is an enhancement of charged particles with $\pt <$~4.0~\GeV\ and a suppression of charged particles with $\pt >$~4.0~\GeV.
Figure~\ref{fig:pttrkdep} shows the \pt\ dependence for selections in \rvar\  for 126--158 GeV and 200--251~\GeV\ jets in the following centrality intervals: 0--10\%, 30--40\% and 60--80\%.
Interestingly, there is no significant suppression of the yields in \pbpb\ collisions for $\rvar < 0.05$ at all measured \pt.
For larger \rvar\ values the yields are enhanced for charged particles with $\pt <$~4~\GeV\ and suppressed for higher \pt\ charged particles in both the 0--10\% and 30--40\% centrality selections and both \ptjet\  ranges presented here.
The magnitude of the enhancement increases for decreasing \pt\ below 4 GeV while the suppression is enhanced with increasing \pt\ for 4--10 GeV, after which it is approximately constant.
At fixed \pt\ the magnitude of the deviation from unity is largest for $0.3< \rvar < 0.4$ and $0.5< \rvar < 0.6$.
In the 60--80\% peripheral collisions, the same trend remains true (but with smaller magnitude modifications) for \mbox{$126 < \ptjet < 158$ GeV}; for the higher \ptjet\ selection the larger uncertainties do not allow a clear conclusion to be drawn for peripheral collisions.

The enhancement of charged particles in the kinematic region of \mbox{$\pt < 4$ GeV} has two common explanations.
First, gluon radiation from the hard scattered parton as it propagates through the QGP would lead to extra soft particles \cite{Chien:2015vja, Kang:2017frl}.
Second, the interactions of a jet with the QGP and its hydrodynamic response could induce a wake that manifests itself as an enhancement of low \pt\ particles \cite{Tachibana:2017syd}.

The observed modification at \mbox{$\pt > 4$ GeV} can be explained on the basis of the larger expected energy loss of gluon-initiated jets, resulting in a relative enhancement of quark jets in \pbpb\ collisions compared to \pp\ collisions at a given \ptjet\ value~\cite{PhysRevC.98.024908, Spousta:2015fca}.
Since gluon jets have a broader distribution of particle transverse momentum with respect to the jet direction compared to quark-initiated jets \cite{OPAL:1995ab}, such an effect could describe the narrowing of the particle distribution around the jet direction for particles with $\pt >$~4.0~\GeV\ that is observed here, though no calculations of this are available.


\begin{figure}[h]
\centerline{
\begin{tabular}{ccc}
\includegraphics[width=0.36\textwidth]{figures/main/results/RDpT_trkpt_jet7_cent0} &
\includegraphics[width=0.36\textwidth]{figures/main/results/RDpT_trkpt_jet7_cent3} &
\includegraphics[width=0.36\textwidth]{figures/main/results/RDpT_trkpt_jet7_cent5} \\
\includegraphics[width=0.36\textwidth]{figures/main/results/RDpT_trkpt_jet9_cent0} &
\includegraphics[width=0.36\textwidth]{figures/main/results/RDpT_trkpt_jet9_cent3} &
\includegraphics[width=0.36\textwidth]{figures/main/results/RDpT_trkpt_jet9_cent5} \\
\end{tabular}}
\caption{\RDptr\ as a function of \pt\ for  0--10\% (left), 30--40\% (middle), and 60--80\% (right) \PbPb\ collisions in two different \ptjet\ selections: 126--158~\GeV\ (top) and 200--251~\GeV\ (bottom).
The different colors indicate different angular distances from the jet axis.
The vertical bars on the data points indicate statistical uncertainties while the shaded boxes indicate systematic uncertainties.
The widths of the boxes are not indicative of the bin size and the points are shifted horizontally for better visibility.}
\label{fig:pttrkdep}
\end{figure}


%%%%%%%    Jet pT-RDptr distributions    %%%%%%%
The \RDptr\ distributions for low and high \pt\ particles in the different \ptjet\ selections are directly overlaid in Figure~\ref{fig:ptjetdep}.
These distributions are for the 0--10\% most central collisions, and show a hint of enhancement in \RDptr\ with increasing \ptjet\  for $r < 0.25$ for low  \pt\ charged particles.
No significant \ptjet\ dependence is seen at larger \rvar\ values, or for high-\pt\ charged particles at any \rvar.
This \ptjet\ dependence is further explored by defining an integral over the low \pt\ excess and is discussed in Section~\ref{sec:discussion_int}.

\begin{figure}[ht]
\centerline{
\includegraphics[width=0.36\textwidth]{figures/main/results/RDpT_dR_trk3_trk6_cent0}}
\caption{\RDptr\ as a function of \rvar\ for 0--10\% collisions for charged particles with 1.6~$< \pt <$~2.5~\GeV\ (closed symbols) and 6.3~$< \pt <$10.0~\GeV\ (open symbols) for different \ptjet\ selections.
The vertical bars on the data points indicate statistical uncertainties while the shaded boxes indicate systematic uncertainties.
The widths of the boxes are not indicative of the bin size and the points are shifted horizontally for better visibility.}
\label{fig:ptjetdep}
\end{figure}



%%%%%%%    Delta DPtr distributions    %%%%%%%
\subsection{\DeltaDptr\ distributions}
\label{sec:delta_dptr}
In addition to the ratios of the \Dptr\ distributions, differences between the unfolded charged-particle yields are also evaluated as \DeltaDptr\ to quantify the modification in terms of the particle density.

These differences are presented as a function of $r$ for different \pt\ selections in 0--10\% central collisions in Figure~\ref{fig:deltadptr}.
These distributions show an excess in the charged-particle yield density for \pbpb\ collisions compared to \pp\ collisions for charged particles with $\pt <4.0$ GeV.
This ranges from 0.5 to 4 particles per unit area per GeV for 1 \GeV\ charged particles in 126--158~\GeV\ jets for 0--10\% central \pbpb\ collisions and increases with increasing \ptjet.
The largest excess for charged particles with $\pt <$~4.0~\GeV\ is within the jet cone.
For large \rvar\ values, the difference decreases, but remains positive.
A depletion for higher \pt\ particles of approximately 0.5 particles per unit area per GeV is seen for 126--158~\GeV\ jets in 0--10\% central \pbpb\ collisions.
The magnitude of this depletion increases for higher \ptjet.
There is a minimum in the \DeltaDptr\ distributions of charged particles with \mbox{$ 4.0 < \pt <  25.1$}~\GeV\ at $0.05 < \rvar < 0.10$ that is seen at many \ptjet\ ranges under investigation.
The magnitudes of the excesses and deficits discussed here are dependent on the selected charged-particle \pt.

\begin{figure}
\centerline{
\begin{tabular}{cc}
\includegraphics[width=0.36\textwidth]{figures/main/results/DeltaDpT_dR_jet7_cent0} &
\includegraphics[width=0.36\textwidth]{figures/main/results/DeltaDpT_dR_jet8_cent0} \\
\includegraphics[width=0.36\textwidth]{figures/main/results/DeltaDpT_dR_jet9_cent0} &
\includegraphics[width=0.36\textwidth]{figures/main/results/DeltaDpT_dR_jet10_cent0} \\
\end{tabular} }
\caption{\DeltaDptr\ as a function of \rvar\ in central collisions for all \pt\ ranges in four \ptjet\ selections: 126--158~\GeV, 158--200~\GeV, 200--251~\GeV, and 251--316~\GeV.
The vertical bars on the data points indicate statistical uncertainties while the shaded boxes indicate systematic uncertainties.
The widths of the boxes are not indicative of the bin size and the points are shifted horizontally for better visibility.}
\label{fig:deltadptr}
\end{figure}





%%%%%%%%%%%%%
\subsection{\pt\ integrated distributions}
\label{sec:discussion_int}
Motivated by similar studies of the enhancement of soft fragments in jet fragmentation functions in \pbpb\ compared to \pp\ collisions from Ref.~\cite{PhysRevC.98.024908}, the unfolded \Dptr\ distributions are integrated for charged particles with \pt\ < 4 GeV to construct the quantities $\Theta(\rvar)$ and $P(\rvar)$ defined as:

\begin{align*}
\Theta(\rvar) &= \int_{1 \text{ GeV}}^{4 \text{ GeV}} \Dptr  \fd \pt \\
P(\rvar) &= \int_0^r \int_{1 \text{ GeV}}^{4 \text{ GeV}} D(\pt, r') \fd \pt \fd r'
\end{align*}
The $\Theta(\rvar)$ values are integrated over the charged-particle \pt\ interval of 1--4~\GeV\ to provide a summary look at the \pt\ region of enhancement discussed above.
The $P(\rvar)$ values further add a running integral over \rvar\ and provide information about the jet shape.
Both of these quantities are compared between the \pp\ and \pbpb\ systems to give the following distributions:

\begin{align*}
\Delta_{\Theta(\rvar)} &= \Theta(\rvar)_{\mathrm{Pb+Pb}} - \Theta(\rvar)_{pp} \\
R_{\Theta(\rvar)} &= \frac{\Theta(\rvar)_{\mathrm{Pb+Pb}}}{\Theta(\rvar)_{\mathrm{pp}}} \\
R_{P(\rvar)} &= \frac{P(\rvar)_{\mathrm{Pb+Pb}}}{P(\rvar)_{pp}}
\end{align*}
These integrated quantities are intended to provide some summary information about the location with respect to the jet axis, magnitude, and \ptjet\ dependence of the low-\pt\ charged-particle excess discussed above.
The ratio quantities are useful for comparisons to other \pbpb\ measurements; $\Delta_{\Theta(\rvar)}$ is very similar to $\DeltaDptr$, however it is integrated over charged-particle \pt\ in the 1--4~\GeV\ interval \cite{PhysRevC.98.024908}.

Figure~\ref{fig:deltaPdeltaT} shows the \DeltaTheta\ distributions as a function of \rvar\ in centrality intervals: 0--10\%, 30--40\%, 60--80\%.
In the most central collisions, a significant \ptjet\ dependence to \DeltaTheta\ is observed; for $\rvar <$~0.4 (particles within the jet cone) \DeltaTheta\ increases with increasing \ptjet.
The value of \DeltaTheta\ decreases in more peripheral collisions and the \ptjet\ dependence is no longer significant.

%%%
%Now, the \ptjet\ dependence to the excess in charged-particle density can be seen clearly; 
%in the most central collisions
%there is an increase in \DeltaTheta\ with increasing \ptjet, but in the mid-central and peripheral collisions this is no longer
%observed within the uncertainties.
%%%%

\begin{figure}
\centerline{
\begin{tabular}{ccc}
\includegraphics[width=0.36\textwidth]{figures/main/results/DeltaDpT_lowpt_integ_cent0} &
\includegraphics[width=0.36\textwidth]{figures/main/results/DeltaDpT_lowpt_integ_cent3} &
\includegraphics[width=0.36\textwidth]{figures/main/results/DeltaDpT_lowpt_integ_cent5} \\
\end{tabular} }
\caption{\DeltaTheta\ as a function of \rvar\ for charged particles with \pt\ < 4 GeV  in four \ptjet\ selections: 126--158~\GeV, 158--200~\GeV, 200--251~\GeV, and 251--316~\GeV and three centrality selections: 0--10\% (left), 30--40\% (middle) and 60--80\% (right).
The vertical bars on the data points indicate statistical uncertainties while the shaded boxes indicate systematic uncertainties.
The widths of the boxes are not indicative of the bin size and the points are shifted horizontally for better visibility.}
\label{fig:deltaPdeltaT}
\end{figure}


\begin{figure}
\centerline{
\begin{tabular}{ccc}
\includegraphics[width=0.36\textwidth]{figures/main/results/RDpT_lowpt_integ_cent0} &
\includegraphics[width=0.36\textwidth]{figures/main/results/RDpT_lowpt_integ_cent3} &
\includegraphics[width=0.36\textwidth]{figures/main/results/RDpT_lowpt_integ_cent5} \\
\includegraphics[width=0.36\textwidth]{figures/main/results/RDpT_jetshape_cent0} &
\includegraphics[width=0.36\textwidth]{figures/main/results/RDpT_jetshape_cent3} &
\includegraphics[width=0.36\textwidth]{figures/main/results/RDpT_jetshape_cent5} \\
\end{tabular} }
\caption{\RTheta\ (top) and \RP\ (bottom) as a function of \rvar\ for charged particles with $\pt < 4$ GeV ranges in four \ptjet\ selections: 126--158~\GeV, 158--200~\GeV, 200--251~\GeV, and 251--316~\GeV\ and three centrality selections: 0--10\% (left), 30--40\% (middle) and 60--80\% (rights).
The vertical bars on the data points indicate statistical uncertainties while the shaded boxes indicate systematic uncertainties.
The widths of the boxes are not indicative of the bin size and the points are shifted horizontally for better visibility.}
\label{fig:RPRT}
\end{figure}


Figure~\ref{fig:RPRT} shows \RTheta\ and \RP\ for the following centrality intervals: 0--10\%, 30--40\% and 60--80\%.
The \RTheta\ distributions of the most central collisions show a maximum for $\rvar \sim 0.4$ and a flattening or a decrease for larger \rvar.
However, since \RTheta\ remains at or above unity for the full range of \rvar\ values presented, \RP\ shows no suppression with increasing \rvar\ over the entire measured range.
In more peripheral collisions the magnitude of the excess is reduced and the trends in \RTheta\ are less clear, however the slow increase of \RP\ is clearly seen for the 30--40\% central collisions.
The flattening of the \RP\ distributions at large distances demonstrates what while wider jets have a softer fragmentation and contain more particles with less \pt\ in \pbpb\ compared to \pp\ collisions \cite{Chesler:2015nqz, Hulcher:2017cpt}, this effect flattens out for jets with radius larger than 0.6.


%%%%
%These measurements show that the excess of particles with $\pt <$~4.0~\GeV\ observed in~\cite{PhysRevC.98.024908} extends
%outside the \RFour\ jet cone.
%The measured dependence of \RDptr\ suggests that the energy lost by jets through the jet quenching process is being transferred to particles with $\pt <$~4.0~\GeV\ at larger radial distances from the jet axis.
%This is qualitatively consistent with theoretical calculations \mbox{\cite{Blaizot:2014ula}}.
%Additionally, these observations are in agreement with the previous measurement of jet fragmentation functions \cite{Chatrchyan:2014ava, Sirunyan:2018jqr, Aaboud:2017bzv, PhysRevC.98.024908} and may indicate the dependence of the response of the hot dense matter to the momentum of a jet passing through it.
%%%%%

\FloatBarrier



%\appendix
%\section{Appendix}
%\label{sec:appendixA}
%% !TEX root = thesis-ex.tex

\begin{figure}
	\centering
	\includegraphics[width=1.0\textwidth]{figures/c.pdf} 
	\caption{ $\Delta\phi$ distributions for truth and reco (left). Response Matrix $M_{ij}$ (center). Correction factors with errors (right). }	
	\label{fig:plots}
\end{figure}

In Figure~\ref{fig:plots}, we have truth and reconstructed $\Delta\phi$ distributions on the left-most plot, the response matrix $M_{ij}$ where $\Delta\phi_{Reco}$ is along the x-axis, along the j-index, and $\Delta\phi_{Truth}$ is along the y-axis, along the i-index, and resulting correction factors with errors on the right-most. 

Define $T_{i}$ as the total number of entries in the $i^{th}$ bin of the Truth distribution (blue points on left plot), and $R_{i}$ as the total number of entries in the $i^{th}$ bin of the Reconstructed distribution (red points on left plot). 

In terms of the response matrix, $R_{j}$ is

\begin{eqnarray} \label{eq:rj}
R_j = \sum_{i}^{}M_{ij} = M_{jj} + \sum_{i\neq j}^{}M_{ij} 
\end{eqnarray}

The last part is just the diagonal element plus the off-diagonal vertical elements of the $i^{th}$ bin (on the x-axis).

Similarly, in terms of the response matrix, $T_{i}$ is

\begin{eqnarray} \label{eq:ti}
T_i = \sum_{j}^{}M_{ij} = M_{ii} + \sum_{j\neq i}^{}M_{ij} 
\end{eqnarray}

For some bin $i^{th}$ reconstructed bin,

\begin{eqnarray} \label{eq:leavearrive}
R_{i} = T_{i} - N_{Leaving} + N_{Arriving} = T_{i} - \sum_{k\neq i}^{}M_{ik} + \sum_{j\neq i}^{}M_{ji}
\end{eqnarray}

We can express the number leaving and number arriving in terms of off-diagonal row or column elements of $M_{ij}$, or in terms of  $T_{i}$, $R_{i}$, and diagonal elements of  $M_{ij}$.

\begin{eqnarray} \label{eq:leavearrivediagonal}
N_{Leaving} = T_{i} - M_{ii} \\
N_{Arriving} = R_{i} - M_{ii} 
\end{eqnarray}

Now, $T_{i}$ is taken as a constant. This means that reconstructed distribution can be different time to time, but the truth distribution stays the same. In the language of a toy MC, this is equivalent to generating one Truth distribution, and smearing it many different times, each time (or for each new "experiment") getting new results.

When $T_{i}$ is taken as constant, the bin migration of leaving and arriving is different. The distribution of $N_{Leaving}$ is binomial, while $N_{Arriving}$ is Poisson. If $T_{i}$ is fixed, there is only a certain number of entries that can leave, while the number that arrives depends on, and is a mix of the entries leaving neighboring bins. 

In a toy MC \footnote{A Toy MC with a randomly generated exponential was generated for the truth distribution 5000 times, with smearing from the ATLAS MC response matrix applied to the reconstructed distribution. The experiment was then repeated 10,000 times to get some good statistics on correction factors, their errors, bin migration, etc.}, for the case where the truth distribution was generated one time, but smearing applied to the reconstructed (case with "fixed" $T_{i}$), it is clear from Figure~\ref{fig:mig_same} that the migration where entries are leaving is narrower than where the arrive. In the same toy MC, when for every experiment a new truth distribution was used, it is evident that the migration to and from is the same.

\begin{figure}
	\centering
	\includegraphics[width=1.0\textwidth]{figures/c3_same.pdf} 
	\caption{ For the case where for every experiment a the same generated truth distribution but differently smeared reconstructed distribution, histogram of migration between $\Delta\phi$ bins (x and y axes) for entries arriving (right) and entries leaving (left).Migration where entries leave has a binomial (narrower) distribution, while entries arriving is Poisson. }	
	\label{fig:mig_same}
\end{figure}

\begin{figure}
	\centering
	\includegraphics[width=1.0\textwidth]{figures/c3_diff.pdf} 
	\caption{ For the case where for every experiment a new truth distribution is generated and the reconstructed is smeared from that, histogram of migration between $\Delta\phi$ bins (x and y axes) for entries arriving (right) and entries leaving (left). Both migrations have Poisson distributions. }	
	\label{fig:mig_diff}
\end{figure}

Correction factors $C_{i}$, which relate $T_{i}$ and $R_{i}$ are

\begin{eqnarray} \label{eq:cfactors}
C_{i} = \frac{T_{i}}{R_{i}}
\end{eqnarray} 

and their respective errors $\sigma_{C_{i}}$ are

\begin{eqnarray} \label{eq:cfactorerrors}
\sigma_{C_{i}}^{2} = \frac{C_{i}^{2}}{R_{i}^{2}}\sigma_{R_{i}}^{2}
\end{eqnarray}	

Now since $T_{i}$ is constant, and the entries leaving a $T_{i}$ bin follow binomial statistics, while entries arriving are Poisson, we continue from Equation~\ref{eq:leavearrive}. The error in $R_{i}$ is 

\begin{eqnarray}
\sigma_{R_{i}}^{2} = \sigma_{N_{Leave}}^{2} + \sigma_{N_{Arrive}}^{2} \\ 
\sigma_{R_{i}}^{2} = T_{i}\frac{T_{i} - M_{ii}}{T_{i}}\big(1 - \frac{T_{i} - M_{ii}}{T_{i}}\big)+(R_{i}-M_{ii}) \\
\sigma_{R_{i}}^{2} = T_{i} + R_{i} - 2M_{ii} - \frac{(T_{i} - M_{ii})^{2}}{T_{i}}
\end{eqnarray}	

From this, plugging into Equation~\ref{eq:cfactorerrors}, the error on the correction factor is

\begin{eqnarray} \label{eq:cfactorerror}
\sigma_{C_{i}}^{2} = \frac{T_{i}^{2}}{R_{i}^{4}}\Big(T_{i} + R_{i} -2M_{ii} - \frac{(T_{i} - M_{ii})^{2}}{T_{i}}\Big) \\
\sigma_{C_{i}}^{2} = \frac{T_{i}^{2}}{R_{i}^{3}}\Big(1 - \frac{M_{ii}^{2}}{T_{i}R_{i}}\Big).
\end{eqnarray}	

%\clearpage

%\clearpage

%\appendix
%\chapter{HI JER Uncertainty}
%\label{sec:appendix_hijerDerivation}
%% !TEX root = thesis-ex.tex
\subsection{Deriving the HI JER}

\begin{figure}
\includegraphics[page=1,width=0.7\textwidth]{figures/appendixHIJERDerivation/JERUncertaintyNote} 
\end{figure}

\begin{figure}
\includegraphics[page=2,width=0.7\textwidth]{figures/appendixHIJERDerivation/JERUncertaintyNote} 
\end{figure}

\begin{figure}
\includegraphics[page=3,width=0.7\textwidth]{figures/appendixHIJERDerivation/JERUncertaintyNote} 
\end{figure}

%\clearpage










%\chapter{Data Sets}
%\label{sec:appendixdata}
%\begin{table}[h]
	\centering
	\begin{tabular}{|| c | c || } 
		\hline
		2016 \pPb\ Data Samples & Number of Events \\ 
		\hline
		\verb|data16_hip5TeV.00312649.physics_Main.recon.AOD.f784_m1741| & 8.96e6 \\
		\verb|data16_hip5TeV.00312796.physics_Main.recon.AOD.f784_m1741| & 4.32e7 \\
		\verb|data16_hip5TeV.00312837.physics_Main.recon.AOD.f774_m1736| & 8.50e7 \\
		\verb|data16_hip5TeV.00312937.physics_Main.recon.AOD.f774_m1736| & 2.60e7 \\
		\verb|data16_hip5TeV.00312945.physics_Main.recon.AOD.f774_m1736| & 2.87e7 \\ 
		\verb|data16_hip5TeV.00312968.physics_Main.recon.AOD.f774_m1736| & 3.66e7 \\
		\verb|data16_hip5TeV.00314199.physics_Main.recon.AOD.f781_m1741| & 2.40e8 \\
		\hline \hline
		2015 \pp\ Data Samples & Number of Events \\ 
		\hline
		\verb|data15_5TeV.periodK.physics_Main.PhysCont.AOD.repro20_v03| & 1.15e8 \\ 
		\verb|data15_5TeV.periodVdM.physics_Main.PhysCont.AOD.repro20_v03| & 
		1.27e6 \\
		\hline	
	\end{tabular}
	\caption{\label{tab:datasamples} Data samples from \sqrtsnn=5.02~TeV \pp\ and \pPb\ collisions collected during the 2015 and 2016 heavy ion runs, respectively. }
	\bigskip
	\bigskip
	\begin{tabular}{|| c | c | c || } 
		\hline
		J & 2015 \pp\ \pythiaeight\ MC Samples & Number of Events \\ 
		\hline
		1 & \verb|mc15_5TeV.420011.Pythia8EvtGen_A14NNPDF23LO_jetjet_| & 5.88e6\\
		& \verb|JZ1R04.merge.AOD.e4108_s2860_r7792_r7676| & \\
		\hline
		2 & \verb|mc15_5TeV.420012.Pythia8EvtGen_A14NNPDF23LO_jetjet_| & 5.84e6\\
		& \verb|JZ2R04.merge.AOD.e4108_s2860_r7792_r7676| & \\
		\hline 
		\hline
		J & 2016 \pPb\ \pythiaeight\ MC Samples & Number of Events \\ 
		\hline
		1 & \verb|mc15_5TeV.420018.Pythia8EvtGen_A14NNPDF23LO_jetjet_| & 1.98e6\\
		& \verb|JZ1R04_MaxEta_m3p0.merge.AOD.e6114_d1462_r10136_r9647| & \\
		\hline
		2 & \verb|mc15_5TeV.420019.Pythia8EvtGen_A14NNPDF23LO_jetjet_| & 1.00e6\\
		& \verb|JZ2R04_MaxEta_m3p0.merge.AOD.e6114_d1462_r10136_r9647| & \\
		\hline 
		\hline
		J & 2015 \pp\ \herwig\ MC Samples & Number of Events \\ 
		\hline
		1 & \verb|mc15_5TeV.420031.HerwigppEvtGen_UEEE5_CTEQ6L1_jetjet_| & 2.82e6\\
		& \verb|JZ1R04.merge.AOD.e4929_s2860_r7792_r7676| & \\
		\hline
		2 & \verb|mc15_5TeV.420032.HerwigppEvtGen_UEEE5_CTEQ6L1_jetjet_| & 2.80e6\\
		& \verb|JZ2R04.merge.AOD.e4929_s2860_r7792_r7676| & \\
		\hline 
	\end{tabular}
	\caption{ 2015 \pp\ \pythiaeight\ MC Samples (top). 2016 \pPb\ \pythiaeight\ MC samples with data overlay (middle). 2015 \pp\ \herwig\ Monte Carlo samples (bottom).  }
	\label{tab:mcsamples}
\end{table}

%\clearpage
%%%%%%%%%%%%%%%%%%%%%%%%%%%%
%\chapter{Bin-by-bin Unfolding Procedure}
%\label{sec:appendixbbb}
%% !TEX root = thesis-ex.tex

\begin{figure}
	\centering
	\includegraphics[width=1.0\textwidth]{figures/c.pdf} 
	\caption{ $\Delta\phi$ distributions for truth and reco (left). Response Matrix $M_{ij}$ (center). Correction factors with errors (right). }	
	\label{fig:plots}
\end{figure}

In Figure~\ref{fig:plots}, we have truth and reconstructed $\Delta\phi$ distributions on the left-most plot, the response matrix $M_{ij}$ where $\Delta\phi_{Reco}$ is along the x-axis, along the j-index, and $\Delta\phi_{Truth}$ is along the y-axis, along the i-index, and resulting correction factors with errors on the right-most. 

Define $T_{i}$ as the total number of entries in the $i^{th}$ bin of the Truth distribution (blue points on left plot), and $R_{i}$ as the total number of entries in the $i^{th}$ bin of the Reconstructed distribution (red points on left plot). 

In terms of the response matrix, $R_{j}$ is

\begin{eqnarray} \label{eq:rj}
R_j = \sum_{i}^{}M_{ij} = M_{jj} + \sum_{i\neq j}^{}M_{ij} 
\end{eqnarray}

The last part is just the diagonal element plus the off-diagonal vertical elements of the $i^{th}$ bin (on the x-axis).

Similarly, in terms of the response matrix, $T_{i}$ is

\begin{eqnarray} \label{eq:ti}
T_i = \sum_{j}^{}M_{ij} = M_{ii} + \sum_{j\neq i}^{}M_{ij} 
\end{eqnarray}

For some bin $i^{th}$ reconstructed bin,

\begin{eqnarray} \label{eq:leavearrive}
R_{i} = T_{i} - N_{Leaving} + N_{Arriving} = T_{i} - \sum_{k\neq i}^{}M_{ik} + \sum_{j\neq i}^{}M_{ji}
\end{eqnarray}

We can express the number leaving and number arriving in terms of off-diagonal row or column elements of $M_{ij}$, or in terms of  $T_{i}$, $R_{i}$, and diagonal elements of  $M_{ij}$.

\begin{eqnarray} \label{eq:leavearrivediagonal}
N_{Leaving} = T_{i} - M_{ii} \\
N_{Arriving} = R_{i} - M_{ii} 
\end{eqnarray}

Now, $T_{i}$ is taken as a constant. This means that reconstructed distribution can be different time to time, but the truth distribution stays the same. In the language of a toy MC, this is equivalent to generating one Truth distribution, and smearing it many different times, each time (or for each new "experiment") getting new results.

When $T_{i}$ is taken as constant, the bin migration of leaving and arriving is different. The distribution of $N_{Leaving}$ is binomial, while $N_{Arriving}$ is Poisson. If $T_{i}$ is fixed, there is only a certain number of entries that can leave, while the number that arrives depends on, and is a mix of the entries leaving neighboring bins. 

In a toy MC \footnote{A Toy MC with a randomly generated exponential was generated for the truth distribution 5000 times, with smearing from the ATLAS MC response matrix applied to the reconstructed distribution. The experiment was then repeated 10,000 times to get some good statistics on correction factors, their errors, bin migration, etc.}, for the case where the truth distribution was generated one time, but smearing applied to the reconstructed (case with "fixed" $T_{i}$), it is clear from Figure~\ref{fig:mig_same} that the migration where entries are leaving is narrower than where the arrive. In the same toy MC, when for every experiment a new truth distribution was used, it is evident that the migration to and from is the same.

\begin{figure}
	\centering
	\includegraphics[width=1.0\textwidth]{figures/c3_same.pdf} 
	\caption{ For the case where for every experiment a the same generated truth distribution but differently smeared reconstructed distribution, histogram of migration between $\Delta\phi$ bins (x and y axes) for entries arriving (right) and entries leaving (left).Migration where entries leave has a binomial (narrower) distribution, while entries arriving is Poisson. }	
	\label{fig:mig_same}
\end{figure}

\begin{figure}
	\centering
	\includegraphics[width=1.0\textwidth]{figures/c3_diff.pdf} 
	\caption{ For the case where for every experiment a new truth distribution is generated and the reconstructed is smeared from that, histogram of migration between $\Delta\phi$ bins (x and y axes) for entries arriving (right) and entries leaving (left). Both migrations have Poisson distributions. }	
	\label{fig:mig_diff}
\end{figure}

Correction factors $C_{i}$, which relate $T_{i}$ and $R_{i}$ are

\begin{eqnarray} \label{eq:cfactors}
C_{i} = \frac{T_{i}}{R_{i}}
\end{eqnarray} 

and their respective errors $\sigma_{C_{i}}$ are

\begin{eqnarray} \label{eq:cfactorerrors}
\sigma_{C_{i}}^{2} = \frac{C_{i}^{2}}{R_{i}^{2}}\sigma_{R_{i}}^{2}
\end{eqnarray}	

Now since $T_{i}$ is constant, and the entries leaving a $T_{i}$ bin follow binomial statistics, while entries arriving are Poisson, we continue from Equation~\ref{eq:leavearrive}. The error in $R_{i}$ is 

\begin{eqnarray}
\sigma_{R_{i}}^{2} = \sigma_{N_{Leave}}^{2} + \sigma_{N_{Arrive}}^{2} \\ 
\sigma_{R_{i}}^{2} = T_{i}\frac{T_{i} - M_{ii}}{T_{i}}\big(1 - \frac{T_{i} - M_{ii}}{T_{i}}\big)+(R_{i}-M_{ii}) \\
\sigma_{R_{i}}^{2} = T_{i} + R_{i} - 2M_{ii} - \frac{(T_{i} - M_{ii})^{2}}{T_{i}}
\end{eqnarray}	

From this, plugging into Equation~\ref{eq:cfactorerrors}, the error on the correction factor is

\begin{eqnarray} \label{eq:cfactorerror}
\sigma_{C_{i}}^{2} = \frac{T_{i}^{2}}{R_{i}^{4}}\Big(T_{i} + R_{i} -2M_{ii} - \frac{(T_{i} - M_{ii})^{2}}{T_{i}}\Big) \\
\sigma_{C_{i}}^{2} = \frac{T_{i}^{2}}{R_{i}^{3}}\Big(1 - \frac{M_{ii}^{2}}{T_{i}R_{i}}\Big).
\end{eqnarray}	

%\clearpage
%%%%%%%%%%%%%%%%%%%%%%%%%%%%
%\chapter{\Dphi\ Correction Factors From \pp\ MC Samples}
%\label{sec:appendixcfactorpp}
%% !TEX root = thesis-ex.tex

\begin{figure}[ht]
	\centerline{
		\begin{tabular}{ccc}
			\includegraphics[width=0.33\textwidth]{output/output_pp_mc_pythia8/h_dPhi_cFactor_All_40_Ystar1_27_28_Pt1_35_28_Pt2_35_40_Ystar2_27.pdf} &
			\includegraphics[width=0.33\textwidth]{output/output_pp_mc_pythia8/h_dPhi_cFactor_All_40_Ystar1_27_28_Pt1_35_28_Pt2_35_27_Ystar2_18.pdf} &
			\includegraphics[width=0.33\textwidth]{output/output_pp_mc_pythia8/h_dPhi_cFactor_All_40_Ystar1_27_28_Pt1_35_28_Pt2_35_18_Ystar2_0.pdf} \\
			\includegraphics[width=0.33\textwidth]{output/output_pp_mc_pythia8/h_dPhi_cFactor_All_40_Ystar1_27_28_Pt1_35_28_Pt2_35_0_Ystar2_18.pdf} &
			\includegraphics[width=0.33\textwidth]{output/output_pp_mc_pythia8/h_dPhi_cFactor_All_40_Ystar1_27_28_Pt1_35_28_Pt2_35_18_Ystar2_40.pdf} &
			\includegraphics[width=0.33\textwidth]{output/output_pp_mc_pythia8/h_dPhi_cFactor_All_40_Ystar1_27_35_Pt1_45_28_Pt2_35_40_Ystar2_27.pdf} \\
			\includegraphics[width=0.33\textwidth]{output/output_pp_mc_pythia8/h_dPhi_cFactor_All_40_Ystar1_27_35_Pt1_45_28_Pt2_35_27_Ystar2_18.pdf} &
			\includegraphics[width=0.33\textwidth]{output/output_pp_mc_pythia8/h_dPhi_cFactor_All_40_Ystar1_27_35_Pt1_45_28_Pt2_35_18_Ystar2_0.pdf} &
			\includegraphics[width=0.33\textwidth]{output/output_pp_mc_pythia8/h_dPhi_cFactor_All_40_Ystar1_27_35_Pt1_45_28_Pt2_35_0_Ystar2_18.pdf} \\
		\end{tabular}
	}
	\caption{Corretion factors derived from \pp\ MC samples.}
\end{figure}

\begin{figure}[ht]
	\centerline{
		\begin{tabular}{ccc}
			\includegraphics[width=0.33\textwidth]{output/output_pp_mc_pythia8/h_dPhi_cFactor_All_40_Ystar1_27_35_Pt1_45_28_Pt2_35_18_Ystar2_40.pdf} &
			\includegraphics[width=0.33\textwidth]{output/output_pp_mc_pythia8/h_dPhi_cFactor_All_40_Ystar1_27_35_Pt1_45_35_Pt2_45_40_Ystar2_27.pdf} &
			\includegraphics[width=0.33\textwidth]{output/output_pp_mc_pythia8/h_dPhi_cFactor_All_40_Ystar1_27_35_Pt1_45_35_Pt2_45_27_Ystar2_18.pdf} \\
			\includegraphics[width=0.33\textwidth]{output/output_pp_mc_pythia8/h_dPhi_cFactor_All_40_Ystar1_27_35_Pt1_45_35_Pt2_45_18_Ystar2_0.pdf} &
			\includegraphics[width=0.33\textwidth]{output/output_pp_mc_pythia8/h_dPhi_cFactor_All_40_Ystar1_27_35_Pt1_45_35_Pt2_45_0_Ystar2_18.pdf} &
			\includegraphics[width=0.33\textwidth]{output/output_pp_mc_pythia8/h_dPhi_cFactor_All_40_Ystar1_27_35_Pt1_45_35_Pt2_45_18_Ystar2_40.pdf} \\
			\includegraphics[width=0.33\textwidth]{output/output_pp_mc_pythia8/h_dPhi_cFactor_All_40_Ystar1_27_45_Pt1_90_28_Pt2_35_40_Ystar2_27.pdf} &
			\includegraphics[width=0.33\textwidth]{output/output_pp_mc_pythia8/h_dPhi_cFactor_All_40_Ystar1_27_45_Pt1_90_28_Pt2_35_27_Ystar2_18.pdf} &
			\includegraphics[width=0.33\textwidth]{output/output_pp_mc_pythia8/h_dPhi_cFactor_All_40_Ystar1_27_45_Pt1_90_28_Pt2_35_18_Ystar2_0.pdf} \\
			\includegraphics[width=0.33\textwidth]{output/output_pp_mc_pythia8/h_dPhi_cFactor_All_40_Ystar1_27_45_Pt1_90_28_Pt2_35_0_Ystar2_18.pdf} &
			\includegraphics[width=0.33\textwidth]{output/output_pp_mc_pythia8/h_dPhi_cFactor_All_40_Ystar1_27_45_Pt1_90_28_Pt2_35_18_Ystar2_40.pdf} &
			\includegraphics[width=0.33\textwidth]{output/output_pp_mc_pythia8/h_dPhi_cFactor_All_40_Ystar1_27_45_Pt1_90_35_Pt2_45_40_Ystar2_27.pdf} \\
		\end{tabular}
	}
	\caption{Corretion factors derived from \pp\ MC samples.}
\end{figure}
\begin{figure}[ht]
		\centerline{
			\begin{tabular}{ccc}
			\includegraphics[width=0.33\textwidth]{output/output_pp_mc_pythia8/h_dPhi_cFactor_All_40_Ystar1_27_45_Pt1_90_35_Pt2_45_27_Ystar2_18.pdf} &
			\includegraphics[width=0.33\textwidth]{output/output_pp_mc_pythia8/h_dPhi_cFactor_All_40_Ystar1_27_45_Pt1_90_35_Pt2_45_18_Ystar2_0.pdf} &
			\includegraphics[width=0.33\textwidth]{output/output_pp_mc_pythia8/h_dPhi_cFactor_All_40_Ystar1_27_45_Pt1_90_35_Pt2_45_0_Ystar2_18.pdf} \\
			\includegraphics[width=0.33\textwidth]{output/output_pp_mc_pythia8/h_dPhi_cFactor_All_40_Ystar1_27_45_Pt1_90_35_Pt2_45_18_Ystar2_40.pdf} &
			\includegraphics[width=0.33\textwidth]{output/output_pp_mc_pythia8/h_dPhi_cFactor_All_40_Ystar1_27_45_Pt1_90_45_Pt2_90_40_Ystar2_27.pdf} &
			\includegraphics[width=0.33\textwidth]{output/output_pp_mc_pythia8/h_dPhi_cFactor_All_40_Ystar1_27_45_Pt1_90_45_Pt2_90_27_Ystar2_18.pdf} \\
			\includegraphics[width=0.33\textwidth]{output/output_pp_mc_pythia8/h_dPhi_cFactor_All_40_Ystar1_27_45_Pt1_90_45_Pt2_90_18_Ystar2_0.pdf} &
			\includegraphics[width=0.33\textwidth]{output/output_pp_mc_pythia8/h_dPhi_cFactor_All_40_Ystar1_27_45_Pt1_90_45_Pt2_90_0_Ystar2_18.pdf} &
			\includegraphics[width=0.33\textwidth]{output/output_pp_mc_pythia8/h_dPhi_cFactor_All_40_Ystar1_27_45_Pt1_90_45_Pt2_90_18_Ystar2_40.pdf} \\
		\end{tabular}
	}
	\caption{Corretion factors derived from \pp\ MC samples.}
\end{figure}


%\clearpage
%%%%%%%%%%%%%%%%%%%%%%%%%%%%
%\chapter{\Dphi\ Correction Factors From \pPb\ MC Samples}
%\label{sec:appendixcfactorpPb}
%\input{appendixY.tex}
%\clearpage
%%%%%%%%%%%%%%%%%%%%%%%%%%%%
%\chapter{Effect of Isolation Cuts}
%\label{sec:appendixisolation}
%
\begin{figure}[ht]
	\centerline{
		\begin{tabular}{ccc}
			\includegraphics[width=0.33\textwidth]{output/output_pp_data/h_dPhi_unfolded_width_All_40_Ystar1_27_28_Pt1_35_28_Pt2_35_NoIsoR.pdf} &
			\includegraphics[width=0.33\textwidth]{output/output_pp_data/h_dPhi_unfolded_width_All_40_Ystar1_27_35_Pt1_45_28_Pt2_35_NoIsoR.pdf} &
			\includegraphics[width=0.33\textwidth]{output/output_pp_data/h_dPhi_unfolded_width_All_40_Ystar1_27_35_Pt1_45_35_Pt2_45_NoIsoR.pdf} \\
			\includegraphics[width=0.33\textwidth]{output/output_pp_data/h_dPhi_unfolded_width_All_40_Ystar1_27_45_Pt1_90_28_Pt2_35_NoIsoR.pdf} &
			\includegraphics[width=0.33\textwidth]{output/output_pp_data/h_dPhi_unfolded_width_All_40_Ystar1_27_45_Pt1_90_35_Pt2_45_NoIsoR.pdf} &
			\includegraphics[width=0.33\textwidth]{output/output_pp_data/h_dPhi_unfolded_width_All_40_Ystar1_27_45_Pt1_90_45_Pt2_90_NoIsoR.pdf} \\
		\end{tabular}
	}
	\caption{Comparison of \conetwo\ distributions with and without isolation requirement in \pp\ data.}
\end{figure}

\begin{figure}[ht]
	\centerline{
		\begin{tabular}{ccc}
			\includegraphics[width=0.33\textwidth]{output/output_pPb_data/h_dPhi_unfolded_yield_All_40_Ystar1_27_28_Pt1_35_28_Pt2_35_NoIsoR.pdf} &
			\includegraphics[width=0.33\textwidth]{output/output_pPb_data/h_dPhi_unfolded_yield_All_40_Ystar1_27_35_Pt1_45_28_Pt2_35_NoIsoR.pdf} &
			\includegraphics[width=0.33\textwidth]{output/output_pPb_data/h_dPhi_unfolded_yield_All_40_Ystar1_27_35_Pt1_45_35_Pt2_45_NoIsoR.pdf} \\
			\includegraphics[width=0.33\textwidth]{output/output_pPb_data/h_dPhi_unfolded_yield_All_40_Ystar1_27_45_Pt1_90_28_Pt2_35_NoIsoR.pdf} &
			\includegraphics[width=0.33\textwidth]{output/output_pPb_data/h_dPhi_unfolded_yield_All_40_Ystar1_27_45_Pt1_90_35_Pt2_45_NoIsoR.pdf} &
			\includegraphics[width=0.33\textwidth]{output/output_pPb_data/h_dPhi_unfolded_yield_All_40_Ystar1_27_45_Pt1_90_45_Pt2_90_NoIsoR.pdf} \\
		\end{tabular}
	}
	\caption{Comparison of \ionetwo\ distributions with and without isolation requirement in \pPb\ data.}
\end{figure}
%\clearpage
%%%%%%%%%%%%%%%%%%%%%%%%%%%%
%\chapter{Effect of New JES Systematic Uncertainties}
%\label{sec:appendixjes}
%% !TEX root = thesis-ex.tex

\begin{figure}[ht]
	\centerline{
		\begin{tabular}{ccc}
			\includegraphics[width=0.33\textwidth]{output/output_pPb_data/h_width_JEScomp_final_40_Ystar1_27_28_Pt1_35_28_Pt2_35.pdf} &
			\includegraphics[width=0.33\textwidth]{output/output_pPb_data/h_width_JEScomp_final_40_Ystar1_27_35_Pt1_45_28_Pt2_35.pdf} &
			\includegraphics[width=0.33\textwidth]{output/output_pPb_data/h_width_JEScomp_final_40_Ystar1_27_35_Pt1_45_35_Pt2_45.pdf} \\
			\includegraphics[width=0.33\textwidth]{output/output_pPb_data/h_width_JEScomp_final_40_Ystar1_27_45_Pt1_90_28_Pt2_35.pdf} &
			\includegraphics[width=0.33\textwidth]{output/output_pPb_data/h_width_JEScomp_final_40_Ystar1_27_45_Pt1_90_35_Pt2_45.pdf} &
			\includegraphics[width=0.33\textwidth]{output/output_pPb_data/h_width_JEScomp_final_40_Ystar1_27_45_Pt1_90_45_Pt2_90.pdf} \\
		\end{tabular}
	}
	\caption{Effect on total systematic uncertainty on \conetwo\ after adding new JES uncertainties. Generally the effect is below 10\%.}
	\label{fig:jessyswidth}
\end{figure}

\begin{figure}[ht]
	\centerline{
		\begin{tabular}{ccc}
			\includegraphics[width=0.33\textwidth]{output/output_pPb_data/h_yield_JEScomp_final_40_Ystar1_27_28_Pt1_35_28_Pt2_35.pdf} &
			\includegraphics[width=0.33\textwidth]{output/output_pPb_data/h_yield_JEScomp_final_40_Ystar1_27_35_Pt1_45_28_Pt2_35.pdf} &
			\includegraphics[width=0.33\textwidth]{output/output_pPb_data/h_yield_JEScomp_final_40_Ystar1_27_35_Pt1_45_35_Pt2_45.pdf} \\
			\includegraphics[width=0.33\textwidth]{output/output_pPb_data/h_yield_JEScomp_final_40_Ystar1_27_45_Pt1_90_28_Pt2_35.pdf} &
			\includegraphics[width=0.33\textwidth]{output/output_pPb_data/h_yield_JEScomp_final_40_Ystar1_27_45_Pt1_90_35_Pt2_45.pdf} &
			\includegraphics[width=0.33\textwidth]{output/output_pPb_data/h_yield_JEScomp_final_40_Ystar1_27_45_Pt1_90_45_Pt2_90.pdf} \\
		\end{tabular}
	}
	\caption{Effect on total systematic uncertainty on \ionetwo\ after adding new JES uncertainties. Generally the effect is below 10\%, with some bins reaching 25\%. }
	\label{fig:jessysyield}
\end{figure}

\begin{figure}[ht]
	\centerline{
		\begin{tabular}{ccc}
			\includegraphics[width=0.33\textwidth]{output/All/pp_data_0/h_width_JEScomp_final_40_Ystar1_27_28_Pt1_35_28_Pt2_35.pdf} &
			\includegraphics[width=0.33\textwidth]{output/All/pp_data_0/h_width_JEScomp_final_40_Ystar1_27_35_Pt1_45_28_Pt2_35.pdf} &
			\includegraphics[width=0.33\textwidth]{output/All/pp_data_0/h_width_JEScomp_final_40_Ystar1_27_35_Pt1_45_35_Pt2_45.pdf} \\
			\includegraphics[width=0.33\textwidth]{output/All/pp_data_0/h_width_JEScomp_final_40_Ystar1_27_45_Pt1_90_28_Pt2_35.pdf} &
			\includegraphics[width=0.33\textwidth]{output/All/pp_data_0/h_width_JEScomp_final_40_Ystar1_27_45_Pt1_90_35_Pt2_45.pdf} &
			\includegraphics[width=0.33\textwidth]{output/All/pp_data_0/h_width_JEScomp_final_40_Ystar1_27_45_Pt1_90_45_Pt2_90.pdf} \\
		\end{tabular}
	}
	\caption{Effect on total systematic uncertainty on ratio \cppb\ after adding new JES uncertainties. The total systematic uncertainty on \cppb\ before the addition of the new JES uncertainties is shown as the dotted red line, after the addition of the new JES uncertainties in the solid black line.Generally the absolute difference is below 2\%.}
	\label{fig:jessysratwidth}
\end{figure}

\begin{figure}[ht]
	\centerline{
		\begin{tabular}{ccc}
			\includegraphics[width=0.33\textwidth]{output/All/pp_data_0/h_yield_JEScomp_final_40_Ystar1_27_28_Pt1_35_28_Pt2_35.pdf} &
			\includegraphics[width=0.33\textwidth]{output/All/pp_data_0/h_yield_JEScomp_final_40_Ystar1_27_35_Pt1_45_28_Pt2_35.pdf} &
			\includegraphics[width=0.33\textwidth]{output/All/pp_data_0/h_yield_JEScomp_final_40_Ystar1_27_35_Pt1_45_35_Pt2_45.pdf} \\
			\includegraphics[width=0.33\textwidth]{output/All/pp_data_0/h_yield_JEScomp_final_40_Ystar1_27_45_Pt1_90_28_Pt2_35.pdf} &
			\includegraphics[width=0.33\textwidth]{output/All/pp_data_0/h_yield_JEScomp_final_40_Ystar1_27_45_Pt1_90_35_Pt2_45.pdf} &
			\includegraphics[width=0.33\textwidth]{output/All/pp_data_0/h_yield_JEScomp_final_40_Ystar1_27_45_Pt1_90_45_Pt2_90.pdf} \\
		\end{tabular}
	}
	\caption{Effect on total systematic uncertainty on ratio \ippb\ after adding new JES uncertainties. The total systematic uncertainty on \ippb\ before the addition of the new JES uncertainties is shown as the dotted red line, after the addition of the new JES uncertainties in the solid black line. Generally the absolute difference is below 2\%, with a one bin in the most negative \ystartwo\ region having an effect of 5\%. }
	\label{fig:jessysratyield}
\end{figure}
\FloatBarrier

%\clearpage
%%%%%%%%%%%%%%%%%%%%%%%%%%%%
%%\section{Appendix: Comparison of Results Before and After HEC/JES Fix}
%%\label{sec:appendixhec}
%%\begin{figure}[ht]
	\centerline{
		\begin{tabular}{cc}
			\includegraphics[width=0.5\textwidth]{figures/All/h_width_final_40_Ystar1_27_28_Pt1_35.pdf} &
			\includegraphics[width=0.5\textwidth]{output/All/pp_data_0/h_width_final_40_Ystar1_27_28_Pt1_35.pdf} \\
			\includegraphics[width=0.5\textwidth]{figures/All/h_width_final_40_Ystar1_27_35_Pt1_45.pdf} &
			\includegraphics[width=0.5\textwidth]{output/All/pp_data_0/h_width_final_40_Ystar1_27_35_Pt1_45.pdf} \\
			\includegraphics[width=0.5\textwidth]{figures/All/h_width_final_40_Ystar1_27_45_Pt1_90.pdf} &
			\includegraphics[width=0.5\textwidth]{output/All/pp_data_0/h_width_final_40_Ystar1_27_45_Pt1_90.pdf} \\
		\end{tabular}
	}
	\caption{Comparison of \wonetwo\ distributions in  \pp\ (open symbols) and \pPb\ (closed symbols) collisions for different selections of \ptone\ and \pttwo\ as a function of \ystartwo. Left column is results before the HEC fix in \pPb\, right row is after the HEC fix plus additional uncertainties from such. The shaded boxes indicate systematic uncertainties, vertical error bars represent statistical uncertainties. Besides small changes in statistical uncertainties, there is not a significant difference in results.}
	\label{fig:finalplotscompwidth}
\end{figure}

\begin{figure}[ht]
	\centerline{
		\begin{tabular}{cc}
			\includegraphics[width=0.5\textwidth]{figures/All/h_yield_final_40_Ystar1_27_28_Pt1_35.pdf} &
			\includegraphics[width=0.5\textwidth]{output/All/pp_data_0/h_yield_final_40_Ystar1_27_28_Pt1_35.pdf} \\
			\includegraphics[width=0.5\textwidth]{figures/All/h_yield_final_40_Ystar1_27_35_Pt1_45.pdf} &
			\includegraphics[width=0.5\textwidth]{output/All/pp_data_0/h_yield_final_40_Ystar1_27_35_Pt1_45.pdf} \\
			\includegraphics[width=0.5\textwidth]{figures/All/h_yield_final_40_Ystar1_27_45_Pt1_90.pdf} &
			\includegraphics[width=0.5\textwidth]{output/All/pp_data_0/h_yield_final_40_Ystar1_27_45_Pt1_90.pdf} \\
		\end{tabular}
	}
	\caption{Comparison of \ionetwo\ distributions in  \pp\ (open symbols) and \pPb\ (closed symbols) collisions for different selections of \ptone\ and \pttwo\ as a function of \ystartwo. Left column is results before the HEC fix in \pPb\, right row is after the HEC fix plus additional uncertainties from such. The shaded boxes indicate systematic uncertainties, vertical error bars represent statistical uncertainties. Most negative \ystartwo\ bins show clear difference in in the yields, which increased as a result of the unfolding accounting for the faulty HEC region.}
	\label{fig:finalplotscompyield}
\end{figure}

%%\clearpage
%%%%%%%%%%%%%%%%%%%%%%%%%%%%
%\chapter{Fitting Systematic Uncertainties}
%\label{sec:appendixfitting}
%% !TEX root = thesis-ex.tex

\begin{figure}[ht]
    \centerline{
        \begin{tabular}{ccc}
            \includegraphics[width=0.33\textwidth]{output/output_pp_data/h_dPhi_unfolded_width_All_40_Ystar1_27_28_Pt1_35_28_Pt2_35.pdf} &
            \includegraphics[width=0.33\textwidth]{output/output_pp_data/h_dPhi_unfolded_width_All_40_Ystar1_27_35_Pt1_45_28_Pt2_35.pdf} &
            \includegraphics[width=0.33\textwidth]{output/output_pp_data/h_dPhi_unfolded_width_All_40_Ystar1_27_35_Pt1_45_35_Pt2_45.pdf} \\
            \includegraphics[width=0.33\textwidth]{output/output_pp_data/h_dPhi_unfolded_width_All_40_Ystar1_27_45_Pt1_90_28_Pt2_35.pdf} &
            \includegraphics[width=0.33\textwidth]{output/output_pp_data/h_dPhi_unfolded_width_All_40_Ystar1_27_45_Pt1_90_35_Pt2_45.pdf} &
            \includegraphics[width=0.33\textwidth]{output/output_pp_data/h_dPhi_unfolded_width_All_40_Ystar1_27_45_Pt1_90_45_Pt2_90.pdf} \\
        \end{tabular}
    }
    \caption{ For \pp\ data, comparison of fits in default range (black) and extended range (red) and their ratios, which represent the systematic uncertainty on the fits. Due to large statistical fluctuations in some points, the ratios are fitted to a constant.  Empty black points show result of statistical RMS calculation. }
    \label{fig:ppsystfits}
\end{figure}

\begin{figure}[ht]
	\centerline{
		\begin{tabular}{ccc}
			\includegraphics[width=0.33\textwidth]{output/output_pPb_data/h_dPhi_unfolded_width_All_40_Ystar1_27_28_Pt1_35_28_Pt2_35.pdf} &
			\includegraphics[width=0.33\textwidth]{output/output_pPb_data/h_dPhi_unfolded_width_All_40_Ystar1_27_35_Pt1_45_28_Pt2_35.pdf} &
			\includegraphics[width=0.33\textwidth]{output/output_pPb_data/h_dPhi_unfolded_width_All_40_Ystar1_27_35_Pt1_45_35_Pt2_45.pdf} \\
			\includegraphics[width=0.33\textwidth]{output/output_pPb_data/h_dPhi_unfolded_width_All_40_Ystar1_27_45_Pt1_90_28_Pt2_35.pdf} &
			\includegraphics[width=0.33\textwidth]{output/output_pPb_data/h_dPhi_unfolded_width_All_40_Ystar1_27_45_Pt1_90_35_Pt2_45.pdf} &
			\includegraphics[width=0.33\textwidth]{output/output_pPb_data/h_dPhi_unfolded_width_All_40_Ystar1_27_45_Pt1_90_45_Pt2_90.pdf} \\
		\end{tabular}
	}
	\caption{ For \pPb\ data, comparison of fits in default range (black) and extended range (red) and their ratios, which represent the systematic uncertainty on the fits. Due to large statistical fluctuations in some points, the ratios are fitted to a constant.  Empty black points show result of statistical RMS calculation.}
	\label{fig:pPbsystfits}
\end{figure}


%\clearpage
%%%%%%%%%%%%%%%%%%%%%%%%%%%%
%\chapter{Unfolded \conetwo\ Distributions from Data with Systematic Uncertainties}
%\label{sec:appendixfinalplots}
%\section{\conetwo\ distributions with no $\Delta \pt$ requirement }
\begin{figure}[ht]
	\centerline{
		\begin{tabular}{ccc}
			\includegraphics[width=0.33\textwidth]{output/All/pp_data_0/h_dPhi_final_40_Ystar1_27_28_Pt1_35_28_Pt2_35_40_Ystar2_27.pdf} &
			\includegraphics[width=0.33\textwidth]{output/All/pp_data_0/h_dPhi_final_40_Ystar1_27_28_Pt1_35_28_Pt2_35_27_Ystar2_18.pdf} &
			\includegraphics[width=0.33\textwidth]{output/All/pp_data_0/h_dPhi_final_40_Ystar1_27_28_Pt1_35_28_Pt2_35_18_Ystar2_0.pdf} \\
			\includegraphics[width=0.33\textwidth]{output/All/pp_data_0/h_dPhi_final_40_Ystar1_27_28_Pt1_35_28_Pt2_35_0_Ystar2_18.pdf} &
			\includegraphics[width=0.33\textwidth]{output/All/pp_data_0/h_dPhi_final_40_Ystar1_27_28_Pt1_35_28_Pt2_35_18_Ystar2_40.pdf} &
			\includegraphics[width=0.33\textwidth]{output/All/pp_data_0/h_dPhi_final_40_Ystar1_27_35_Pt1_45_28_Pt2_35_40_Ystar2_27.pdf} \\
			\includegraphics[width=0.33\textwidth]{output/All/pp_data_0/h_dPhi_final_40_Ystar1_27_35_Pt1_45_28_Pt2_35_27_Ystar2_18.pdf} &
			\includegraphics[width=0.33\textwidth]{output/All/pp_data_0/h_dPhi_final_40_Ystar1_27_35_Pt1_45_28_Pt2_35_18_Ystar2_0.pdf} &
			\includegraphics[width=0.33\textwidth]{output/All/pp_data_0/h_dPhi_final_40_Ystar1_27_35_Pt1_45_28_Pt2_35_0_Ystar2_18.pdf} \\
		\end{tabular}
	}
	\caption{Unfolded \conetwo\ distributions in  \pp\ (red symbols) and \pPb\ (black symbols) collisions for different selections of \ptone, \pttwo, and \ystartwo\ as a function of \Dphi. Lines represent results of the fit (for more details see the text). Open boxes represent correlated systematic uncertainties and vertical error bars represent statistical uncertainties. Results are shown with no $\Delta\pt$ requirement.}
\end{figure}

\begin{figure}[ht]
	\centerline{
		\begin{tabular}{ccc}
			\includegraphics[width=0.33\textwidth]{output/All/pp_data_0/h_dPhi_final_40_Ystar1_27_35_Pt1_45_28_Pt2_35_18_Ystar2_40.pdf} &
			\includegraphics[width=0.33\textwidth]{output/All/pp_data_0/h_dPhi_final_40_Ystar1_27_35_Pt1_45_35_Pt2_45_40_Ystar2_27.pdf} &
			\includegraphics[width=0.33\textwidth]{output/All/pp_data_0/h_dPhi_final_40_Ystar1_27_35_Pt1_45_35_Pt2_45_27_Ystar2_18.pdf} \\
			\includegraphics[width=0.33\textwidth]{output/All/pp_data_0/h_dPhi_final_40_Ystar1_27_35_Pt1_45_35_Pt2_45_18_Ystar2_0.pdf} &
			\includegraphics[width=0.33\textwidth]{output/All/pp_data_0/h_dPhi_final_40_Ystar1_27_35_Pt1_45_35_Pt2_45_0_Ystar2_18.pdf} &
			\includegraphics[width=0.33\textwidth]{output/All/pp_data_0/h_dPhi_final_40_Ystar1_27_35_Pt1_45_35_Pt2_45_18_Ystar2_40.pdf} \\
			\includegraphics[width=0.33\textwidth]{output/All/pp_data_0/h_dPhi_final_40_Ystar1_27_45_Pt1_90_28_Pt2_35_40_Ystar2_27.pdf} &
			\includegraphics[width=0.33\textwidth]{output/All/pp_data_0/h_dPhi_final_40_Ystar1_27_45_Pt1_90_28_Pt2_35_27_Ystar2_18.pdf} &
			\includegraphics[width=0.33\textwidth]{output/All/pp_data_0/h_dPhi_final_40_Ystar1_27_45_Pt1_90_28_Pt2_35_18_Ystar2_0.pdf} \\
			\includegraphics[width=0.33\textwidth]{output/All/pp_data_0/h_dPhi_final_40_Ystar1_27_45_Pt1_90_28_Pt2_35_0_Ystar2_18.pdf} &
			\includegraphics[width=0.33\textwidth]{output/All/pp_data_0/h_dPhi_final_40_Ystar1_27_45_Pt1_90_28_Pt2_35_18_Ystar2_40.pdf} &
			\includegraphics[width=0.33\textwidth]{output/All/pp_data_0/h_dPhi_final_40_Ystar1_27_45_Pt1_90_35_Pt2_45_40_Ystar2_27.pdf} \\
		\end{tabular}
	}
	\caption{Unfolded \conetwo\ distributions in  \pp\ (red symbols) and \pPb\ (black symbols) collisions for different selections of \ptone, \pttwo, and \ystartwo\ as a function of \Dphi. Lines represent results of the fit (for more details see the text). Open boxes represent correlated systematic uncertainties and vertical error bars represent statistical uncertainties. Results are shown with no $\Delta\pt$ requirement.}
\end{figure}

\begin{figure}[ht]
	\centerline{
		\begin{tabular}{ccc}
			\includegraphics[width=0.33\textwidth]{output/All/pp_data_0/h_dPhi_final_40_Ystar1_27_45_Pt1_90_35_Pt2_45_27_Ystar2_18.pdf} &
			\includegraphics[width=0.33\textwidth]{output/All/pp_data_0/h_dPhi_final_40_Ystar1_27_45_Pt1_90_35_Pt2_45_18_Ystar2_0.pdf} &
			\includegraphics[width=0.33\textwidth]{output/All/pp_data_0/h_dPhi_final_40_Ystar1_27_45_Pt1_90_35_Pt2_45_0_Ystar2_18.pdf} \\
			\includegraphics[width=0.33\textwidth]{output/All/pp_data_0/h_dPhi_final_40_Ystar1_27_45_Pt1_90_35_Pt2_45_18_Ystar2_40.pdf} &
			\includegraphics[width=0.33\textwidth]{output/All/pp_data_0/h_dPhi_final_40_Ystar1_27_45_Pt1_90_45_Pt2_90_40_Ystar2_27.pdf} &
			\includegraphics[width=0.33\textwidth]{output/All/pp_data_0/h_dPhi_final_40_Ystar1_27_45_Pt1_90_45_Pt2_90_27_Ystar2_18.pdf} \\
			\includegraphics[width=0.33\textwidth]{output/All/pp_data_0/h_dPhi_final_40_Ystar1_27_45_Pt1_90_45_Pt2_90_18_Ystar2_0.pdf} &
			\includegraphics[width=0.33\textwidth]{output/All/pp_data_0/h_dPhi_final_40_Ystar1_27_45_Pt1_90_45_Pt2_90_0_Ystar2_18.pdf} &
			\includegraphics[width=0.33\textwidth]{output/All/pp_data_0/h_dPhi_final_40_Ystar1_27_45_Pt1_90_45_Pt2_90_18_Ystar2_40.pdf} \\
		\end{tabular}
	}
	\caption{Unfolded \conetwo\ distributions in  \pp\ (red symbols) and \pPb\ (black symbols) collisions for different selections of \ptone, \pttwo, and \ystartwo\ as a function of \Dphi. Lines represent results of the fit (for more details see the text). Open boxes represent correlated systematic uncertainties and vertical error bars represent statistical uncertainties. Results are shown with no $\Delta\pt$ requirement.}
\end{figure}

\FloatBarrier
\section{\conetwo\ distributions with a requirement of $\Delta\pt > 3$ GeV }
\begin{figure}[ht]
	\centerline{
		\begin{tabular}{ccc}
			\includegraphics[width=0.33\textwidth]{output.3pT/All/pp_data_0/h_dPhi_final_40_Ystar1_27_28_Pt1_35_28_Pt2_35_40_Ystar2_27.pdf} &
			\includegraphics[width=0.33\textwidth]{output.3pT/All/pp_data_0/h_dPhi_final_40_Ystar1_27_28_Pt1_35_28_Pt2_35_27_Ystar2_18.pdf} &
			\includegraphics[width=0.33\textwidth]{output.3pT/All/pp_data_0/h_dPhi_final_40_Ystar1_27_28_Pt1_35_28_Pt2_35_18_Ystar2_0.pdf} \\
			\includegraphics[width=0.33\textwidth]{output.3pT/All/pp_data_0/h_dPhi_final_40_Ystar1_27_28_Pt1_35_28_Pt2_35_0_Ystar2_18.pdf} &
			\includegraphics[width=0.33\textwidth]{output.3pT/All/pp_data_0/h_dPhi_final_40_Ystar1_27_28_Pt1_35_28_Pt2_35_18_Ystar2_40.pdf} &
			\includegraphics[width=0.33\textwidth]{output.3pT/All/pp_data_0/h_dPhi_final_40_Ystar1_27_35_Pt1_45_28_Pt2_35_40_Ystar2_27.pdf} \\
			\includegraphics[width=0.33\textwidth]{output.3pT/All/pp_data_0/h_dPhi_final_40_Ystar1_27_35_Pt1_45_28_Pt2_35_27_Ystar2_18.pdf} &
			\includegraphics[width=0.33\textwidth]{output.3pT/All/pp_data_0/h_dPhi_final_40_Ystar1_27_35_Pt1_45_28_Pt2_35_18_Ystar2_0.pdf} &
			\includegraphics[width=0.33\textwidth]{output.3pT/All/pp_data_0/h_dPhi_final_40_Ystar1_27_35_Pt1_45_28_Pt2_35_0_Ystar2_18.pdf} \\
		\end{tabular}
	}
	\caption{Unfolded \conetwo\ distributions in  \pp\ (red symbols) and \pPb\ (black symbols) collisions for different selections of \ptone, \pttwo, and \ystartwo\ as a function of \Dphi. Lines represent results of the fit (for more details see the text). Open boxes represent correlated systematic uncertainties and vertical error bars represent statistical uncertainties. Results are presented with a  requirement of $\Delta\pt > 3$ GeV.}
\end{figure}

\begin{figure}[ht]
	\centerline{
		\begin{tabular}{ccc}
			\includegraphics[width=0.33\textwidth]{output.3pT/All/pp_data_0/h_dPhi_final_40_Ystar1_27_35_Pt1_45_28_Pt2_35_18_Ystar2_40.pdf} &
			\includegraphics[width=0.33\textwidth]{output.3pT/All/pp_data_0/h_dPhi_final_40_Ystar1_27_35_Pt1_45_35_Pt2_45_40_Ystar2_27.pdf} &
			\includegraphics[width=0.33\textwidth]{output.3pT/All/pp_data_0/h_dPhi_final_40_Ystar1_27_35_Pt1_45_35_Pt2_45_27_Ystar2_18.pdf} \\
			\includegraphics[width=0.33\textwidth]{output.3pT/All/pp_data_0/h_dPhi_final_40_Ystar1_27_35_Pt1_45_35_Pt2_45_18_Ystar2_0.pdf} &
			\includegraphics[width=0.33\textwidth]{output.3pT/All/pp_data_0/h_dPhi_final_40_Ystar1_27_35_Pt1_45_35_Pt2_45_0_Ystar2_18.pdf} &
			\includegraphics[width=0.33\textwidth]{output.3pT/All/pp_data_0/h_dPhi_final_40_Ystar1_27_35_Pt1_45_35_Pt2_45_18_Ystar2_40.pdf} \\
			\includegraphics[width=0.33\textwidth]{output.3pT/All/pp_data_0/h_dPhi_final_40_Ystar1_27_45_Pt1_90_28_Pt2_35_40_Ystar2_27.pdf} &
			\includegraphics[width=0.33\textwidth]{output.3pT/All/pp_data_0/h_dPhi_final_40_Ystar1_27_45_Pt1_90_28_Pt2_35_27_Ystar2_18.pdf} &
			\includegraphics[width=0.33\textwidth]{output.3pT/All/pp_data_0/h_dPhi_final_40_Ystar1_27_45_Pt1_90_28_Pt2_35_18_Ystar2_0.pdf} \\
			\includegraphics[width=0.33\textwidth]{output.3pT/All/pp_data_0/h_dPhi_final_40_Ystar1_27_45_Pt1_90_28_Pt2_35_0_Ystar2_18.pdf} &
			\includegraphics[width=0.33\textwidth]{output.3pT/All/pp_data_0/h_dPhi_final_40_Ystar1_27_45_Pt1_90_28_Pt2_35_18_Ystar2_40.pdf} &
			\includegraphics[width=0.33\textwidth]{output.3pT/All/pp_data_0/h_dPhi_final_40_Ystar1_27_45_Pt1_90_35_Pt2_45_40_Ystar2_27.pdf} \\
		\end{tabular}
	}
	\caption{Unfolded \conetwo\ distributions in  \pp\ (red symbols) and \pPb\ (black symbols) collisions for different selections of \ptone, \pttwo, and \ystartwo\ as a function of \Dphi. Lines represent results of the fit (for more details see the text). Open boxes represent correlated systematic uncertainties and vertical error bars represent statistical uncertainties. Results are presented with a  requirement of $\Delta\pt > 3$ GeV.}
\end{figure}

\begin{figure}[ht]
	\centerline{
		\begin{tabular}{ccc}
			\includegraphics[width=0.33\textwidth]{output.3pT/All/pp_data_0/h_dPhi_final_40_Ystar1_27_45_Pt1_90_35_Pt2_45_27_Ystar2_18.pdf} &
			\includegraphics[width=0.33\textwidth]{output.3pT/All/pp_data_0/h_dPhi_final_40_Ystar1_27_45_Pt1_90_35_Pt2_45_18_Ystar2_0.pdf} &
			\includegraphics[width=0.33\textwidth]{output.3pT/All/pp_data_0/h_dPhi_final_40_Ystar1_27_45_Pt1_90_35_Pt2_45_0_Ystar2_18.pdf} \\
			\includegraphics[width=0.33\textwidth]{output.3pT/All/pp_data_0/h_dPhi_final_40_Ystar1_27_45_Pt1_90_35_Pt2_45_18_Ystar2_40.pdf} &
			\includegraphics[width=0.33\textwidth]{output.3pT/All/pp_data_0/h_dPhi_final_40_Ystar1_27_45_Pt1_90_45_Pt2_90_40_Ystar2_27.pdf} &
			\includegraphics[width=0.33\textwidth]{output.3pT/All/pp_data_0/h_dPhi_final_40_Ystar1_27_45_Pt1_90_45_Pt2_90_27_Ystar2_18.pdf} \\
			\includegraphics[width=0.33\textwidth]{output.3pT/All/pp_data_0/h_dPhi_final_40_Ystar1_27_45_Pt1_90_45_Pt2_90_18_Ystar2_0.pdf} &
			\includegraphics[width=0.33\textwidth]{output.3pT/All/pp_data_0/h_dPhi_final_40_Ystar1_27_45_Pt1_90_45_Pt2_90_0_Ystar2_18.pdf} &
			\includegraphics[width=0.33\textwidth]{output.3pT/All/pp_data_0/h_dPhi_final_40_Ystar1_27_45_Pt1_90_45_Pt2_90_18_Ystar2_40.pdf} \\
		\end{tabular}
	}
	\caption{Unfolded \conetwo\ distributions in  \pp\ (red symbols) and \pPb\ (black symbols) collisions for different selections of \ptone, \pttwo, and \ystartwo\ as a function of \Dphi. Lines represent results of the fit (for more details see the text). Open boxes represent correlated systematic uncertainties and vertical error bars represent statistical uncertainties. Results are presented with a  requirement of $\Delta\pt > 3$ GeV.}
\end{figure}


%\clearpage
%%%%%%%%%%%%%%%%%%%%%%%%%%%%%
%%\chapter{Comparison of Results From \pp\ Data and MC Samples}
%%\label{sec:appendixppdatavsmc}
%%
\section{Comparison of Results From Data and MC Samples}
It is interesting to look at a comparison of results between data and MC samples in both collision systems. The \pp\ and \pPb\ MC samples are simulated using the same \pythiaeight\ tune which do not include saturation effects. Thus, in the \pp\ collision system, where nuclear effects are not expected to be present, there should not be a difference between results of data and MC simulations. The comparison between data and MC will be shown with no $\Delta\pt$ requirement, as it was seen to make no significant effect on previous measurements. A comparison of \conetwo\ distributions between data and MC simulations is plotted in the left and right columns of Figure~\ref{fig:finalplotsMCwidth} for \pp\ and \pPb\ systems respectively. There is no statistically significant difference between the results from data and MC samples for either collision system. A comparison of \ionetwo\ distributions between data and MC simulations is plotted in the left and right columns of Figure~\ref{fig:finalplotsMCyield} for \pp\ and \pPb\ systems respectively. The results in the proton going direction indicate that there are could be differences between the models and measured quantities. However, these differences are stronger in the \pPb\ system. This could mean that there are effects not yet described by the MC generator.

\begin{figure}[ht]
	\centerline{
		\begin{tabular}{cc}
			\includegraphics[width=0.45\textwidth]{output/output_pp_data/h_width_final_40_Ystar1_27_28_Pt1_35.pdf} &
			\includegraphics[width=0.45\textwidth]{output/output_pPb_data/h_width_final_40_Ystar1_27_28_Pt1_35.pdf} \\
			\includegraphics[width=0.45\textwidth]{output/output_pp_data/h_width_final_40_Ystar1_27_35_Pt1_45.pdf} &
			\includegraphics[width=0.45\textwidth]{output/output_pPb_data/h_width_final_40_Ystar1_27_35_Pt1_45.pdf} \\
			\includegraphics[width=0.45\textwidth]{output/output_pp_data/h_width_final_40_Ystar1_27_45_Pt1_90.pdf} &
			\includegraphics[width=0.45\textwidth]{output/output_pPb_data/h_width_final_40_Ystar1_27_45_Pt1_90.pdf} \\
		\end{tabular}
	}
	\caption{Comparison of \wonetwo\ distributions in data (closed symbols) and MC (open symbols) samples in \pp\ (left) and \pPb\ (right) collisions in different selections of \ptone\ and \pttwo\ as a function of \ystartwo. The shaded boxes indicate systematic uncertainties, vertical error bars represent statistical uncertainties. The results show good closure between MC and data.  Results are shown with no $\Delta\pt$ requirement.}
	\label{fig:finalplotsMCwidth}
\end{figure}

\begin{figure}[ht]
	\centerline{
		\begin{tabular}{cc}
			\includegraphics[width=0.45\textwidth]{output/output_pp_data/h_yield_final_40_Ystar1_27_28_Pt1_35.pdf} &
			\includegraphics[width=0.45\textwidth]{output/output_pPb_data/h_yield_final_40_Ystar1_27_28_Pt1_35.pdf} \\
			\includegraphics[width=0.45\textwidth]{output/output_pp_data/h_yield_final_40_Ystar1_27_35_Pt1_45.pdf} &
			\includegraphics[width=0.45\textwidth]{output/output_pPb_data/h_yield_final_40_Ystar1_27_35_Pt1_45.pdf} \\
			\includegraphics[width=0.45\textwidth]{output/output_pp_data/h_yield_final_40_Ystar1_27_45_Pt1_90.pdf} &
			\includegraphics[width=0.45\textwidth]{output/output_pPb_data/h_yield_final_40_Ystar1_27_45_Pt1_90.pdf} \\
		\end{tabular}
	}
	\caption{Comparison of \wonetwo\ distributions in data (closed symbols) and MC (open symbols) samples in \pp\ (left) and \pPb\ (right) collisions in different selections of \ptone\ and \pttwo\ as a function of \ystartwo. The shaded boxes indicate systematic uncertainties, vertical error bars represent statistical uncertainties. The results in the proton going direction indicate that there are differences in the models and measured quantities. However, these differences are stronger in the \pPb\ system. This could mean that there are effects not yet described by the MC generator. Results are shown with no $\Delta\pt$ requirement.}
	\label{fig:finalplotsMCyield}
\end{figure}

\FloatBarrier

%%\clearpage
%%%%%%%%%%%%%%%%%%%%%%%%%%%%%
%%\section{Appendix: $r_{\mathrm{trk}}$ in \pPb}
%%\label{sec:appendixG}
%%\input{appendixG.tex}
\clearpage%---------------

\printbibliography
%[heading=bibintoc,title={References}]

\end{document}
