% !TEX encoding = UTF-8 Unicode
% !TEX root = thesis-ex.tex

This section gives an overview of the sources of systematic uncertainties on the \pp\ and \pbpb\ charged particle spectra associated with jet.
These include:

\begin{itemize}
\item Jet energy scale
\item Jet energy resolution
\item Tracking selections
%\item Truth track definition
%\item Detector material description in simulation
%\item Tracking in dense environments
%\item Fake track subtraction
%\item Track momentum
\item Unfolding
\item Underlying event contribution
\item MC non-closure
\end{itemize}

The systematic uncertainties are evaluated separately for \Dptr\ distributions and for their ratios as a function of jet \pT\ for \pp\ and \pbpb\ collisions.
For each systematic variation, the entire analysis procedure is repeated to ensure that the jets are treated in a consistent manner throughout the analysis.
The positive relative shift was used to calculate the upper bound of the systematic uncertainty, whereas the negative relative shift was used to calculate the lower bound.
All uncertainties except the unfolding and the MC non-closure are assumed to be correlated and are evaluated by comparing the \Rdptr\ distributions for the various systematic variations to the nominal \Rdptr\ distribution.
For uncorrelated systematic uncertainties, the uncertainty on the \RDptr\ distribution is evaluated by adding the uncertainties on the \pp\ and \pbpb\ \Dptr\ distributions in quadrature.
The total systematic uncertainties on the \Rdptr\ distributions for a selection of track \pt\ ranges (1.0--1.6 \GeV, 2.5--4.0 \GeV, 6.3--10 \GeV) in jets with \pt\ in the 126--158 \GeV\ range are shown in Figures~\ref{fig:rdptr_sys_uncert1} and \ref{fig:rdptr_sys_uncert2}. 
% Figure~\ref{fig:rdptr_sys_uncert1}-\ref{fig:rdptr_sys_uncert2}.
%The systematic uncertainties for other jet \pT\ intervals as show in appendix \ref{sec:appendixA}.



\begin{figure}
\centering
\begin{subfigure}[b]{\textwidth}
    \centering
    \includegraphics[page=1, width=\textwidth]{figures/main/systematics/Summary_ChPS_dR_sys_PbPb_error}
    \caption{}
    \label{fig:rdptr_sys_uncert1a}
\end{subfigure} \\
\begin{subfigure}[b]{\textwidth}
    \centering
    \includegraphics[page=3, width=\textwidth]{figures/main/systematics/Summary_ChPS_dR_sys_PbPb_error}
    \caption{}
    \label{fig:rdptr_sys_uncert1b}
\end{subfigure}\hfill
   \caption{A summary of the systematic uncertainties on \RDptr\ distributions for different track \mbox{$1.0 < \pt < 1.6$ GeV} (top) and \mbox{$2.5 < \pt < 4.0$ GeV} (bottom), for jets with \pt\ 126--158 \GeV, as a function of \rvar\ for different centrality bins.
Different panels are different centrality bins.
The total systematic uncertainty and its individual contributions are shown.}
\label{fig:rdptr_sys_uncert1}
\end{figure}


\begin{figure}
\centering
\begin{subfigure}[b]{\textwidth}
    \centering
    \includegraphics[page=5, width=\textwidth]{figures/main/systematics/Summary_ChPS_dR_sys_PbPb_error}
    \caption{}
    \label{fig:rdptr_sys_uncert2a}
\end{subfigure} \\
\begin{subfigure}[b]{\textwidth}
    \centering
    \includegraphics[page=6, width=\textwidth]{figures/main/systematics/Summary_ChPS_dR_sys_PbPb_error}
    \caption{}
    \label{fig:rdptr_sys_uncert2b}
\end{subfigure}\hfill
   \caption{A summary of the systematic uncertainties on \RDptr\ distributions for different track \mbox{$6.3 < \pt < 10.0$ GeV} (top) and \mbox{$10.0 < \pt < 25.1$ GeV} (bottom), for jets with \pt\ 126--158 \GeV, as a function of \rvar\ for different centrality bins.
Different panels are different centrality bins.
The total systematic uncertainty and its individual contributions are shown.}
\label{fig:rdptr_sys_uncert2}
\end{figure}

\subsection{Jet energy scale uncertainty}

The uncertainty on the JES for heavy ion jets has two parts.
The first is taken from \pp\ JES uncertainties for jets in \pp\ collisions while the second is specific to the heavy ion jets.
For the \pp\ part we use the strongly reduced set of 4 nuisance parameters using Scenario 1 as described in Ref.~\cite{JESuncertaintytwiki}.
Nuisance parameters that are not applicable for HI jet collections (pileup, b-jets, flavor and MC non closure) are removed or replaced (flavor uncertainties).
The heavy ion specific components are from the cross calibration~\cite{cc2015} and the jet flavor uncertainties at 5.02~TeV~\cite{2015392}.
For each component of the variation the response matrices are regenerated with the shifted \ptjet:

\begin{equation}
\pT^{\star,\mathrm{reco}} = \pT^{\mathrm{reco}} (1\pm U^{\mathrm{JES}}(\pT , \eta)).
\end{equation}
The data is then re-unfolded with these response matrices and the variation in the fragmentation functions is taken as the systematic uncertainty.

The centrality dependent uncertainty on the JES was evaluated by shifting the jet \pt\ of all measured jets up and down by shift between 0\% and 0.5\%.
The magnitude of the shift depends on the centrality in the way that the uncertainty on the jet \pt\ is 0.5\% in 1\% most central collisions and than linearly decreases to 0\% in 60\% peripheral bin.
The size of the shift reflects the uncertainty on the JES evaluated as using the $r-$track study where the sum of \pT\ of the tracks associated to a reconstructed jet is compared to the reconstructed jet \pT\ in ratio that is than compared between PbPb data and MC~\cite{HIjesnote,Aad:2014bxa}.



\subsection{Jet energy resolution}
To account for systematic uncertainties due to disagreement between the jet energy resolution in data and MC, the unfolding procedure was repeated with a modified response matrix.
The matrix was generated by repeating the MC study with modifications to the $\Delta \pt$ for each matched truth-reconstructed jet pair.
The procedure to generate modified migration matrices follows the standard procedure applied in \pp\ jet measurements and is used for both the \pp\ and \pbpb\ collisions.
The $\texttt{JetEnergyResolutionProvider}$ tool~\cite{JERUncertaintyProviderRun2} was used to retrieve uncertainty on the fractional resolution, $\sigma^{\mathrm{syst}}_{\mathrm{JER}}$ as a function of jet $\pt$ and $\eta$.
An additional HI jet specific uncertainty from the cross calibration of the HI jet collections~\cite{cc2015} is applied to jets in both \pp\ and \pbpb\ collisions.
The full JER uncertainty on 2015 \pp\ data is shown also in Ref.~\cite{Aad:1696485}
The jet $\pt^{\mathrm{reco}}$ was then smeared by

\begin{align}
\pt^{\star, \mathrm{reco}} = \pt^{\mathrm{reco}}\times \mathcal{N}(1,\sigma^{\mathrm{eff}}_{\mathrm{JER}})\,,
\end{align}
where $\mathcal{N}(1,\sigma^{\mathrm{eff}}_{\mathrm{JER}})$ is the normal distribution with the effective resolution $\sigma^{\mathrm{eff}}_{\mathrm{JER}}=\sqrt{(\sigma_{\mathrm{JER}} + \sigma^{\mathrm{syst}}_{\mathrm{JER}})^{2} - \sigma_{\mathrm{JER}}^{2}}$.

%%%%%%%%%%%%%%%%%
%The systematic uncertainties on the \Dptr\ distributions decreases with decreasing \pt\ and increasing jet \pT.
%The typical systematic uncertainty originating from JER changes varies from 10\% to 1\% depending on the jet \ET\ and $z$.
%%%%%%%%%%%%%%%%%


\subsection{Tracking selections}
\paragraph{Track selection}
This uncertainty was estimated by tightening the tracking cuts by adding the cuts on the significance of $d_0$ and $z_0$ as described in the Section~\ref{sec:trackselection}. 
The entire analysis is redone with these track selections (including re-deriving the tracking efficiencies and the $\eta-\phi$ maps for the UE estimation) and the difference from the nominal analysis is taken as the systematic uncertainty.

\paragraph{Truth track definition}  
This uncertainty quantifies the robustness of the matching of reconstructed to truth particles.
The uncertainty is taken as a difference in the final results obtained with  $\mcprob > 0.3$ and results obtained with $ \mcprob > 0.5$.
This systematic included a re-derivation of the $\eta-\phi$ maps for UE estimation.
%The change in tracking efficiency for these two selections is negligible.

\paragraph{Detector material description in simulation}
The uncertainty on the inner detector material varies with \pttrk\ and \etatrk\ from 0.5\% to 2.0\%~\cite{ref:tracktwiki} on the efficiency correction.
This systematic also included a re-derivation of the $\eta-\phi$ maps for UE estimation.

\paragraph{Tracking in dense environments}
There is a 0.4\% uncertainty on the efficiency due to tracking in dense environments (the core of the jet)~\cite{ref:tracktwiki}.
This systematic also included a re-derivation of the $\eta-\phi$ maps for UE estimation.

\paragraph{Fake rate and secondaries}
The uncertainty on the rate of fake tracks and secondaries is taken to be 30\% independent of \pttrk\ and \etatrk~\cite{ref:tracktwiki, Nachman:2259091}.
This uncertainty is conservatively symmetrized.

\paragraph{Uncertainty on the track momentum}
To account for a possible misalignment in \pp\ and \PbPb\ data, the reconstructed \pT\ of each track (corrected first as described in section~\ref{Sec:Trackmomentumcorrection}) was changed according to~\cite{TrackingRec}:

\begin{equation}
\pt \rightarrow \pt \times (1 + q \times \pt \delta_{sagitta}(\eta, \phi))^{-1},
\end{equation}
where $q$ is charge of the track and $\delta_{sagitta}(\eta, \phi)$ is uncertainty on the track curvature.
The uncertainty derived for 5.02~TeV \pp\ and \PbPb\ data is included in InDetTrackSystematicsTools-00-00-19.
Due to statistical origin of the uncertainty the resulting systematic uncertainty is symmetrized.
This systematic also included a re-derivation of the $\eta-\phi$ maps for UE estimation.

%%%%%%%%%%%%%%%%%
%The resulting systematic uncertainty is $<<1$\% for low and intermediate $z$ and \pT\ and reaches up to 4\% at high $z$.
%As the source of the shift is present both in \pp\ and \PbPb\ it does partially cancel in the ratios. 
%%%%%%%%%%%%%%%%%


\subsection{Systematic uncertainty due to unfolding}
The systematic uncertainty associated with the unfolding is connected with the sensitivity of the unfolding procedure to the choice of the input distributions.
The systematic is evaluated by generating response matrices from the MC distributions without the reweighting factor that is used to match the jet spectrum and \Dptr\ distributions in data, and then unfolding the data using these response matrices.
This has minimal effect on track \pt\ because of the good track momentum resolution in the kinematic region of interest.
The uncertainty is evaluated by comparing the nominal result with the un-reweighed result, and is considered to be uncorrelated between \pbpb\ and \pp.


\subsection{Systematic uncertainty due to the UE event subtraction}
The systematic uncertainty associated with the estimation of the UE has two main components: one is the statistical uncertainty on the $\eta-\phi$ maps used in the map method (described in section~\ref{sec:map_method}), and the other is the comparison of the map method to the alternative cone method (discussed in section~\ref{sec:cone_method}.
More details on the cone method can be found in Ref.~\cite{PhysRevC.98.024908}.
The contributions of both components to the underlying event uncertainty can be seen in Figure~\ref{fig:UE_sys_contrib}, with the uncertainty from the map statistic dominating in central collisions.
The uncertainty on the underlying event convolutes with the signal to background ratio to produce the uncertainty on the charged particle spectra.

\begin{figure}
\centering
\includegraphics[page=1,width=1.\textwidth]{figures/main/systematics/Summary_UE_RDpT_dR_sys_error}
\caption{Size of the individual contributions to the underlying event systematic uncertainty as a function for \rvar\ for 0-10\% \pbpb\ collisions, in 126-158 GeV jets, 1-1.6 GeV tracks.}
\label{fig:UE_sys_contrib}
\end{figure}


\paragraph{Uncertainty from map statistic:} 
The $\eta-\phi$ maps used in the estimation of the underlying event are sparsely populated for high track \pt\ and high \ptjet, and are susceptible to statistical fluctuations.
To take this into account, 100 pseudo-experiments are conducted to re-estimate the set of maps, with a bin-by-bin gaussian variation where the mean and standard deviation were taken to be the bin content and bin error from the nominal set of maps.
The distribution of the relative difference between each estimation of the shifted underlying event and and the nominal value is fit to a gaussian.
The width of this gaussian is taken to be the systematic uncertainty.
This uncertainty is symmetrized to be conservative.
A few examples of the distribution of normalized relative differences can be seen in Figure~\ref{fig:gaus_diff}.
The size of the systematic from this can be seen in Fig.\ref{fig:mapstat_corr}.


\begin{figure}
\begin{subfigure}{0.5\textwidth}
\centering
\includegraphics[width=1\textwidth]{figures/main/systematics/map_stat_gaus}
\caption{}
\label{fig:gaus_diff}
\end{subfigure}
\begin{subfigure}{0.5\textwidth}
\centering
\includegraphics[width=1\textwidth]{figures/main/systematics/map_stat_size}
\caption{}
\label{fig:mapstat_corr}
\end{subfigure}
\caption{(Left) An example of the relative difference between the nominal and shifted values of the UE, fit to a gaussian. The width is taken as the systematic uncertainty.
Wider distributions larger statistical uncertainty on the bin content in the $\eta-\phi$ map used to estimate the UE.
(Right) Size of the systematic uncertainty from the map statistic component, as a function for \pttrk\ and \ptjet\ for 0-10\% \pbpb\ collisions, $0.15 < r < 0.20$ away from the jet axis.}
\end{figure}

%\begin{figure}[h]
%\centering
%\includegraphics[width=0.75\textwidth]{figures/main/systematics/map_stat_size}
%\caption{Size of the systematic uncertainty from the map statistic component, as a function for \pttrk\ and \ptjet\ for 0-10\% \pbpb\ collisions, $0.15 < r < 0.20$ away from the jet axis.}
%\label{fig:mapstat_corr}
% \end{figure}


\paragraph{Uncertainty from cone method: } The difference between the UE from the two methods is discussed in section \ref{sec:cone_method} and is shown in Figure~\ref{fig:conemethod_mapmethod}.
The effect of the different UE estimation methods on the charged particle spectra is seen in Fig.\ref{fig:conemethod_chps_comparison}.
This uncertainty is conservatively symmetrized.
While the absolute size of the uncertainty on the UE is typically small, the small signal-to-background ratio makes this the dominant systematic uncertainty in central collisions for lowest \pT\ tracks and large \rvar.

\begin{figure}
\centerline{\includegraphics[page=2,width=1.\textwidth]{figures/main/systematics/ChPS_UE_Comparison}}
\caption{Ratio of the charged particle spectra as determined using two different UE estimation methods as a function for \rvar\ for 0-10\% \pbpb\ collisions in 126-158 GeV jets and 1-1.6 GeV tracks.
Deviations from unity are a combination of the difference between the two methods and the signal to background ratio.
The largest differences between the spectra are seen at large \rvar, where the signal to background is the smallest.
Points are offset along the x-axis for ease of viewing.}
\label{fig:conemethod_chps_comparison}
\end{figure}




\subsection{MC non-closure}
To make sure that all the sources of systematic uncertainties were covered, the systematic uncertainty from the non closure in the MC was also evaluated.
It was calculated using the technical closure (done using non-reweighed response matrices) between the fully corrected and reconstructed charged particle distributions in MC to the charged particle distributions evaluated at the truth level.
This uncertainty can be considered a measure of unknowns in the analysis, but it also includes fluctuations due to the finite statistics in the MC which are used to evaluate it (especially in high \pttrk\ regions of the analysis.
The non-closure can be seen in Figure~\ref{fig:pbpbclosure}.
The systematic uncertainty is taken to be uncorrelated between \pbpb\ and \pp 

\begin{figure}
\centerline{\includegraphics[page=1,width=1.\textwidth]{figures/main/systematics/ChPS_final_dR_PbPb_MC.pdf}}
\caption{Size of the non-closure as a function for \rvar\ for 0-10\% \pbpb\ collisions, in 126-158 GeV jets for different \pttrk\ ranges.
Points in the bottom panel are offset along the x-axis for ease of viewing.}
\label{fig:pbpbclosure}
\end{figure}



\subsection{Correlations between the systematic uncertainties in \pbpb\ and \pp\ collisions}
Due to the common analysis and reconstruction procedure, and detector conditions, the systematic uncertainties are correlated between the \pp\ and \pbpb\ collisions in most cases.
Table~\ref{tab:systematics} summarizes correlations between \pp\ and \PbPb\ and also point-to-point correlations of individual distributions.
The unfolding uncertainty is uncorrelated between the two systems because it comes from the sensitivity of the unfolding to the starting MC distribution.
In \pbpb\ collisions where the fragmentation is modified by the presence of the QGP, this sensitivity could be different than in \pp\ collisions where the fragmentation functions are quite similar to those in \pythiaeight~\cite{201865}.
The impact of the modification of the fragmentation process in \PbPb\ compared to \pp\ and MC simulations is account for in the HI specific data-driven and centrality dependent uncertainty on the JES.

\begin{table}[h]
\centering
\begin{tabular}{ | m{3cm} | m{3cm} | m{3cm} | m{3cm} |}
\hline
\textbf{Uncertainty} & \textbf{\pp\ and \PbPb\ correlated} & \textbf{Point-to-point correlated} & \textbf{One/two sided or symmetrized} \\ \hline
JES (\pp) & yes & yes & two sided \\ \hline
JES (HI) & no & yes & two sided \\ \hline
JER & yes & yes & symmetrized \\ \hline
Track selection & yes & yes & one sided \\ \hline
\mcprob & yes & yes & one sided \\ \hline
Material & yes & yes & one sided \\ \hline
Dense environment & yes & yes & one sided \\ \hline
Fake rate & yes & yes & symmetrized \\ \hline
Track momentum & yes & no & two sided \\ \hline
Unfolding & no & yes & symmetrized \\ \hline
UE subtraction & no & yes & symmetrized \\ \hline
MC non-closure & no & no & symmetrized \\ \hline
\end{tabular}
\caption{Summary of correlation of different systematic uncertainties.}
\label{tab:systematics}
\end{table}

In the case where the systematic uncertainties are correlated, we evaluate \Rdptr\ ratios using the systematic variation from the nominal distributions in both \pp\ and \pbpb.
The variation in the ratio is used as the systematic uncertainty.
The variations in the ratios are summed in quadrature to get the total systematic uncertainty on the ratio.


