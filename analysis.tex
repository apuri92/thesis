\subsection{Overview}
In both the \pp\ and \pPb\ MC and data samples, two highest \pt\ jets are used to study azimuthal angular correlations. The measurement uses jets with transverse momentum between 28~GeV to 90~GeV. Due to the jet radius \RFour, the full coverage of the forward detector up to $|\eta|=4.9$ is reduced to cover only up to $4.5$ in pseudorapidity. Furthermore, due to the center-of-mass rapidity shift of $\Delta y$=0.465 in the \pPb\ collision system, the corresponding \ystar\ interval that is studied is approximately $-4.0<\ystar<4.0$. The \ystar\ interval used in the measurement is consistent in the \pp\ and \pPb\ collision systems. The final observables in this analysis are widths of dijet \conetwo\ distributions and conditional yields. The widths are sensitive to broadening between the leading and sub-leading jets and the yields show the number of dijets, given a leading jet in each \pT\ and \ystar\ kinematic region. 

The binning of this measurement is summarized in  Table~\ref{tab:binning} and is composed of different combinations of \ystarone, \ystartwo, \ptone, and \pttwo, where \ystarone\ and \ptone\ is the position and transverse energy of the leading jet, and \ystartwo\ and \pttwo is the position and transverse energy of the sub-leading jet. Since the measurement aims to probe low-x partons, only the interval $2.7<\ystarone<4.0$, which is the proton going direction in \pPb\ is used. The \ystar\ binning is chosen to be consistent with the center of mass rapidity boundary between forward and central triggers in \pPb\ data taking. The transverse momentum binning was chosen to be on the boundaries of the \pt\ intervals used for different triggers in \pp\ data taking. 

The \conetwo\ distributions are evaluated as a function of \Dphi\ in combinations of \ystarone, \ystartwo, \ptone, and \pttwo\ bins, unfolded, and normalized by the leading jet \pt\ spectra. Leading jet \ptone\ spectra are estimated in different \ystarone\ bins and are also unfolded. The azimuthal correlation distributions are fitted to extract their widths  \wonetwo\ and integrated of over their full range to extract the conditional yields \ionetwo. The correct normalization by number of leading jets is important for the measurement of \ionetwo\ and thus must be analyzed carefully. 

\begin{table}[h]
	\centering
	\begin{tabular}{|| c | c | c || } 
		\hline
		\ptone Bins [GeV] & \pttwo Bins [GeV] & \ystartwo Bins \\ 
		\hline
		$28<\ptone<35$   & $28<\pttwo<35$  & $2.7<\ystarjet<4.0$ \\ 
		$35<\ptone<45$   & $35<\pttwo<45$  & $1.8<\ystarjet<2.7$ \\ 
		$45<\ptone<90$   & $45<\pttwo<90$  & $0.0<\ystarjet<1.8$ \\
						 & 				   & $-1.8<\ystarjet<0.0$ \\
						 &				   & $-4.0<\ystarjet<-1.8$ \\
		\hline
	\end{tabular}
	\caption{\label{tab:binning} Transverse momentum and \ystar\ binning for leading and sub-leading jets. For the leading jet, only the $2.7<\ystarone<4.0$ bin is used. }
\end{table}

To account for detector affects, the distributions in data have to be unfolded using MC information. The method used is the bin-by-bin unfolding which relies on MC information about the relationship between any truth and reconstructed quantity. This type of unfolding is sensitive to differences in the shapes of data and MC distributions and requires a re-weighting of the MC before unfolding factors can be evaluated. The re-weighting is done in two steps: 1) weights for jet \pt\ spectra are evaluated; 2) when deriving weights for \conetwo\ distributions, the dependence on the jet \pt\ spectra is removed by applying the weights from the previous step. The final weight is the product of the two weights.

To better match UE levels to the data, the \pPb\ MC is re-weighted at the event level. The total FCal $E_{T}$ distribution in MC is divided by the total FCal $E_{T}$ in data to derive the event weights which are then applied to the MC. The total FCal $E_{T}$ distributions in \pPb\ MC and data, along with the ratio between the two distributions are shown in Fig.~\ref{fig:fcalet}. 

\begin{figure}
	\centering
	\includegraphics[width=0.95\textwidth]{output/output_pPb_data/pPbFCalWeight.pdf} 
	\caption{ Total FCal $E_{T}$ distributions in \pPb\ MC and data (left), and ratio MC/Data (right). }	
	\label{fig:fcalet}
\end{figure}

\subsection{Unfolding Procedure}
\label{sec:unfolding}

Detector effects affecting the leading jet \pT\ spectra and $\mathrm{d}N_{1,2}/\mathrm{d}\Dphi$ distributions in \pp\ and \pPb\ collisions are corrected using a bin-by-bin unfolding procedure. For more information on the this procedure see Appendix~\ref{sec:appendixbbb}. The unfolding procedure corrects for the effect of the migration due to the finite JER, JAR, and the jet reconstruction efficiency. The jets excluded due to the disabled HEC region in \pPb\ data and MC samples are naturally accounted-for using the same procedure. Two corresponding MC distributions for each of the two observables are evaluated, one using generator level jets and the other using reconstructed jets after the detector simulation. The MC response matrices are also filled using the same procedure. The diagonal elements of these matrices represent pairs of truth and reconstructed jets agree in momentum and position intervals of the measurement. The response matrix is always a multidimensional object with twice the number of dimensions used in the phase space of the measurement. The ratio of these two MC distributions provides correction factors which are then applied to the data. The correction factors $C_{i}$ are defined as:  
\begin{eqnarray}
C_{i} = \frac{T_{i}}{R_{i}},
\label{eqn:factors}
\end{eqnarray}

where $T_{i}$ and $R_{i}$ are the number of truth and reconstructed dijets, respectively.  However, The reconstructed and generated distributions are manifestations of each other since they former is actually a detector reconstruction of its respective truth event. Thus, $T_{i}$ and $R_{i}$ are partially correlated, the resulting errors on the correction factors are defined as:
\begin{eqnarray}
\delta C_{i}^{2} = \frac{T_{i}^{2}}{R_{i}^{3}}\bigg(1-\frac{M_{ii}^{2}}{T_{i}R_{i}}\bigg),
\label{eqn:factorserrors}
\end{eqnarray}

where $M_{ii}$ are the diagonal elements of the response matrix. These errors take into account the correlation between the truth and reconstructed quantities. Errors on correlated quantities will be smaller than those on purely uncorrelated distributions because if there is no migration, i.e. the reconstructed quantities perfectly resemble their generator level counterparts, $M_{ii}=T_{i}=R_{i}$ and therefore $\delta C_{i}^{2} = 0$. However, there is insignificant migration in energy and position, so the diagonal matrix elements are rarely similar to either the reconstructed or generated counts.

As mentioned previously, bin-by-bin unfolding procedure is sensitive to the shapes of the distributions from which the correction factors are derived. This method works when the shape of the data distribution matches the shape of the MC distributions. Since both the\pt\ spectra and \conetwo\ distributions are unfolded with correction factors, both distributions must first be re-weighted. The weights are estimated as ratios of distributions of $\mathrm{Data/MC_{Reco}}$. The value of the weight for a given truth and reconstructed jet pair is obtained from the truth jet kinematics. This procedure is done for all jet measurements and is motivated by the need to re-weight the prior (truth) distribution. Further, re-weighting using reconstructed kinematics could introduce inefficiency to the response matrix. In the following procedure, jet \pt\ spectra weights are derived first. Then \conetwo\ weights are derived with the \pt\ spectra weight applied. With this intermediate re-weighting in jet \pt\ spectra, it is found that the \conetwo weights are invariant in \pT, allowing extrapolation into underflow and overflow bins in \pT, and reducing statistical fluctuations. Final \conetwo\ weights are derived only as a function of \Dphi\ in bins of \ystar, removing the \pT\ dependence. The product of \pt\ spectra weights and the \conetwo\ weights is applied to the final MC distributions when deriving the correction factors.

From the re-weighted MC truth and reconstructed distributions, correction factors are derived and applied to data both for the \pt\ spectra and $\mathrm{d}N_{1,2}/\mathrm{d}\Dphi$ distributions. The unfolded $\mathrm{d}N_{1,2}/\mathrm{d}\Dphi$  data distributions are scaled by the unfolded leading jet \pt\ spectra information to obtain \conetwo\ and are then fitted to the exponentially modified Gaussian function. The widths are extracted from fit results, and the conditional yields are extracted by integrating these \conetwo\ distributions.

\subsection{Jet Spectra}

Jets in \pp\ and \pPb\ data are required to have a trigger fired, and any jet(s) are required to be in the trigger's pseudorapidity range and transverse momentum interval where the trigger efficiency is above $99\%$. The jets are entered with prescale weights given by the ATLAS Lumi-Calc for each trigger and run. For the $2.7<\ystarone<4.0$ rapidity range, the contribution of different triggers to the final spectra is shown for \pp\ data in Fig.~\ref{fig:ppspectrawithtrig}. The leading jet \pt\ spectra for \pp\ data are presented in different forward \ystar\ bins on the left of Fig.~\ref{fig:spectra} and for \pPb\ data on the right of Fig.~\ref{fig:spectra}. In \pPb\ data, only one trigger with no pre-scale is used, thus, unlike the \pp\ spectra, where there are many trigger contributions, the final spectra is composed entirely of one trigger. The \pT\ binning is consistent with what is shown in Table~\ref{tab:binning} because these spectra will eventually be used for normalization of \Dphi\ distributions.

\begin{figure}
	\centering
	\includegraphics[width=0.49\textwidth]{output/output_pp_data/ystar_spect_All.pdf} 
	\includegraphics[width=0.49\textwidth]{output/output_pPb_data/ystar_spect_All.pdf} 
	
	\caption{ Single-jet \pt\ spectra for jets in \pp\ data (left) and \pPb\ data (right) in bins of \ystar. }	
	\label{fig:spectra}
\end{figure}

\begin{figure}
	\centering
	\includegraphics[width=0.65\textwidth]{output/output_pp_data/ystar_spect_fine_40_Ystar1_27.pdf} 
	\caption{ Individual triggers, and resulting jet \pT\ spectra for \pp\ data for the $2.7<\ystarone<4.0$ rapidity range. }	
	\label{fig:ppspectrawithtrig}
\end{figure}

In MC, jet \pt\ spectra are filled separately for each cross setction weighted (JZx) sample, and then combined using the cross section weights and filtering efficiencies. If no cross section weighted recombination is performed, the spectra will not be smooth and will have jumps at the jet \pt\ corresponding to the boundaries covered by the individual $JZ$ samples. The smoothly falling spectra  from MC show that the cross section weighted recombination is working correctly.  Reconstructed and truth leading jet \pt\ spectra for the \pp\ MC are shown in Fig.~\ref{fig:ppmcrecospectra} and for the \pPb\ MC in Fig.~\ref{fig:pPbmcrecospectra}. 

\begin{figure}
	\centerline{
		\begin{tabular}{cc}
			\includegraphics[width=0.45\textwidth]{output/output_pp_mc_pythia8/ystar_spect_reco_All.pdf} & 
			\includegraphics[width=0.45\textwidth]{output/output_pp_mc_pythia8/ystar_spect_truth_All.pdf}  \\
		\end{tabular}
	}
	\caption{ Reconstructed  (left) and truth (right) level leading jet \pt\ spectra in \pp\ MC in bins of \ystar.}	
	\label{fig:ppmcrecospectra}
\end{figure}

\begin{figure}
	\centerline{
		\begin{tabular}{cc}
			\includegraphics[width=0.45\textwidth]{output/output_pPb_mc_pythia8/ystar_spect_reco_All.pdf} & 
			\includegraphics[width=0.45\textwidth]{output/output_pPb_mc_pythia8/ystar_spect_truth_All.pdf} \\
		\end{tabular}
	}
	\caption{ Reconstructed  (left) and truth (right) level leading jet \pt\ spectra in \pPb\ MC in bins of \ystar.} \label{fig:pPbmcrecospectra}
\end{figure}

\FloatBarrier
\subsection{Jet Spectra Re-weighting}
The leading jet \pt\ spectra weights in both the \pp\ and \pPb\ MCs are derived as the ratio of $\mathrm{Data/MC_{Reco}}$ leading jet \pt\ spectra. The weights are derived by first scaling the Data and MC spectra to a common integral and then taking their quotient in bins of \ystar. Jet \pt\ spectra with fine binning are used to have better sensitivity to the shape. Scaled jet \pt\ spectra from data and reconstructed level MC are shown as the black and red points, respectively, on the top plots of Fig.~\ref{fig:spectweights}. Their ratio, which represents the jet \pt\ spectra re-weighting factors, is show by the blue points in the bottom plots of Fig.~\ref{fig:spectweights}. Jet \pt\ spectra weights are consistent with unity in \pp\ and \pPb\ collisions.

The shape of the re-weighted reconstructed level MC jet spectra should match the shape of the reconstructed level jet spectra from data. To check this, reconstructed jet spectra from data are compared to reconstructed jet spectra before and after re-weighting in MC. The ratio of data to re-weighted MC is consistent with unity for \pp\ and \pPb. The ratio and reconstructed jet \pt\ spectra as shown as the red points in Fig.~\ref{fig:spectweights}.

\begin{figure}[ht]
	\centerline{
		\begin{tabular}{cc}
			\includegraphics[width=0.5\textwidth]{output/output_pp_data/hSpectMC_40_Ystar1_27.pdf} &
			\includegraphics[width=0.5\textwidth]{output/output_pPb_data/hSpectMC_40_Ystar1_27.pdf} \\
		\end{tabular}
	}
	\caption{Reconstructed level data (black) and re-weighted (red) and default (blue) reconstructed jet spectra from MC, with ratios. Jet \pt\ spectra re-weighting factors are represented by the ratio of Data to reco MC (blue poitns in ratio).  The ratio to data to re-weighted MC (red points in ratio) is consistent with unity for \pp\ (left) and \pPb\ (right). Shown for $2.7<\ystarone<4.0$, which is the only \ystarone\ bin used in the analysis.}
	\label{fig:spectweights}
\end{figure}

Jet spectra are not re-weighted in \ystar\ because the effect from the JAR is much smaller than from JER and additionally, wide bins in rapidity are used. Putiry matrices for \pp\ and \pPb\ MC showing migration in \ystar\ are shown in Fig.~\ref{fig:ystarrespmat}. There is minor migration, with a purity of at least 97\% indicating no significant change in the shape of the distribution as a function of \ystar.

\begin{figure}[ht]
	\centerline{
		\begin{tabular}{cc}
			\includegraphics[width=0.5\textwidth]{output/output_pp_mc_pythia8/h_yStarRespMat_28_Pt_35.pdf} &
			\includegraphics[width=0.5\textwidth]{output/output_pPb_mc_pythia8/h_yStarRespMat_28_Pt_35.pdf} \\
		\end{tabular}
	}
	\caption{Purity matrices for \ystar, shown for \pp\ (left) and \pPb\ MCs. High purity indicates very minor effect on the shape of the distribution. Shown for the $28<\pt<35$ GeV interval.}
	\label{fig:ystarrespmat}
\end{figure}

\FloatBarrier
\subsection{Jet Spectra Unfolding}
To unfold the leading jet \pT\ spectra, the unfolding procedure described in~\ref{sec:unfolding} is used with correction factors obtained from the ratio the truth to reconstructed leading jet \pt\ spectra. The response matrix describes the bin migration between \pttruth\ and \ptreco. The \pp\ reconstructed and truth jet \pt\ spectra, with the response matrix and resulting correction factors are shown on the left of Fig.~\ref{fig:spectCFrespmat}. Similarly, the \pPb\ reconstructed and truth jet \pt\ spectra, with the response matrix and resulting correction factors are shown on the right of Fig.~\ref{fig:spectCFrespmat}. The correction factors and ratios of unfolded to reconstructed MC are shown as a check that the unfolding procedure is working correctly, not as a check of closure.

\begin{figure}[ht]
	\centerline{
		\begin{tabular}{cc}
			\includegraphics[width=0.5\textwidth]{output/output_pp_mc_pythia8/h_ystar_spect_unfolded_All_MUT_40_Ystar1_27.pdf} &
			\includegraphics[width=0.5\textwidth]{output/output_pPb_mc_pythia8/h_ystar_spect_unfolded_All_MUT_40_Ystar1_27.pdf} \\
			\includegraphics[width=0.5\textwidth]{output/output_pp_mc_pythia8/h_ystar_spect_respMat_All_40_Ystar1_27.pdf} &
			\includegraphics[width=0.5\textwidth]{output/output_pPb_mc_pythia8/h_ystar_spect_respMat_All_40_Ystar1_27.pdf} \\
		\end{tabular}
	}
	\caption{ Reconstructed and truth jet \pt\ spectra distributions (top plot), the resulting correction factors (middle plot) and the \pT\ response matrix (bottom plot). Results shown for \pp\ MC samples (left) and \pPb\ MC samples (right).}
	\label{fig:spectCFrespmat}
\end{figure}

\subsection{Dijet Azimuthal Distributions \conetwo}
Distributions of the azimuthal correlations \conetwo\ of two jets are constructed from the leading and sub-leading jet kinematics. In \pp\ and \pPb\ data, a trigger is required, and the leading jet is required to be in the trigger's pseudorapidity and transverse momentum range. In the dijet system there is a combinatoric contribution which comes from multi-parton scattering in both \pp\ and \pPb. This is corrected for by fitting to a constant in the range $0<|\Dphi|<1$, and subtracting the result on the full range $0<|\Dphi|<\pi$. The effect of the combinatoric subtraction (CS) is small, as can be seen in Fig.~\ref{fig:combsubt}, where \conetwo\ distributions with and without subtraction are shown, along with \wonetwo\ and \ionetwo\ results respectively. This is done at the reconstructed and truth levels in the same manner. The \Dphi\ distributions are then normalized by the leading jet \pt\ spectra counts, fitted to measure the widths, and integrated to measure the yields.

\begin{figure}[ht]
	\centerline{
		\begin{tabular}{cc}
			\includegraphics[width=0.5\textwidth]{output/output_pp_data/h_dPhi_unfolded_All_40_Ystar1_27_35_Pt1_45_28_Pt2_35_40_Ystar2_27.pdf} &
			\includegraphics[width=0.5\textwidth]{output/output_pPb_data/h_dPhi_unfolded_All_40_Ystar1_27_35_Pt1_45_28_Pt2_35_40_Ystar2_27.pdf} \\
		\end{tabular}
	}
	\caption{ \conetwo\ distributions for \pp\ and \pPb\ data showing the effects of the combinatoric subtraction. Red points are \conetwo\ distributions without combinatoric subtraction, black points are the same distributions with combinatoric subtration. Shown from $0<\Dphi<1$ is the fit to the tail of the \conetwo\ distribution. The analysis uses combinatoric subtraction by default. }
	\label{fig:combsubt}
\end{figure}


\subsection{ Re-weighting \conetwo\ Distributions }
The weights for \conetwo distributions in both \pp\ and \pPb\ MCs are derived as the ratios of Data to MC \conetwo\ distributions. This way, the \pT\ dependence of the azimuthal correlation distributions can be eliminated and only residual differences in shapes between data and MC distributions need to be accounted for. The \pp\ MC \conetwo\ weights in all combinations of \ptone\ and \pttwo\ and increasing bins in \ystartwo\ are shown in Fig.~\ref{fig:ppIndividualDphiWeights} as a function of \Dphi. In such fine binning the weights have very high statistical fluctuations but they are invariant in \pT, so they can be combined to form weights only depending on \ystartwo, as shown on the left of  Fig.~\ref{fig:alldphiweights}. To account for the still high statistical fluctuations in the tail of the distributions, the points are also smoothed. The \pPb\ \conetwo\ weights are evaluated with the same method. The \pPb\ MC \conetwo\ weights in all combinations of \ptone\ and \pttwo\ in increasing bins in \ystartwo\ are shown in Fig.~\ref{fig:pPbIndividualDphiWeights}, and the combined weights are shown on the right of Fig.~\ref{fig:alldphiweights}, all as a function of \Dphi. The \conetwo\ weights are consistent with unity near the peak of \conetwo\ distributions, where the effect of re-weighting is largest.  

\begin{figure}[ht]
	\centerline{
		\begin{tabular}{cc}
			\includegraphics[width=0.5\textwidth]{output/output_pp_mc_pythia8/cw_40_Ystar1_27_40_Ystar2_27.pdf} &
			\includegraphics[width=0.5\textwidth]{output/output_pp_mc_pythia8/cw_40_Ystar1_27_27_Ystar2_18.pdf} \\
		\end{tabular}
	}
	\caption{ \pp\ MC \conetwo\ weights shown for increasing bins of \ystartwo\ and all possible combinations of \ptone\ and \pttwo. Weights have high statistical fluctuations but are invariant in \pT. }
	\label{fig:ppIndividualDphiWeights}
\end{figure}

\begin{figure}[ht]
	\centerline{
		\begin{tabular}{cc}
			\includegraphics[width=0.5\textwidth]{output/output_pPb_mc_pythia8/cw_40_Ystar1_27_40_Ystar2_27.pdf} &
			\includegraphics[width=0.5\textwidth]{output/output_pPb_mc_pythia8/cw_40_Ystar1_27_27_Ystar2_18.pdf} \\
		\end{tabular}
	}
	\caption{ \pPb\ MC \conetwo\ weights shown for increasing bins of \ystartwo and all possible combinations of \ptone\ and \pttwo. Weights have high statistical fluctuations but are invariant in \pT. }
	\label{fig:pPbIndividualDphiWeights}
\end{figure}

\begin{figure}[ht]
	\centerline{
		\begin{tabular}{cc}
			\includegraphics[width=0.55\textwidth]{output/output_pp_mc_pythia8/h_dPhi_weights_All.pdf} &
			\includegraphics[width=0.55\textwidth]{output/output_pPb_mc_pythia8/h_dPhi_weights_All.pdf}\\
		\end{tabular}
	}
	\caption{ \pp\ (left) and \pPb\ (right) MC samples \conetwo\ weights for combined \pT\ bins, now shown only in bins of \ystartwo.  }
	\label{fig:alldphiweights}
\end{figure}

To properly use the re-weighting in the unfolding procedures, the shapes of re-weighted reconstructed MC distributions should be checked against those in data. There is not expected to be a complete match because the re-weighting is done as a function of truth kinematics, but it should pull the reconstructed distribution towards the data. Comparisons of the re-weighted and default MC distributions to the data are shown in Fig.~\ref{fig:ppweightscomp} for \pp\ and Fig.~\ref{fig:pPbweightscomp} for \pPb\ \conetwo\ distributions. The ratio of the data to re-weighted MC is constant in \Dphi, indicating a consistent shape. The ratio is fitted to a constant in a range where there is sufficient statistical precision ($2.5<\Dphi<\pi$). In order to test fit quality, probability distributions of the fit results are shown for \pp\ and \pPb\ in Fig.~\ref{fig:weightscompfitsflat}. The probability distributions are flat indicating a good fit to a constant function. 

\begin{figure}[ht]
	\centerline{
		\begin{tabular}{ccc}
			\includegraphics[width=0.33\textwidth]{output/output_pp_data/hMC_dPhi_40_Ystar1_27_28_Pt1_35_28_Pt2_35_40_Ystar2_27.pdf} &			\includegraphics[width=0.33\textwidth]{output/output_pp_data/hMC_dPhi_40_Ystar1_27_35_Pt1_45_28_Pt2_35_40_Ystar2_27.pdf} &
			\includegraphics[width=0.33\textwidth]{output/output_pp_data/hMC_dPhi_40_Ystar1_27_45_Pt1_90_45_Pt2_90_18_Ystar2_0.pdf} \\
		\end{tabular}
	}
	\caption{ \Dphi\ distributions for \pp\ data and MC. For MC, both re-weighted and default reconstructed distributions ares shown. The re-weighting makes the shapes flat in \Dphi\ as indicated by the constant ratio.}
	\label{fig:ppweightscomp}
\end{figure}

\begin{figure}[ht]
	\centerline{
		\begin{tabular}{ccc}
			\includegraphics[width=0.33\textwidth]{output/output_pPb_data/hMC_dPhi_40_Ystar1_27_28_Pt1_35_28_Pt2_35_40_Ystar2_27.pdf} &
			\includegraphics[width=0.33\textwidth]{output/output_pPb_data/hMC_dPhi_40_Ystar1_27_35_Pt1_45_28_Pt2_35_40_Ystar2_27.pdf} &
			\includegraphics[width=0.33\textwidth]{output/output_pPb_data/hMC_dPhi_40_Ystar1_27_45_Pt1_90_45_Pt2_90_18_Ystar2_0.pdf} \\
		\end{tabular}
	}
	\caption{ \Dphi\ distributions for \pPb\ data and MC. For MC, both re-weighted and default reconstructed distributions ares shown. The re-weighting makes the shapes flat in \Dphi\ as indicated by the constant ratio.}
	\label{fig:pPbweightscomp}
\end{figure}

\begin{figure}[ht]
	\centerline{
		\begin{tabular}{cc}
			\includegraphics[width=0.40\textwidth]{output/output_pp_data/h_probWeights_pp.pdf} &			\includegraphics[width=0.40\textwidth]{output/output_pPb_data/h_probWeights_pPb.pdf} \\
		\end{tabular}
	}
	\caption{ Probability distribution for constant fits to ratio of re-weighted reco MC to data \Dphi\ distributions. Shown for \pp\ (left) and \pPb\ (right) MCs. }
	\label{fig:weightscompfitsflat}
\end{figure}

\subsection{ Fitting of \conetwo\ Distributions } \label{sec:fitting}


The unfolded jet \pT\ spectra and $\mathrm{d}N_{1,2}/\mathrm{d}\Dphi$ are further used to evaluate \conetwo\ distributions both in \pp\ and \pPb\ collisions using the procedure described until this point. The \conetwo\ distributions are then fitted by a double-exponential distribution convoluted with a Gaussian function:
\begin{eqnarray}
f(\Dphi) = \int_{-\infty}^{\infty}\mathrm{d}\delta\frac{e^{-\delta^{2}/2\sigma^{2}}}{\sqrt{8\pi\sigma^{2}\tau^{2}}}e^{-|\Dphi -\delta|/\tau}
\label{eq:convolution}
\end{eqnarray}

where $\tau$ is the inverse slope of the exponential component and $\sigma$ is the width of the Gaussian distribution. All parameters are required to be positive. Evaluating the convolution of the Gaussian and double exponential functions, the resulting fit function used in the analysis is:
\begin{eqnarray}
f(\Dphi) = A\frac{e^{\sigma^2/2\tau^2}}{2\tau}\bigg(\frac{1}{2}e^{\Dphi/ \tau}\mathrm{Erfc}\bigg(\frac{1}{\sqrt{2}}\bigg[\frac{\Dphi}{\sigma}+\frac{\sigma}{\tau}\bigg]\bigg)+e^{-\Dphi/\tau}\bigg[1-\frac{1}{2}\mathrm{Erfc}\bigg(\frac{1}{\sqrt{2}}\bigg[\frac{\Dphi}{\sigma}-\frac{\sigma}{\tau}\bigg]\bigg)\bigg]\bigg),
\end{eqnarray} 

where $A$ is the overall scaling factor. From the fit function, the quantity chosen as the width is the second moment, or root-mean-square (RMS) of the probability density function in Eq~\ref{eq:convolution}: 
\begin{eqnarray}
\wonetwo = RMS(\conetwo) =  \sqrt{2\tau^2 + \sigma^{2}}.
\end{eqnarray}

The fitting procedure is performed for $2.5<\Dphi<\pi$, and is similar to the one used in a previous dijet  measurement~\cite{Chatrchyan:2014hqa}. However, the convolution of the Gaussian and double exponential functions is found to better describe the data around the peak of the \conetwo\ distributions than a pure exponential function. A fitting procedure is chosen rather than directly evaluating the RMS relative to $\pi$ in order to minimize the impact of statistical fluctuations.
The fit is performed for $2.5<\Dphi<\pi$, similarly to the phase-space used in a previous dijet measurement~\cite{Chatrchyan:2014hqa}. Fitting is chosen rather than directly evaluating the RMS relative to $\pi$ in order to minimize the impact of statistical fluctuations.  

%Di-jet azimuthal angular correlation distributions are fitted to an exponentially modified gaussion function. This fit function is obtained from a convolution of an exponential and a Gaussian, shown in Equation~\ref{eqn:conv}.

%Expanding the convolution of the Gaussian and exponential functions, the resulting formula used in the analysis is shown in Equation~\ref{eqn:fit}. 

%In the formula, the exponential component is $\tau$, the Gaussian component is $\sigma$, and $A$ is the overall multiplicative scaling factor.


%The fit function is not able to describe the \Dphi\ distributions in their full range, especially at large \Dphi\ away from $\pi$ due to multi-jets contributions which are not of interest to this analysis. The fit range is chosen from $2.5<\Dphi<\pi$, similar to the phase-space used in a previous dijet transverse momentum balance measurement~\cite{Chatrchyan:2014hqa}. The resulting width, which is defined as $RMS =  \sqrt{2\tau^2 + \sigma^{2}}$, is then extracted and plotted in bins of \ystarone, \ystartwo, \ptone, and \pttwo.      

\FloatBarrier
\subsection{ Unfolding \conetwo\ Distributions }
When filling the truth and reconstructed distributions in either \pp\ or \pPb, the leading jet weights shown in Fig.~\ref{fig:spectweights}, in addition to the \pT\ invariant \conetwo\ weights shown in Fig.s~\ref{fig:alldphiweights} for \pp\ and \pPb\ samples are applied in product. Using the re-weighted truth and reconstructed \conetwo\ distributions, along with the respective re-weighted response matrices, new correction factors are then derived using the bin-by-bin procedure described earlier. \conetwo\ distributions for truth, reconstructed, and unfolded \pp\ MC in two different bins of \ptone\ are shown on the left of Fig.~\ref{fig:spectunfoldingmc}, along with the correction factors and respective response matrices. Similarly, two different azimuthal correlation distributions for truth, reco, and unfolded \pPb\ MC distributions in different bins of \ptone\ are shown on the right of Fig.~\ref{fig:spectunfoldingmc}, along with the correction factors and respective response matrices. 
All of the correction factors derived from \pp\ and \pPb\ MC samples are shown in Appendices~\ref{sec:appendixcfactorpp} and~\ref{sec:appendixcfactorpPb}, respectively.

\begin{figure}[ht]
	\centerline{
		\begin{tabular}{cc}
			\includegraphics[width=0.5\textwidth]{output/output_pp_mc_pythia8/h_dPhi_unfolded_All_MUT_40_Ystar1_27_28_Pt1_35_28_Pt2_35_40_Ystar2_27.pdf} &
			\includegraphics[width=0.5\textwidth]{output/output_pp_mc_pythia8/h_dPhi_unfolded_All_MUT_40_Ystar1_27_35_Pt1_45_28_Pt2_35_40_Ystar2_27.pdf} \\
			\includegraphics[width=0.5\textwidth]{output/output_pp_mc_pythia8/h_dPhi_respMat_All_40_Ystar1_27_28_Pt1_35_28_Pt2_35_40_Ystar2_27.pdf} &
			\includegraphics[width=0.5\textwidth]{output/output_pp_mc_pythia8/h_dPhi_respMat_All_40_Ystar1_27_35_Pt1_45_28_Pt2_35_40_Ystar2_27.pdf} \\
		\end{tabular}
	}
	\caption{ MC reconstructed, truth, and unfolded \conetwo\ distributions for two different bins of \ptone, with correction factors (top row) and respective response matrices (bottom row) for \pp\ MC samples (left) and \pPb\ MC samples (right). }
	\label{fig:spectunfoldingmc}
\end{figure}

\subsection{MC Closure Test}
As a check of the unfolding procedure, the MC reconstructed results are unfolded using the derived correction factors. Unfolded reconstructed MC distributions should resemble those at the generator level (truth level). The comparison of the \pp\ MC truth and unfolded \conetwo\ distributions, and the respective ratios are shown in Fig.~\ref{fig:ppwidthsTruthUF} in bins of \ptone\ and \pttwo. Similarly, a comparison of \ionetwo\ in the \pp\ MC truth and unfolded distributions is shown in Fig.~\ref{fig:ppyieldsTruthUF}. The ratios between unfolded and truth results are consistent with unity within statistical uncertainties indicating there is good closure between the unfolded and truth results. The comparison of the \pPb\ MC truth and unfolded widths, and the respective ratios are shown in Fig.~\ref{fig:pPbwidthsTruthUF} in bins of \ptone\ and \pttwo. The similar comparison of conditional yields is shown in Fig.~\ref{fig:pPbyieldsTruthUF}. As in the case of the \pp\ system, the ratios between unfolded and truth results in the \pPb\ system are consistent with unity within statistical uncertainties indicating there is good closure between the unfolded and truth results. The few fluctuations seen in the ratios are statistical in origin.

\begin{figure}[ht]
	\centerline{
		\begin{tabular}{ccc}
			\includegraphics[width=0.33\textwidth]{output/All/pp_mc_pythia8_0/h_dPhi_width_40_Ystar1_27_28_Pt1_35.pdf} &
			\includegraphics[width=0.33\textwidth]{output/All/pp_mc_pythia8_0/h_dPhi_width_40_Ystar1_27_35_Pt1_45.pdf} &
			\includegraphics[width=0.33\textwidth]{output/All/pp_mc_pythia8_0/h_dPhi_width_40_Ystar1_27_45_Pt1_90.pdf} \\
		\end{tabular}
	}
	\caption{ Comparison of widths from fits to \conetwo\ distributions between unfolded and truth results for the \pp\ MC. Ratios are consistent with unity, indicating good unfolding closure. }
	\label{fig:ppwidthsTruthUF}
\end{figure}


\begin{figure}[ht]
	\centerline{
		\begin{tabular}{ccc}
			\includegraphics[width=0.33\textwidth]{output/All/pPb_mc_pythia8_0/h_dPhi_width_40_Ystar1_27_28_Pt1_35.pdf} &
			\includegraphics[width=0.33\textwidth]{output/All/pPb_mc_pythia8_0/h_dPhi_width_40_Ystar1_27_35_Pt1_45.pdf} &
			\includegraphics[width=0.33\textwidth]{output/All/pPb_mc_pythia8_0/h_dPhi_width_40_Ystar1_27_45_Pt1_90.pdf} \\
		\end{tabular}
	}
	\caption{ Comparison of widths fits to \conetwo\ distributions between unfolded and truth results for the \pPb\ MC. Ratios are consistent with unity, indicating good unfolding closure. }
	\label{fig:pPbwidthsTruthUF}
\end{figure}

\begin{figure}[ht]
	\centerline{
		\begin{tabular}{ccc}
			\includegraphics[width=0.33\textwidth]{output/All/pp_mc_pythia8_0/h_dPhi_yield_40_Ystar1_27_28_Pt1_35.pdf} &
			\includegraphics[width=0.33\textwidth]{output/All/pp_mc_pythia8_0/h_dPhi_yield_40_Ystar1_27_35_Pt1_45.pdf} &
			\includegraphics[width=0.33\textwidth]{output/All/pp_mc_pythia8_0/h_dPhi_yield_40_Ystar1_27_45_Pt1_90.pdf} \\
		\end{tabular}
	}
	\caption{ Comparison of \ionetwo\ between unfolded and truth results for the \pp\ MC. Ratios are consistent with unity, indicating good unfolding closure. }
	\label{fig:ppyieldsTruthUF}
\end{figure}

\begin{figure}[ht]
	\centerline{
		\begin{tabular}{ccc}
			\includegraphics[width=0.33\textwidth]{output/All/pPb_mc_pythia8_0/h_dPhi_yield_40_Ystar1_27_28_Pt1_35.pdf} &
			\includegraphics[width=0.33\textwidth]{output/All/pPb_mc_pythia8_0/h_dPhi_yield_40_Ystar1_27_35_Pt1_45.pdf} &
			\includegraphics[width=0.33\textwidth]{output/All/pPb_mc_pythia8_0/h_dPhi_yield_40_Ystar1_27_45_Pt1_90.pdf} \\
		\end{tabular}
	}
	\caption{ Comparison of \ionetwo\ between unfolded and truth results for the \pPb\ MC. Ratios are consistent with unity, indicating good unfolding closure. }
	\label{fig:pPbyieldsTruthUF}
\end{figure}


As an additional closure test, the jet \pT\ spectra and \conetwo\ correction factors derived from the \pythiaeight\ MC were applied to reconstructed jets from the \herwig\ MC. A comparison of unfolded and truth \conetwo\ and \ionetwo\ between the \pp\ \herwig\ and \pythiaeight\ MCs are shown in Fig.~\ref{fig:herwigpythiaclosure}. For the \pPb\ results, there is no additional MC so this test was only done on the \pp\ MC. Ratios of unfolded to truth distributions indicate good closure. From Table~\ref{tab:mcsamples} in the appendix, it is evident that the statistics in the \pp\ \herwig\ MC is roughly 50\% of the \pp\ \pythiaeight\ MC, so the resulting fluctuations are seen as statistical. 


\begin{figure}[ht]
	\centerline{
		\begin{tabular}{cc}
			\includegraphics[width=0.5\textwidth]{output/All/pp_mc_herwig_0/h_dPhi_width_40_Ystar1_27_35_Pt1_45.pdf} &
			\includegraphics[width=0.5\textwidth]{output/All/pp_mc_herwig_0/h_dPhi_yield_40_Ystar1_27_35_Pt1_45.pdf} \\
		\end{tabular}
	}
	\caption{ Comparison of \conetwo\ (left) and \ionetwo\ (right) between unfolded and truth results for the \pp\ \herwig\ MC. Unfolding is done using correction factors derived from the \pythiaeight\ MC. Ratios are consistent with unity, indicating good unfolding closure. }
	\label{fig:herwigpythiaclosure}
\end{figure}

\subsection{Isolation Requirements}
Initially, jets were required to be isolated such that if two jets were separated by a distance of $\Delta R < 0.2$, they were not considered in the event. This was done to avoid potential split jet contributions to the result. However, when comparing with NLO QCD, isolation requirements cause complications and as a result they were removed. The effect of the isolation requirement on \conetwo\ and \ionetwo\ distributions is very minor, as shown in Appendix~\ref{sec:appendixisolation}.


\FloatBarrier