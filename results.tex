% !TEX root = thesis-ex.tex

\subsection{\conetwo\ Distributions and Their Fits}
\begin{figure}[ht]
	\centerline{
		\begin{tabular}{ccc}
			\includegraphics[width=0.33\textwidth]{output/All/pp_data_0/h_dPhi_final_40_Ystar1_27_28_Pt1_35_28_Pt2_35_40_Ystar2_27.pdf} &
			\includegraphics[width=0.33\textwidth]{output/All/pp_data_0/h_dPhi_final_40_Ystar1_27_28_Pt1_35_28_Pt2_35_27_Ystar2_18.pdf} &
			\includegraphics[width=0.33\textwidth]{output/All/pp_data_0/h_dPhi_final_40_Ystar1_27_28_Pt1_35_28_Pt2_35_18_Ystar2_0.pdf} \\
			\includegraphics[width=0.33\textwidth]{output.3pT/All/pp_data_0/h_dPhi_final_40_Ystar1_27_28_Pt1_35_28_Pt2_35_40_Ystar2_27.pdf} &
			\includegraphics[width=0.33\textwidth]{output.3pT/All/pp_data_0/h_dPhi_final_40_Ystar1_27_28_Pt1_35_28_Pt2_35_27_Ystar2_18.pdf} &
			\includegraphics[width=0.33\textwidth]{output.3pT/All/pp_data_0/h_dPhi_final_40_Ystar1_27_28_Pt1_35_28_Pt2_35_18_Ystar2_0.pdf} \\
		\end{tabular}
	}
	\caption{ Unfolded \conetwo\ distributions in  \pp\ (red symbols) and \pPb\ (black symbols) collisions for different selections of \ptone, \pttwo, and \ystartwo\ as a function of \Dphi. Lines represent results of the fit (for more details see the text). Open boxes represent correlated systematic uncertainties and vertical error bars represent statistical uncertainties. Results are presented with no $\Delta\pt$ requirement (top row) and with a requirement of $\Delta\pt > 3$ GeV (bottom row). }
	\label{fig:dphipPbpp}
\end{figure}

This section presents results for \wonetwo\ and the \ionetwo\ distributions, and ratios, \cppb\ and \ippb, of these distributions in \pPb\ and \pp\ collisions in order to explore the effects of saturation of gluon distribution functions. These distributions are measured for pairs of leading and sub-leading jets in transverse momentum range of $28 < \pT < 90$~GeV. Leading jets are measured in the center-of-mass rapidity region $2.7<\ystarone<4.0$ and sub-leading jets in the center-of-mass rapidity of $-4.0<\ystartwo<4.0$. 

Examples of unfolded \conetwo\ distributions with systematic uncertainties in different intervals of \ystartwo, \ptone, and \pttwo\ evaluated in \pp\ and \pPb\ collisions are shown in Fig.~\ref{fig:dphipPbpp} together with the fit results. These results are presented with and without a requirement of $\Delta\pt > 3$ GeV. All the \conetwo\ distributions used in the analysis, with systematic uncertainties and fit result are shown for \pp\ and \pPb\ collisions with and without the $\Delta \pt$ requirement in Appendix~\ref{sec:appendixfinalplots}. The \conetwo\ distributions exhibit an exponential behavior, with a flattening, described by the Gaussian, near the peak at $\Dphi = \pi$. Fit quality is validated from the $\chi^{2}/NDF$ probability distribution shown in Figure~\ref{fig:finalfitsquality}. Since there is no physics motivation behind the fit function, a uniform probability distribution is not expected. 


\begin{figure}[t]
	\centerline{
		\begin{tabular}{c}
			\includegraphics[width=0.85\textwidth]{output/All/pp_data_0/h_chi2_prob.pdf} 
		\end{tabular}
	}
	\caption{Comparison of fit quality of unfolded \pp\ (red) and \pPb\ (black) results.}
	\label{fig:finalfitsquality}
\end{figure}

\subsection{Widths and Conditional Yields With no $\Delta \pt$ Requirement}
The results of measurements of \wonetwo\ in \pPb\ collisions and \pp\ collisions for different ranges of \ptone\ and \pttwo\ as a function of \ystartwo\ are  presented in left panels of Fig.~\ref{fig:finalplots}. The \wonetwo\ distribution increases with increasing rapidity separation between the leading and sub-leading jets both in the \pp\ and \pPb\ collisions. Further, the \wonetwo\ increases with imbalance in \pT\ between the leading and sub-leading jets. From the pQCD BFKL equation, the probability of additional soft radiation increases with larger rapidity separation between dijets, leading to a stronger \Dphi\ decorrelation~\cite{Orr:1998ps}. The results of the measurement of conditional yields \ionetwo\ in \pPb\ and \pp\ collisions are shown in the right panels of Fig.~\ref{fig:finalplots}. The \ionetwo\ distribution increases with the increasing rapidity separation between the two jets reaching a maximum for sub-leading jets in the $0<\ystartwo<1.8$ interval and decreases for larger rapidity separations between the two jets. This is attributed to the dijet cross section falling off faster at forward rapidities compared to the inclusive jet cross section. The shapes of the \ionetwo\ distributions for \pp\ and \pPb\ collisions are similar for all \ptone\ and \pttwo\ combinations. 

The ratios \cppb\ between \pPb\ collisions and \pp\ collisions for different rangess of \ptone\ and \pttwo\ as a function of \ystartwo\ are  consistent with unity and are presented in the left panel of Fig.~\ref{fig:finalratios}. The ratios \ippb\ between \pPb\ collisions and \pp\ collisions in the same bins of rapidity and transverse momenta are shown in the right graph of Fig.~\ref{fig:finalratios}. The uncertainty on both \cppb\ and \ippb\ is dominated by systematic uncertainties, which are correlated in jet \pt\ and \ystar. The ratios \ippb\ are consistent with unity for sub-leading jets in the lead-going direction. However, the ratio of conditional yields of jet pairs in the proton-going direction in \pPb\ collisions compared to \pp\ collisions is suppressed by approximately 20\%, with no significant dependence on jet \pt\ and rapidity of the sub-leading jet \ystartwo. In the most forward-forward configuration, with both jets in the lowest jet \pt\ interval $28<\ptone, \pttwo< 35$, the approximate $x$ range probed is $1.5 \times 10^{-4}<\xb< 10^{-3}$. The suppression is an indication of possible nuclear effects including saturation.

\subsection{Widths and Conditional Yields With a $\Delta \pt > 3$ GeV Requirement}
Results for the \wonetwo\ and the \ionetwo\ distributions from \pPb\ collisions and \pp\ collisions with a $\Delta\pt>3$ GeV requirement are shown in Fig.~\ref{fig:finalplotswithcuts}. The \conetwo\ distributions are unaffected by the $\Delta\pt$ cut, but the conditional yields \ionetwo\ are smaller than the results with no $\Delta\pt$ cut. This is expected because in bins of \pt\ with a width of 7~GeV to 10~GeV, a requirement that the leading and sub-leading jets have a $\Delta\pt > 3$ GeV will affect a significant portion of the statistics. Results for the ratios \cppb\ and \ippb\ with a $\Delta\pt>3$ GeV requirement are shown in Fig.~\ref{fig:finalplotswithcuts}. The ratios \cppb\ and \ippb\ are both unaffected by the $\Delta\pt$ cut indicating that having such a requirement does not have an impact on the study of possible saturation effects since both \pp\ and \pPb\ collisions are identically affected.


\begin{figure}[ht]
	\centerline{
		\begin{tabular}{cc}
			\includegraphics[width=0.5\textwidth]{output/All/pp_data_0/h_width_dist_final.pdf} &
			\includegraphics[width=0.5\textwidth]{output/All/pp_data_0/h_yield_dist_final.pdf} \\
		\end{tabular}
	}
	\caption{Comparison of \wonetwo\ (left) and \ionetwo\ (right) distributions in  \pp\ (open symbols) and \pPb\ (closed symbols) collisions for different selections of \ptone\ and \pttwo\ as a function of \ystartwo. The shaded and empty boxes indicate systematic uncertainties and vertical error bars represent statistical uncertainties. Results are presented with no $\Delta\pt$ requirement. }
	\label{fig:finalplots}
\end{figure}

\begin{figure}[ht]
	\centerline{
		\begin{tabular}{cc}
			\includegraphics[width=0.5\textwidth]{output.3pT/All/pp_data_0/h_width_dist_final.pdf} &
			\includegraphics[width=0.5\textwidth]{output.3pT/All/pp_data_0/h_yield_dist_final.pdf} \\
		\end{tabular}
	}
	\caption{Comparison of \wonetwo\ (left) and \ionetwo\ (right) distributions in  \pp\ (open symbols) and \pPb\ (closed symbols) collisions for different selections of \ptone\ and \pttwo\ as a function of \ystartwo. The shaded and empty boxes indicate systematic uncertainties and vertical error bars represent statistical uncertainties. Some points have been removed due to high statistical uncertainties. Results are presented with a  requirement of $\Delta\pt > 3$ GeV.}
	\label{fig:finalplotswithcuts}
\end{figure}



\begin{figure}[ht]
	\centerline{
		\begin{tabular}{c}
			\includegraphics[width=1.0\textwidth]{output/All/pp_data_0/h_width_ratio_together_final.pdf} \\
			\includegraphics[width=1.0\textwidth]{output/All/pp_data_0/h_yield_ratio_together_final.pdf} \\
		\end{tabular}
	}
	\caption{Ratios \cppb\ of \wonetwo\ (too) and \ippb\ of \ionetwo\ (bottom) between \pPb\ collisions and \pp\ collisions for different selections of \ptone\ and \pttwo\ as a function of \ystartwo. The open boxes indicate systematic uncertainties and vertical error bars represent statistical uncertainties. Results are presented with no $\Delta\pt$ requirement.}
	\label{fig:finalratios}
\end{figure}


\begin{figure}[ht]
	\centerline{
		\begin{tabular}{c}
			\includegraphics[width=1.0\textwidth]{output.3pT/All/pp_data_0/h_width_ratio_together_final.pdf} \\
			\includegraphics[width=1.0\textwidth]{output.3pT/All/pp_data_0/h_yield_ratio_together_final.pdf} \\
		\end{tabular}
	}
	\caption{Ratios \cppb\ of \conetwo\ (top) and  \ippb\ of \ionetwo\ (bottom) between \pPb\ collisions and \pp\ collisions for different selections of \ptone\ and \pttwo\ as a function of \ystartwo. The open boxes indicate systematic uncertainties and vertical error bars represent statistical uncertainties. Data points in the rapidity interval of $-4.0<\ystartwo<1.8$ are not presented due to limited statistical precision. Results are presented with a  requirement of $\Delta\pt > 3$ GeV.}
	\label{fig:finalratioswithcuts}
\end{figure}

