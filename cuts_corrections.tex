\subsection{Overview}

In both the \pp\ and \pPb\ MC and data samples, two highest \pt\ jets are used to study azimuthal angular correlations. This measurement uses jets with a transverse momentum from 28~GeV to 90~GeV, in a \ystar\ range from -4.0 to 4.0. The final observables in this analysis are widths of di-jet \Dphi\ distributions and conditional yields. The widths are sensitive to broadening between the leading and sub-leading jets and the yields show the number of di-jets, given a leading jet in each \pT\ and \ystar\ kinematic region. 

The binning of this measurement is summarized in  Table~\ref{tab:binning} and is composed of different combinations of \ystarone, \ystartwo, \ptone, and \pttwo, where (\ystarone, \ptone) is the position and transverse energy of the leading jet, and (\ystartwo, \pttwo) the position and transverse energy of the sub-leading jet. Since the measurement aims to probe low-x partons, only the interval $2.7<\ystarone<4.0$, which is the proton going direction in \pPb\ is used. In the 2016 \pPb\ Transverse momentum binning was chosen on the edges of \pt\ intervals used for different triggers in \pp. 

Leading jet \ptone\ spectra are estimated in different \ystarone\ bins, unfolded, and used as a normalization of \Dphi\ distributions. Di-jet azimuthal angular correlation distributions are evaluated as a function of \Dphi\ in combinations of \ystarone, \ystartwo, \ptone, and \pttwo\ bins, unfolded, and normalized by the leading jet \pt\ spectra. The \Dphi\ distributions are fitted to extract the widths, which do not depend on the overall normalization. Conditional yields are obtained by integrating the \Dphi\ distributions over their full range so the correct normalization by number of leading jets is important. 

\begin{table}
	\centering
	\begin{tabular}{|| c | c | c || } 
		\hline
		\ptone Bins [GeV] & \pttwo Bins [GeV] & \ystartwo Bins \\ 
		\hline
		$28<\ptone<35$   & $28<\pttwo<35$  & $2.7<\ystarjet<4.0$ \\ 
		$35<\ptone<45$   & $35<\pttwo<45$  & $1.8<\ystarjet<2.7$ \\ 
		$45<\ptone<90$   & $45<\pttwo<90$  & $0.0<\ystarjet<1.8$ \\
						 & 				   & $-1.8<\ystarjet<0.0$ \\
						 &				   & $-4.0<\ystarjet<-1.8$ \\
		\hline
	\end{tabular}
	\caption{\label{tab:binning} Transverse momentum and \ystar\ binning for leading and sub-leading jets. For the leading jet, only the $2.7<\ystarone<4.0$ bin is used. }
\end{table}

To account for detector affects, the distributions in data have to be unfolded using MC information. The unfolding method used is the bin-by-bin unfolding which relies on MC information about the relationship between any truth and reconstructed quantity. This type of unfolding is sensitive to differences in the shapes of data and MC distributions and requires a re-weighting of the MC before unfolding factors can be evaluated. 

\subsection{Unfolding Procedure}
\label{sec:unfolding}
Due to effects of bin migration from JER and position resolution, it is necessary to perform an unfolding to account for these effects. Bayesian unfolding was first attempted, but the sensitivity to statistic fluctuations did not give good convergence. As a result, the bin-by-bin unfolding is the method used throughout the analysis. With this procedure, migration along multiple \ystar\ and \pT\ bins can be accounted for, more information can be found in Appendix~\ref{sec:appendixA}. Pairs of truth and reconstructed jets are used to fill the respective distributions and response matrices. The diagonal elements of these matrices represent pairs of truth and reconstructed jets agree in momentum and position intervals of the measurement. The response matrix is always a multidimensional object with twice the number of dimensions used in the phase space of the measurement. In \Dphi\ bins with index $i$, the correction factors $C_{i}$ are defined as  

\begin{eqnarray}
C_{i} = \frac{T_{i}}{R_{i}}
\label{eqn:factors}
\end{eqnarray}

where $T_{i}$ and $R_{i}$ are the number of truth and reconstructed di-jets, respectively.  Due to the fact that $T_{i}$ and $R_{i}$ are partially correlated, the resulting errors on the correction factors are defined as

\begin{eqnarray}
\delta C_{i}^{2} = \frac{T_{i}^{2}}{R_{i}^{3}}\bigg(1-\frac{M_{ii}^{2}}{T_{i}R_{i}}\bigg)
\label{eqn:factorserrors}
\end{eqnarray}

where $M_{ii}$ are the diagonal elements of the response matrix. These errors take into account the correlation between the truth and reconstructed quantities.

The bin-by-bin unfolding procedure is sensitive to the shapes of the distributions from which the correction factors are derived. This method works when the shape of the data distribution matches the shape of the MC distributions. Since both the spectra and \Dphi\ distributions are unfolded with correction factors, both the MC spectra and MC \Dphi\ distributions must first be re-weighted. The weights are estimated as ratios of distributions of $\mathrm{Data/MC_{Reco}}$. The value of the weight for a given truth and reconstructed jet pair is obtained from the truth jet kinematics. This procedure is done for all jet measurements and is motivated by the need to re-weight the prior (truth) distribution. Further, re-weighting using reconstructed kinematics could introduce inefficiency to the response matrix. In the following procedure, jet \pt\ spectra weights are derived first. Then \Dphi\ weights are derived with the spectra weight applied. With this intermediate re-weighting in jet \pt\ spectra, it is found that the \Dphi\ weights are invariant in \pT, allowing extrapolation into underflow and overflow bins in \pT, and reducing statistical fluctuations. Final \Dphi\ weights are derived only as a function of \Dphi\ in bins of \ystar, removing the \pT\ dependence. The product of spectra weights and the \Dphi\ weights is applied to the final MC distributions when deriving the correction factors.

From the re-weighted MC truth and reconstructed distributions, correction factors are derived and applied to data both for the spectra and \Dphi\ distributions. The unfolded \Dphi\ data distributions are scaled by the unfolded leading jet \pt\ spectra information, and fitted to the exponentially modified Gaussian function. 


\subsection{Jet Spectra}

Jets in \pp\ and \pPb\ data are required to have a trigger fired, and any jet(s) are required to be in the trigger's pseudorapidity range and transverse momentum interval where the trigger efficiency is above $99\%$. The jets are entered with prescale weights given by the ATLAS Lumi-Calc for each trigger and run. For the $2.7<\ystarone<4.0$ rapidity range, the contribution of different triggers to the final spectra is shown for \pp\ data in Figure~\ref{fig:ppspectrawithtrig}. The leading jet \pt\ spectra for \pp\ data are presented in different forward \ystar\ bins in Figure~\ref{fig:ppspectra} and for \pPb\ data in Figure~\ref{fig:pPbspectra}. In \pPb\ data, only one trigger with no pre-scale is used, thus, unlike the \pp\ spectra, where there are many trigger contributions, the final spectra is composed entirely of one trigger. The \pT\ binning is consistent with what is shown in Table~\ref{tab:binning} because these spectra will eventually be used for normalization of \Dphi\ distributions.

\begin{figure}
	\centering
	\includegraphics[width=0.65\textwidth]{output/output_pp_data/ystar_spect_All.pdf} 
	\caption{ Single-jet \pt\ spectra for jets in \pp\ data in bins of \ystar. }	
	\label{fig:ppspectra}
\end{figure}

\begin{figure}
	\centering
	\includegraphics[width=0.65\textwidth]{output/output_pp_data/ystar_spect_fine_40_Ystar1_27.pdf} 
	\caption{ Individual triggers, and resulting jet \pT\ spectra for \pp\ data for the $2.7<\ystarone<4.0$ rapidity range. }	
	\label{fig:ppspectrawithtrig}
\end{figure}

\begin{figure}	\centering
	\includegraphics[width=0.65\textwidth]{output/output_pPb_data/ystar_spect_All.pdf} 
	\caption{ Single-jet \pt\ spectra for jets in \pPb\ data in bins of pseudorapidity. }	
	\label{fig:pPbspectra}
\end{figure}

In MC, jet \pt\ spectra are filled separately for each cross setction weighted (JZx) sample, and then combined using the cross section weights and filtering efficiencies. Reconstructed and truth leading jet \pt\ spectra for the \pp\ MC are shown in Figure~\ref{fig:ppmcrecospectra} and for the \pPb\ MC in Figure~\ref{fig:pPbmcrecospectra}. 

\begin{figure}
	\centerline{
		\begin{tabular}{cc}
			\includegraphics[width=0.45\textwidth]{output/output_pp_mc_pythia8/ystar_spect_reco_All.pdf} & 
			\includegraphics[width=0.45\textwidth]{output/output_pp_mc_pythia8/ystar_spect_truth_All.pdf}  \\
		\end{tabular}
	}
	\caption{ Reconstructed  (left) and truth (right) level leading jet \pt\ spectra in \pp\ MC in bins of \ystar.}	
	\label{fig:ppmcrecospectra}
\end{figure}

\begin{figure}
	\centerline{
		\begin{tabular}{cc}
			\includegraphics[width=0.45\textwidth]{output/output_pPb_mc_pythia8/ystar_spect_reco_All.pdf} & 
			\includegraphics[width=0.45\textwidth]{output/output_pPb_mc_pythia8/ystar_spect_truth_All.pdf} \\
		\end{tabular}
	}
	\caption{ Reconstructed  (left) and truth (right) level leading jet \pt\ spectra in \pPb\ MC in bins of \ystar.} \label{fig:pPbmcrecospectra}
\end{figure}

\FloatBarrier
\subsection{Jet Spectra Re-weighting}
The leading jet \pt\ spectra weights in both the \pp\ and \pPb\ MCs are derived as the ratio of $Data/MC_{Reco}$ leading jet \pt\ spectra. Jet spectra with fine \pT\ binning are used to have better sensitivity to the shape. The data and MC leading jet \pt\ spectra with fine \pT\ binning are shown for \pp\ in Figure~\ref{fig:ppspectfine}, and for \pPb\ in Figure~\ref{fig:pPbspectfine}. The weights are derived by first scaling the Data and MC spectra to a common integral and then taking their quotient in bins of \ystar. The spectra weights are smoothed to avoid introducing statistical fluctuations. The smoothed \pp\ and \pPb\ leading jet \pt\ spectra weights as a function of \ptone\ are shown in Figure~\ref{fig:spectweights}.

\begin{figure}[ht]
	\centerline{
		\begin{tabular}{cc}
			\includegraphics[width=0.5\textwidth]{output/output_pp_data/ystar_spect_fine_All.pdf} &
			\includegraphics[width=0.5\textwidth]{output/output_pp_mc_pythia8/ystar_spect_fine_reco_All.pdf} \\
		\end{tabular}
	}
	\caption{Leading jet \pt\ spectra in fine bins if \pT\ for \pp\ data (left) and MC (right). }
	\label{fig:ppspectfine}
\end{figure}


\begin{figure}[ht]
	\centerline{
		\begin{tabular}{cc}
			\includegraphics[width=0.5\textwidth]{output/output_pPb_data/ystar_spect_fine_All.pdf} &
			\includegraphics[width=0.5\textwidth]{output/output_pPb_mc_pythia8/ystar_spect_fine_reco_All.pdf} \\
		\end{tabular}
	}
	\caption{Leading jet \pt\ spectra in fine bins if \pT\ for \pPb\ data (left) and MC (right). }
	\label{fig:pPbspectfine}
\end{figure}

\begin{figure}[ht]
	\centerline{
		\begin{tabular}{cc}
			\includegraphics[width=0.5\textwidth]{output/output_pp_mc_pythia8/h_spect_weights_All.pdf} &
			\includegraphics[width=0.5\textwidth]{output/output_pPb_mc_pythia8/h_spect_weights_All.pdf} \\
		\end{tabular}
	}
	\caption{Leading jet \pt\ spectra weights for \pp\ (left) and \pPb\ (right). Only the $2.7<\ystarone<4.0$ bin is used in the analysis but the other \ystarone\ bins are shown in \pp\ for comparison.}
	\label{fig:spectweights}
\end{figure}

The shape of the re-weighted reconstructed level MC jet spectra should match the shape of the reconstructed level jet spectra from data. To check this, reconstructed jet spectra from data are compared to reconstructed jet spectra before and after re-weighting in MC. The ratio of data to re-weighted MC is consistent with unity for \pp\ and \pPb\ reconstructed jet spectra as shown in Figure~\ref{fig:spectwithwithoutweight}.

\begin{figure}[ht]
	\centerline{
		\begin{tabular}{cc}
			\includegraphics[width=0.5\textwidth]{output/output_pp_data/hSpectMC_40_Ystar1_27.pdf} &
			\includegraphics[width=0.5\textwidth]{output/output_pPb_data/hSpectMC_40_Ystar1_27.pdf} \\
		\end{tabular}
	}
	\caption{Reconstructed level data (black) and re-weighted (red) and default (blue) reconstructed jet spectra from MC, with ratios. The ratio of re-weighted MC to data is consistent with unity for \pp\ (left) and \pPb\ (right). Shown for  $2.7<\ystarone<4.0$.}
	\label{fig:spectwithwithoutweight}
\end{figure}

Jet spectra are not re-weighted in \ystar\ because the effect from the JAR is much smaller than from JER and additionally, wide bins in rapidity are used. Response matrices for \pp\ and \pPb\ MC showing migration in \ystar\ are shown in Figure~\ref{fig:ystarrespmat}. There is very minor migration, with a purity of over 99\% indicating no change in the shape of the distribution as a function of \ystar.

\begin{figure}[ht]
	\centerline{
		\begin{tabular}{cc}
			\includegraphics[width=0.5\textwidth]{output/output_pp_mc_pythia8/h_yStarRespMat_28_Pt_35.pdf} &
			\includegraphics[width=0.5\textwidth]{output/output_pPb_mc_pythia8/h_yStarRespMat_28_Pt_35.pdf} \\
		\end{tabular}
	}
	\caption{Response matrices for \ystar, shown for \pp\ (left) and \pPb\ MCs. High purity indicates very minor effect on the shape of the distribution. Shown for the $28<\pt<35$ GeV interval.}
	\label{fig:ystarrespmat}
\end{figure}

\FloatBarrier
\subsection{Jet Spectra Unfolding}
To unfold the leading jet \pT\ spectra, the unfolding procedure described in~\ref{sec:unfolding} is used with correction factors obtained from the ratio the truth to reconstructed leading jet \pt\ spectra. The response matrix describes the bin migration between \pttruth\ and \ptreco. The \pp\ reconstructed and truth jet \pt\ spectra, with the response matrix and resulting correction factors are shown in Figure~\ref{fig:ppspectCFrespmat}. Similarly, the \pPb\ reconstructed and truth jet \pt\ spectra, with the response matrix and resulting correction factors are shown in Figure~\ref{fig:pPbspectCFrespmat}. The correction factors and ratios of unfolded to reconstructed MC are shown as a check that the unfolding procedure is working correctly, not as a check of closure.

\begin{figure}[ht]
	\centerline{
		\begin{tabular}{c}
			\includegraphics[width=0.6\textwidth]{output/output_pp_mc_pythia8/h_ystar_spect_unfolded_All_MUT_40_Ystar1_27.pdf} \\
			\includegraphics[width=0.6\textwidth]{output/output_pp_mc_pythia8/h_ystar_spect_respMat_All_40_Ystar1_27.pdf} \\
		\end{tabular}
	}
	\caption{ \pp\ MC reconstructed and truth jet \pt\ spectra distributions (top plot), the resulting correction factors (middle plot) and the \pT\ response matrix (bottom plot). }
	\label{fig:ppspectCFrespmat}
\end{figure}

\begin{figure}[ht]
	\centerline{
		\begin{tabular}{cc}
			\includegraphics[width=0.6\textwidth]{output/output_pPb_mc_pythia8/h_ystar_spect_unfolded_All_MUT_40_Ystar1_27.pdf} \\
			\includegraphics[width=0.6\textwidth]{output/output_pPb_mc_pythia8/h_ystar_spect_respMat_All_40_Ystar1_27.pdf} \\
		\end{tabular}
	}
	\caption{ \pPb\ MC reconstructed and truth jet \pt\ spectra distributions with correction factors (top plot), and the \pT\ response matrix (bottom plot). }
	\label{fig:pPbspectCFrespmat}
\end{figure}

\FloatBarrier
\subsection{Di-Jet Azimuthal Angular Distributions}
Distributions of the azimuthal angular correlations |\Dphi| of two jets are constructed from the leading and sub-leading jet kinematics. In \pp\ and \pPb\ data, a trigger is required, and the leading jet is required to be in the trigger's pseudorapidity and transverse momentum range. In the di-jet system there is a combinatoric contribution which can come from split jets or multi-parton scattering in both \pp\ and \pPb, as well as hard scattering \pPb. This is corrected for by fitting to a constant in the range $0<|\Dphi|<1$, and subtracting the result on the full range $0<|\Dphi|<\pi$. This is done at the reconstructed and truth levels in the same manner.	 The \Dphi\ distributions are then normalized by the leading jet \pt\ spectra counts, fitted to measure the widths, and integrated to measure the yields.

\subsection{ Re-weighting \Dphi\ Distributions }
The weights for \Dphi\ distributions in both \pp\ and \pPb\ MCs are derived as the ratios of Data to MC \Dphi\ distributions. This way, the \pT\ dependence of the \Dphi\ distributions can be eliminated and only residual differences in shapes of \Dphi\ distributions between Data and MC need to be accounted for. The \pp\ MC \Dphi\ weights in all combinations of \ptone\ and \pttwo\ and increasing bins in \ystartwo\ are shown in Figure~\ref{fig:ppIndividualDphiWeights} as a function of \Dphi. In such fine binning the weights have very high statistical fluctuations but they are invariant in \pT, so they can be combined and smoothed to form weights only only depending on \ystartwo, as shown in Figure~\ref{fig:ppAllDphiWeights}. The \pPb\ \Dphi\ weights are evaluated with the same method. The \pPb\ MC \Dphi\ weights in all combinations of \ptone\ and \pttwo\ in increasing bins in \ystartwo\ are shown in Figure~\ref{fig:pPbIndividualDphiWeights}, and the combined and smoothed weights are shown in Figure~\ref{fig:pPbAllDphiWeights}, all as a function of \Dphi.

\begin{figure}[ht]
	\centerline{
		\begin{tabular}{cc}
			\includegraphics[width=0.5\textwidth]{output/output_pp_mc_pythia8/cw_40_Ystar1_27_40_Ystar2_27.pdf} &
			\includegraphics[width=0.5\textwidth]{output/output_pp_mc_pythia8/cw_40_Ystar1_27_27_Ystar2_18.pdf} \\
		\end{tabular}
	}
	\caption{ \pp\ MC \Dphi\ weights shown for increasing bins of \ystartwo\ and all possible combinations of \ptone\ and \pttwo. Weights have high statistical fluctuations but are invariant in \pT. }
	\label{fig:ppIndividualDphiWeights}
\end{figure}

\begin{figure}[ht]
	\centerline{
		\begin{tabular}{cc}
			\includegraphics[width=0.5\textwidth]{output/output_pPb_mc_pythia8/cw_40_Ystar1_27_40_Ystar2_27.pdf} &
			\includegraphics[width=0.5\textwidth]{output/output_pPb_mc_pythia8/cw_40_Ystar1_27_27_Ystar2_18.pdf} \\
		\end{tabular}
	}
	\caption{ \pPb\ MC \Dphi\ weights shown for increasing bins of \ystartwo and all possible combinations of \ptone\ and \pttwo. Weights have high statistical fluctuations but are invariant in \pT. }
	\label{fig:pPbIndividualDphiWeights}
\end{figure}

\begin{figure}[ht]
	\centerline{
		\begin{tabular}{c}
			\includegraphics[width=0.75\textwidth]{output/output_pp_mc_pythia8/h_dPhi_weights_All.pdf}\\
		\end{tabular}
	}
	\caption{ \pp\ MC \Dphi\ weights for combined \pT\ bins, now shown only in bins of \ystartwo.  }
	\label{fig:ppAllDphiWeights}
\end{figure}

\begin{figure}[ht]
	\centerline{
		\begin{tabular}{c}
			\includegraphics[width=0.75\textwidth]{output/output_pPb_mc_pythia8/h_dPhi_weights_All.pdf}\\
		\end{tabular}
	}
	\caption{ \pPb\ MC \Dphi\ weights for combined \pT\ bins, now shown only in bins of \ystartwo.  }
	\label{fig:pPbAllDphiWeights}
\end{figure}

\FloatBarrier

To properly use re-weighting in the unfolding procedure, the re-weighted reconstructed MC and data distributions should have a similar shape. There is not expected to be a complete match between Data and re-weighted MC because the re-weighting is done as a function of truth kinematics. Comparisons of the re-weighted and default MC distributions to the data are shown in Figure~\ref{fig:ppweightscomp} for \pp\ and Figure~\ref{fig:pPbweightscomp} for \pPb. The ratio of the data to re-weighted MC is constant in \Dphi, indicating a consistent shape. The ratio is fitted in the same range as \Dphi\ distributions ($2.5<\Dphi<\pi$) to a constant, and in order to test fit quality, probability distributions of the fit results are shown for \pp\ and \pPb\ in Figure~\ref{fig:weightscompfitsflat}. The probability distributions are flat indicating a good fit to constant. 

\begin{figure}[ht]
	\centerline{
		\begin{tabular}{ccc}
			\includegraphics[width=0.33\textwidth]{output/output_pp_data/hMC_dPhi_40_Ystar1_27_28_Pt1_35_28_Pt2_35_40_Ystar2_27.pdf} &			\includegraphics[width=0.33\textwidth]{output/output_pp_data/hMC_dPhi_40_Ystar1_27_35_Pt1_45_28_Pt2_35_40_Ystar2_27.pdf} &
			\includegraphics[width=0.33\textwidth]{output/output_pp_data/hMC_dPhi_40_Ystar1_27_45_Pt1_90_45_Pt2_90_18_Ystar2_0.pdf} \\
		\end{tabular}
	}
	\caption{ \Dphi\ distributions for \pp\ data and MC. For MC, both re-weighted and default reconstructed distributions ares shown. The re-weighting makes the shapes flat in \Dphi\ as indicated by the constant ratio.}
	\label{fig:ppweightscomp}
\end{figure}

\begin{figure}[ht]
	\centerline{
		\begin{tabular}{ccc}
			\includegraphics[width=0.33\textwidth]{output/output_pPb_data/hMC_dPhi_40_Ystar1_27_28_Pt1_35_28_Pt2_35_40_Ystar2_27.pdf} &
			\includegraphics[width=0.33\textwidth]{output/output_pPb_data/hMC_dPhi_40_Ystar1_27_35_Pt1_45_28_Pt2_35_40_Ystar2_27.pdf} &
			\includegraphics[width=0.33\textwidth]{output/output_pPb_data/hMC_dPhi_40_Ystar1_27_45_Pt1_90_45_Pt2_90_18_Ystar2_0.pdf} \\
		\end{tabular}
	}
	\caption{ \Dphi\ distributions for \pPb\ data and MC. For MC, both re-weighted and default reconstructed distributions ares shown. The re-weighting makes the shapes flat in \Dphi\ as indicated by the constant ratio.}
	\label{fig:pPbweightscomp}
\end{figure}

\begin{figure}[ht]
	\centerline{
		\begin{tabular}{cc}
			\includegraphics[width=0.45\textwidth]{output/output_pp_data/h_probWeights_pp.pdf} &			\includegraphics[width=0.45\textwidth]{output/output_pPb_data/h_probWeights_pPb.pdf} \\
		\end{tabular}
	}
	\caption{ Probability distribution for constant fits to ratio of re-weighted reco MC to data \Dphi\ distributions. Shown for \pp\ (left) and \pPb\ (right) MCs. }
	\label{fig:weightscompfitsflat}
\end{figure}


\subsection{ Fitting of \Dphi\ Distributions } \label{sec:fitting}

The unfolded jet \pT\ spectra and $\mathrm{d}N_{1,2}(\Dphi)/\mathrm{d}\Dphi$ are further used to evaluate \conetwo\ distributions both in \pp\ and \pPb\ collisions. The \conetwo\ distributions are then fitted by an a double-exponential distribution smeared by a Gaussian function.  This fit function is obtained from a convolution of a double-exponential and a Gaussian:

\begin{eqnarray}
f(x) = \int_{-\infty}^{\infty}d\delta\frac{e^{-\delta^{2}/2\sigma^{2}}}{\sqrt{8\pi\sigma^{2}\tau^{2}}}e^{-|x-\delta|/\tau}.
\end{eqnarray}

Expanding the convolution of the Gaussian and double exponential functions, the resulting formula used in the analysis is:

\begin{eqnarray}
f(x) = A\frac{e^{\sigma^2/2\tau^2}}{2\tau}\bigg(\frac{1}{2}e^{\frac{x}{\tau}}Erfc\bigg(\frac{1}{\sqrt{2}}\bigg[\frac{x}{\sigma}+\frac{\sigma}{\tau}\bigg]\bigg)+e^{\frac{-x}{\tau}}\bigg[1-\frac{1}{2}Erfc\bigg(\frac{1}{\sqrt{2}}\bigg[\frac{x}{\sigma}-\frac{\sigma}{\tau}\bigg]\bigg)\bigg]\bigg)
\end{eqnarray} 

where $\tau$ is the inverse slope of the exponential component, $\sigma$ the width of the Gaussian distribution, and $A$ is the overall scaling factor. The widths of \conetwo\ distributions are calculated as 

\begin{eqnarray}
RMS(\conetwo) =  \sqrt{2\tau^2 + \sigma^{2}}.
\end{eqnarray}
%The fit function is not able to describe the \Dphi\ distributions in their full range, especially at large \Dphi\ away from $\pi$ due to multi-jets contributions which are not of interest to this analysis. 
where \conetwo\ is fitted in the interval $2.5<\Dphi<\pi$, similarly to the phase-space used in a previous di-jet measurement~\cite{Chatrchyan:2014hqa}.

%Di-jet azimuthal angular correlation distributions are fitted to an exponentially modified gaussion function. This fit function is obtained from a convolution of an exponential and a Gaussian, shown in Equation~\ref{eqn:conv}.

%Expanding the convolution of the Gaussian and exponential functions, the resulting formula used in the analysis is shown in Equation~\ref{eqn:fit}. 

%In the formula, the exponential component is $\tau$, the Gaussian component is $\sigma$, and $A$ is the overall multiplicative scaling factor.


%The fit function is not able to describe the \Dphi\ distributions in their full range, especially at large \Dphi\ away from $\pi$ due to multi-jets contributions which are not of interest to this analysis. The fit range is chosen from $2.5<\Dphi<\pi$, similar to the phase-space used in a previous di-jet transverse momentum balance measurement~\cite{Chatrchyan:2014hqa}. The resulting width, which is defined as $RMS =  \sqrt{2\tau^2 + \sigma^{2}}$, is then extracted and plotted in bins of \ystarone, \ystartwo, \ptone, and \pttwo.      

\FloatBarrier
\subsection{ Unfolding \Dphi\ Distributions }
When filling the truth and reconstructed distributions in either \pp\ or \pPb, the leading jet weights shown in Figure~\ref{fig:spectweights}, in addition to the \pT\ invariant \Dphi\ weights shown in Figures~\ref{fig:ppAllDphiWeights} and ~\ref{fig:pPbAllDphiWeights} for \pp\ and \pPb\ are applied as product. Using the re-weighted truth and reconstructed \Dphi\ distributions, along with the respective re-weighted response matrices, new correction factors are then derived using the bin-by-bin procedure described earlier. \Dphi\ distributions for truth, reconstructed, and unfolded \pp\ MC in two different bins of \ptone\ are shown in Figure~\ref{fig:ppUnfoldingMC}, along with the correction factors and respective response matrices. Similarly, two different \Dphi\ distributions for truth, reco, and unfolded \pPb\ MC distributions in two different bins of \ptone\ are shown in Figure~\ref{fig:pPbUnfoldingMC}, along with the correction factors and respective response matrices. 
All the \Dphi\ distributions from truth MC, unfolded reconstructed MC, and data, along with correction factors are shown in Appendix~\ref{sec:appendixB}.

\begin{figure}[ht]
	\centerline{
		\begin{tabular}{cc}
			\includegraphics[width=0.5\textwidth]{output/output_pp_mc_pythia8/h_dPhi_unfolded_All_MUT_40_Ystar1_27_28_Pt1_35_28_Pt2_35_40_Ystar2_27.pdf} &
			\includegraphics[width=0.5\textwidth]{output/output_pp_mc_pythia8/h_dPhi_unfolded_All_MUT_40_Ystar1_27_35_Pt1_45_28_Pt2_35_40_Ystar2_27.pdf} \\
			\includegraphics[width=0.5\textwidth]{output/output_pp_mc_pythia8/h_dPhi_respMat_All_40_Ystar1_27_28_Pt1_35_28_Pt2_35_40_Ystar2_27.pdf} &
			\includegraphics[width=0.5\textwidth]{output/output_pp_mc_pythia8/h_dPhi_respMat_All_40_Ystar1_27_35_Pt1_45_28_Pt2_35_40_Ystar2_27.pdf} \\
		\end{tabular}
	}
	\caption{ \pp\ MC truth, reconstructed, and unfolded \Dphi\ distributions for two different bins of \ptone, with correction factors (top row) and respective response matrices (bottom row). }
	\label{fig:ppUnfoldingMC}
\end{figure}

\begin{figure}[ht]
	\centerline{
		\begin{tabular}{ccc}
			\includegraphics[width=0.5\textwidth]{output/output_pPb_mc_pythia8/h_dPhi_unfolded_All_MUT_40_Ystar1_27_28_Pt1_35_28_Pt2_35_40_Ystar2_27.pdf} &
			\includegraphics[width=0.5\textwidth]{output/output_pPb_mc_pythia8/h_dPhi_unfolded_All_MUT_40_Ystar1_27_35_Pt1_45_28_Pt2_35_40_Ystar2_27.pdf} \\
			\includegraphics[width=0.5\textwidth]{output/output_pPb_mc_pythia8/h_dPhi_respMat_All_40_Ystar1_27_28_Pt1_35_28_Pt2_35_40_Ystar2_27.pdf} &
			\includegraphics[width=0.5\textwidth]{output/output_pPb_mc_pythia8/h_dPhi_respMat_All_40_Ystar1_27_35_Pt1_45_28_Pt2_35_40_Ystar2_27.pdf} \\
		\end{tabular}
	}
	\caption{ \pPb\ MC truth, reconstructed, and unfolded \Dphi\ distributions for two different bins of \ptone, with correction factors (top row) and respective response matrices (bottom row). }
	\label{fig:pPbUnfoldingMC}
\end{figure} 

\FloatBarrier
\subsection{MC Closure Test}
As a check, the MC reconstructed results are unfolded using the derived correction factors. The comparison of the \pp\ MC truth and unfolded widths, and the respective ratios are shown in Figure~\ref{fig:ppwidthsTruthUF} in bins of \ptone\ and \pttwo. The ratios between unfolded and truth results are consistent with unity within statistical uncertainties indicating there is good closure between the unfolded and truth results.  Similarly, comparison of the \pPb\ MC truth and unfolded widths, and the respective ratios are shown in Figure~\ref{fig:pPbwidthsTruthUF} in bins of \ptone\ and \pttwo. The ratios between unfolded and truth results are consistent with unity within statistical uncertainties indicating there is good closure between the unfolded and truth results.  

The comparison of the \pp\ MC truth and unfolded yields, and the respective ratios are shown in Figure~\ref{fig:ppyieldsTruthUF} in bins of \ptone\ and \pttwo. The ratios between unfolded and truth results are consistent with unity within statistical uncertainties indicating there is good closure between the unfolded and truth results.  Similarly, comparison of the \pPb\ MC truth and unfolded yields, and the respective ratios are shown in Figure~\ref{fig:pPbyieldsTruthUF} in bins of \ptone\ and \pttwo. The ratios between unfolded and truth results are consistent with unity within statistical uncertainties indicating there is good closure between the unfolded and truth results.  

\begin{figure}[ht]
	\centerline{
		\begin{tabular}{ccc}
			\includegraphics[width=0.33\textwidth]{output/All/pp_mc_pythia8_0/h_dPhi_width_40_Ystar1_27_28_Pt1_35.pdf} &
			\includegraphics[width=0.33\textwidth]{output/All/pp_mc_pythia8_0/h_dPhi_width_40_Ystar1_27_35_Pt1_45.pdf} &
			\includegraphics[width=0.33\textwidth]{output/All/pp_mc_pythia8_0/h_dPhi_width_40_Ystar1_27_45_Pt1_90.pdf} \\
		\end{tabular}
	}
	\caption{ Comparison of widths from \Dphi\ fits between unfolded and truth results for the \pp\ MC. Ratios are consistent with unity, indicating good unfolding closure. }
	\label{fig:ppwidthsTruthUF}
\end{figure}


\begin{figure}[ht]
	\centerline{
		\begin{tabular}{ccc}
			\includegraphics[width=0.33\textwidth]{output/All/pPb_mc_pythia8_0/h_dPhi_width_40_Ystar1_27_28_Pt1_35.pdf} &
			\includegraphics[width=0.33\textwidth]{output/All/pPb_mc_pythia8_0/h_dPhi_width_40_Ystar1_27_35_Pt1_45.pdf} &
			\includegraphics[width=0.33\textwidth]{output/All/pPb_mc_pythia8_0/h_dPhi_width_40_Ystar1_27_45_Pt1_90.pdf} \\
		\end{tabular}
	}
	\caption{ Comparison of widths from \Dphi\ fits between unfolded and truth results for the \pPb\ MC. Ratios are consistent with unity, indicating good unfolding closure. }
	\label{fig:pPbwidthsTruthUF}
\end{figure}

\begin{figure}[ht]
	\centerline{
		\begin{tabular}{ccc}
			\includegraphics[width=0.33\textwidth]{output/All/pp_mc_pythia8_0/h_dPhi_yield_40_Ystar1_27_28_Pt1_35.pdf} &
			\includegraphics[width=0.33\textwidth]{output/All/pp_mc_pythia8_0/h_dPhi_yield_40_Ystar1_27_35_Pt1_45.pdf} &
			\includegraphics[width=0.33\textwidth]{output/All/pp_mc_pythia8_0/h_dPhi_yield_40_Ystar1_27_45_Pt1_90.pdf} \\
		\end{tabular}
	}
	\caption{ Comparison of yields from \Dphi\ distributions between unfolded and truth results for the \pp\ MC. Ratios are consistent with unity, indicating good unfolding closure. }
	\label{fig:ppyieldsTruthUF}
\end{figure}

\begin{figure}[ht]
	\centerline{
		\begin{tabular}{ccc}
			\includegraphics[width=0.33\textwidth]{output/All/pPb_mc_pythia8_0/h_dPhi_yield_40_Ystar1_27_28_Pt1_35.pdf} &
			\includegraphics[width=0.33\textwidth]{output/All/pPb_mc_pythia8_0/h_dPhi_yield_40_Ystar1_27_35_Pt1_45.pdf} &
			\includegraphics[width=0.33\textwidth]{output/All/pPb_mc_pythia8_0/h_dPhi_yield_40_Ystar1_27_45_Pt1_90.pdf} \\
		\end{tabular}
	}
	\caption{ Comparison of yields from \Dphi\ distributions between unfolded and truth results for the \pPb\ MC. Ratios are consistent with unity, indicating good unfolding closure. }
	\label{fig:pPbyieldsTruthUF}
\end{figure}


As an additional closure test, the jet \pT\ spectra and \Dphi\ correction factors derived from the \pythiaeight\ MC were applied to reconstructed jets from the \herwig\ MC. A comparison of unfolded to truth \conetwo\ and \ionetwo\ fromthe \pp\ \herwig\ are shown in Figure~\ref{fig:herwigpythiaclosure}. For \pPb\ there is no additional MC so this test was only done on the \pp\ MC. Ratios of unfolded to truth distributions indicate good closure. From Tables~\ref{tab:mcsamplespp},~\ref{tab:mcsamplesppherwig} it is clear that the statistics in the \pp\ \herwig\ MC is roughly 50\% of the \pp\ \pythiaeight\ MC, and the resulting fluctuations can be taken as statistical. 


\begin{figure}[ht]
	\centerline{
		\begin{tabular}{ccc}
			\includegraphics[width=0.33\textwidth]{output/All/pp_mc_herwig_0/h_dPhi_width_40_Ystar1_27_35_Pt1_45.pdf} &
			\includegraphics[width=0.33\textwidth]{output/All/pp_mc_herwig_0/h_dPhi_yield_40_Ystar1_27_35_Pt1_45.pdf} \\
		\end{tabular}
	}
	\caption{ Comparison of \conetwo\ (left) and \ionetwo\ (right) between unfolded and truth results for the \pp\ \herwig\ MC. Unfolding is done using correction factors derived from the \pythiaeight\ MC. Ratios are consistent with unity, indicating good unfolding closure. }
	\label{fig:herwigpythiaclosure}
\end{figure}

\FloatBarrier